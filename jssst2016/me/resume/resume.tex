% 以下の3行は変更しないこと.
\documentclass[T]{compsoft}
\taikai{2016}
\pagestyle {empty}

\usepackage[dvipdfmx]{graphicx}

% ユーザが定義したマクロなどはここに置く.ただし学会誌のスタイルの
% 再定義は原則として避けること.

\begin{document}

% 論文のタイトル
\title{多段階 let 挿入を行うコード生成言語の型システムの設計}

% 著者
% 和文論文の場合,姓と名の間には半角スペースを入れ,
% 複数の著者の間は全角スペースで区切る
%
\author{大石 純平 亀山 幸義
%
% ここにタイトル英訳 (英文の場合は和訳) を書く.
%
\ejtitle{let insertion language and type system}
%
% ここに著者英文表記 (英文の場合は和文表記) および
% 所属 (和文および英文) を書く.
% 複数著者の所属はまとめてよい.
%
\shozoku{Junpei Ohishi}{筑波大学システム情報学科コンピュータ・サイエンス専攻}%
{Dept.\ of Information and Computer Science, Tuskuba University}}

% 和文アブストラクト
\Jabstract{%
  コード生成法は,プログラムの実行性能の高さと保守性・再利用性を両立でき
るプログラミング手法として有力なものである.本研究は,コード生成法で必
要とされる「多段階let挿入」等を簡潔に表現できるコントロールオペレータ
である shift0/reset0を持つコード生成言語とその型システムを構築し,
生成されたコードの型安全性を保証する.多段階let挿入は,入れ子になった
forループを飛び越えたコード移動を許す仕組みであり,ループ不変式の移動
などのために必要である.コード生成言語の型安全性に関して,破壊的代入
を持つ体系に対するSudoらの研究等があるが,本研究は,彼らの環境識別子
に対する小さな拡張により,shift0/reset0 に対する型システムが構築で
きることを示した.
}

\maketitle \thispagestyle {empty}

\section{はじめに}
コード生成法は,プログラムの生産性・保守性と実行性能の高さを両立させられるプログラミング手法として有力なものである.
% 本研究室では,コード生成法をサポートするプログラム言語の信頼性・安全性を高める研究を行ってきている.
本研究は,コード生成法で必要とされる「多段階let挿入」等を簡潔に表現できるコントロールオペレータである shift0/reset0を持つコード生成言語とその型システムを構築し,生成されたコードの型安全性を保証する.

% すなわち,プログラム生成を行うことによって生成されるプログラムが安全に
% 実行されることを,プログラムの生成段階より早い段階,
% すなわちプログラム生成を行うプログラムのコンパイル段階で検査することの
% できる言語体系およびシステムを構築することを目標としている.

多段階let挿入は,入れ子になったforループを飛び越えたコード移動を許す仕組みであり,ループ不変式の移動などのために必要である.

ここでいう安全性は,構文的に正しいプログラムであること,
文字列同士の加算や乗算を決して行わない等の通常の型安全性を満たすことのほか,
自由変数やプログラム特化後において利用できない変数に依存したプログラム
を生成しないという,変数や変数スコープに関する性質を含む概念である.

この研究での大きな課題は,従来のコード生成のためのプログラミング言語の多くが,純粋なラムダ計算に基づく関数型プログラミング言語を基礎としており,効率の良いコードを生成する多くの技法をカバーしていないことである.これを克服する体系,すなわち,効率良いプログラムを記述するための表現力を高めつつ,安全性が保証された体系が求められている.

本研究の目的は,安全性が厳密に保証される計算体系の理論を構築し,さらにそれを実現する処理系を実装することを目的とする.このため,比較的最近になって理論的性質が解明されつつあるshift0/reset0 というコントロールオペレータに着目し,これをコード生成の体系に追加して得られた体系を構築して,上記の課題を解決することを狙いとする.

コード生成言語の型安全性に関して,破壊的代入を持つ体系に対する須藤らの研究\cite{Sudo2014}等があるが,本研究は,彼らの環境識別子に対する小さな拡張により,shift0/reset0 に対する型システムが構築できることを示す.

\section{関連研究}


\section{まとめと今後の課題}

{\bf 謝辞}\
ほげほげ

\bibliographystyle {jssst}
\bibliography {../../../bib/bibfile}
% Sudo2014のbibfile の書き方を note 使ってるのを変更する必要あり
\end{document}


%%% Local Variables:
%%% mode: japanese-latex
%%% TeX-master: t
%%% End:
