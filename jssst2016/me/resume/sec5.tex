% section 5

\section{対象言語: 構文と意味論}

本研究における対象言語は,ラムダ計算にコード生成機能とコントロールオペ
レータshift0/reset0を追加したものに型システムを導入したものである.

本稿では,最小限の言語のみについて考えるため,コード生成機能の
「ステージ(段階)」は,コード生成段階(レベル0,現在ステージ)と
生成されたコードの実行段階(レベル1,将来ステージ)の2ステージのみを考える.

前述したように,本研究の言語では,
コードコンビネータ(Code Combinator)方式を使い,
コードコンビネータは,
$\cPlus$ や $\cIf$のように下線を引いて表す.

\subsection{構文の定義}

対象言語の構文を定義する.

変数は,レベル0変数($x$), レベル1変数($u$),
(レベル0の)継続変数($k$)の3種類がある.
レベル0項($e^0$),レベル1項($e^1$)およびレベル0の値($v$)を下の通り定義する.

\begin{align*}
  c & ::= i \mid b \mid \cint
         \mid \cat \mid + \mid \cPlus \mid \cIf \\
  v & ::= x \mid c \mid \fun{x}{e^0} \mid \code{e^1} \\
  e^0 & ::=  v  \mid e^0~ e^0 \mid \ift{e^0}{e^0}{e^0} \\
    & \mid \cfun{x}{e^0}
      \mid \ccfun{u}{e^0} \\
    & \mid \resetz{e^0}
      \mid \shiftz{k}{e^0}
      \mid \throw{k}{v} \\
  e^1 & ::=  u \mid c \mid \fun{u}{e^1} \mid e^1~ e^1
      \mid \ift{e^1}{e^1}{e^1} \\
\end{align*}
ここで$i$は整数の定数,$b$は真理値定数である.

定数のうち,下線がついているものはコードコンビネータである.
変数は,ラムダ抽象(下線なし,下線つき,二重下線つき)および shift0 により束縛され,
$\alpha$同値な項は同一視する.
$\letin{x}{e_1}{e_2}$および$\clet{x}{e_1}{e_2}$は,
それぞれ,$(\fun{x}{e_2})e_1$
$(\cfun{x}{e_2})\cat e_1$の省略形である.
前述の例でのべた$\cFor$は,
コード構築定数とコードレベル適用$\cap$を用いて導入することとし,
(この導入にあたっての型システムの拡張は容易なので)ここでは省略する.

\subsection{操作的意味論}

対象言語は,値呼びで left-to-rightの操作的意味論を持つ.
ここでは評価文脈に基づく定義を与える.

評価文脈を以下のように定義する.
\begin{align*}
  E & ::= [~] \mid E~ e^0 \mid v~ E \\
    & \mid \ift{E}{e^0}{e^0} \mid \Resetz~ E \mid \ccfun{u}{E}
\end{align*}
コード生成言語で特徴的なことは,
コードレベルのラムダ抽象の内部で評価が進行する点である.実際,
上記の定義には,$\ccfun{u}{E}$が含まれている.
たとえば,$\ccfun{u}{u \cPlus [~]}$ は評価文脈である.

この評価文脈$E$と次に述べる計算規則$r \to l$ により,
評価関係$e \lto e'$ を次のように定義する.
\[
  \infer{E[r] \lto E[l]}{r \to l}
\]

計算規則は以下の通り定義する.
\begin{align*}
  (\fun{x}{e})~v &\to e\{ x := v \} \\
  \ift{true}{e_1}{e_2} &\to e_1 \\
  \ift{else}{e_1}{e_2} &\to e_2 \\
  \cfun{x}{e} &\to \ccfun{u}{(e\{ x := \code{u} \})} \\
  \ccfun{u}{\code{e}} &\to \code{\fun{u}{e}} \\
  \Resetz~ v &\to v \\
  \Resetz (E[\Shiftz~ k \to e]) &\to e \ksubst{k}{E}
\end{align*}
ただし,4行目の$u$はフレッシュなコードレベル変数とし,
最後の行の$E$は穴の周りに{\Resetz}を含まない評価文脈とする.
また,この行の右辺のトップレベルに{\Resetz}がない点が,
shift/reset の振舞いとの違いである.すなわち,shift0 を1回計算すると,
reset0 が1つはずれるため,shift0 をN個入れ子にすることにより,
N個分外側のreset0 までアクセスすることができ,多段階let挿入を実現でき
るようになる.

上記における継続変数に対する代入$e\ksubst{k}{E}$は次の通り定義する.
\begin{align*}
  (\throw{k}{v})\ksubst{k}{E} &\equiv \Resetz (E[v]) \\
  (\throw{k'}{v})\ksubst{k}{E} &\equiv \throw{k'}{(v\ksubst{k}{E})}
\\
& \text{ただし}~k \not= k'
\end{align*}
上記以外の$e$に対する代入の定義は透過的であるとする.
上記の定義の1行目で\Resetz を挿入しているのは{\Shiftz}の意味論に対応し
ており,これを挿入しない場合は別のコントロールオペレータ(Felleisenの
control/promptに類似した control0/prompt0)の振舞いとなる.

コードコンビネータ定数の振舞い(ラムダ計算における$\delta$規則に相当)は
以下のように定義する.

\begin{align*}
  \cint~ n &\to \code{n} \\
  \code{e_1}~ \cat~ \code{e_2} &\to \code{e_1~ e_2} \\
  \code{e_1}~ \cPlus~ \code{e_2} &\to \code{e_1 + e_2} \\
  \cif{e_1}{e_2}{e_3} &\to \code{\ift{e_1}{e_2}{e_3}}
\end{align*}
%計算の例を以下に示す.
%\begin{align*}
%  e_1 & = \Resetz ~~\cLet~x_1=\csp{3}~\cIn \\
%      & \phantom{=}~~ \Resetz ~~\cLet~x_2=\csp{5}~\cIn \\
%      & \phantom{=}~~ \Shiftz~k~\to~\cLet~y=t~\cIn \\
%      & \phantom{=}~~ \Throw~ k~ (x_1~\cPlus~x_2~\cPlus~y) \\
%\end{align*}
%
%\begin{align*}
%  [ e_1 ] &\lto [ \Resetz (\cLet~x_1=\csp{3}~\cIn \\
%          &\Resetz~ \cLet~x_2=\csp{5}~\cIn \\
%          &[ \Shiftz~ k~ \to~ \cLet~ y=t~ \cIn \\
%          &[ \Throw~ k~(x_1~\cPlus~x_2~\cPlus~y) ] ] ) ] \\
%          &\lto [ \cLet~ y=t~ \cIn \\
%          &[ \cfun{x}{\Resetz~ (\cLet~x_1=\csp{3}~ \cIn~ \Resetz~ (\cLet~ x_2=\csp{5}~ \cIn [x]))} (x_1~\cPlus~x_2~\cPlus~y) ]] \\
%          &\lto [ \cfun{y}{(\cfun{x}{\Resetz~ (\cLet~x_1=\csp{3}~ \cIn~ \Resetz~ (\cLet~ x_2=\csp{5}~ \cIn [x]))} (x_1~\cPlus~x_2~\cPlus~y))}~ \cat~ t ] \\
%          &\lto [[\cfun{y}{(\cfun{x}{\Resetz~ (\cLet~x_1=\csp{3}~ \cIn~ \Resetz~ (\cLet~ x_2=\csp{5}~ \cIn [x]))} (x_1~\cPlus~x_2~\cPlus~y))}]~ \cat~ t] \\
%          &\lto [[\ccfun{y_1}{(\cfun{x}{\Resetz~ (\cLet~x_1=\csp{3}~ \cIn~ \Resetz~ (\cLet~ x_2=\csp{5}~ \cIn [x]))} (x_1~\cPlus~x_2~\cPlus~ \code{y_1}))}]~ \cat~ t] \\
%          &\lto
%\end{align*}


\section{型システム}

本研究での型システムについて述べる.

基本型$b$,環境識別子(Environment Classifier)$\gamma$を以下の通り定義する.
\begin{align*}
  b & ::= \intT \mid \boolT \\
  \gamma & ::= \gamma_x \mid \gamma \cup \gamma
\end{align*}
$\gamma$の定義における$\gamma_x$は環境識別子の変数を表す.
すなわち,環境識別子は,変数であるかそれらを$\cup$で結合した形である.
以下では,メタ変数と変数を区別せず$\gamma_x$を$\gamma$と表記する.
ここで環境識別子として$\cup$を導入した理由は後述する.

$L ::= \empty \mid \gamma$ は現在ステージと将来ステージをまとめ
て表す記号である.たとえば,$\Gamma \vdash^L
e:t~;~\sigma$は,$L=\empty$のとき現在ステージの判断で,
$L=\gamma$のとき将来ステージの判断となる.

レベル0の型$t^0$,レベル1の型$t^1$,(レベル0の)型の有限列$\sigma$,
(レベル0の)継続の型$\kappa$を次の通り定義する.
\begin{align*}
  t^0 & ::= b \mid \funT{t^0}{t^0}{\sigma} \mid \codeT{t^1}{\gamma} \\
  t^1 & ::= b \mid t^1 \to t^1 \\
  \sigma & ::= \epsilon \mid t^0, \sigma \\
  \kappa^0 & ::= \contT{\codeT{t^1}{\gamma}}{\codeT{t^1}{\gamma}}{\sigma}
\end{align*}

レベル0の関数型$\funT{t^0}{t^0}{\sigma}$は,
エフェクトをあらわす列$\sigma$を伴っている.これは,その関数型をもつ項
を引数に適用したときに生じる計算エフェクトであり,具体的には,
\Shiftz の answer type の列である.前述したようにshift0 は多段
階の reset0 にアクセスできるため,$n$個先のreset0 の answer typeまで記
憶するため,このように型の列$\sigma$で表現している.
ただし,本研究の範囲では,answer type modification に対応する必要はな
いので,エフェクトはシンプルに型の列($n$個先の reset0 のanswer type を
$n=1,\cdots,k$に対して並べた列)で表現している.
この型システムの詳細は,Materzokら\cite{Materzok2011}の研究を参照されたい.

本稿の範囲では,コントロールオペレータは現在ステージのみにあらわれ,生
成されるコードの中にはあらわないため,レベル1の関数型は,エフェクトを
表す列を持たない.
また,本項では,shift0/reset0 はコードを操作する目的にのみ使うため,継
続の型は,コードからコードへの関数の形をしている.
ここでは,後の定義を簡略化するため,継続を,通常の関数とは区別しており,
そのため,継続の型も通常の関数の型とは区別して二重の横線で表現している.

型判断は,以下の2つの形である.
\begin{align*}
  \Gamma \vdash^{L} e : t ;~\sigma \\
  \Gamma \models \gamma \ord \gamma
\end{align*}
ここで,型文脈$\Gamma$は次のように定義される.
\begin{align*}
  \Gamma ::= \emptyset
  \mid \Gamma, (\gamma \ord \gamma)
  \mid \Gamma, (x : t)
  \mid \Gamma, (u : t)^{\gamma}
\end{align*}

型判断の導出規則を与える.まず,$\Gamma \models \gamma \ord \gamma$の
形に対する規則である.

\[
  \infer
  {\Gamma \models \gamma_1 \ord \gamma_1}
  {}
\quad
  \infer
  {\Gamma, \gamma_1 \ord \gamma_2 \models \gamma_1 \ord \gamma_2}
  {}
\]

\[
  \infer
  {\Gamma \models \gamma_1 \ord \gamma_3}
  {\Gamma \models \gamma_1 \ord \gamma_2 & \Gamma \models \gamma_2 \ord \gamma_3}
\]

\[
  \infer
  {\Gamma \models \gamma_1 \cup \gamma_2 \ord \gamma_1}
  {}
\quad
  \infer
  {\Gamma \models \gamma_1 \cup \gamma_2 \ord \gamma_2}
  {}
\]

\[
  \infer
  {\Gamma \models \gamma_3 \ord \gamma_1 \cup \gamma_2}
  {\Gamma \models \gamma_3 \ord \gamma_1
  &\Gamma \models \gamma_3 \ord \gamma_2}
\]



次に,$\Gamma \vdash^{L} e : t ;~\sigma$ の形に対する導出規則を与える.
まずは,レベル0における単純な規則である.

\[
  \infer
  {\Gamma, x : t \vdash x : t ~;~ \sigma}
  {}
\quad
  \infer
  {\Gamma, (u : t)^\gamma \vdash^\gamma u : t ~;~ \sigma}
  {}
\]

\[
  \infer
  {\Gamma \vdash^{L} c : t^c ~;~\sigma}
  {}
\]

\[
  \infer
  {\Gamma \vdash^{L} e_1~ e_2 : t_1 ; \sigma}
  {\Gamma \vdash^{L} e_1 : t_2 \to t_1 ; \sigma
    & \Gamma \vdash^{L} e_2 : t_2  ; \sigma
  }
\]

\[
  \infer
  {\Gamma \vdash \fun{x}{e} : \funT{t_1}{t_2}{\sigma} ~;~\sigma'}
  {\Gamma,~x : t_1 \vdash e : t_2 ~;~ \sigma}
\quad
  \infer
  {\Gamma \vdash^\gamma \fun{x}{e} : \funT{t_1}{t_2}{} ~;~\sigma'}
  {\Gamma,~(u : t_1)^\gamma \vdash^\gamma e : t_2 ~;~ \sigma}
\]

\[
  \infer
  {\Gamma \vdash^{L} \ift{e_1}{e_2}{e_3} : t ~;~ \sigma}
  {\Gamma \vdash^{L} e_1 : \boolT ;~ \sigma
    & \Gamma \vdash^{L} e_2 : t ; \sigma
    & \Gamma \vdash^{L} e_3 : t ; \sigma}
\]

次にコードレベル変数に関するラムダ抽象の規則である.

\[
  \infer[(\gamma_1~\text{is eigen var})]
  {\Gamma \vdash \cfun{x}{e} : \codeT{t_1\to t_2}{\gamma} ~;~ \sigma}
  {\Gamma,~\gamma_1 \ord \gamma,~x:\codeT{t_1}{\gamma_1} \vdash e
    : \codeT{t_2}{\gamma_1}; \sigma}
\]

\[
  \infer
  {\Gamma \vdash \ccfun{u^1}{e} : \codeT{t_1 \to t_2}{\gamma} ; \sigma}
  {\Gamma, \gamma_1 \ord \gamma, x : (u : t_1)^{\gamma_1} \vdash e : \codeT{t_2}{\gamma_1} ; \sigma}
\]

コントロールオペレータに対する型導出規則である.

\[
  \infer{\Gamma \vdash \resetz{e} : \codeT{t}{\gamma} ~;~ \sigma}
  {\Gamma \vdash e : \codeT{t}{\gamma} ~;~ \codeT{t}{\gamma}, \sigma}
\]

\[
  \infer{\Gamma \vdash \shiftz{k}{e} : \codeT{t_1}{\gamma_1} ~;~ \codeT{t_0}{\gamma_0},\sigma}
  {\Gamma,~k:\contT{\codeT{t_1}{\gamma_1}}{\codeT{t_0}{\gamma_0}}{\sigma}
    \vdash e : \codeT{t_0}{\gamma_0} ; \sigma
    & \Gamma \models \gamma_1 \ord \gamma_0
  }
\]

\[
  \infer
  {\Gamma,~k:\contT{\codeT{t_1}{\gamma_1}}{\codeT{t_0}{\gamma_0}}{\sigma}
    \vdash \throw{k}{v} : \codeT{t_0}{\gamma_2} ; \sigma}
  {\Gamma
    \vdash v : \codeT{t_1}{\gamma_1 \cup \gamma_2} ; \sigma
    & \Gamma \models \gamma_2 \ord \gamma_0
  }
\]

コード生成に関する補助的な規則として,Subsumptionに相当する規則等がある.
\[
  \infer
  {\Gamma \vdash e : \codeT{t}{\gamma_2} ; \sigma}
  {\Gamma \vdash e : \codeT{t}{\gamma_1} ; \sigma
    & \Gamma \models \gamma_2 \ord \gamma_1
  }
\]

\[
  \infer
  {\Gamma \vdash^{\gamma_2} e : t ~;~ \sigma}
  {\Gamma \vdash^{\gamma_1} e : t ~;~ \sigma
    & \Gamma \models \gamma_2 \ord \gamma_1
  }
\]


\[
  \infer
  {\Gamma \vdash \code{e} : \codeT{t^1}{\gamma} ; \sigma}
  {\Gamma \vdash^{\gamma} e : t^1 ; \sigma}
\]


\subsection{型付け例}

上記の型システムのもとで,いくつかの項の型付けについて述べる.

\begin{align*}
  e_1 & = \Resetz ~~\cLet~x_1=\csp{3}~\cIn \\
      & \phantom{=}~~ \Resetz ~~\cLet~x_2=\csp{5}~\cIn \\
      & \phantom{=}~~ \Shiftz~k~\to~\cLet~y=t~\cIn \\
      & \phantom{=}~~ \Throw~k~(x_1~\cPlus~x_2~\cPlus~y)
\end{align*}

この式$e_1$に対して,もし,$t=\csp{7}$ あるいは $t=x_1$であれば,
$e_1$ は型付け可能である.
一方,$t=x_2$ であれば,$e_1$ は型付けできない.

\begin{align*}
  e_2 & = \Resetz ~~\cLet~x_1=\csp{3}~\cIn \\
      & \phantom{=}~~ \Resetz ~~\cLet~x_2=\csp{5}~\cIn \\
      & \phantom{=}~~ \Shiftz~k_2~\to~ \Shiftz~k_1~\to~ \cLet~y=t~\cIn \\
      & \phantom{=}~~ \Throw~k_1~(\Throw~k_2~(x_1~\cPlus~x_2~\cPlus~y))
\end{align*}

この式$e_2$に対して,もし$t=\csp{7}$であれば$e_1$は型付け可能である.
一方,$t=x_2$ あるいは $t=x_1$であれば,$e_1$は型付けできない.

このように,(少なくとも)上記の例については安全な式と危険な式を正しく峻
別できていることがわかった.

\subsection{型安全性について}

本研究の型システムに対する型保存(Subject Reduction)定理について述べる.
型保存定理は,(証明できれば)
進行(Progress)定理とあわせて型システムの健全性を導く定理である.

\begin{quote}
(型保存性)
$\vdash e:t~;~\sigma$ かつ $e \lto e'$ であれば,$\vdash e':t~;~\sigma$
である.
\end{quote}

この定理は reset0-shift0の計算規則が多相性を持たない場合には容易に証明
できるが,多相性については精密な扱いが必要であり,
現段階では,型保存定理の証明は進行中である.

%\begin{lemm}[不要な仮定の除去]
%  $\Gamma_1,\gamma_2 \ord \gamma_1 \vdash e : t_1 ~;~\sigma$
%  かつ,$\gamma_2$が $\Gamma_1, e, t_1, \sigma$に出現しないなら,
%  $\Gamma_1 \vdash e : t_1 ~;~\sigma$ である.
%\end{lemm}
%
%\begin{lemm}[値に関する性質]
%  $\Gamma_1 \vdash v : t_1 ~;~\sigma$
%  ならば,
%  $\Gamma_1 \vdash v : t_1 ~;~\sigma'$
%  である.
%\end{lemm}
%
%\begin{lemm}[代入]
%  $\Gamma_1, \Gamma_2, x : t_1 \vdash e : t_2 ~;~\sigma$
%  かつ
%  $\Gamma_1 \vdash v : t_1 ~;~\sigma$
%  ならば,
%  $\Gamma_1, \Gamma_2 \vdash e\{x := v\} : t_2~;~\sigma$
%\end{lemm}
%
%これらをもとに型保存定理を証明する.
%本研究の対象言語は,コントロールオペレータが操作する対象となる式の型を
%コード型に限定するなど,注意深く設計しているので,ほとんどのケースの証
%明はスムーズであるが,reset0-shift0 に関する計算規則(shift0 が評価文脈
%を捕捉して継続変数$k$に渡す規則)とthrowに関する計算規則では,
%サブタイプ多相性に相当する性質を使っているので,以下の技術的な補題が必
%要である.
%
%\begin{lemm}[識別子に関する多相性]
%  穴の周りにreset0を含まない評価文脈$E$,変数$x$,
%  そして$\Gamma = (u_1:t_1)^{\gamma_1}, \cdots, (u_n:t_n)^{\gamma_n}$
%      かつ$i=1,\cdots,n$に対して$\Gamma \models \gamma_0 \ord \gamma_i$であるとする.
%  このとき,
%  $\Gamma, x:\codeT{t_0}{\gamma'} \vdash E[x] : \codeT{t_1}{\gamma_0} ~;~\sigma$
%  であれば,フレッシュな$\gamma$に対して,
%  $\Gamma, x:\codeT{t_0}{\gamma'\cup \gamma} \vdash
%   E[x] : \codeT{t_1}{\gamma_0 \cup \gamma} ~;~\sigma$
%  である.
%\end{lemm}
%
%この補題は,評価文脈$E$に対して,穴の型が$\codeT{t_0}{\gamma'}$で
%評価文脈全体の型が$\codeT{t_1}{\gamma_0}$であれば,
%それぞれの環境識別子に$\gamma_2$を加えて,
%$\codeT{t_0}{\gamma'\cup \gamma}$型から,
%$\codeT{t_1}{\gamma_0\cup \gamma}$型への評価文脈として使ってもよい,
%ということを主張している.ここで $\gamma \ord \gamma_0$ なので,
%$\gamma$と$\gamma_0 \cup \gamma$は$\ord$の意味で等しくなり,
%$\codeT{t_1}{\gamma_0\cup\gamma}$型を持つ項は,
%$\codeT{t_1}{\gamma}$型も持つことがわかる.
%この定理により,shift0が捕捉した継続を(環境識別子について)多相的に使う
%ことが可能となり,reset0-shift0 の計算規則が正当化される.
%
%上記の補題を証明すれば,型保存定理の証明の残りのケースは比較的容易であ
%る.なお,この補題を使うケースにおいて,定理の言明にあらわれる項$e$が
%閉じた項であること(環境識別子に関する
%
%
%進行定理については
%精密な定式化が必要(reset0がない式でshift0を実行した時など)が必要なので,

%\subsection{進行}
%\begin{theo}[進行]
%  $\vdash e:t$ が導出可能であれば,$e$ は 値 $v$ である.または,$e \lto e'$ であるような 項 $e'$ が存在する
%\end{theo}
%
%\paragraph{証明}
%$\vdash e:t$ の導出に関する帰納法による.\\
%Const, Abs, Code 規則の場合 $e$ は値である.\\
%Var 規則の場合 $\vdash e:t$ は導出可能でない.\\
%Throw 規則の場合 $\vdash e:t$ は導出可能でない.\\
%Reset0 規則の場合 $e = \Resetz~ e_1$ とする.
%帰納法の仮定より評価文脈における $\Resetz E$ より簡約が進み,\\
%$e_1$ が値のとき,$e \lto v$ となるような $v$ が存在する.\\
%$e_1$ が値でないとき,

%%% Local Variables:
%%% mode: latex
%%% TeX-master: "paper"
%%% End: