% section 2

\section{コード生成とlet挿入}

コード生成、すなわち、プログラムによるプログラム(コード)の生成の手法は、
対象領域に関する知識、実行環境、利用可能な計算機リソースなどのパラメー
タに特化した(実行性能の高い)プログラムを生成する目的で広く利用されている。
生成されるコードを文字列として表現する素朴なコード生成法では、
構文エラーなどのエラーを含むコードを生成してしまう危険があり、さらに、
生成されたコードのディバッグが非常に困難であるという問題がある。

これらの問題を解決するため、
コード生成器(コード生成をするプログラム)を記述するためのプログラム言語
の研究が行わており、特に静的な型システムのサポートを持つ言語として、
MetaOCaml, Template Haskell, Scala LMSなどがある。


\kam{ここにコード例をいっぱいかく}

\subsection{プログラム例}
本研究におけるプログラムとどのように簡約が行われるかの説明を掲載する.
言語,および意味論については後に説明を行う.

\begin{align}
  & (\cint 3)~ \cPlus~ (\cint 5) \\
  & \too \code{3 + 5} \nonumber
\end{align}
$3$ と $5$ がそれぞれ,$\cint$ により,整数のコード $\code{3},\code{5}$ に簡約され,
$\cPlus$ により,それらを足しあわせ,$\code{3 + 5}$ というコードが得られる.

\begin{align}
  & \cLet~ x = (\cint{3})~ \cIn~ x~ \cPlus~ (\cint 7)  \\
  & \too \code{\Let~ x = 3~ \In~ x + 7} \nonumber
\end{align}


\begin{align}
  & (\cfun{x}{x~ \cPlus~ (\cint 7))~}{(\cint 3)} \\
  & \too \code{\Let~ x = 3~ \In~ x + 7} \nonumber
\end{align}

\begin{align}
  & (\cfun{y}{\Let~ x = y~ \In~ \cfun{y}{x~ \cPlus~ y}{}}{}) \\
  & \too \code{\fun{y}{\cfun{y'}{}{}}{y~ \cPlus~ y'}} \nonumber
\end{align}

以下で,我々の言語体系におけるshift0/reset0 による多段階let挿入の例を掲載する.

\begin{align*}
    e &= \red{\Resetz} ~~\cLet~x_1=\csp{3}~\cIn \\
      &\phantom{=}~~ \blue{\Resetz} ~~\cLet~x_2=\csp{5}~\cIn \\
      &\phantom{=}~~ \blue{\Shiftz}~\blue{k_2}~\to~ \red{\Shiftz}~\red{k_1}~\to~ \magenta{\cLet~y=t~\cIn} \\
      &\phantom{=}~~ \cThrow~\red{k_1}~(\cThrow~\blue{k_2}~(x_1~\cPlus~x_2~\cPlus~y))
\end{align*}
とする.

$e$ を計算すると,
$\blue{\Resetz}$によって,切り取られた継続 $\cLet~x_2=\csp{5}~\cIn$ が,
$\blue{\Shiftz}$ によって,$\blue{k_2}$へと捕獲され,
次に,
$\red{\Resetz}$によって,切り取られた継続 $\cLet~x_2=\csp{3}~\cIn$ が,
$\red{\Shiftz}$ によって,$\red{k_1}$へと捕獲される.

わかりやすいところまで計算を進めると以下のようになり,
\begin{align*}
  e & \too \magenta{\cLet~y=t~\cIn} \\
    & \phantom{\too}~~ \cThrow~\red{k_1}~(\cThrow~\blue{k_2}~(x_1~\cPlus~x_2~\cPlus~y))
\end{align*}

$\magenta{\cLet~y=t~\cIn}$ がトップに挿入されたことが分かる.
$\cThrow$ は,切り取られた継続を引数に適用するための演算子である.
つまり,
\begin{align*}
  e & \too \magenta{\cLet~y=t~\cIn} \\
    & \cLet~x_1=\csp{3}~\cIn \\
    & \cLet~x_2=\csp{5}~\cIn \\
    & (x_1~\cPlus~x_2~\cPlus~y)
\end{align*}

となり,$\magenta{\cLet~y=t~\cIn}$ が 二重の $\cLet$を飛び越えて,挿入された事が分かる.
これが多段階 let 挿入である.

また, 項 $t$ の種類によっては,型が付いていてほしくない場合と付いて欲しい場合とがある.
例えば,$t$ が $\code{7}$ のときは,型がつき,
$t$ が $x_1$ や $x_2$ のとき型が付かないようにしたい.
つまり,この例においては,項 $t$ の種類によって,安全なコードか,安全でないコードかが変わるので,それを型で判断したい.このような型システムを構築することを考える.

%%% Local Variables:
%%% mode: latex
%%% TeX-master: "paper"
%%% End:
