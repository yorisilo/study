% section 2

\section{コード生成とlet挿入}

コード生成、すなわち、プログラムによるプログラム(コード)の生成の手法は、
対象領域に関する知識、実行環境、利用可能な計算機リソースなどのパラメー
タに特化した(実行性能の高い)プログラムを生成する目的で広く利用されている。
生成されるコードを文字列として表現する素朴なコード生成法では、
構文エラーなどのエラーを含むコードを生成してしまう危険があり、さらに、
生成されたコードのディバッグが非常に困難であるという問題がある。

これらの問題を解決するため、
コード生成器(コード生成をするプログラム)を記述するためのプログラム言語
の研究が行わており、特に静的な型システムのサポートを持つ言語として、
MetaOCaml, Template Haskell, Scala LMSなどがある。


\kam{ここにコード例をいっぱいかく}

