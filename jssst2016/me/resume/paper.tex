% 以下の3行は変更しないこと.
\documentclass[T]{compsoft}
\taikai{2016}
\pagestyle {empty}

\usepackage[dvipdfmx]{graphicx,color}
\usepackage{ascmac}

\usepackage{theorem}
\usepackage{amsmath,amssymb}
\usepackage{ascmac}
\usepackage{mathtools}
\usepackage{proof}
\usepackage{stmaryrd}
\usepackage{listings,jlisting}

% ユーザが定義したマクロなどはここに置く.ただし学会誌のスタイルの
% 再定義は原則として避けること.

\definecolor{DarkGreen}{rgb}{0,0.5,0}
\definecolor{Magenta}{rgb}{1.0, 0.0, 1.0}

\newcommand\too{\leadsto^*}
\newcommand\pink[1]{\textcolor{pink}{#1}}
\newcommand\red[1]{\textcolor{red}{#1}}
\newcommand\green[1]{\textcolor{green}{#1}}
\newcommand\magenta[1]{\textcolor{magenta}{#1}}
\newcommand\blue[1]{\textcolor{blue}{#1}}

\newcommand\fun[2]{\lambda{#1}.{#2}}

\newcommand\Resetz{\textbf{reset0}}
\newcommand\Shiftz{\textbf{shift0}}
\newcommand\Throw{\textbf{throw}}
\newcommand\resetz[1]{\Resetz~{#1}}
\newcommand\shiftz[2]{\Shiftz~{#1}\to{#2}}
\newcommand\throw[2]{\Throw~{#1}~{#2}}

\newcommand\cfun[2]{\underline{\lambda}{#1}.{#2}}
\newcommand\ccfun[2]{\underline{\underline{\lambda}}{#1}.{#2}}

\newcommand\cResetz{\underline{\textbf{reset0}}}
\newcommand\cShiftz{\underline{\textbf{shift0}}}
\newcommand\cThrow{\underline{\textbf{throw}}}
\newcommand\cresetz[1]{\cResetz~{#1}}
\newcommand\cshiftz[2]{\cShiftz~{#1}\to{#2}}
\newcommand\cthrow[2]{\cThrow~{#1}~{#2}}

\newcommand\cPlus{\underline{\textbf{+}}}

\newcommand\cLet{\underline{\textbf{let}}}
\newcommand\cIn{\underline{\textbf{in}}}
\newcommand\clet[3]{\cLet~{#1}={#2}~\cIn~{#3}}
\newcommand\csp[1]{\texttt{\%}{#1}}
\newcommand\cint{\underline{\textbf{int}}}
\newcommand\code[1]{\texttt{<}{#1}\texttt{>}}

\newcommand\intT{\mbox{\texttt{int}}}
\newcommand\boolT{\mbox{\texttt{bool}}}

\newcommand\codeT[2]{\langle{#1}\rangle^{#2}}
\newcommand\funT[3]{{#1} \stackrel{#3}{\rightarrow} {#2}}
\newcommand\contT[3]{{#1} \stackrel{#3}{\Rightarrow} {#2}}

\newcommand\ord{\ge}

\newcommand\Let{\textbf{let}}
\newcommand\In{\textbf{in}}
\newcommand\letin[3]{\Let~{#1}={#2}~\In~{#3}}

\newcommand\ift[3]{\textbf{if}~{#1}~\textbf{then}~{#2}~\textbf{else}~{#3}}
\newcommand\cif[3]{\underline{\textbf{if}}~\code{{#1}}~\code{{#2}}~\code{{#3}}}
\newcommand\cIf{\underline{\textbf{if}}}

\newcommand\fix{\textbf{fix}}
\newcommand\cfix{\underline{\textbf{fix}}}

\newcommand\lto{\leadsto}
\newcommand\cat{~\underline{@}~}

\newcommand\ksubst[2]{\{{#1}\Leftarrow{#2}\}}

\newcommand\forin[2]{\textbf{for}~{#1}~\textbf{to}~{#2}~\textbf{in}}
\newcommand\cforin[2]{\underline{\textbf{for}}~{#1}~\underline{\textbf{to}}~{#2}~\underline{\textbf{in}}}
\newcommand\cArray[1]{\underline{[{#1}]}}
\newcommand\cArrays[2]{\underline{[{#1}][{#2}]}}

% コメントマクロ
\newcommand\kam[1]{\red{{#1}}}
\newcommand\ooi[1]{\blue{{#1}}}

\theoremstyle{break}

\newtheorem{theo}{定理}[section]
\newtheorem{defi}{定義}[section]
\newtheorem{lemm}{補題}[section]

\begin{document}

% 論文のタイトル
\title{多段階 let 挿入を行うコード生成言語の型システムの設計}

% 著者
% 和文論文の場合,姓と名の間には半角スペースを入れ,
% 複数の著者の間は全角スペースで区切る
%
\author{大石 純平 亀山 幸義
%
% ここにタイトル英訳 (英文の場合は和訳) を書く.
%
\ejtitle{Type-Safe Code Generation with Multi-level Let-insertion}
%
% ここに著者英文表記 (英文の場合は和文表記) および
% 所属 (和文および英文) を書く.
% 複数著者の所属はまとめてよい.
%
\shozoku{Junpei Ohishi, Yukiyoshi Kameyama}{筑波大学システム情報工学研究科コンピュータ・サイエンス専攻}%
{Department of Computer Science, University of Tsukuba}}

% 和文アブストラクト
\Jabstract{%
コード生成法は,プログラムの実行性能の高さと保守性・再利用性を両立でき
るプログラミング手法として有力なものである.本研究は,コード生成法で必
要とされる「多段階let挿入」等を簡潔に表現できるコントロールオペレータ
である shift0/reset0を持つコード生成言語とその型システムを構築し,
生成されたコードの型安全性を保証する.多段階let挿入は,入れ子になった
forループを飛び越えたコード移動を許す仕組みであり,ループ不変式の移動
などのために必要である.コード生成言語の型安全性に関して,破壊的代入
を持つ体系に対するSudoらの研究等があるが,本研究は,彼らの環境識別子
に対する小さな拡張により,shift0/reset0 に対する型システムが構築で
きることを示した.
}

\maketitle \thispagestyle {empty}

% section 1

\section{はじめに}
コード生成法は,プログラムの生産性・保守性と実行性能の高さを両立させら
れるプログラミング手法として有力なものである.
本研究は,コード生成法で必要とされる「多段階let挿入」等を簡潔に表現で
きるコントロールオペレータである shift0/reset0を持つコード生成言語とそ
の型システムを構築し,生成されたコードの型安全性を静的に保証する言語体
系および型システムを設計する.
これにより,コード生成器のコンパイル段階,すなわち,実際にコードが生成
されてコンパイルされるより遥かに前の段階でのエラーの検出が可能となると
いう利点がある.

コード生成におけるlet挿入は,生成されたコードを移動して効率良いコード
に変形するための機能であり,ループ不変式をforループの外側に移動したり,
コードの計算結果を共有するなどのコード変換(コード最適化)において必要な機能である.
多段階let挿入は,入れ子になったforループ等を飛び越えて,コードを移動す
る機能である.

% ここでいう安全性は,構文的に正しいプログラムであること,
% 文字列同士の加算や乗算を決して行わない等の通常の型安全性を満たすことのほか,
% 自由変数やプログラム特化後において利用できない変数に依存したプログラム
% を生成しないという,変数や変数スコープに関する性質を含む概念である.

% この研究での大きな課題は,従来のコード生成のためのプログラミング言語の多くが,純粋なラムダ計算に基づく関数型プログラミング言語を基礎としており,効率の良いコードを生成する多くの技法をカバーしていないことである.これを克服する体系,すなわち,効率良いプログラムを記述するための表現力を高めつつ,安全性が保証された体系が求められている.

本研究は,
多段階let挿入を可能とするコード生成体系の構築のため,
比較的最近になって理論的性質が解明されたshift0/reset0\cite{Materzok2011} という
コントロールオペレータに着目する.
このコントロールオペレータに対する型規則を適切に設計することにより,
型安全性を解決することを目的とする.
コントロールオペレータを含む項の計算について分析した結果、
スコープの包含関係が逆転することや2つのスコープの合流があることから、
変数スコープを表す識別子にジョイン(和集合)を追加すればよいという着想を
得て、型システムを設計することに成功した。

本研究に関連した従来研究としては,
束縛子を越えない範囲でのコントロールオペレータを許した研究や,
局所的な代入可能変数を持つ体系に対する須藤らの研究\cite{Sudo2014},
後者を,グローバルな代入可能変数を持つ体系に拡張した研究
\cite{Aplas2016}などがある.
しかし,いずれの研究でも 多段階のforループを飛び越えたlet挿入は許していない.
本研究は,須藤らの研究をベースに,
shift0/reset0 を持つコード生成体系を設計した点に新規性がある.

% section 2

\section{コード生成とlet挿入}

コード生成、すなわち、プログラムによるプログラム(コード)の生成の手法は、
対象領域に関する知識、実行環境、利用可能な計算機リソースなどのパラメー
タに特化した(実行性能の高い)プログラムを生成する目的で広く利用されている。
生成されるコードを文字列として表現する素朴なコード生成法では、
構文エラーなどのエラーを含むコードを生成してしまう危険があり、さらに、
生成されたコードのディバッグが非常に困難であるという問題がある。

これらの問題を解決するため、
コード生成器(コード生成をするプログラム)を記述するためのプログラム言語
の研究が行わており、特に静的な型システムのサポートを持つ言語として、
MetaOCaml, Template Haskell, Scala LMSなどがある。

本研究は、
MetaOCaml などの値呼び関数型言語に基づいたコード生成言語を対象としているが、
言語のプレゼンテーションでは、先行研究にならい
コードコンビネータ(Code Combinator)方式を使う。MetaML/MetaOCamlなどにおける擬似引用
(Quasi-quotation)方式は、コード生成に関する言語要素として
「ブラケット(コード生成,quotation)」と
「エスケープ(コード合成,anti-quotation)」を用いるのに対して、
コードコンビネータ方式では、
各演算子に対して、「コード生成版の演算子(コードコンビネータ)」を用意してコード生成器を記述する。
たとえば、加算$e_1+e_2$に対して、
コードコンビネータ版は$e_1 \cPlus e_2$というように、
演算子名に下線をつけてあらわす。

本章では、例に基づいてコード生成器とlet挿入について説明する。
対象言語の構文・意味論などの形式的体系の説明は後に行う.

\subsection{コードコンビネータ方式のプログラム例}

まず、(完成した)コードは、$\code{3}$や$\code{3+5}$のようにブラケットで囲んで表す。
次のコードは、これらを生成するプログラムである。
\begin{align*}
(\cint~3)   & \too \code{3} \\
(\cint~3)~ \cPlus~ (\cint~5) & \too \code{3 + 5}
\end{align*}
$\cint$ は整数を整数のコードに変換し、
$\cPlus$は、整数のコード2つをもらって、それらの加算をおこなうコードを
生成するコードコンビネータである。
なお、$\too$は0ステップ以上の簡約を表す。

$\cfun{x}{e}$と$\cap$ はそれぞれラムダ抽象と関数適用のコードを生成する。
\begin{align*}
\cfun{x}{x \cPlus (\cint~3)}   & \too \code{\fun{x}{x+3}} \\
(\cfun{x}{x \cPlus (\cint~3)}) \cat (\cint~5) & \too 
\code{(\fun{u}{u+3})~5}
\end{align*}
ラムダ抽象のコードコンビネータにおいて、$x$は「(コードレベルの)変数」その
ものを表すのではなく、「変数のコード」をあらわす。
上記の例の計算過程で、$x$は
$\code{u}$(ここで$u$は新たに作成されたコードレベルの変数)に簡約され、計算が進む。

$\cLet$はlet式のコードを生成する。
\begin{align*}
& \clet{x}{(\cint~3)}{x~ \cPlus~ (\cint~7)} \\
  & \too \code{\Let~ x = 3~ \In~ x + 7} 
\end{align*}
実は、$\cLet$は、コードコンビネータとしてのラムダ抽象と適用によりマ
クロ定義され、上記の式は、以下の式と同じである。
\begin{align*}
& (\cfun{x}{~x~ \cPlus~ (\cint~7)~}) \cat (\cint~3)  \\
& \too \code{\Let~ x = 3~ \In~ x + 7} 
\end{align*}

本研究の対象言語は、MetaML や MetaOCaml と同様、静的束縛の言語であり、
以下の例では、束縛変数の名前が正しく付け換えられる。
\begin{align*}
\cfun{y}{\Let~ x = y~ \In~ \cfun{y}{~x~ \cPlus~ y}{}}{} 
& \too \code{\fun{y}{\fun{y'}{y~ +~ y'}}}
\end{align*}
この例では、2つのラムダ抽象が$y$という変数をもっているが、これらは異な
る束縛変数であるので、計算の過程で衝突が起きるときは名前換えが発生する。

\subsection{コード生成におけるlet挿入}

$\cFor$はfor式を生成するコードコンビネータである。
ここで、(コードレベルの)配列$A$の第$n$要素に対する代入を
$\aryset{A}{n}{e}$と表し、
$\caryset{a}{e_1}{e_2}$は対応するコードコンビネータであると仮定する。
また、$A$は適宜$n$次元のものを考えることにする。
\begin{align*}
& \cforin{x=(\cint~3)}{(\cint~7)} \\
& \qquad \caryset{\code{A}}{x}{(\cint~0)} \\
& \too \code{\forin{i=3}{7}~\aryset{A}{i}{0}}
\end{align*}
$\cFor$を入れ子にすると、入れ子のfor式が生成できる。
\begin{align*}
& \cforin{x=(\cint~3)}{(\cint~7)} \\
& \cforin{y=(\cint~1)}{(\cint~9)} \\
& \qquad \caryset{\code{A}}{(x,y)}{(\cint~0)} \\
& \too \codebegin \forin{i=3}{7} \\
& \phantom{\too \codebegin} \forin{j=1}{9} \\
& \phantom{\too \codebegin} ~~\aryset{A}{i,j}{0} \codeend
\end{align*}

この二重ループの中で、複雑な計算をするループ不変式があったとする。たと
えば、配列の初期値として$0$でなく、(何らかの複雑な)計算結果を代入する
が、その計算にはループ変数$i,j$を使わない場合を考える。
それを$e$とすると、
\begin{align*}
& \codebegin \forin{i=3}{7} \\
& \phantom{\codebegin} \forin{j=1}{9} \\
& \phantom{\codebegin} ~~\aryset{A}{i,j}{e} \codeend
\end{align*}
というコードの代わりに
\begin{align*}
& \codebegin \Let~z=e~\In \\
& \phantom{\codebegin \Let} \forin{i=3}{7} \\
& \phantom{\codebegin \Let} \forin{j=1}{9} \\
& \phantom{\codebegin \Let} ~~\aryset{A}{i,j}{z} \codeend
\end{align*}
というコードを生成した方がよい。

このように、生成するコードの上部(よりトップレベルに近い方)にlet式を挿
入することができれば、早い段階で値を計算できたり、また、同一の部分式が
ある場合は計算結果を再利用できたり、という利点がある。

なお、この変形・最適化は、コードを生成してから行なうのでよければ技術的
に難しいものではない。しかし、
コード生成においては、生成されるコード量の爆発が問題になることが多く、
無駄なコードはできるだけ早い段階で除去したい、すなわち、
コードを生成してから最適化するのではなく生成段階でコードを変形・最適化したいという
強い要求がある。

そこで、コード生成器にlet挿入を組み込むことが考えられる。
let挿入は部分計算(partial evaluation)で研究されてきたものであり、
コントロールオペレータを適切に用いることで実現できることが知られている。
本研究では、shift0/reset0 というコントロールオペレータを用いて上記のlet
挿入を実現する。(従来用いられてきたshift/reset でなく shift0/reset0を
用いる理由は後述する。)

\begin{align*}
    e &= \red{\Resetz} ~~\cLet~x_1=\csp{3}~\cIn \\
      &\phantom{=}~~ \blue{\Resetz} ~~\cLet~x_2=\csp{5}~\cIn \\
      &\phantom{=}~~ \blue{\Shiftz}~\blue{k_2}~\to~ \red{\Shiftz}~\red{k_1}~\to~ \magenta{\cLet~y=t~\cIn} \\
      &\phantom{=}~~ \cThrow~\red{k_1}~(\cThrow~\blue{k_2}~(x_1~\cPlus~x_2~\cPlus~y))
\end{align*}
とする.

$e$ を計算すると,
$\blue{\Resetz}$によって,切り取られた継続 $\cLet~x_2=\csp{5}~\cIn$ が,
以下で,我々の言語体系におけるshift0/reset0 による多段階let挿入の例を掲載する.

\begin{align*}
    e &= \red{\Resetz} ~~\cLet~x_1=\csp{3}~\cIn \\
      &\phantom{=}~~ \blue{\Resetz} ~~\cLet~x_2=\csp{5}~\cIn \\
      &\phantom{=}~~ \blue{\Shiftz}~\blue{k_2}~\to~ \red{\Shiftz}~\red{k_1}~\to~ \magenta{\cLet~y=t~\cIn} \\
      &\phantom{=}~~ \cThrow~\red{k_1}~(\cThrow~\blue{k_2}~(x_1~\cPlus~x_2~\cPlus~y))
\end{align*}
とする.

$e$ を計算すると,
$\blue{\Resetz}$によって,切り取られた継続 $\cLet~x_2=\csp{5}~\cIn$ が,
$\blue{\Shiftz}$ によって,$\blue{k_2}$へと捕獲され,
次に,
$\red{\Resetz}$によって,切り取られた継続 $\cLet~x_2=\csp{3}~\cIn$ が,
$\red{\Shiftz}$ によって,$\red{k_1}$へと捕獲される.

わかりやすいところまで計算を進めると以下のようになり,
\begin{align*}
  e & \too \magenta{\cLet~y=t~\cIn} \\
    & \phantom{\too}~~ \cThrow~\red{k_1}~(\cThrow~\blue{k_2}~(x_1~\cPlus~x_2~\cPlus~y))
\end{align*}

$\magenta{\cLet~y=t~\cIn}$ がトップに挿入されたことが分かる.
$\cThrow$ は,切り取られた継続を引数に適用するための演算子である.
つまり,
\begin{align*}
  e & \too \magenta{\cLet~y=t~\cIn} \\
    & \cLet~x_1=\csp{3}~\cIn \\
    & \cLet~x_2=\csp{5}~\cIn \\
    & (x_1~\cPlus~x_2~\cPlus~y)
\end{align*}

となり,$\magenta{\cLet~y=t~\cIn}$ が 二重の $\cLet$を飛び越えて,挿入された事が分かる.
これが多段階 let 挿入である.

また, 項 $t$ の種類によっては,型が付いていてほしくない場合と付いて欲しい場合とがある.
例えば,$t$ が $\code{7}$ のときは,型がつき,
$t$ が $x_1$ や $x_2$ のとき型が付かないようにしたい.
つまり,この例においては,項 $t$ の種類によって,安全なコードか,安全でないコードかが変わるので,それを型で判断したい.このような型システムを構築することを考える.

%%% Local Variables:
%%% mode: latex
%%% TeX-master: "paper"
%%% End:

% section 3

\section{環境識別子とその精密化}

\subsection{環境識別子による変数のスコープ表現}
\kam{ここに須藤研究と大石研究の図をいれる(大石研究の図は、次の章に移動
  する予定だが、とりあえず、ここにいれておいてください)}

環境識別子 Environment Classifier による型による変数のスコープのアイデア\cite{Taha:2003:EC:604131.604134,Sudo2014}を拡張することによって,shift0/reset0の型安全性を保証する.
今まではプログラムがネストしていけば,その分だけ使える自由変数が増えていったのだが, shift0/reset0 が導入されたことにより,計算の順序が変わり,単純にネストした分だけ使える自由変数が増えていくという訳にはいかない.
そこで,環境識別子の結合を意味する $\cup$ を導入することにより上記の問題を解決することにした.


コードレベルの変数スコープと,型によるスコープの表現をあらわしたのが,以下の図である.
\begin{center}
  \includegraphics[clip,height=4cm]{./img/ec_let.png}
\end{center}

環境識別子とは,ある項のスコープの範囲において使える自由変数の集合である.
プログラムのネストが深くなると,使える自由変数が増えていくことが上図で分かるだろう.

\begin{center}
  \includegraphics[clip,height=4cm]{./img/ec_for.png}
\end{center}

$\gamma$は変数のスコープを表し,そのスコープ内で使える自由変数の集合と思ってもらえば良い.
$\gamma$には,包含関係があり,それを $\gamma_1 \ord \gamma_0$ というような順序で表す.
直感的には$\gamma_0$より$\gamma_1$のほうが使える自由変数が多いという意味である.

このように EC を型に付加することで,その型がどのスコープに存在するかどうかが分かる.
つまり,型同士の相対的な位置を型レベルで判断できるようになる.

% \begin{tikzpicture}
%   \node (s) {$\gamma_0$};
%   \node[above right=of s] (a) {$\gamma_1$};
%   \node[below right=of s] (b) {$\gamma_2$};
%   \node[below right=of a] (t) {$\gamma_3 = \gamma_1 \cup \gamma_2$};

%   \foreach \u / \v in {s/a,s/b,b/t,a/t}
%   \draw[->] (\u) -- (\v);
% \end{tikzpicture}

\subsection{環境識別子の精密化}

\begin{center}
  \includegraphics[clip,height=5.5cm]{./img/ecex_let.png}
\end{center}

しかし,shift0/reset0 が導入されることにより,計算の順序が

\begin{center}
  \includegraphics[clip,height=5.5cm]{./img/ecex_let_non_gamma.png}
\end{center}

\begin{center}
  \includegraphics[clip,height=4cm]{./img/ecex_for.png}
\end{center}

\begin{center}
  \includegraphics[clip,height=3cm]{./img/ecex_for_non_gamma.png}
\end{center}

\begin{center}
  \includegraphics[clip,width=7cm]{./img/gamma.png}
\end{center}

\begin{center}
  \includegraphics[clip,width=4cm]{./img/gamma_normal.png}
\end{center}

%%% Local Variables:
%%% mode: japanese-latex
%%% TeX-master: "paper"
%%% End:

% section 4

\section{本研究のアイディア}


% section 5

\section{対象言語: 構文と意味論}

本研究における対象言語は,ラムダ計算にコード生成機能とコントロールオペ
レータshift0/reset0を追加したものに型システムを導入したものである.

本稿では,最小限の言語のみについて考えるため,コード生成機能の
「ステージ(段階)」は,コード生成段階(レベル0,現在ステージ)と
生成されたコードの実行段階(レベル1,将来ステージ)の2ステージのみを考える.

前述したように,本研究の言語では,
コードコンビネータ(Code Combinator)方式を使い,
コードコンビネータは,
$\cPlus$ や $\cIf$のように下線を引いて表す.

\subsection{構文の定義}

対象言語の構文を定義する.

変数は,レベル0変数($x$), レベル1変数($u$),
(レベル0の)継続変数($k$)の3種類がある.
レベル0項($e^0$),レベル1項($e^1$)およびレベル0の値($v$)を下の通り定義する.

\begin{align*}
  c & ::= i \mid b \mid \cint
         \mid \cat \mid + \mid \cPlus \mid \cIf \\
  v & ::= x \mid c \mid \fun{x}{e^0} \mid \code{e^1} \\
  e^0 & ::=  v  \mid e^0~ e^0 \mid \ift{e^0}{e^0}{e^0} \\
    & \mid \cfun{x}{e^0}
      \mid \ccfun{u}{e^0} \\
    & \mid \resetz{e^0}
      \mid \shiftz{k}{e^0}
      \mid \throw{k}{v} \\
  e^1 & ::=  u \mid c \mid \fun{u}{e^1} \mid e^1~ e^1
      \mid \ift{e^1}{e^1}{e^1} \\
\end{align*}
ここで$i$は整数の定数,$b$は真理値定数である.

定数のうち,下線がついているものはコードコンビネータである.
変数は,ラムダ抽象(下線なし,下線つき,二重下線つき)および shift0 により束縛され,
$\alpha$同値な項は同一視する.
$\letin{x}{e_1}{e_2}$および$\clet{x}{e_1}{e_2}$は,
それぞれ,$(\fun{x}{e_2})e_1$
$(\cfun{x}{e_2})\cat e_1$の省略形である.
前述の例でのべた$\cFor$は,
コード構築定数とコードレベル適用$\cap$を用いて導入することとし,
(この導入にあたっての型システムの拡張は容易なので)ここでは省略する.

\subsection{操作的意味論}

対象言語は,値呼びで left-to-rightの操作的意味論を持つ.
ここでは評価文脈に基づく定義を与える.

評価文脈を以下のように定義する.
\begin{align*}
  E & ::= [~] \mid E~ e^0 \mid v~ E \\
    & \mid \ift{E}{e^0}{e^0} \mid \Resetz~ E \mid \ccfun{u}{E}
\end{align*}
コード生成言語で特徴的なことは,
コードレベルのラムダ抽象の内部で評価が進行する点である.実際,
上記の定義には,$\ccfun{u}{E}$が含まれている.
たとえば,$\ccfun{u}{u \cPlus [~]}$ は評価文脈である.

この評価文脈$E$と次に述べる計算規則$r \to l$ により,
評価関係$e \lto e'$ を次のように定義する.
\[
  \infer{E[r] \lto E[l]}{r \to l}
\]

計算規則は以下の通り定義する.
\begin{align*}
  (\fun{x}{e})~v &\to e\{ x := v \} \\
  \ift{true}{e_1}{e_2} &\to e_1 \\
  \ift{else}{e_1}{e_2} &\to e_2 \\
  \cfun{x}{e} &\to \ccfun{u}{(e\{ x := \code{u} \})} \\
  \ccfun{u}{\code{e}} &\to \code{\fun{u}{e}} \\
  \Resetz~ v &\to v \\
  \Resetz (E[\Shiftz~ k \to e]) &\to e \ksubst{k}{E}
\end{align*}
ただし,4行目の$u$はフレッシュなコードレベル変数とし,
最後の行の$E$は穴の周りに{\Resetz}を含まない評価文脈とする.
また,この行の右辺のトップレベルに{\Resetz}がない点が,
shift/reset の振舞いとの違いである.すなわち,shift0 を1回計算すると,
reset0 が1つはずれるため,shift0 をN個入れ子にすることにより,
N個分外側のreset0 までアクセスすることができ,多段階let挿入を実現でき
るようになる.

上記における継続変数に対する代入$e\ksubst{k}{E}$は次の通り定義する.
\begin{align*}
  (\throw{k}{v})\ksubst{k}{E} &\equiv \Resetz (E[v]) \\
  (\throw{k'}{v})\ksubst{k}{E} &\equiv \throw{k'}{(v\ksubst{k}{E})}
\\
& \text{ただし}~k \not= k'
\end{align*}
上記以外の$e$に対する代入の定義は透過的であるとする.
上記の定義の1行目で\Resetz を挿入しているのは{\Shiftz}の意味論に対応し
ており,これを挿入しない場合は別のコントロールオペレータ(Felleisenの
control/promptに類似した control0/prompt0)の振舞いとなる.

コードコンビネータ定数の振舞い(ラムダ計算における$\delta$規則に相当)は
以下のように定義する.

\begin{align*}
  \cint~ n &\to \code{n} \\
  \code{e_1}~ \cat~ \code{e_2} &\to \code{e_1~ e_2} \\
  \code{e_1}~ \cPlus~ \code{e_2} &\to \code{e_1 + e_2} \\
  \cif{e_1}{e_2}{e_3} &\to \code{\ift{e_1}{e_2}{e_3}}
\end{align*}
%計算の例を以下に示す.
%\begin{align*}
%  e_1 & = \Resetz ~~\cLet~x_1=\csp{3}~\cIn \\
%      & \phantom{=}~~ \Resetz ~~\cLet~x_2=\csp{5}~\cIn \\
%      & \phantom{=}~~ \Shiftz~k~\to~\cLet~y=t~\cIn \\
%      & \phantom{=}~~ \Throw~ k~ (x_1~\cPlus~x_2~\cPlus~y) \\
%\end{align*}
%
%\begin{align*}
%  [ e_1 ] &\lto [ \Resetz (\cLet~x_1=\csp{3}~\cIn \\
%          &\Resetz~ \cLet~x_2=\csp{5}~\cIn \\
%          &[ \Shiftz~ k~ \to~ \cLet~ y=t~ \cIn \\
%          &[ \Throw~ k~(x_1~\cPlus~x_2~\cPlus~y) ] ] ) ] \\
%          &\lto [ \cLet~ y=t~ \cIn \\
%          &[ \cfun{x}{\Resetz~ (\cLet~x_1=\csp{3}~ \cIn~ \Resetz~ (\cLet~ x_2=\csp{5}~ \cIn [x]))} (x_1~\cPlus~x_2~\cPlus~y) ]] \\
%          &\lto [ \cfun{y}{(\cfun{x}{\Resetz~ (\cLet~x_1=\csp{3}~ \cIn~ \Resetz~ (\cLet~ x_2=\csp{5}~ \cIn [x]))} (x_1~\cPlus~x_2~\cPlus~y))}~ \cat~ t ] \\
%          &\lto [[\cfun{y}{(\cfun{x}{\Resetz~ (\cLet~x_1=\csp{3}~ \cIn~ \Resetz~ (\cLet~ x_2=\csp{5}~ \cIn [x]))} (x_1~\cPlus~x_2~\cPlus~y))}]~ \cat~ t] \\
%          &\lto [[\ccfun{y_1}{(\cfun{x}{\Resetz~ (\cLet~x_1=\csp{3}~ \cIn~ \Resetz~ (\cLet~ x_2=\csp{5}~ \cIn [x]))} (x_1~\cPlus~x_2~\cPlus~ \code{y_1}))}]~ \cat~ t] \\
%          &\lto
%\end{align*}


\section{型システム}

本研究での型システムについて述べる.

基本型$b$,環境識別子(Environment Classifier)$\gamma$を以下の通り定義する.
\begin{align*}
  b & ::= \intT \mid \boolT \\
  \gamma & ::= \gamma_x \mid \gamma \cup \gamma
\end{align*}
$\gamma$の定義における$\gamma_x$は環境識別子の変数を表す.
すなわち,環境識別子は,変数であるかそれらを$\cup$で結合した形である.
以下では,メタ変数と変数を区別せず$\gamma_x$を$\gamma$と表記する.
ここで環境識別子として$\cup$を導入した理由は後述する.

$L ::= \empty \mid \gamma$ は現在ステージと将来ステージをまとめ
て表す記号である.たとえば,$\Gamma \vdash^L
e:t~;~\sigma$は,$L=\empty$のとき現在ステージの判断で,
$L=\gamma$のとき将来ステージの判断となる.

レベル0の型$t^0$,レベル1の型$t^1$,(レベル0の)型の有限列$\sigma$,
(レベル0の)継続の型$\kappa$を次の通り定義する.
\begin{align*}
  t^0 & ::= b \mid \funT{t^0}{t^0}{\sigma} \mid \codeT{t^1}{\gamma} \\
  t^1 & ::= b \mid t^1 \to t^1 \\
  \sigma & ::= \epsilon \mid t^0, \sigma \\
  \kappa^0 & ::= \contT{\codeT{t^1}{\gamma}}{\codeT{t^1}{\gamma}}{\sigma}
\end{align*}

レベル0の関数型$\funT{t^0}{t^0}{\sigma}$は,
エフェクトをあらわす列$\sigma$を伴っている.これは,その関数型をもつ項
を引数に適用したときに生じる計算エフェクトであり,具体的には,
\Shiftz の answer type の列である.前述したようにshift0 は多段
階の reset0 にアクセスできるため,$n$個先のreset0 の answer typeまで記
憶するため,このように型の列$\sigma$で表現している.
ただし,本研究の範囲では,answer type modification に対応する必要はな
いので,エフェクトはシンプルに型の列($n$個先の reset0 のanswer type を
$n=1,\cdots,k$に対して並べた列)で表現している.
この型システムの詳細は,Materzokら\cite{Materzok2011}の研究を参照されたい.

本稿の範囲では,コントロールオペレータは現在ステージのみにあらわれ,生
成されるコードの中にはあらわないため,レベル1の関数型は,エフェクトを
表す列を持たない.
また,本項では,shift0/reset0 はコードを操作する目的にのみ使うため,継
続の型は,コードからコードへの関数の形をしている.
ここでは,後の定義を簡略化するため,継続を,通常の関数とは区別しており,
そのため,継続の型も通常の関数の型とは区別して二重の横線で表現している.

型判断は,以下の2つの形である.
\begin{align*}
  \Gamma \vdash^{L} e : t ;~\sigma \\
  \Gamma \models \gamma \ord \gamma
\end{align*}
ここで,型文脈$\Gamma$は次のように定義される.
\begin{align*}
  \Gamma ::= \emptyset
  \mid \Gamma, (\gamma \ord \gamma)
  \mid \Gamma, (x : t)
  \mid \Gamma, (u : t)^{\gamma}
\end{align*}

型判断の導出規則を与える.まず,$\Gamma \models \gamma \ord \gamma$の
形に対する規則である.

\[
  \infer
  {\Gamma \models \gamma_1 \ord \gamma_1}
  {}
\quad
  \infer
  {\Gamma, \gamma_1 \ord \gamma_2 \models \gamma_1 \ord \gamma_2}
  {}
\]

\[
  \infer
  {\Gamma \models \gamma_1 \ord \gamma_3}
  {\Gamma \models \gamma_1 \ord \gamma_2 & \Gamma \models \gamma_2 \ord \gamma_3}
\]

\[
  \infer
  {\Gamma \models \gamma_1 \cup \gamma_2 \ord \gamma_1}
  {}
\quad
  \infer
  {\Gamma \models \gamma_1 \cup \gamma_2 \ord \gamma_2}
  {}
\]

\[
  \infer
  {\Gamma \models \gamma_3 \ord \gamma_1 \cup \gamma_2}
  {\Gamma \models \gamma_3 \ord \gamma_1
  &\Gamma \models \gamma_3 \ord \gamma_2}
\]



次に,$\Gamma \vdash^{L} e : t ;~\sigma$ の形に対する導出規則を与える.
まずは,レベル0における単純な規則である.

\[
  \infer
  {\Gamma, x : t \vdash x : t ~;~ \sigma}
  {}
\quad
  \infer
  {\Gamma, (u : t)^\gamma \vdash^\gamma u : t ~;~ \sigma}
  {}
\]

\[
  \infer
  {\Gamma \vdash^{L} c : t^c ~;~\sigma}
  {}
\]

\[
  \infer
  {\Gamma \vdash^{L} e_1~ e_2 : t_1 ; \sigma}
  {\Gamma \vdash^{L} e_1 : t_2 \to t_1 ; \sigma
    & \Gamma \vdash^{L} e_2 : t_2  ; \sigma
  }
\]

\[
  \infer
  {\Gamma \vdash \fun{x}{e} : \funT{t_1}{t_2}{\sigma} ~;~\sigma'}
  {\Gamma,~x : t_1 \vdash e : t_2 ~;~ \sigma}
\quad
  \infer
  {\Gamma \vdash^\gamma \fun{x}{e} : \funT{t_1}{t_2}{} ~;~\sigma'}
  {\Gamma,~(u : t_1)^\gamma \vdash^\gamma e : t_2 ~;~ \sigma}
\]

\[
  \infer
  {\Gamma \vdash^{L} \ift{e_1}{e_2}{e_3} : t ~;~ \sigma}
  {\Gamma \vdash^{L} e_1 : \boolT ;~ \sigma
    & \Gamma \vdash^{L} e_2 : t ; \sigma
    & \Gamma \vdash^{L} e_3 : t ; \sigma}
\]

次にコードレベル変数に関するラムダ抽象の規則である.

\[
  \infer[(\gamma_1~\text{is eigen var})]
  {\Gamma \vdash \cfun{x}{e} : \codeT{t_1\to t_2}{\gamma} ~;~ \sigma}
  {\Gamma,~\gamma_1 \ord \gamma,~x:\codeT{t_1}{\gamma_1} \vdash e
    : \codeT{t_2}{\gamma_1}; \sigma}
\]

\[
  \infer
  {\Gamma \vdash \ccfun{u^1}{e} : \codeT{t_1 \to t_2}{\gamma} ; \sigma}
  {\Gamma, \gamma_1 \ord \gamma, x : (u : t_1)^{\gamma_1} \vdash e : \codeT{t_2}{\gamma_1} ; \sigma}
\]

コントロールオペレータに対する型導出規則である.

\[
  \infer{\Gamma \vdash \resetz{e} : \codeT{t}{\gamma} ~;~ \sigma}
  {\Gamma \vdash e : \codeT{t}{\gamma} ~;~ \codeT{t}{\gamma}, \sigma}
\]

\[
  \infer{\Gamma \vdash \shiftz{k}{e} : \codeT{t_1}{\gamma_1} ~;~ \codeT{t_0}{\gamma_0},\sigma}
  {\Gamma,~k:\contT{\codeT{t_1}{\gamma_1}}{\codeT{t_0}{\gamma_0}}{\sigma}
    \vdash e : \codeT{t_0}{\gamma_0} ; \sigma
    & \Gamma \models \gamma_1 \ord \gamma_0
  }
\]

\[
  \infer
  {\Gamma,~k:\contT{\codeT{t_1}{\gamma_1}}{\codeT{t_0}{\gamma_0}}{\sigma}
    \vdash \throw{k}{v} : \codeT{t_0}{\gamma_2} ; \sigma}
  {\Gamma
    \vdash v : \codeT{t_1}{\gamma_1 \cup \gamma_2} ; \sigma
    & \Gamma \models \gamma_2 \ord \gamma_0
  }
\]

コード生成に関する補助的な規則として,Subsumptionに相当する規則等がある.
\[
  \infer
  {\Gamma \vdash e : \codeT{t}{\gamma_2} ; \sigma}
  {\Gamma \vdash e : \codeT{t}{\gamma_1} ; \sigma
    & \Gamma \models \gamma_2 \ord \gamma_1
  }
\]

\[
  \infer
  {\Gamma \vdash^{\gamma_2} e : t ~;~ \sigma}
  {\Gamma \vdash^{\gamma_1} e : t ~;~ \sigma
    & \Gamma \models \gamma_2 \ord \gamma_1
  }
\]


\[
  \infer
  {\Gamma \vdash \code{e} : \codeT{t^1}{\gamma} ; \sigma}
  {\Gamma \vdash^{\gamma} e : t^1 ; \sigma}
\]


\subsection{型付け例}

上記の型システムのもとで,いくつかの項の型付けについて述べる.

\begin{align*}
  e_1 & = \Resetz ~~\cLet~x_1=\csp{3}~\cIn \\
      & \phantom{=}~~ \Resetz ~~\cLet~x_2=\csp{5}~\cIn \\
      & \phantom{=}~~ \Shiftz~k~\to~\cLet~y=t~\cIn \\
      & \phantom{=}~~ \Throw~k~(x_1~\cPlus~x_2~\cPlus~y)
\end{align*}

この式$e_1$に対して,もし,$t=\csp{7}$ あるいは $t=x_1$であれば,
$e_1$ は型付け可能である.
一方,$t=x_2$ であれば,$e_1$ は型付けできない.

\begin{align*}
  e_2 & = \Resetz ~~\cLet~x_1=\csp{3}~\cIn \\
      & \phantom{=}~~ \Resetz ~~\cLet~x_2=\csp{5}~\cIn \\
      & \phantom{=}~~ \Shiftz~k_2~\to~ \Shiftz~k_1~\to~ \cLet~y=t~\cIn \\
      & \phantom{=}~~ \Throw~k_1~(\Throw~k_2~(x_1~\cPlus~x_2~\cPlus~y))
\end{align*}

この式$e_2$に対して,もし$t=\csp{7}$であれば$e_1$は型付け可能である.
一方,$t=x_2$ あるいは $t=x_1$であれば,$e_1$は型付けできない.

このように,(少なくとも)上記の例については安全な式と危険な式を正しく峻
別できていることがわかった.

\subsection{型安全性について}

本研究の型システムに対する型保存(Subject Reduction)定理について述べる.
型保存定理は,(証明できれば)
進行(Progress)定理とあわせて型システムの健全性を導く定理である.

\begin{quote}
(型保存性)
$\vdash e:t~;~\sigma$ かつ $e \lto e'$ であれば,$\vdash e':t~;~\sigma$
である.
\end{quote}

この定理は reset0-shift0の計算規則が多相性を持たない場合には容易に証明
できるが,多相性については精密な扱いが必要であり,
現段階では,型保存定理の証明は進行中である.

%\begin{lemm}[不要な仮定の除去]
%  $\Gamma_1,\gamma_2 \ord \gamma_1 \vdash e : t_1 ~;~\sigma$
%  かつ,$\gamma_2$が $\Gamma_1, e, t_1, \sigma$に出現しないなら,
%  $\Gamma_1 \vdash e : t_1 ~;~\sigma$ である.
%\end{lemm}
%
%\begin{lemm}[値に関する性質]
%  $\Gamma_1 \vdash v : t_1 ~;~\sigma$
%  ならば,
%  $\Gamma_1 \vdash v : t_1 ~;~\sigma'$
%  である.
%\end{lemm}
%
%\begin{lemm}[代入]
%  $\Gamma_1, \Gamma_2, x : t_1 \vdash e : t_2 ~;~\sigma$
%  かつ
%  $\Gamma_1 \vdash v : t_1 ~;~\sigma$
%  ならば,
%  $\Gamma_1, \Gamma_2 \vdash e\{x := v\} : t_2~;~\sigma$
%\end{lemm}
%
%これらをもとに型保存定理を証明する.
%本研究の対象言語は,コントロールオペレータが操作する対象となる式の型を
%コード型に限定するなど,注意深く設計しているので,ほとんどのケースの証
%明はスムーズであるが,reset0-shift0 に関する計算規則(shift0 が評価文脈
%を捕捉して継続変数$k$に渡す規則)とthrowに関する計算規則では,
%サブタイプ多相性に相当する性質を使っているので,以下の技術的な補題が必
%要である.
%
%\begin{lemm}[識別子に関する多相性]
%  穴の周りにreset0を含まない評価文脈$E$,変数$x$,
%  そして$\Gamma = (u_1:t_1)^{\gamma_1}, \cdots, (u_n:t_n)^{\gamma_n}$
%      かつ$i=1,\cdots,n$に対して$\Gamma \models \gamma_0 \ord \gamma_i$であるとする.
%  このとき,
%  $\Gamma, x:\codeT{t_0}{\gamma'} \vdash E[x] : \codeT{t_1}{\gamma_0} ~;~\sigma$
%  であれば,フレッシュな$\gamma$に対して,
%  $\Gamma, x:\codeT{t_0}{\gamma'\cup \gamma} \vdash
%   E[x] : \codeT{t_1}{\gamma_0 \cup \gamma} ~;~\sigma$
%  である.
%\end{lemm}
%
%この補題は,評価文脈$E$に対して,穴の型が$\codeT{t_0}{\gamma'}$で
%評価文脈全体の型が$\codeT{t_1}{\gamma_0}$であれば,
%それぞれの環境識別子に$\gamma_2$を加えて,
%$\codeT{t_0}{\gamma'\cup \gamma}$型から,
%$\codeT{t_1}{\gamma_0\cup \gamma}$型への評価文脈として使ってもよい,
%ということを主張している.ここで $\gamma \ord \gamma_0$ なので,
%$\gamma$と$\gamma_0 \cup \gamma$は$\ord$の意味で等しくなり,
%$\codeT{t_1}{\gamma_0\cup\gamma}$型を持つ項は,
%$\codeT{t_1}{\gamma}$型も持つことがわかる.
%この定理により,shift0が捕捉した継続を(環境識別子について)多相的に使う
%ことが可能となり,reset0-shift0 の計算規則が正当化される.
%
%上記の補題を証明すれば,型保存定理の証明の残りのケースは比較的容易であ
%る.なお,この補題を使うケースにおいて,定理の言明にあらわれる項$e$が
%閉じた項であること(環境識別子に関する
%
%
%進行定理については
%精密な定式化が必要(reset0がない式でshift0を実行した時など)が必要なので,

%\subsection{進行}
%\begin{theo}[進行]
%  $\vdash e:t$ が導出可能であれば,$e$ は 値 $v$ である.または,$e \lto e'$ であるような 項 $e'$ が存在する
%\end{theo}
%
%\paragraph{証明}
%$\vdash e:t$ の導出に関する帰納法による.\\
%Const, Abs, Code 規則の場合 $e$ は値である.\\
%Var 規則の場合 $\vdash e:t$ は導出可能でない.\\
%Throw 規則の場合 $\vdash e:t$ は導出可能でない.\\
%Reset0 規則の場合 $e = \Resetz~ e_1$ とする.
%帰納法の仮定より評価文脈における $\Resetz E$ より簡約が進み,\\
%$e_1$ が値のとき,$e \lto v$ となるような $v$ が存在する.\\
%$e_1$ が値でないとき,

%%% Local Variables:
%%% mode: latex
%%% TeX-master: "paper"
%%% End:

\section{まとめと今後の課題}

{\bf 謝辞} 本研究は、JSPS 科研費 15K12007 の支援を受けている。

\bibliographystyle {jssst}
\bibliography {bibfile}

\end{document}

%%% Local Variables:
%%% mode: japanese-latex
%%% TeX-master: t
%%% End:

\subsection{プログラム例}

\section{型システム}
\ooi{answer タイプがお尻に付いた型システムを書く}

\section{型安全性の証明}
\ooi{ここは途中までになりそう}

\section{まとめと今後の課題}
\ooi{途中までになったところを今後やりたいこととする?そうすると,研究結果としてはまとまっていないので,タイトルを ``多段階 let 挿入を行うコード生成言語の型システムの設計の試み''に変更するかも}

{\bf 謝辞}
ほげほげ

% Sudo2014のbibfile の書き方を note 使ってるのを変更する必要あり -> 変更しなくておkぽい
\end{document}


%%% Local Variables:
%%% mode: japanese-latex
%%% TeX-master: t
%%% End:
