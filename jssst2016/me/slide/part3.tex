% 先の解決方法で実際になにができるのかの説明

\subsection{型付けの例}

\begin{frame}
  \frametitle{型が付く例/付かない例}
  コード生成器
  \begin{align*}
    e & = \red{\Resetz} ~~\cfordo{i = 0}{n} \\
      & \phantom{=}~~~~\blue{\Resetz} ~~\cfordo{j = 0}{m} \\
      & \phantom{=}~~~~~~ \blue{\Shiftz}~\blue{k_2}~\to~ \red{\Shiftz}~\red{k_1}~\to~ \magenta{\cLet~y=\green{t}~\cIn} \\
      & \phantom{=}~~~~~~~~ \red{k_1}~(\blue{k_2}~(\caryset{\code{a}}{(i,j)}{b[i] + y}))
  \end{align*}

  \pause
  生成されるコード
  \begin{columns}
    \begin{column}{0.5\textwidth}%% [横幅] 0.2\textwidth = ページ幅の 20 %
      \center
      \includegraphics[clip,height=1cm]{./img/batsu.png}
      \begin{align*}
        e & \too \magenta{\Let ~y ~= ~\green{a[i][j]} ~\In} \\
          & \phantom{\too}~~~~ \fordo{i = 0}{n} \\
          & \phantom{\too}~~~~~~\fordo{j = 0}{m} \\
          & \phantom{\too}~~~~~~~~\aryset{a}{i,j}{b[i] + y} \\
      \end{align*}
    \end{column}

    \begin{column}{0.5\textwidth}%% [横幅] 0.2\textwidth = ページ幅の 20 %
      \center
      \includegraphics[height=1cm]{./img/maru.png}
      \begin{align*}
        e & \too \magenta{\Let ~y ~= ~\green{7} ~\In} \\
          & \phantom{\too}~~~~ \fordo{i = 0}{n} \\
          & \phantom{\too}~~~~~~\fordo{j = 0}{m} \\
          & \phantom{\too}~~~~~~~~\aryset{a}{i,j}{b[i] + y} \\
      \end{align*}
    \end{column}
  \end{columns}
\end{frame}

\newcommand\boxterm{\framebox{
    \only<2>{\green{\cint{3}}}
    \only<3>{\red{x~\cPlus~(\cint{3})}}
    \phantom{A}
  }}

\begin{frame}
  \frametitle{型付けの例(1)}
  \begin{align*}
    e1 & = \Resetz ~~(\cfordo{x = e1}{e2} \\
       & \phantom{=}~~~ \Shiftz~k~\to~
         {\cLet~u=\green{\boxterm}~\cIn}~\Throw~k~u)
  \end{align*}

  \[
    \infer{\vdash e1 : \codeTs{t}{\gamma0};~\epsilon}
    {\infer{\vdash \cfor{x=...} :
        \codeTs{t}{\gamma0};~\codeTs{t}{\gamma0}}
      {\infer{\gamma1 \ord \gamma0,~x:\codeTs{t}{\gamma1}
          \vdash \Shiftz~k~\to~... :
          \codeTs{t}{\gamma1};~\codeTs{t}{\gamma0}}
        {\infer{\Gamma1 \vdash \cLet~u=...:\codeTs{t}{\gamma0};~\epsilon}
          {\infer{\Gamma2 \vdash
              \Throw~k~u:\codeTs{t}{\gamma2};~\epsilon }
            {\infer{\Gamma2 \vdash
                u:\codeTs{t}{\gamma1 \cup \gamma2};~ \sigma}
              {}
            }
            &\infer*{\Gamma1 \vdash
              \green{\boxterm}:\codeTs{t}{\red{\gamma0}};~\epsilon}
            {}
          }
        }
      }
    }
  \]

  {\footnotesize
    \begin{align*}
      \Gamma1 &= \gamma1 \ord \gamma0,~x:\codeTs{t}{\red{\gamma1}},
                ~k: \contT{\codeT{t}{\gamma_1}}{\codeT{t}{\gamma_0}} \\
      \Gamma2 &= \Gamma1,~\gamma2 \ord \gamma0,~u:\codeTs{t}{\gamma2}
    \end{align*}
  }


\end{frame}

% \begin{frame}
%   \frametitle{型付けの例(2)}
%   \begin{align*}
%     e1 & = \red{\Resetz} ~~\cfordo{i = 0}{n} \\
%        & \phantom{=}~~~~\blue{\Resetz} ~~\cfordo{j = 0}{m} \\
%        & \phantom{=}~~~~~~ \blue{\Shiftz}~\blue{k_2}~\to~ \red{\Shiftz}~\red{k_1}~\to~ \magenta{\cLet~y=\green{t}~\cIn} \\
%        & \phantom{=}~~~~~~~~ \red{k_1}~(\blue{k_2}~(\caryset{\code{a}}{(i,j)}{b[i] + y}))
%   \end{align*}


% \end{frame}

\section{まとめと今後の課題}

\begin{frame}
  \frametitle{まとめと今後の課題}
  まとめ
  \begin{itemize}
    % \item コードの言語にshift0 reset0 を組み込んだ言語の設計を行った
  \item コード生成言語の型システムに shift0/reset0 を組み込んだ 型システムの設計を完成させた.
  \item 安全なコードの場合に型が付くこと,安全でないコードの場合には型が付かないように意図通りに型システムが設計できていることをみた
  \end{itemize}

  % \vspace{1in}
  \vspace{\baselineskip}

  今後の課題
  \begin{itemize}
    % \item answer type modification に対応した型システムを設計し,(subject reduction 等の)健全性の証明を行う
  \item 設計した型システムの健全性の証明(Subject recudtion等)
  \item 型推論アルゴリズムの開発
  \item 言語の拡張
    \begin{itemize}
    \item グローバルな参照 (OCamlのlet ref)
    \item 生成したコードの実行 (MetaOCamlの run)
    \end{itemize}
  \end{itemize}
\end{frame}

%%% Local Variables:
%%% mode: japanese-latex
%%% TeX-master: "slide"
%%% End:
