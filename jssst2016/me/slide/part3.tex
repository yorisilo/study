% 先の解決方法で実際になにができるのかの説明

\subsection{型付けの例}

% \begin{frame}
%   \frametitle{型が付く例/付かない例}
%   コード生成器
%   \begin{align*}
%     e & = \red{\Resetz} ~~\cfordo{i = 0}{n} \\
%     & \phantom{=}~~~~\blue{\Resetz} ~~\cfordo{j = 0}{m} \\
%     & \phantom{=}~~~~~~ \blue{\Shiftz}~\blue{k_2}~\to~ \red{\Shiftz}~\red{k_1}~\to~ \magenta{\cLet~y=\green{t}~\cIn} \\
%     & \phantom{=}~~~~~~~~ \red{k_1}~(\blue{k_2}~(\caryset{\code{a}}{(i,j)}{b[i] + y}))
%   \end{align*}

%   \pause
%   生成されるコード
%   \begin{columns}
%     \begin{column}{0.5\textwidth}%% [横幅] 0.2\textwidth = ページ幅の 20 %
%       \center
%       \includegraphics[clip,height=1cm]{./img/batsu.png}
%       \begin{align*}
%         e & \too \magenta{\Let ~y ~= ~\green{a[i][j]} ~\In} \\
%           & \phantom{\too}~~~~ \fordo{i = 0}{n} \\
%           & \phantom{\too}~~~~~~\fordo{j = 0}{m} \\
%           & \phantom{\too}~~~~~~~~\aryset{a}{i,j}{b[i] + y} \\
%       \end{align*}
%     \end{column}

%     \begin{column}{0.5\textwidth}%% [横幅] 0.2\textwidth = ページ幅の 20 %
%       \center
%       \includegraphics[height=1cm]{./img/maru.png}
%       \begin{align*}
%         e & \too \magenta{\Let ~y ~= ~\green{7} ~\In} \\
%           & \phantom{\too}~~~~ \fordo{i = 0}{n} \\
%           & \phantom{\too}~~~~~~\fordo{j = 0}{m} \\
%           & \phantom{\too}~~~~~~~~\aryset{a}{i,j}{b[i] + y} \\
%       \end{align*}
%     \end{column}
%   \end{columns}
% \end{frame}

\newcommand\boxterm{\framebox{
    \only<2>{\green{$\cint{3}$}}
    \only<3>{\red{$x~\cPlus~(\cint{3})$}}
    \phantom{A}
  }}

\begin{frame}
  \frametitle{型付けの例(1)}
  \begin{align*}
    e & = \Resetz ~~(\cfordo{x = e1}{e2} \\
       & \phantom{=}~~~ \Shiftz~k~\to~
         {\cLet~u=\green{\boxterm}~\cIn}~\Throw~k~u)
  \end{align*}

  \[
    \infer{\vdash e : \codeTs{t}{\gamma0};~\epsilon}
    {\infer[(\gamma1^*)]{\vdash \cfor{x=...} :
        \codeTs{t}{\gamma0};~\codeTs{t}{\gamma0}}
      {\infer{\gamma1 \ord \gamma0,~x:\codeTs{t}{\gamma1}
          \vdash \Shiftz~k~\to~... :
          \codeTs{t}{\gamma1};~\codeTs{t}{\gamma0}}
        {\infer[(\gamma2^*)]{\Gamma a \vdash \cLet~u=...:\codeTs{t}{\gamma0};~\epsilon}
          {\infer{\Gamma b \vdash
              \Throw~k~u:\codeTs{t}{\gamma2};~\epsilon }
            {\infer{\Gamma b \vdash
                u:\codeTs{t}{\gamma1 \cup \gamma2};~ \sigma}
              {}
            }
            &\infer*{\Gamma a \vdash
              \green{\boxterm}:\codeTs{t}{\red{\gamma0}};~\epsilon}
            {}
          }
        }
      }
    }
  \]

  % $[(\gamma1^*)]$ は eigen variable (固有変数)

  {\footnotesize
    \begin{align*}
      \Gamma a &= \gamma1 \ord \gamma0,~x:\codeTs{t}{\red{\gamma1}},
                ~k: \contT{\codeTs{t}{\gamma_1}}{\codeTs{t}{\gamma_0}} \\
      \Gamma b &= \Gamma a1,~\gamma2 \ord \gamma0,~u:\codeTs{t}{\gamma2}
    \end{align*}
  }

\end{frame}

\begin{frame}
  \frametitle{型付けの例(2)}

\newcommand\gammaa{\gamma1 \ord \gamma0,~x:\codeTs{t}{\gamma1}}
\newcommand\gammab{\Gamma a,\gamma2 \ord \gamma1,~y:\codeTs{t}{\gamma2}}
\newcommand\gammac{\Gamma b,\blue{k_2}:\contT{\codeTs{t}{\gamma2}}{\codeTs{t}{\gamma1}}}
\newcommand\gammad{\Gamma c,\red{k_1}:\contT{\codeTs{t}{\gamma1}}{\codeTs{t}{\gamma0}}}
\newcommand\gammae{\Gamma d,\gamma3 \ord \gamma0,\magenta{u}:\codeTs{t}{\gamma3}}

\footnotesize
  \begin{align*}
    e'= & \red{\Resetz}~(\cfordo{x = e1}{e2}~\blue{\Resetz} ~(\cfordo{y = e3}{e4} \\
       & \blue{\Shiftz}~\blue{k_2}\to~ \red{\Shiftz}~\red{k_1}\to~\magenta{\cLet~u=\green{X}~\cIn} 
         ~\red{\Throw~k_1}~(\blue{\Throw~k_2}~e5)))
  \end{align*}

  \[
    \infer{\vdash e'=\red{\Resetz}\cdots : \codeTs{t}{\gamma0};~~~\epsilon}
    {\infer[(\gamma1^*)]
      {\vdash \cfor{x=...} : \codeTs{t}{\gamma0};~~~\codeTs{t}{\gamma0}}
      {\infer{\Gamma a=\gammaa\vdash \blue{\Resetz}\cdots : \codeTs{t}{\gamma1};~~~\codeTs{t}{\gamma0}}
             {\infer[(\gamma2^*)]
                 {\Gamma a\vdash \cfor{y=...}: \codeTs{t}{\gamma1};~~~\codeTs{t}{\gamma1},\codeTs{t}{\gamma0}}
                 {\infer{\Gamma b=\gammab\vdash \blue{\Shiftz~k_2}... :\codeTs{t}{\gamma2}
                               ;~~~\codeTs{t}{\gamma1},\codeTs{t}{\gamma0}}
                        {\infer{\Gamma c=\gammac\vdash \red{\Shiftz~k_1}... :\codeTs{t}{\gamma1}
                                      ;~~~\codeTs{t}{\gamma0}}
                               {\infer[(\gamma3)]
                                  {\Gamma d=\gammad\vdash \magenta{\cLet~u=...} : \codeTs{t}{\gamma0};~~\epsilon}
                                  {\infer
                                     {\Gamma e=\gammae\vdash \red{\Throw~k_1}~... : \codeTs{t}{\gamma3};~~\epsilon}
                                     {\infer{\Gamma e\vdash \blue{\Throw~k_2}~e5 : 
                                               \codeTs{t}{\gamma1\cup\gamma3};~~\epsilon}
                                            {\infer*{\Gamma e\vdash e5:
                                               \codeTs{t}{\gamma2\cup\gamma1\cup\gamma3};~~~\epsilon}
                                                    {}
                                            }
                                     }
                                  &\infer*{\Gamma d\vdash X:\codeTs{t}{\gamma0};\epsilon}{}
                                  }
                               }
                        }
                 }
             }
      }
    }
  \]

\end{frame}

\section{まとめと今後の課題}

\begin{frame}
  \frametitle{まとめと今後の課題}
  まとめ
  \begin{itemize}
    % \item コードの言語にshift0 reset0 を組み込んだ言語の設計を行った
  \item コード生成言語の型システムに shift0/reset0 を組み込んだ 型システムの設計を完成させた.
  \item 安全なコードの場合に型が付くこと,安全でないコードの場合には型が付かないように意図通りに型システムが設計できていることをみた
  \end{itemize}

  % \vspace{1in}
  \vspace{\baselineskip}

  今後の課題
  \begin{itemize}
    % \item answer type modification に対応した型システムを設計し,(subject reduction 等の)健全性の証明を行う
  \item 設計した型システムの健全性の証明(Subject recudtion等)
  \item 型推論アルゴリズムの開発
  \item 言語の拡張
    \begin{itemize}
    \item グローバルな参照 (OCamlのlet ref)
    \item 生成したコードの実行 (MetaOCamlの run)
    \end{itemize}
  \end{itemize}
\end{frame}

%%% Local Variables:
%%% mode: japanese-latex
%%% TeX-master: "slide"
%%% End:
