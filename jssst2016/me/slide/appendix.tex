%% ===============================================
%% 予備スライド
%% ===============================================

%% 予備スライドは appendix 環境の中に書きましょう.
\begin{appendix}
  %% 予備スライドの先頭に APPENDIX と書かれたページを挟む(お好みで消去しても良い)
  \frame[plain]{\centerline{\Huge\bfseries\color{structure}APPENDIX}}


\section{健全性の証明}

\begin{frame}[fragile]
  \frametitle{健全性の証明 (Subject Reduction)}

  型安全性(型システムの健全性; Subject Reduction等の性質)を厳密に証明する.

  \begin{block}{Subject Redcution Property}
    $\Gamma \vdash M: \tau$ が導ければ(プログラム$M$が型検査を通れば),
    $M$を計算して得られる任意の$N$に対して,
    $\Gamma \vdash N: \tau$ が導ける($N$も型検査を通り,$M$と同じ型,
    同じ自由変数を持つ)
  \end{block}
\end{frame}

\section{shift/reset と shift0/reset0 の意味論}

\begin{frame}[fragile]
  \frametitle{shift/reset と shift0/reset0 の意味論}

  shift/reset
  \noindent
  \begin{align*}
    \Resett ~(E[\Shiftt~ k \to e]) &~\leadsto~ \Resett~ e \ksubst{k}{E} \\
    \uncover<2->{\Resett ~(E[\Shiftt~ k \to e]) &~\leadsto~ \Resett~ e[k := \fun{x}{\Resett~ E[x]}]}
  \end{align*}

  shift0/reset0
  \noindent
  \begin{align*}
    \Resetz ~(E[\Shiftz~ k \to e]) &~\leadsto~ e \ksubst{k}{E} \\
    \uncover<2->{\Resetz ~(E[\Shiftz~ k \to e]) &~\leadsto~ e[k := \fun{x}{\Resetz~ E[x]}]}
  \end{align*}


\end{frame}

\end{appendix}

% なぜ shift0/reset0 なの? indexのあるshift/reset とか control/prompt とかは使えないの?

% 証明はどこまで進んでるの?

% answer type は型の列 こいつの説明

%%% Local Variables:
%%% mode: japanese-latex
%%% TeX-master: "slide"
%%% End:
