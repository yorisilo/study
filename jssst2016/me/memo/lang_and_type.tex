\documentclass[10pt,a4j]{jarticle}

\usepackage{amsmath,amssymb}
\usepackage{ascmac}
\usepackage{mathtools}
\usepackage{proof}
\usepackage{stmaryrd}
\usepackage[top=1cm, bottom=1cm, left=1cm, right=1cm, includefoot]{geometry}
\usepackage{theorem}

% \usepackage[margin=1.8cm]{geometry}

\usepackage{setspace}
% \setstretch{1} % ページ全体の行間を設定
% \setlength{\textheight}{38\baselineskip}
\setlength{\columnsep}{10mm}


% \renewcommand{\baselinestretch}{2}

% \overfullrule=0pt

\newcommand\fun[2]{\lambda{#1}.{#2}}

\newcommand\Resetz{\textbf{reset0}}
\newcommand\Shiftz{\textbf{shift0}}
\newcommand\Throw{\textbf{throw}}
\newcommand\resetz[1]{\Resetz~{#1}}
\newcommand\shiftz[2]{\Shiftz~{#1}\to{#2}}
\newcommand\throw[2]{\Throw~{#1}~{#2}}

\newcommand\cfun[2]{\underline{\lambda}{#1}.{#2}}
\newcommand\ccfun[2]{\underline{\underline{\lambda}}{#1}.{#2}}

\newcommand\cResetz{\underline{\textbf{reset0}}}
\newcommand\cShiftz{\underline{\textbf{shift0}}}
\newcommand\cThrow{\underline{\textbf{throw}}}
\newcommand\cresetz[1]{\cResetz~{#1}}
\newcommand\cshiftz[2]{\cShiftz~{#1}\to{#2}}
\newcommand\cthrow[2]{\cThrow~{#1}~{#2}}

\newcommand\cPlus{\underline{\textbf{+}}}

\newcommand\cLet{\underline{\textbf{clet}}}
\newcommand\cIn{\underline{\textbf{in}}}
\newcommand\clet[3]{\cLet~{#1}={#2}~\cIn~{#3}}
\newcommand\csp[1]{\texttt{\%}{#1}}
\newcommand\code[1]{\texttt{<}{#1}\texttt{>}}

\newcommand\codeT[2]{\langle{#1}\rangle^{#2}}
\newcommand\contT[2]{({#1} \Rightarrow {#2})}

\newcommand\ord{\ge}

\newcommand\Let{\textbf{let}}
\newcommand\In{\textbf{in}}
\newcommand\letin[3]{\Let~{#1}={#2}~\In~{#3}}

\newcommand\ift[3]{\textbf{if}~{#1}~\textbf{then}~{#2}~\textbf{else}~{#3}}
\newcommand\cif[3]{\underline{\textbf{cif}}~\code{{#1}}~\code{{#2}}~\code{{#3}}}
\newcommand\cIf{\underline{\textbf{cif}}}

\newcommand\fix{\textbf{fix}}
\newcommand\cfix{\underline{\textbf{fix}}}

\newcommand\lto{\leadsto}
\newcommand\cat{~\underline{@}~}

\theoremstyle{break}

\newtheorem{theo}{定理}[section]
\newtheorem{defi}{定義}[section]
\newtheorem{lemm}{補題}[section]

\title{コード生成 + Shift0/Reset0 の型システム}
\date{\today}
\author{大石純平}

\begin{document}
\maketitle

answer type は考えていない.\\
後で,answer type を加えたやつを考える.\\
answer type modificationについては考えない

\section{Syntax}
\begin{align*}
  v & ::= c \mid \fun{x}{e} \mid \code{e} \\
  e & ::=  x \mid c \mid \fun{x}{e} \mid e~e \\
    & \mid \cfun{x}{e}
      \mid \ccfun{x}{e}
      \mid \resetz{e}
      \mid \shiftz{k}{e}
      \mid \throw{k}{e} \\
    & \mid \clet{x}{e}{e}
      \mid \ift{e_1}{e_2}{e_3}
      \mid \cdots \\
  c & ::= N \mid B \mid \csp \mid \cat \mid + \mid \cPlus \mid \cIf \mid \fix \mid \cfix
\end{align*}

% fix http://d.hatena.ne.jp/tanakh/20040813

N is Integer numeric, B is Bool (true or false)

\section{Semantics}
left-to-right, call-by-value

\subsection{Evaluation Context}

\begin{align*}
  E & ::= [~] \mid E~ e \mid v~ E \mid \Resetz~ E \mid \ccfun{x}{E} \mid \ift{E}{e_1}{e_2}
\end{align*}

\subsection{Operation Semantics}
underline 付きのものは,コードコンビネータであり,なにか値を受け取ってコードを出すもの

underline がないもの: present stage で動く

underline があるもの: present stage で動かない

$\Shiftz$ $\Resetz$ $\Throw$ は コードの型を持つ $e$ のみを引数に取ることにする? $\Rightarrow$ する

コードレベルで shift0/reset0 throw は出てこないようにする? $\Rightarrow$ する

$\Throw~ k~ e$ ってあるけど,これ,$\Throw~ e$ にしたほうがいい? $\Rightarrow$ 良くない.

\begin{align*}
  & \infer{e \lto e'}
    {E[e] \lto E[e']} \\
  E[(\fun{x}{e})~v] &\lto E[e\{ x := v \}] \\
  E[\letin{x}{v}{e}] &\lto E[e\{ x := v \}]\\
  % E[\cResetz~\code{e}] &\lto E[\code{\resetz e}] \\
  E[\Resetz~ v] &\lto E[v] \\
  E[\cfun{x}{e}] &\lto E[\ccfun{y}{e\{ x := \code{y} \}}] \\
                    &y \text{ is fresh variable} \\
  E[\ccfun{y}{\code{e}}] &\lto E[\code{\fun{y}{e}}] \\
  % E[\cResetz (E'[\cShiftz ~k \to E''[\cThrow~ k~ e]])] &\lto E[E''[k e\{k := \cfun{x}{\cResetz~ (E'[x])} \}]] \\
  % E[\Resetz (E'[\Shiftz~ k \to E''[\Throw~ k~ e]])] &\lto E[E''[(k~ e)\{k := \cfun{x}{\Resetz~ (E'[x])} \}]] \\
  % &x \text{ is fresh variable} \\
  E[\Resetz (E'[\Shiftz~ k \to e]])] &\lto E[e \{k := \cfun{x}{\Resetz~ (E'[x])} \}] \\
                    &x \text{ is fresh variable} \\
  E[\Throw~ k~ v] &\lto E[k~ v] \\
  E[\code{e_1} \cat \code{e_2}] &\lto E[\code{e_1~ e_2}] \\
  % E[\clet{x}{e_1}{e_2}] &\lto E[\cfun{x}{e_2} \cat e_1] \\
  E[\clet{x}{e_1}{e_2}] &\lto E[\code{\letin{x}{e_1}{e_2}}] \\
  E[\% n] &\lto E[\code{n}] \\
  E[\code{e_1}~ \cPlus~ \code{e_2}] &\lto E[\code{e_1 + e_2}] \\
  E[\cif{e_1}{e_2}{e_3}] &\lto E[\code{\ift{e_1}{e_2}{e_3}}] \\
  E[\ift{true}{e_1}{e_2}] &\lto e_1 \\
  E[\ift{else}{e_1}{e_2}] &\lto e_2 \\
  E[\fix~ e_1] &\lto E[e_1~ (\fix~ e_1)] \\
\end{align*}
% Y M = M (Y M)
% Y = λf.(λx.f(x x)) (λx.f(x x))
% Y M => M (Y M) => M (M (Y M)) => ...
\paragraph{簡約例}

\begin{align*}
  e_1 & = \Resetz ~~\cLet~x_1=\csp{3}~\cIn \\
      & \phantom{=}~~ \Resetz ~~\cLet~x_2=\csp{5}~\cIn \\
      & \phantom{=}~~ \Shiftz~k~\to~\cLet~y=t~\cIn \\
      & \phantom{=}~~ \Throw~ k~ (x_1~\cPlus~x_2~\cPlus~y) \\
\end{align*}

\begin{align*}
  [ e_1 ] &\lto [ \Resetz (\cLet~x_1=\csp{3}~\cIn \\
          &\Resetz~ \cLet~x_2=\csp{5}~\cIn \\
          &[ \Shiftz~ k~ \to~ \cLet~ y=t~ \cIn \\
          &[ \Throw~ k~(x_1~\cPlus~x_2~\cPlus~y) ] ] ) ] \\
          &\lto [ \cLet~ y=t~ \cIn \\
          &[ \cfun{x}{\Resetz~ (\cLet~x_1=\csp{3}~ \cIn~ \Resetz~ (\cLet~ x_2=\csp{5}~ \cIn [x]))} (x_1~\cPlus~x_2~\cPlus~y) ]] \\
          &\lto [ \cfun{y}{(\cfun{x}{\Resetz~ (\cLet~x_1=\csp{3}~ \cIn~ \Resetz~ (\cLet~ x_2=\csp{5}~ \cIn [x]))} (x_1~\cPlus~x_2~\cPlus~y))} \cat t ] \\
          &\lto [[\cfun{y}{(\cfun{x}{\Resetz~ (\cLet~x_1=\csp{3}~ \cIn~ \Resetz~ (\cLet~ x_2=\csp{5}~ \cIn [x]))} (x_1~\cPlus~x_2~\cPlus~y))}] \cat t] \\
          &\lto [[\ccfun{y_1}{(\cfun{x}{\Resetz~ (\cLet~x_1=\csp{3}~ \cIn~ \Resetz~ (\cLet~ x_2=\csp{5}~ \cIn [x]))} (x_1~\cPlus~x_2~\cPlus~ \code{y_1}))}] \cat t] \\
          &\lto
\end{align*}

let ref の e1 e2 の制限 scope extrusion問題への対処
shift reset で同じようなことをかけるので,これについて考える


\section{Type System}

\begin{align*}
  t & ::= \textrm{BasicType} \mid t \to t \mid \codeT{t}{\gamma}
\end{align*}

Typing rule for code-level lambda:
\[
  \infer[(\gamma_1~\text{is eigen var})]
  {\Gamma \vdash \cfun{x}{e} ~:~ \codeT{t_1\to t_2}{\gamma}}
  {\Gamma,~\gamma_1 \ord \gamma,~x:\codeT{t_1}{\gamma_1} \vdash e
    ~:~ \codeT{t_2}{\gamma_1}}
\]

Typing rule for code-level let (derived rule):
\[
  \infer[(\gamma_1~\text{is eigen var})]
  {\Gamma \vdash \clet{x}{e_1}{e_2} ~:~ \codeT{t_2}{\gamma}}
  {\Gamma \vdash e_1 ~:~ \codeT{t_1}{\gamma}
    &\Gamma,~\gamma_1 \ord \gamma,~x:\codeT{t_1}{\gamma_1} \vdash
    e_2 ~:~ \codeT{t_2}{\gamma_1}
  }
\]

reset0, shift0, throw のアンダーラインは取る?
$\rightarrow$ present stage で shift reset throw も動くので.

Typing rule for code-level reset0:
\[
  \infer{\Gamma \vdash \resetz{e} ~:~ \codeT{t}{\gamma}}
  {\Gamma \vdash e ~:~ \codeT{t}{\gamma}}
\]

Typing rule for code-level shift0:
\[
  \infer{\Gamma \vdash \shiftz{k}{e} ~:~ \codeT{t_1}{\gamma_1}}
  {\Gamma,~k:\contT{\codeT{t_1}{\gamma_1}}{\codeT{t_0}{\gamma_0}}
    \vdash e ~:~ \codeT{t_0}{\gamma_0}
    & \Gamma \models \gamma_1 \ord \gamma_0
  }
\]

Typing rule for code-level throw:
\[
  \infer[(\gamma_3~\text{is eigen var})]
  {\Gamma,~k:\contT{\codeT{t_1}{\gamma_1}}{\codeT{t_0}{\gamma_0}}
    \vdash \throw{k}{e} ~:~ \codeT{t_0}{\gamma_2}}
  {\Gamma,~
    \gamma_3 \ord \gamma_1,~
    \gamma_3 \ord \gamma_2
    \vdash e ~:~ \codeT{t_1}{\gamma_3}
    & \Gamma \models \gamma_2 \ord \gamma_0
  }
\]

Typing rule for Subs-0:
\[
  \infer
  {\Gamma,~k:\contT{\codeT{t_1}{\gamma_1}}{\codeT{t_0}{\gamma_0}}
    \vdash \throw{k}{e} ~:~ \codeT{t_0}{\gamma_2}}
  {\Gamma,~
    \gamma_3 \ord \gamma_1,~
    \gamma_3 \ord \gamma_2
    \vdash e ~:~ \codeT{t_1}{\gamma_3}
    & \Gamma \models \gamma_2 \ord \gamma_0
  }
\]

\section{Example}

\begin{align*}
  e_1 & = \Resetz ~~\cLet~x_1=\csp{3}~\cIn \\
      & \phantom{=}~~ \Resetz ~~\cLet~x_2=\csp{5}~\cIn \\
      & \phantom{=}~~ \Shiftz~k~\to~\cLet~y=t~\cIn \\
      & \phantom{=}~~ \Throw~k~(x_1~\cPlus~x_2~\cPlus~y)
\end{align*}

If $t=\csp{7}$ or $t=x_1$, then $e_1$ is typable.

If $t=x_2$, then $e_1$ is not typable.

\begin{align*}
  e_2 & = \Resetz ~~\cLet~x_1=\csp{3}~\cIn \\
      & \phantom{=}~~ \Resetz ~~\cLet~x_2=\csp{5}~\cIn \\
      & \phantom{=}~~ \Shiftz~k_2~\to~ \Shiftz~k_1~\to~ \cLet~y=t~\cIn \\
      & \phantom{=}~~ \Throw~k_1~(\Throw~k_2~(x_1~\cPlus~x_2~\cPlus~y))
\end{align*}

If $t=\csp{7}$, then $e_1$ is typable.

If $t=x_2$ or $t=x_1$, then $e_1$ is not typable.

\section{型安全性の証明}
型システムの健全性を型保存定理,進行定理によって証明する

\subsection{型保存}
\begin{theo}[型保存]
  $\vdash e:t$ かつ $e \lto e'$ であれば,$\vdash e':t$ である
\end{theo}

\subsection{進行}
\begin{theo}[進行]
  $\vdash e:t$ が導出可能であれば,$e$ は 値 $v$ である.または,$e \lto e'$ であるような 項 $e'$ が存在する
\end{theo}

\paragraph{証明}
$\vdash e:t$ の導出に関する帰納法による.\\
Const, Abs, Code 規則の場合 $e$ は値である.\\
Var 規則の場合 $\vdash e:t$ は導出可能でない.\\
Throw 規則の場合 $\vdash e:t$ は導出可能でない.\\
Reset0 規則の場合 $e = \Resetz~ e_1$ とする.
帰納法の仮定より評価文脈における $\Resetz E$ より簡約が進み,\\
$e_1$ が値のとき,$e \lto v$ となるような $v$ が存在する.\\
$e_1$ が値でないとき,

\end{document}

%%% Local Variables:
%%% mode: japanese-latex
%%% TeX-master: t
%%% End:
