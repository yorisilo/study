\documentclass[10pt,a4j]{jarticle}

\usepackage{amsmath,amssymb}
\usepackage{ascmac}
\usepackage{mathtools}
\usepackage{proof}
\usepackage{stmaryrd}
\usepackage[top=1cm, bottom=1cm, left=1cm, right=1cm, includefoot]{geometry}

% \usepackage[margin=1.8cm]{geometry}

\usepackage{setspace}
% \setstretch{1} % ページ全体の行間を設定
% \setlength{\textheight}{38\baselineskip}
\setlength{\columnsep}{10mm}


% \renewcommand{\baselinestretch}{2}

% \overfullrule=0pt

\newcommand\fun[2]{\lambda{#1}.{#2}}

\newcommand\Resetz{\textbf{reset0}}
\newcommand\Shiftz{\textbf{shift0}}
\newcommand\Throw{\textbf{throw}}
\newcommand\resetz[1]{\Resetz~{#1}}
\newcommand\shiftz[2]{\Shiftz~{#1}.{#2}}
\newcommand\throw[2]{\Throw~{#1}~{#2}}

\newcommand\cfun[2]{\underline{\lambda}{#1}.{#2}}
\newcommand\ccfun[2]{\underline{\underline{\lambda}}{#1}.{#2}}

\newcommand\cResetz{\underline{\textbf{reset0}}}
\newcommand\cShiftz{\underline{\textbf{shift0}}}
\newcommand\cThrow{\underline{\textbf{throw}}}
\newcommand\cresetz[1]{\cResetz~{#1}}
\newcommand\cshiftz[2]{\cShiftz~{#1}\to{#2}}
\newcommand\cthrow[2]{\cThrow~{#1}~{#2}}

\newcommand\cPlus{\underline{\textbf{+}}}

\newcommand\cLet{\underline{\textbf{clet}}}
\newcommand\cIn{\underline{\textbf{in}}}
\newcommand\clet[3]{\cLet~{#1}={#2}~\cIn~{#3}}
\newcommand\csp[1]{\texttt{\%}{#1}}
\newcommand\code[1]{\texttt{<}{#1}\texttt{>}}

\newcommand\codeT[2]{\langle{#1}\rangle^{#2}}
\newcommand\contT[2]{({#1} \Rightarrow {#2})}

\newcommand\ord{\ge}

\newcommand\lto{\leadsto}
\newcommand\cat{~\underline{@}~}

\title{コード生成 + Shift0/Reset0 の型システム}
\date{\today}
\author{大石純平}

\begin{document}
\maketitle

answer type は考えていない.\\
後で,answer type を加えたやつを考える.\\
answer type modificationについては考えない

\section{Syntax}
\begin{align*}
  v & ::= c \mid \fun{x}{e} \mid \code{e} \\
  e & ::=  x \mid c \mid \fun{x}{e} \mid e~e \\
    & \mid \cfun{x}{e}
      \mid \cresetz{e}
      \mid \cshiftz{k}{e}
      \mid \cthrow{k}{e} \\
    & \mid \clet{x}{e}{e}
      \mid \cdots \\
  c & ::= N \mid B \mid \% \mid \cat \mid + \mid \cPlus
\end{align*}

N is Integer numeric, B is Bool (true or false)

\section{Semantics}
left-to-right, call-by-value
\subsection{Evaluation Context}

\begin{align*}
  E & ::= [~] \mid E~ e \mid v~ E \mid \cResetz~ E \mid \ccfun{x}{E}
\end{align*}

\subsection{Operation Semantics}

\begin{align*}
  E[(\fun{x}{e})~v] &\lto E[e\{ x := v \}] \\
  E[\cResetz~\code{e}] &\lto E[\code{e}] \\
  E[\cfun{x}{e}] &\lto E[\ccfun{y}{e\{ x := \code{y} \}}] ~~y \text{ is fresh variable} \\
  E[\ccfun{y}{\code{e}}] &\lto E[\code{\fun{y}{e}}] \\
  % E[\cResetz (E'[\cShiftz ~k \to E''[\cThrow~ k~ e]])] &\lto E[E''[k e\{k := \cfun{x}{\cResetz~ (E'[x])} \}]] \\
  E[\cResetz (E'[\cShiftz~ k \to E''[\cThrow~ k~ e]])] &\lto E[E''[(k~ e)\{k := \cfun{x}{\cResetz~ (E'[x])} \}]] \\
                    &x \text{ is fresh variable} \\
  E[\code{e_1} \cat \code{e_2}] &\lto E[\code{e_1~ e_2}] \\
  E[\clet{x}{e_1}{e_2}] &\lto E[\cfun{x}{e_2} \cat e_1] \\
  E[\% n] &\lto E[\code{n}] \\
  E[\code{e_1}~ \cPlus~ \code{e_2}] &\lto E[\code{e_1 + e_2}]
\end{align*}

\section{Type System}

\begin{align*}
  t & ::= \textrm{BasicType} \mid t \to t \mid \codeT{t}{\gamma}
\end{align*}

Typing rule for code-level lambda:
\[
  \infer[(\gamma_1~\text{is eigen var})]
  {\Gamma \vdash \cfun{x}{e} ~:~ \codeT{t_1\to t_2}{\gamma}}
  {\Gamma,~\gamma_1 \ord \gamma,~x:\codeT{t_1}{\gamma_1} \vdash e
    ~:~ \codeT{t_2}{\gamma_1}}
\]

Typing rule for code-level let (derived rule):
\[
  \infer[(\gamma_1~\text{is eigen var})]
  {\Gamma \vdash \clet{x}{e_1}{e_2} ~:~ \codeT{t_2}{\gamma}}
  {\Gamma \vdash e_1 ~:~ \codeT{t_1}{\gamma}
    &\Gamma,~\gamma_1 \ord \gamma,~x:\codeT{t_1}{\gamma_1} \vdash
    e_2 ~:~ \codeT{t_2}{\gamma_1}
  }
\]

Typing rule for code-level reset0:
\[
  \infer{\Gamma \vdash \cresetz{e} ~:~ \codeT{t}{\gamma}}
  {\Gamma \vdash e ~:~ \codeT{t}{\gamma}}
\]

Typing rule for code-level shift0:
\[
  \infer{\Gamma \vdash \cshiftz{k}{e} ~:~ \codeT{t_1}{\gamma_1}}
  {\Gamma,~k:\contT{\codeT{t_1}{\gamma_1}}{\codeT{t_0}{\gamma_0}}
    \vdash e ~:~ \codeT{t_0}{\gamma_0}
    & \Gamma \models \gamma_1 \ord \gamma_0
  }
\]

Typing rule for code-level throw:
\[
  \infer[(\gamma_3~\text{is eigen var})]
  {\Gamma,~k:\contT{\codeT{t_1}{\gamma_1}}{\codeT{t_0}{\gamma_0}}
    \vdash \cthrow{k}{e} ~:~ \codeT{t_0}{\gamma_2}}
  {\Gamma,~
    \gamma_3 \ord \gamma_1,~
    \gamma_3 \ord \gamma_2
    \vdash e ~:~ \codeT{t_1}{\gamma_3}
    & \Gamma \models \gamma_2 \ord \gamma_0
  }
\]


\section{Example}

\begin{align*}
  e_1 & = \cResetz ~~\cLet~x_1=\csp{3}~\cIn \\
      & \phantom{=}~~ \cResetz ~~\cLet~x_2=\csp{5}~\cIn \\
      & \phantom{=}~~ \cShiftz~k~\to~\cLet~y=t~\cIn \\
      & \phantom{=}~~ \cThrow~k~(x_1~\cPlus~x_2~\cPlus~y)
\end{align*}

If $t=\csp{7}$ or $t=x_1$, then $e_1$ is typable.

If $t=x_2$, then $e_1$ is not typable.

\begin{align*}
  e_2 & = \cResetz ~~\cLet~x_1=\csp{3}~\cIn \\
      & \phantom{=}~~ \cResetz ~~\cLet~x_2=\csp{5}~\cIn \\
      & \phantom{=}~~ \cShiftz~k_2~\to~ \cShiftz~k_1~\to~ \cLet~y=t~\cIn \\
      & \phantom{=}~~ \cThrow~k_1~(\cThrow~k_2~(x_1~\cPlus~x_2~\cPlus~y))
\end{align*}

If $t=\csp{7}$, then $e_1$ is typable.

If $t=x_2$ or $t=x_1$, then $e_1$ is not typable.

\end{document}

%%% Local Variables:
%%% mode: japanese-latex
%%% TeX-master: t
%%% End:
