\documentclass[10pt,a4j]{jarticle}

\usepackage[dvipdfmx]{graphicx,color}

\usepackage{amsmath,amssymb}
\usepackage{ascmac}
\usepackage{mathtools}
\usepackage{proof}
\usepackage{stmaryrd}
\usepackage[top=1cm, bottom=1cm, left=1cm, right=1cm, includefoot]{geometry}
\usepackage{theorem}

% \usepackage[margin=1.8cm]{geometry}

\usepackage{setspace}
% \setstretch{1} % ページ全体の行間を設定
% \setlength{\textheight}{38\baselineskip}
\setlength{\columnsep}{10mm}


% \renewcommand{\baselinestretch}{2}

% \overfullrule=0pt

\definecolor{DarkGreen}{rgb}{0,0.5,0}
\definecolor{Magenta}{rgb}{1.0, 0.0, 1.0}

\newcommand\too{\leadsto^*}
\newcommand\pink[1]{\textcolor{pink}{#1}}
\newcommand\red[1]{\textcolor{red}{#1}}
\newcommand\green[1]{\textcolor{green}{#1}}
\newcommand\magenta[1]{\textcolor{magenta}{#1}}
\newcommand\blue[1]{\textcolor{blue}{#1}}

\newcommand\fun[2]{\lambda{#1}.{#2}}

\newcommand\Resetz{\textbf{reset0}}
\newcommand\Shiftz{\textbf{shift0}}
\newcommand\Throw{\textbf{throw}}
\newcommand\resetz[1]{\Resetz~{#1}}
\newcommand\shiftz[2]{\Shiftz~{#1}\to{#2}}
\newcommand\throw[2]{\Throw~{#1}~{#2}}

\newcommand\cfun[2]{\underline{\lambda}{#1}.{#2}}
\newcommand\ccfun[2]{\underline{\underline{\lambda}}{#1}.{#2}}

\newcommand\cResetz{\underline{\textbf{reset0}}}
\newcommand\cShiftz{\underline{\textbf{shift0}}}
\newcommand\cThrow{\underline{\textbf{throw}}}
\newcommand\cresetz[1]{\cResetz~{#1}}
\newcommand\cshiftz[2]{\cShiftz~{#1}\to{#2}}
\newcommand\cthrow[2]{\cThrow~{#1}~{#2}}

\newcommand\cPlus{\underline{\textbf{+}}}

\newcommand\cLet{\underline{\textbf{clet}}}
\newcommand\cIn{\underline{\textbf{in}}}
\newcommand\clet[3]{\cLet~{#1}={#2}~\cIn~{#3}}
\newcommand\csp[1]{\texttt{\%}{#1}}
\newcommand\cint{\underline{\textbf{cint}}}
\newcommand\code[1]{\texttt{<}{#1}\texttt{>}}

\newcommand\codeT[2]{\langle{#1}\rangle^{#2}}
\newcommand\contT[2]{({#1} \Rightarrow {#2})}

\newcommand\ord{\ge}

\newcommand\Let{\textbf{let}}
\newcommand\In{\textbf{in}}
\newcommand\letin[3]{\Let~{#1}={#2}~\In~{#3}}

\newcommand\ift[3]{\textbf{if}~{#1}~\textbf{then}~{#2}~\textbf{else}~{#3}}
\newcommand\cif[3]{\underline{\textbf{cif}}~\code{{#1}}~\code{{#2}}~\code{{#3}}}
\newcommand\cIf{\underline{\textbf{cif}}}

\newcommand\fix{\textbf{fix}}
\newcommand\cfix{\underline{\textbf{fix}}}

\newcommand\lto{\leadsto}
\newcommand\cat{\underline{@}}

% コメントマクロ
\newcommand\kam[1]{\red{{#1}}}
\newcommand\ooi[1]{\blue{{#1}}}

\theoremstyle{break}

\newtheorem{theo}{定理}[section]
\newtheorem{defi}{定義}[section]
\newtheorem{lemm}{補題}[section]

\title{コード生成 + Shift0/Reset0 の型システム}
\date{\today}
\author{大石純平}

\begin{document}
\maketitle

answer type は考えていない.\\
後で,answer type を加えたやつを考える.\\
answer type modificationについては考えない

\section{Syntax}
\begin{align*}
  v & ::= c_0 \mid \fun{x}{e} \mid \code{e} \\
  e & ::=  x \mid c_0 \mid c_1~ e \mid c_2~ e_1~ e_2 \mid c_3~ e_1~ e_2~ e_3 \mid \fun{x}{e} \mid e_1~ e_2 \\
    & \mid \cfun{x}{e}
      \mid \ccfun{x}{e}
      \mid \resetz{e}
      \mid \shiftz{k}{e}
      \mid \throw{k}{e} \\
    & \mid \clet{x}{e_1}{e_2}
      \mid \ift{e_1}{e_2}{e_3} \\
  c_0 & ::= N \mid B \\
  & \text{N is Integer numeric (1,2,3,$\cdots$), B is Bool (true or false)}\\
  c_1 & ::= \cint \mid \fix \mid \cfix \\
  c_2 & ::= \cat \mid + \mid \cPlus \\
  c_3 & ::= \cIf
\end{align*}

% fix http://d.hatena.ne.jp/tanakh/20040813

\section{Semantics}
left-to-right, call-by-value

\subsection{Evaluation Context}

\begin{align*}
  E & ::= [~] \mid E~ e \mid v~ E \\
    & \mid c_1~ E \mid c_2~ E~ e \mid c_2~ v~ E \\
    & \mid \ift{E}{e_1}{e_2} \mid \Resetz~ E \mid \ccfun{x}{E}
\end{align*}

\subsection{Operation Semantics}
underline 付きのものは,コードコンビネータであり,なにか値を受け取ってコードを出すもの

underline がないもの: present stage で動く

underline があるもの: present stage で動かない

$\Shiftz$ $\Resetz$ $\Throw$ は コードの型を持つ $e$ のみを引数に取ることにする? $\Rightarrow$ する

コードレベルで shift0/reset0 throw は出てこないようにする? $\Rightarrow$ する

$\Throw~ k~ e$ ってあるけど,これ,$\Throw~ e$ にしたほうがいい? $\Rightarrow$ 良くない.

\begin{align*}
  &\infer{e \lto e'}
    {E[e] \lto E[e']} \\
\end{align*}

\begin{align*}
  (\fun{x}{e})~v &\lto e\{ x := v \} \\
  \letin{x}{v}{e} &\lto e\{ x := v \}\\
  \ift{true}{e_1}{e_2} &\lto e_1 \\
  \ift{else}{e_1}{e_2} &\lto e_2 \\
  \cfun{x}{e} &\lto \ccfun{y}{e\{ x := \code{y} \}} \\
  &y \text{ is fresh variable} \\
  \ccfun{y}{\code{e}} &\lto \code{\fun{y}{e}} \\
  % E[\cResetz (E'[\cShiftz ~k \to E''[\cThrow~ k~ e]])] &\lto E[E''[k e\{k := \cfun{x}{\cResetz~ (E'[x])} \}]] \\
  % E[\Resetz (E'[\Shiftz~ k \to E''[\Throw~ k~ e]])] &\lto E[E''[(k~ e)\{k := \cfun{x}{\Resetz~ (E'[x])} \}]] \\
  % &x \text{ is fresh variable} \\
  \Resetz~ v &\lto v \\
  \Resetz (E[\Shiftz~ k \to e]) &\lto e \{k := \cfun{x}{\Resetz (E[x])} \} \\
  &x \text{ is fresh variable} \\
  \Throw~ k~ v &\lto k~ v \\
\end{align*}
\ooi{$\Throw$ の簡約は,代入に lambda-substitution のような感じで定義することで無くす?}

\begin{align*}
  \fix~ e &\lto e~ (\fix~ e) \\
  \cfix~ e &\lto \code{\fix~ e} \\
  \cint~ n &\lto \code{n} \\
  \code{e_1}~ \cat~ \code{e_2} &\lto \code{e_1~ e_2} \\
  \code{e_1}~ \cPlus~ \code{e_2} &\lto \code{e_1 + e_2} \\
  \cif{e_1}{e_2}{e_3} &\lto \code{\ift{e_1}{e_2}{e_3}} \\
  \clet{x}{e_1}{e_2} &\lto \code{\letin{x}{e_1}{e_2}} \\
\end{align*}
% Y M = M (Y M)
% Y = λf.(λx.f(x x)) (λx.f(x x))
% Y M => M (Y M) => M (M (Y M)) => ...
\paragraph{簡約例}

\begin{align*}
  e_1 & = \Resetz ~~\cLet~x_1=\csp{3}~\cIn \\
      & \phantom{=}~~ \Resetz ~~\cLet~x_2=\csp{5}~\cIn \\
      & \phantom{=}~~ \Shiftz~k~\to~\cLet~y=t~\cIn \\
      & \phantom{=}~~ \Throw~ k~ (x_1~\cPlus~x_2~\cPlus~y) \\
\end{align*}

\begin{align*}
  [ e_1 ] &\lto [ \Resetz (\cLet~x_1=\csp{3}~\cIn \\
          &\Resetz~ \cLet~x_2=\csp{5}~\cIn \\
          &[ \Shiftz~ k~ \to~ \cLet~ y=t~ \cIn \\
          &[ \Throw~ k~(x_1~\cPlus~x_2~\cPlus~y) ] ] ) ] \\
          &\lto [ \cLet~ y=t~ \cIn \\
          &[ \cfun{x}{\Resetz~ (\cLet~x_1=\csp{3}~ \cIn~ \Resetz~ (\cLet~ x_2=\csp{5}~ \cIn [x]))} (x_1~\cPlus~x_2~\cPlus~y) ]] \\
          &\lto [ \cfun{y}{(\cfun{x}{\Resetz~ (\cLet~x_1=\csp{3}~ \cIn~ \Resetz~ (\cLet~ x_2=\csp{5}~ \cIn [x]))} (x_1~\cPlus~x_2~\cPlus~y))}~ \cat~ t ] \\
          &\lto [[\cfun{y}{(\cfun{x}{\Resetz~ (\cLet~x_1=\csp{3}~ \cIn~ \Resetz~ (\cLet~ x_2=\csp{5}~ \cIn [x]))} (x_1~\cPlus~x_2~\cPlus~y))}]~ \cat~ t] \\
          &\lto [[\ccfun{y_1}{(\cfun{x}{\Resetz~ (\cLet~x_1=\csp{3}~ \cIn~ \Resetz~ (\cLet~ x_2=\csp{5}~ \cIn [x]))} (x_1~\cPlus~x_2~\cPlus~ \code{y_1}))}]~ \cat~ t] \\
          &\lto
\end{align*}

let ref の e1 e2 の制限 scope extrusion問題への対処
shift reset で同じようなことをかけるので,これについて考える


\section{Type System}

\begin{align*}
  \textrm{BasicType} & ::= \textrm{Int} \mid \textrm{Bool}
\end{align*}

\begin{align*}
  t & ::= \textrm{BasicType} \mid t \to t \mid \codeT{t}{\gamma} \\
  \sigma & ::= \epsilon \mid t, \sigma
\end{align*}

\begin{align*}
  \Gamma ::= \emptyset
  \mid \Gamma, (\gamma \ord \gamma)
  \mid \Gamma, (x : t)^{L} ; \sigma
  \mid \Gamma, (u^1 : t)^{\gamma} ; \sigma
\end{align*}

Typing judgements take the form:

\begin{align*}
  \Gamma \vdash^{L} e : t ; \sigma \\
  \Gamma \models \gamma_1 \ord \gamma_2
\end{align*}

Environment Classifier の rule:
\[
  \infer
  {\Gamma, \gamma_1 \ord \gamma_2 \models \gamma_1 \ord \gamma_2}
  {}
\]

\[
  \infer
  {\Gamma \models \gamma_1 \ord \gamma_3}
  {\Gamma \models \gamma_1 \ord \gamma_2 & \Gamma \models \gamma_2 \ord \gamma_3}
\]



Typing rule for code-level lambda:
\[
  \infer[(\gamma_1~\text{is eigen var})]
  {\Gamma \vdash \cfun{x}{e} : \codeT{t_1\to t_2}{\gamma} ; \sigma}
  {\Gamma,~\gamma_1 \ord \gamma,~x:\codeT{t_1}{\gamma_1} ; \sigma \vdash e
    : \codeT{t_2}{\gamma_1}; \sigma}
\]

Typing rule for code-level let (derived rule):
\[
  \infer[(\gamma_1~\text{is eigen var})]
  {\Gamma \vdash \clet{x}{e_1}{e_2} : \codeT{t_2}{\gamma} ; \sigma}
  {\Gamma \vdash e_1 : \codeT{t_1}{\gamma} ; \sigma
    &\Gamma,~\gamma_1 \ord \gamma,~x:\codeT{t_1}{\gamma_1} ; \sigma \vdash
    e_2 : \codeT{t_2}{\gamma_1} ; \sigma
  }
\]

\ooi{reset0, shift0, throw は,present-stage で動き,引数はコード型のみ}


Typing rule for code-level reset0:
\[
  \infer{\Gamma \vdash \resetz{e} : \codeT{t}{\gamma} ; \sigma}
  {\Gamma \vdash e : \codeT{t}{\gamma} ; \codeT{t}{\gamma}, \sigma}
\]

Typing rule for code-level shift0:
\[
  \infer{\Gamma \vdash \shiftz{k}{e} : \codeT{t_1}{\gamma_1} ; \sigma}
  {\Gamma,~k:\contT{\codeT{t_1}{\gamma_1}}{\codeT{t_0}{\gamma_0} ; \sigma}
    \vdash e : \codeT{t_0}{\gamma_0} ; \sigma
    & \Gamma \models \gamma_1 \ord \gamma_0
  }
\]

Typing rule for code-level throw:
\[
  \infer[(\gamma_3~\text{is eigen var})]
  {\Gamma,~k:\contT{\codeT{t_1}{\gamma_1}}{\codeT{t_0}{\gamma_0}} ; \sigma
    \vdash \throw{k}{e} : \codeT{t_0}{\gamma_2} ; \sigma}
  {\Gamma,~
    \gamma_3 \ord \gamma_1,~
    \gamma_3 \ord \gamma_2
    \vdash e : \codeT{t_1}{\gamma_3} ; \sigma
    & \Gamma \models \gamma_2 \ord \gamma_0
  }
\]

Typing rule for Subs-0:
\[
  \infer
  {\Gamma \vdash e : \codeT{t}{\gamma_1} ; \sigma}
  {\Gamma \vdash e : \codeT{t}{\gamma_2} ; \sigma
    & \Gamma \models \gamma_1 \ord \gamma_2
  }
\]

Typing rule for Subs-1:
\[
  \infer
  {\Gamma \vdash e : \codeT{t}{\gamma_1} ; \sigma}
  {\Gamma \vdash e : \codeT{t}{\gamma_2} ; \sigma
    & \Gamma \models \gamma_1 \ord \gamma_2
  }
\]

Typing rule for Var:
\[
  \infer
  {\Gamma, (x : t)^{L} ; \sigma \vdash^{L} x : t ; \sigma}
  {}
\]

Typing rule for App:
\[
  \infer
  {\Gamma \vdash^{L} e_1~ e_2 : t_1 ; \sigma}
  {\Gamma \vdash^{L} e_1 : t_2 \to t_1 ; \sigma
    & \Gamma \vdash^{L} e_2 : t_2  ; \sigma
  }
\]

Typing rule for Abs:
\[
  \infer
  {\Gamma \vdash^{L} \fun{x}{e} : t_1 \to t_2 \sigma}
  {\Gamma,~(x : t_1)^{L} ; \sigma \vdash^{L} e : t_2 ; \sigma}
\]

Typing rule for If:
\[
  \infer
  {\Gamma \vdash^{L} \ift{e_1}{e_2}{e_3} : t ; \sigma}
  {\Gamma \vdash^{L} e_1 : \textrm{Bool} ; \sigma
    & \Gamma \vdash^{L} e_3 : t ; \sigma}
\]

Typing rule for CAbs: $\gamma_1 \not \in FCV(\Gamma, \codeT{t_1 \to t_2}{\gamma})$
\[
  \infer
  {\Gamma \vdash \cfun{x}{e} : \codeT{t_1 \to t_2}{\gamma} ; \sigma}
  {\Gamma, \gamma_1 \ord \gamma, x : \codeT{t_1}{\gamma_1} ; \sigma \vdash e : \codeT{t_2}{\gamma_1} ; \sigma}
\]

Typing rule for IAbs: $\gamma_1 \not \in FCV(\Gamma, \codeT{t_1 \to t_2}{\gamma})$
\[
  \infer
  {\Gamma \vdash \ccfun{u^1}{e} : \codeT{t_1 \to t_2}{\gamma} ; \sigma}
  {\Gamma, \gamma_1 \ord \gamma, x : (u : t_1)^{\gamma_1} ; \sigma \vdash e : \codeT{t_2}{\gamma_1} ; \sigma}
\]

Typing rule for Code:
\[
  \infer
  {\Gamma \vdash \code{e^1} : \codeT{t^1}{\gamma} ; \sigma}
  {\Gamma \vdash^{\gamma} e^1 : t^1 ; \sigma}
\]

Typing rule for Const:
\[
  \infer
  {\Gamma \vdash^{L} c : t^c}
  {}
\]


\ooi{コードレベルの変数$u^1$,コードレベルの項 $e^1$,コードレベルの型 $t^1$,あと,$t^c$  についてなぜ必要なのかよく分かっていない...}

\section{Example}

\begin{align*}
  e_1 & = \Resetz ~~\cLet~x_1=\csp{3}~\cIn \\
      & \phantom{=}~~ \Resetz ~~\cLet~x_2=\csp{5}~\cIn \\
      & \phantom{=}~~ \Shiftz~k~\to~\cLet~y=t~\cIn \\
      & \phantom{=}~~ \Throw~k~(x_1~\cPlus~x_2~\cPlus~y)
\end{align*}

If $t=\csp{7}$ or $t=x_1$, then $e_1$ is typable.

If $t=x_2$, then $e_1$ is not typable.

\begin{align*}
  e_2 & = \Resetz ~~\cLet~x_1=\csp{3}~\cIn \\
      & \phantom{=}~~ \Resetz ~~\cLet~x_2=\csp{5}~\cIn \\
      & \phantom{=}~~ \Shiftz~k_2~\to~ \Shiftz~k_1~\to~ \cLet~y=t~\cIn \\
      & \phantom{=}~~ \Throw~k_1~(\Throw~k_2~(x_1~\cPlus~x_2~\cPlus~y))
\end{align*}

If $t=\csp{7}$, then $e_1$ is typable.

If $t=x_2$ or $t=x_1$, then $e_1$ is not typable.

\section{型安全性の証明}
型システムの健全性を型保存定理,進行定理によって証明する

\subsection{型保存 (subject reduction)}
\begin{theo}[型保存]
  $\vdash e:t$ かつ $e \lto e'$ であれば,$\vdash e':t$ である
\end{theo}

\begin{lemm}[代入]
  $\Gamma_1, \Gamma_2, x : t_1 \vdash e : t_2 $
  かつ
  $\Gamma_1 \vdash v : t_1 $
  ならば,
  $\Gamma_1, \Gamma_2 \vdash e\{x := v\} : t_2$
\end{lemm}

% \begin{lemm}
%   $e_1 = \cfun{x}{e_3}$ \\
%   $e_2 = \ccfun{y}{e_3\{x:=\code{y}\}}~~\text{y is fresh}$

%   \[
%     \infer
%     {\Gamma \vdash \cfun{x}{e_3} : \codeT{t_1 \to t_2}{\gamma} ; \sigma}
%     {\Gamma, \gamma_1 \ord \gamma, x : \codeT{t_1}{\gamma_1} ; \sigma \vdash e : \codeT{t_2}{\gamma_1} ; \sigma}
%   \]

%   であれば,

%   \[
%     \infer
%     {\Gamma \vdash \ccfun{y}{e_3\{x:=\code{y}\}} : \codeT{t_1 \to t_2}{\gamma} ; \sigma}
%     {\Gamma, \gamma_1 \ord \gamma, (y : t_1)^{\gamma_1} ; \sigma \vdash e_3\{x:=\code{y}\} : \codeT{t_2}{\gamma_1} ; \sigma}
%   \]

% \end{lemm}

\begin{lemm}[$\underline{\lambda}$]
  $\Gamma, \gamma_1 \ord \gamma, x : \codeT{t_1}{\gamma_1} ; \sigma \vdash e : \codeT{t_2}{\gamma_1} ; \sigma$
  ならば,
  $\Gamma, \gamma_1 \ord \gamma, (y : t_1)^{\gamma_1} ; \sigma \vdash e_3\{x:=\code{y}\} : \codeT{t_2}{\gamma_1} ; \sigma$
\end{lemm}

\begin{lemm}[$\underline{\underline{\lambda}}$]
  $\gamma_1 \ord \gamma, \Gamma, (y : t_1)^{\gamma_1} ; \sigma \vdash^{\gamma_1} e : t_2 ; \sigma$
  ならば,
  $\Gamma, (y : t_1)^{\gamma} \vdash^{\gamma} e : t_2 ; \sigma$
\end{lemm}

\begin{lemm}[$\Resetz - \Shiftz$]
  $\Gamma \models \gamma_1 \ord \gamma_0  \Gamma, k : \contT{\codeT{t_1}{\gamma_1}}{\codeT{t_0}{\gamma_0}} ; \sigma \vdash e : \codeT{t_0}{\gamma_0} ; \codeT{t_0}{\gamma_0}$
  ならば,
  $\vdash v : \codeT{t_1}{\gamma_3}$
\end{lemm}

\ooi{answer type のところ,よく考える}

\paragraph{証明}


\subsection{進行}
\begin{theo}[進行]
  $\vdash e:t$ が導出可能であれば,$e$ は 値 $v$ である.または,$e \lto e'$ であるような 項 $e'$ が存在する
\end{theo}

\paragraph{証明}
$\vdash e:t$ の導出に関する帰納法による.\\
Const, Abs, Code 規則の場合 $e$ は値である.\\
Var 規則の場合 $\vdash e:t$ は導出可能でない.\\
Throw 規則の場合 $\vdash e:t$ は導出可能でない.\\
Reset0 規則の場合 $e = \Resetz~ e_1$ とする.
帰納法の仮定より評価文脈における $\Resetz E$ より簡約が進み,\\
$e_1$ が値のとき,$e \lto v$ となるような $v$ が存在する.\\
$e_1$ が値でないとき,

\end{document}

%%% Local Variables:
%%% mode: japanese-latex
%%% TeX-master: t
%%% End:
