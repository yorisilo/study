%%
% このファイルは、筑波大学大学院システム情報工学研究科の
% 学位論文本体のサンプルです。
% このファイルを書き換えて、この例と同じような書式の論文本体を
% LaTeXを使って作成することができます。
%
% PC環境や、LaTeX環境の設定によっては漢字コードや改行コードを
% 変更する必要があります。
%%
\documentclass[a4paper,11pt]{jreport}

%%【PostScript, JPEG, PNG等の画像の貼り込み】
%% 利用するパッケージを選んでコメントアウトしてください。
%\usepackage{graphicx} % for \includegraphics[width=3cm]{sample.eps}
\usepackage{epsfig} % for \psfig{file=sample.eps,width=3cm}
%\usepackage{epsf} % for \epsfile{file=sample.eps,scale=0.6}
%\usepackage{epsbox} % for \epsfile{file=sample.eps,scale=0.6}

%% dvipdfm を使う場合(dvi->pdfを直接生成する場合)
%\usepackage[dvipdfm]{color,graphicx}
%% dvipdfm を使ってPDFの「しおり」を付ける場合
%\usepackage[dvipdfm,bookmarks=true,bookmarksnumbered=true,bookmarkstype=toc]{hyperref}
%% 参考:dvipdfm 日本語版
%% http://hamilcar.phys.kyushu-u.ac.jp/~hirata/dvipdfm/

\usepackage{times} % use Times Font instead of Computer Modern

\setcounter{tocdepth}{3}
\setcounter{page}{-1}

\setlength{\oddsidemargin}{0.1in}
\setlength{\evensidemargin}{0.1in}
\setlength{\topmargin}{0in}
\setlength{\textwidth}{6in}
%\setlength{\textheight}{10.1in}
\setlength{\parskip}{0em}
\setlength{\topsep}{0em}

%\newcommand{\zu}[1]{{\gt \bf 図\ref{#1}}}

%% タイトル生成用パッケージ(重要)
\usepackage{sie-jp-jis}

%% タイトル
%% 【注意】タイトルの最後に\\ を入れるとエラーになります
\title{筑波大学大学院システム情報工学研究科\\における修士論文の書き方}
%% 著者
\author{筑波 太郎}
%% 学位 (2012/11 追加)
\degree{修士(○○○○)}
%% 指導教員
\advisor{筑波 大二郎}

%% 専攻名 と 年月
%% 年月は必要に応じて書き替えてください。
\majorfield{△△△△} \yearandmonth{2013年 3月}



\begin{document}
\maketitle
\thispagestyle{empty}
\newpage

\thispagestyle{empty}
\vspace*{20pt plus 1fil}
\parindent=1zw
\noindent
%%
%% 論文の概要(Abstract)
%%
\begin{center}
{\bf 概要}
\vspace{5mm}
\end{center}
この文書は、筑波大学大学院システム情報工学研究科の修士論文本体のサンプル
である。このファイルを書き換えて、この例と同じような書式の論文本体を
\LaTeX を使って作成することができる。

このサンプルは、学生諸君が面倒な位置決めをして表紙を作成する手間を軽減す
るために提供している。もちろん、このサンプルで示す表紙は例であり、要項に
準拠していれば、このファイルに頼らずに自分で表紙の位置決めを行ってもよい。

%%%%%
\par
\vspace{0pt plus 1fil}
\newpage

\pagenumbering{roman} % I, II, III, IV
\tableofcontents
\listoffigures
%\listoftables

\pagebreak \setcounter{page}{1}
\pagenumbering{arabic} % 1,2,3


\chapter{はじめに}

修士論文自体は、まとめて製本し保存するため、体裁を大体そろえてもらうこと
になっている。そのため、このような修士論文本体の形式の例を作成した。

研究の内容や分野によっては書き方が異なる場合もあるので、詳しいことは指導教員に聞くと
よい。この文書は主にタイトルの作成方法と、論文の体裁を示すのみであり、どうやっ
たらよい論文になるかの示唆は含まれていない。

\chapter{形式}

ここでは、論文の表紙および本体の記述方法について述べる。

\section{表紙}

表紙は、{\tt $\backslash$maketitle} によって作成するため、以下の項目に相
当する文字列をそれぞれ記述する。

\begin{description} \parskip=1pt
\item{題目: }
題目は{\tt $\backslash$title} に記述する。行替えを行う場合は$\backslash$
	   $\backslash$ を入力する。ただし、題目の最後に$\backslash$
	   $\backslash$ を入力するとコンパイルが通らなくなるので注意する。
	   なお、4行以上の題目の場合、表紙ページがあふれるためスタイルファ
	   イル``sie-jp.sty''を変更する必要がある。
\item{著者名: }
著者名は{\tt $\backslash$author} に記述する。
\item{学位: }
学位名は{\tt $\backslash$degree} に記述する。
\item{指導教員名: }
指導教教員は{\tt $\backslash$advisor} に記述する。
\item{専攻名: }
専攻名は{\tt $\backslash$majorfield} に記述する。
\item{年月: }
年月は{\tt $\backslash$yearandmonth} に記述する。
\end{description}

\section{本体}

本体は1段組で記述する。

図表には番号と説明(caption)を付け、文章中で参照する。表
\ref{table:fundamental_data_type}と図\ref{figure:sample}はそれぞれ表と図
の例である。表の説明は上に、図の説明は下に書くことが多い。図の挿入に用い
るパッケージについては使用環境に合わせて自由に選択してほしい。

\begin{table}[hbt]
\caption{表の例}
\label{table:fundamental_data_type}
\begin{center}
\begin{tabular}{| c | r | r | r | r |}
\hline
年 度 & 1年次 & 2年次 & 3年次 & 4年次 \\
\hline
1995 & 85 & 92 & 86 & 88 \\
1996 & 83 & 89 & 90 & 102 \\
1997 & 88 & 87 & 91 & 112 \\
1998 & 144 & 93 & 90 & 115 \\
\hline
\end{tabular}
\end{center}
\end{table}
\medskip

\begin{figure}[htbp]
\begin{center}
%\includegraphics[width=3cm]{sample.eps}
\psfig{file=sample.eps,scale=0.6}
%\epsfile{file=sample.eps,scale=0.6}
\end{center}
\caption{図の例}
\label{figure:sample}
\end{figure}

詳しくは参考書など(少し古い)\cite{RakRak}\cite{JiyuuJizai}を参照のこと。
また、奥村晴彦氏の「日本語\TeX 情報(Japanese TeX FAQ)」
http://www.matsusaka-u.ac.jp/\~{}okumura/texfaq/ は、日本語の\TeX に関す
る情報が充実している。また、具体的な論文としての文献参照例として
\cite{bryant-ieeetc86}を挙げておく。


\chapter*{謝辞}
\addcontentsline{toc}{chapter}{\numberline{}謝辞}

\newpage

\addcontentsline{toc}{chapter}{\numberline{}参考文献}
\renewcommand{\bibname}{参考文献}

%% 参考文献に jbibtex を使う場合
%\bibliographystyle{junsrt}
%\bibliography{samplebib}
%% [compile] jbibtex sample; platex sample; platex sample;

%% 参考文献を直接ファイルに含めて書く場合
\begin{thebibliography}{1}
\bibitem{RakRak}
野寺隆志.
\newblock 楽々 \LaTeX.
\newblock 共立出版, 1990.

\bibitem{JiyuuJizai}
磯崎秀樹.
\newblock \LaTeX 自由自在.
\newblock サイエンス社, July 1992.

\bibitem{bryant-ieeetc86}
Randal~E. Bryant.
\newblock Graph-based algorithms for {B}oolean function manipulation.
\newblock {\em IEEE Transactions on Computers}, Vol. C-35, No.~8, pp. 677--691,
  August 1986.
\end{thebibliography}

\end{document}

%%% Local Variables:
%%% mode: japanese-latex
%%% TeX-master: t
%%% End:
