%
% Type inference algorithm draft (for Oishi System)
%
\documentclass{jsarticle}
\addtolength{\topmargin}{-2cm}
\addtolength{\textwidth}{3cm}
\addtolength{\oddsidemargin}{-1.5cm}
\usepackage{amsmath,amssymb}
% \usepackage[dvipdfm]{graphicx}

\newenvironment{vq}
{%begin
  \VerbatimEnvironment \begin{screen} \begin{quote} \begin{Verbatim}
      }
      {%end
      \end{Verbatim} \end{quote} \end{screen}
}

% \newcommand*\defeq{\stackrel{\text{def}}{=}}

\overfullrule=0pt

\begin{document}

\begin{center}
  Oishi Type System に対する型推論アルゴリズム
\end{center}

\section{型システム}

いまのところ、2016/09JSSST大会バージョンものとする。

\section{型推論アルゴリズム}

概要: 以下の2ステップから構成
\begin{itemize}
\item 制約生成:与えられた項にたいして、(型およびクラシファイアに関する)制約を返す。
\item 制約を解く。
\end{itemize}

\subsection{制約生成}

これは、もの型システム($T_1$とする)を
「トップダウンでの制約生成向け型システム($T_2$とする)」に変
形することであたえる。

$T_2$の設計指針:
\begin{itemize}
\item $T_1$と$T_2$は「型付けできる」という関係として等価である。
\item $T_2$は、term-oriented である。
  (結論側の式のトップレベルの形だけで、どの型付けルールを適用可能か、一意的にわかる。)
\item $T_2$は、制約生成をする。
  (結論側の式の要素は変数として、
  「それがこういう形でなければいけない」という条件は、制約の形で「生成」する。)
\end{itemize}

以上をどう見たすか? ポイントは、subsumption rule の適用タイミング(な
るべく subsumption rule を適用するのを避けたい)である。

定理1. $T_1$で型付け可能であれば、
「subsumption rule を、コードレベルのラムダ抽象、shift0,reset0,throw0
ルールの直前もしくは、変数ルールの直後にのみ適用した導出」が存在する。


\subsection{制約解消}

\end{document}
