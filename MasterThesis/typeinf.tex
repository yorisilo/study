%
% Type inference algorithm draft (for Oishi System)
%
\documentclass[dvipdfmx]{jsarticle}

\usepackage[dvipdfmx]{graphicx,color}
\addtolength{\topmargin}{-2cm}
\addtolength{\textwidth}{3cm}
\addtolength{\oddsidemargin}{-1.5cm}

\usepackage{theorem}
\usepackage{amsmath,amssymb}
\usepackage{ascmac}
\usepackage{mathtools}
\usepackage{proof}
\usepackage{stmaryrd}
\usepackage{listings,jlisting}
\usepackage{here}
\usepackage{verbatim}
\usepackage{framed}
\usepackage{algorithm}
\usepackage{algpseudocode}

\lstset{
  basicstyle=\ttfamily,
  columns=fullflexible,
  keepspaces=true,
}

\newenvironment{vq}
{%begin
  \VerbatimEnvironment \begin{screen} \begin{quote} \begin{Verbatim}
      }
      {%end
      \end{Verbatim} \end{quote} \end{screen}
}
\newtheorem{theorem}{theorem}[section]

\definecolor{DarkGreen}{rgb}{0,0.5,0}
\definecolor{Magenta}{rgb}{1.0, 0.0, 1.0}

\newcommand\cbra{\texttt{<}}
\newcommand\cket{\texttt{>}}

\newcommand\lam{\lambda}

\newcommand\codeTs[2]{\langle{#1}\rangle{\textbf{\textasciicircum}{#2}}}

\newcommand\fordo[2]{\textbf{for}~{#1}~\textbf{to}~{#2}~\textbf{do}}
\newcommand\cfordo[2]{\underline{\textbf{for}}~{#1}~\underline{\textbf{to}}~{#2}~\underline{\textbf{do}}}
\newcommand\cfor[1]{\underline{\textbf{for}}~{#1}}

\newcommand\too{\leadsto^*}
\newcommand\downtoo{\rotatebox{-90}{$\leadsto^*$}}
\newcommand\pink[1]{\textcolor{pink}{#1}}
\newcommand\red[1]{\textcolor{red}{#1}}
\newcommand\green[1]{\textcolor{green}{#1}}
\newcommand\magenta[1]{\textcolor{magenta}{#1}}
\newcommand\blue[1]{\textcolor{blue}{#1}}

\newcommand\fun[2]{\lambda{#1}.{#2}}

\newcommand\Resetz{\textbf{reset0}}
\newcommand\Shiftz{\textbf{shift0}}
\newcommand\Throw{\textbf{throw}}
\newcommand\resetz[1]{\Resetz~{#1}}
\newcommand\shiftz[2]{\Shiftz~{#1}\to{#2}}
\newcommand\throw[2]{\Throw~{#1}~{#2}}

\newcommand\cfun[2]{\underline{\lambda}{#1}.{#2}}
\newcommand\ccfun[2]{\underline{\underline{\lambda}}{#1}.{#2}}

\newcommand\cResetz{\underline{\textbf{reset0}}}
\newcommand\cShiftz{\underline{\textbf{shift0}}}
\newcommand\cThrow{\underline{\textbf{throw}}}
\newcommand\cresetz[1]{\cResetz~{#1}}
\newcommand\cshiftz[2]{\cShiftz~{#1}\to{#2}}
\newcommand\cthrow[2]{\cThrow~{#1}~{#2}}

\newcommand\cPlus{\underline{\textbf{+}}}
\newcommand\Plus{\textbf{+}}

\newcommand\cLet{\underline{\textbf{let}}}
\newcommand\cIn{\underline{\textbf{in}}}
\newcommand\clet[3]{\cLet~{#1}={#2}~\cIn~{#3}}
\newcommand\csp[1]{\texttt{\%}{#1}}
\newcommand\cint{\underline{\textbf{int}}}
\newcommand\code[1]{\texttt{<}{#1}\texttt{>}}
\newcommand\codebegin{\texttt{<}}
\newcommand\codeend{\texttt{>}}

\newcommand\intT{\mbox{\texttt{int}}}
\newcommand\boolT{\mbox{\texttt{bool}}}

\newcommand\codeT[2]{\langle{#1}\rangle^{#2}}
\newcommand\funT[3]{{#1} \stackrel{#3}{\rightarrow} {#2}}
\newcommand\contT[3]{{#1} \stackrel{#3}{\Rightarrow} {#2}}

\newcommand\ord{\ge}

\newcommand\Let{\textbf{let}}
\newcommand\In{\textbf{in}}
\newcommand\letin[3]{\Let~{#1}={#2}~\In~{#3}}

\newcommand\iif{\textbf{if}}
\newcommand\then{\textbf{then}}
\newcommand\eelse{\textbf{else}}
\newcommand\ift[3]{\textbf{if}~{#1}~\textbf{then}~{#2}~\textbf{else}~{#3}}
\newcommand\cif[3]{\underline{\textbf{if}}~\code{{#1}}~\code{{#2}}~\code{{#3}}}
\newcommand\cIf{\underline{\textbf{if}}}

\newcommand\fix{\textbf{fix}}
\newcommand\cfix{\underline{\textbf{fix}}}

\newcommand\lto{\leadsto}
\newcommand\cat{~\underline{@}~}

\newcommand\ksubst[2]{\{{#1}\Leftarrow{#2}\}}

\newcommand\cFor{\underline{\textbf{for}}}
\newcommand\forin[2]{\textbf{for}~{#1}~\textbf{to}~{#2}~\textbf{do}}
\newcommand\cforin[2]{\underline{\textbf{for}}~{#1}~\underline{\textbf{to}}~{#2}~\underline{\textbf{do}}}
\newcommand\cArray[1]{\underline{[{#1}]}}
\newcommand\cArrays[2]{\underline{[{#1}][{#2}]}}
\newcommand\aryset[3]{{#1}[{#2}]\leftarrow {#3}}
\newcommand\caryset[3]{\underline{\textbf{aryset}}~{#1}~{#2}~{#3}}
\newcommand\set{\underline{\textbf{set}}}

% コメントマクロ
\newcommand\kam[1]{\red{kam said: {#1}}}
\newcommand\oishi[1]{\blue{oishi said: {#1}}}

\theoremstyle{break}

\newtheorem{theo}{定理}[section]
\newtheorem{defi}{定義}[section]
\newtheorem{lemm}{補題}[section]

\algnewcommand\algorithmicforeach{\textbf{for each}}
\algdef{S}[FOR]{ForEach}[1]{\algorithmicforeach\ #1\ \algorithmicdo}


\newcommand\smallerscope[2]{#1 \ord #2}
\newcommand\greaterscope[2]{#2 \ord #1}
\newcommand\longer[2]{{#1} \ord {#2}}
% \newcommand*\defeq{\stackrel{\text{def}}{=}}
\newcommand\Int{\mbox{\texttt{Int}}}
\newcommand\Bool{\mbox{\texttt{Bool}}}

\newcommand\uni{\cup} % !!! 現在の順序では「∪」

\overfullrule=0pt

\begin{document}

\begin{center}
  Oishi Type System に対する型推論アルゴリズム$T_2$ \\
  2016/1/15
\end{center}

\section{型推論用の型システム$T_2$}

(var0)

\[
  \infer[Constr;~\Gamma \models \longer{t}{t'}]
  {\Gamma \vdash x:t;~\sigma}
  {(x:t') \in \Gamma
  }
\]

(var1)

\[
  \infer[Constr;~\Gamma \models \longer{\gamma}{\gamma'}]
  {\Gamma \vdash^{\gamma} u:t;~\sigma}
  {(u:t)^{\gamma'} \in \Gamma
  }
\]

(add)

\[
  \infer[Constr;~ t = \intT]
  {\Gamma \vdash x~ \Plus~ y: t; \sigma}
  {\Gamma \vdash x: \intT ; \sigma & \Gamma \vdash y: \intT ; \sigma}
\]

(code-add)

\[
  \infer[Constr;~ \Gamma \models \longer{t}{\codeT{\intT}{\gamma}}]
  {\Gamma \vdash u~ \cPlus~ w: t; \sigma}
  {\Gamma \vdash u: \codeT{\intT}{\gamma}; \sigma & \Gamma \vdash w: \codeT{\intT}{\gamma}; \sigma}
\]

(const)

\[
  \infer[Constr;~\Gamma \models \longer{t}{t^c}]
  {\Gamma \vdash^{L} c:t;~\sigma}
  {}
\]

(app0)

\[
  \infer[Constr;~\Gamma \models \longer{t}{t_1}]
  {\Gamma \vdash^\gamma e_1 \, e_2 : t;~\sigma}
  {\Gamma \vdash^\gamma e_1 : \funT{t_2}{t_1}{\sigma};~\sigma
    &\Gamma \vdash^\gamma e_2 : t_2;~\sigma
  }
\]

(app1)

\[
  \infer[Constr;~\Gamma \models \longer{t}{t_1}]
  {\Gamma \vdash e_1 \, e_2 : t;~\sigma}
  {\Gamma \vdash e_1 : t_2 \to t_1;~\sigma
    &\Gamma \vdash e_2 : t_2;~\sigma
  }
\]

(lambda0)

\[
  \infer[Constr;~t=\funT{t_1}{t_2}{\sigma'},~ \Gamma \models \sigma \ord \sigma' ]
  {\Gamma \vdash \lambda x.e : t;~\sigma}
  {\Gamma,~x:t_1 \vdash e : t_2;~\sigma'}
\]

(lambda1)
% \oishi{
% コードの中で,shift0/reset0 は使わないので, $\sigma$ は $\epsilon$ なはず.
% なので,$\sigma$ は $\epsilon$ としておいてもよい
% }

\[
  \infer[Constr;~t=\funT{t_1}{t_2}{}]
  {\Gamma \vdash^\gamma \lambda u.e : t;~\sigma}
  {\Gamma,~(u:t_1)^\gamma \vdash^\gamma e : t_2;~\sigma}
\]

(if)

\[
  \infer[Constr;~(none)]
  {\Gamma \vdash^L
    \textbf{if}~e_1 \textbf{then}~e_2 \textbf{else}~e_3 ~:~t; ~ \sigma}
  {\Gamma \vdash^L e_1 : \Bool;~\sigma
    &\Gamma \vdash^L e_2 : t;~\sigma
    &\Gamma \vdash^L e_3 : t;~\sigma
  }
\]

(code-lambda)

\[
  \infer[Constr;~\Gamma \models \longer{t}{\codeT{t_1 \to t_2}{\gamma}}]
  {\Gamma \vdash \underline{\lambda}x.e ~:~t;~\sigma}
  {\Gamma,\longer{\gamma'}{\gamma},x:\codeT{t_1}{\gamma'}
    \vdash e : \codeT{t_2}{\gamma'};~\sigma
  }
\]

(code-let)

\[
  \infer[Constr;~\Gamma \models \longer{t}{\codeT{t'}{\gamma_0}}]
  {\Gamma \vdash \cLet~ x = e_0~ \cIn~ e_1 : t ; \sigma}
  { \Gamma \vdash e_0 : \codeT{t'}{\gamma_0} ; \sigma
    &\Gamma, \longer{\gamma_1}{\gamma_0}, x: \codeT{t'}{\gamma_1} \vdash e_1 : \codeT{t'}{\gamma_1} ; \sigma
  }
\]

(reset0)

\[
  \infer[Constr;~\Gamma \models \longer{t}{\codeT{t'}{\gamma}}]
  {\Gamma \vdash \Resetz~e ~:~ t; ~\sigma}
  {\Gamma \vdash e:\codeT{t'}{\gamma};~\codeT{t'}{\gamma},\sigma
  }
\]

(shift0)

\[
  \infer[Constr;~\Gamma \models \longer{t}{\codeT{t_1}{\gamma_1}},~ t_2 =\codeT{t_0}{\gamma_0},~ \Gamma \models \longer{\gamma_1}{\gamma_0}]
  {\Gamma \vdash \shiftz{k}{e} : t~;~ t_2,\sigma}
  {\Gamma,~k:\contT{\codeT{t_1}{\gamma_1}}{\codeT{t_0}{\gamma_0}}{\sigma}
    \vdash e : \codeT{t_0}{\gamma_0} ; \sigma
  }
\]


(throw0)
% \oishi{
%   throw0 規則にのみ $\sigma$ part の subsumption 規則を適用すればおk?
% }
% \\

\[
  \infer[Constr;~ {\scriptstyle \Gamma \models \longer{t}{\codeT{t_0}{\gamma_2}},~ (\Gamma,~ k:~t') \models \gamma_2 \ord \gamma_0,~  t' = \contT{\codeT{t_1}{\gamma_1}}{\codeT{t_0}{\gamma_0}}{\sigma},~ \Gamma \models \gamma_1 \uni \gamma_2 \ord \gamma'}]
  {\Gamma,~ k:t'
    \vdash \throw{k}{v} : t ; \sigma}
  {\Gamma,~ k:t'
    \vdash v : \codeT{t_1}{\gamma'} ; \sigma
  }
\]

(code)

\[
  \infer[Constr;~\longer{t}{\codeT{t_1}{\gamma}}]
  {\Gamma \vdash \code{e} : t;~\sigma}
  {\Gamma \vdash^\gamma e : t_1;~\sigma}
\]

\subsection{bugってそうなところ}

(code-let)

(add)

(code-add)

の規則を追加したのですが,あってるのかわからない.

\subsection{bugってそうな箇所}

実装は ``study/src/metaS0/typeinf.ml'' にあります.

\begin{lstlisting}
let rec gen_constr (tyct: tycntxtT list) (lv: lvT) (e: expr) (t: tyT) (sgm: sgmT) (cnstl: constrT list) : constrT list =
  match e,lv with
  | (Var x), L0 ->
    let (t', L0) = lookup_tycntxt x tyct in
    (* print_ty' t'; print_lv' l; *)
    let c1 = CModelGtt(tyct, (t, t')) in
    let new_cnstl = c1 :: cnstl in
    new_cnstl
...
\end{lstlisting}
のところで,\lstinline|let (t', l) = lookup_tycntxt x tyct in| とすると,以下の\ref{subsec:ex1}の例は動きます.(あってるかはちょっと不明なのですが).

\lstinline|let (t', L0) = lookup_tycntxt x tyct in| とすると,
動きません.

理由としては,型文脈 tyct の x は レベルが L0 でなく,レベル $\gamma$  であるからです.
パターンマッチとしては, x のレベルは L0 であるべきはずなのですが...

\subsubsection{動いてほしい例}
\label{subsec:ex1}

\begin{lstlisting}
# pp_cnstrl @@ Let_("x", Code(Int 1), PrimOp2("Add_", Code(Int 1), Var "x"))

# pp_cnstrl @@ R0(Let_("x", Code(Int 1), R0(Let_("y", Code(Int 2),
                  S0("k", Let_("z", Var "x", T0("k", App(Var "k", Var "z"))))))))
\end{lstlisting}


\end{document}

%%% Local Variables:
%%% mode: japanese-latex
%%% TeX-master: t
%%% End:
