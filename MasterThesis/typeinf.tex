%
% Type inference algorithm draft (for Oishi System)
%
\documentclass[dvipdfmx]{jsarticle}

\usepackage[dvipdfmx]{graphicx,color}
\addtolength{\topmargin}{-2cm}
\addtolength{\textwidth}{3cm}
\addtolength{\oddsidemargin}{-1.5cm}

\usepackage{theorem}
\usepackage{amsmath,amssymb}
\usepackage{ascmac}
\usepackage{mathtools}
\usepackage{proof}
\usepackage{stmaryrd}
\usepackage{listings,jlisting}
\usepackage{here}

\newenvironment{vq}
{%begin
  \VerbatimEnvironment \begin{screen} \begin{quote} \begin{Verbatim}
      }
      {%end
      \end{Verbatim} \end{quote} \end{screen}
}
\newtheorem{theorem}{theorem}[section]

\usepackage{color}

\definecolor{DarkGreen}{rgb}{0,0.5,0}
\definecolor{Magenta}{rgb}{1.0, 0.0, 1.0}

\newcommand\too{\leadsto^*}
\newcommand\pink[1]{\textcolor{pink}{#1}}
\newcommand\red[1]{\textcolor{red}{#1}}
\newcommand\green[1]{\textcolor{green}{#1}}
\newcommand\magenta[1]{\textcolor{magenta}{#1}}
\newcommand\blue[1]{\textcolor{blue}{#1}}

\newcommand\fun[2]{\lambda{#1}.{#2}}

\newcommand\Resetz{\textbf{reset0}}
\newcommand\Shiftz{\textbf{shift0}}
\newcommand\Throw{\textbf{throw}}
\newcommand\resetz[1]{\Resetz~{#1}}
\newcommand\shiftz[2]{\Shiftz~{#1}\to{#2}}
\newcommand\throw[2]{\Throw~{#1}~{#2}}

\newcommand\cfun[2]{\underline{\lambda}{#1}.{#2}}
\newcommand\ccfun[2]{\underline{\underline{\lambda}}{#1}.{#2}}

\newcommand\cResetz{\underline{\textbf{reset0}}}
\newcommand\cShiftz{\underline{\textbf{shift0}}}
\newcommand\cThrow{\underline{\textbf{throw}}}
\newcommand\cresetz[1]{\cResetz~{#1}}
\newcommand\cshiftz[2]{\cShiftz~{#1}\to{#2}}
\newcommand\cthrow[2]{\cThrow~{#1}~{#2}}

\newcommand\cPlus{\underline{\textbf{+}}}
\newcommand\Plus{\textbf{+}}

\newcommand\cLet{\underline{\textbf{let}}}
\newcommand\cIn{\underline{\textbf{in}}}
\newcommand\clet[3]{\cLet~{#1}={#2}~\cIn~{#3}}
\newcommand\csp[1]{\texttt{\%}{#1}}
\newcommand\cint{\underline{\textbf{int}}}
\newcommand\code[1]{\texttt{<}{#1}\texttt{>}}
\newcommand\codebegin{\texttt{<}}
\newcommand\codeend{\texttt{>}}

\newcommand\intT{\mbox{\texttt{int}}}
\newcommand\boolT{\mbox{\texttt{bool}}}

\newcommand\codeT[2]{\langle{#1}\rangle^{#2}}
\newcommand\funT[3]{{#1} \stackrel{#3}{\rightarrow} {#2}}
\newcommand\contT[3]{{#1} \stackrel{#3}{\Rightarrow} {#2}}

\newcommand\ord{\ge}

\newcommand\Let{\textbf{let}}
\newcommand\In{\textbf{in}}
\newcommand\letin[3]{\Let~{#1}={#2}~\In~{#3}}

\newcommand\ift[3]{\textbf{if}~{#1}~\textbf{then}~{#2}~\textbf{else}~{#3}}
\newcommand\cif[3]{\underline{\textbf{if}}~\code{{#1}}~\code{{#2}}~\code{{#3}}}
\newcommand\cIf{\underline{\textbf{if}}}

\newcommand\fix{\textbf{fix}}
\newcommand\cfix{\underline{\textbf{fix}}}

\newcommand\lto{\leadsto}
\newcommand\cat{~\underline{@}~}

\newcommand\ksubst[2]{\{{#1}\Leftarrow{#2}\}}

\newcommand\cFor{\underline{\textbf{for}}}
\newcommand\forin[2]{\textbf{for}~{#1}~\textbf{to}~{#2}~\textbf{do}}
\newcommand\cforin[2]{\underline{\textbf{for}}~{#1}~\underline{\textbf{to}}~{#2}~\underline{\textbf{do}}}
\newcommand\cArray[1]{\underline{[{#1}]}}
\newcommand\cArrays[2]{\underline{[{#1}][{#2}]}}
\newcommand\aryset[3]{{#1}[{#2}]\leftarrow {#3}}
\newcommand\caryset[3]{\underline{\textbf{aryset}}~{#1}~{#2}~{#3}}

% コメントマクロ
\newcommand\kam[1]{\red{kam said: {#1}}}
\newcommand\oishi[1]{\blue{oishi said: {#1}}}

\theoremstyle{break}

\newtheorem{theo}{定理}[section]
\newtheorem{defi}{定義}[section]
\newtheorem{lemm}{補題}[section]

\algnewcommand\algorithmicforeach{\textbf{for each}}
\algdef{S}[FOR]{ForEach}[1]{\algorithmicforeach\ #1\ \algorithmicdo}

\newcommand\greaterthan{\ge}


\newcommand\smallerscope[2]{#1 \ord #2}
\newcommand\greaterscope[2]{#2 \ord #1}
\newcommand\longer[2]{{#1} \ord {#2}}
% \newcommand*\defeq{\stackrel{\text{def}}{=}}
\newcommand\Int{\mbox{\texttt{Int}}}
\newcommand\Bool{\mbox{\texttt{Bool}}}

\overfullrule=0pt

\begin{document}

\begin{center}
  Oishi Type System に対する型推論アルゴリズム \\
  2016/11/19
\end{center}

\section{型システム}

いまのところ、2016/09JSSST大会バージョンものとする。

\section{型推論アルゴリズム}

概要: 以下の2ステップから構成
\begin{itemize}
\item 制約生成:与えられた項にたいして、(型およびクラシファイアに関する)制約を返す。
\item 制約を解く。
\end{itemize}

\subsection{制約生成}

これは、もの型システム($T_1$とする)を
「トップダウンでの制約生成向け型システム($T_2$とする)」に変
形することであたえる。

$T_2$の設計指針:
\begin{itemize}
\item $T_1$と$T_2$は「型付けできる」という関係として等価である。
\item $T_2$は、term-oriented である。
  (結論側の式のトップレベルの形だけで、どの型付けルールを適用可能か、一意的にわかる。)
\item $T_2$は、制約生成をする。
  (結論側の式の要素は変数として、
  「それがこういう形でなければいけない」という条件は、制約の形で「生成」する。)
\end{itemize}

以上をどう満たすか? ポイントは、subsumption rule の適用タイミング(な
るべく subsumption rule を適用するのを避けたい)である。

\subsection{型システム$T_2$の導入}

subsumption rule が出現する場所を限定することができる.
特に,ルールと,その直後に subsumption がつかわれる場合を考えてみよう.
以下で,「var1」等といった表記は,
「もともとあるver1ルールを subsumption規則と組み合わせた形に改訂したも
の」である.
また,横棒の右に書いてあるConstr;... は(ルールを下から上にむけて
使うとき),Constr 以下の制約が生成される,という意味である.

\oishi{型付け規則の適用の直後に subsumption を適用するのならば,以下のような形になるのでは?
  \\
  (型付け規則 X1)
  \[
    \infer[Constr;~\Gamma \models \gamma_2 \greaterthan \gamma_1, ...]
    {\Gamma \vdash e : \codeT{t}{\gamma_2} ; \sigma}
    {\Gamma \vdash e : \codeT{t}{\gamma_1} ; \sigma
      % & \Gamma \models \gamma_2 \ord \gamma_1
    }
  \]
  (型付け規則 X2)
  \[
    \infer[Constr;~\Gamma \models \gamma_2 \greaterthan \gamma_1, ...]
    {\Gamma \vdash^{\gamma_2} e : t ~;~ \sigma}
    {\Gamma \vdash^{\gamma_1} e : t ~;~ \sigma
      % & \Gamma \models \gamma_2 \ord \gamma_1
    }
  \]
}

また、型$t_1, t_2$に対する$\longer{}{}$の記号は以下の意味であるが、
とりあえず、(以下の意味にしたがって分解はせずに)
$\longer{t_1}{t_2}$ の形のまま、制約として生成する。
\begin{itemize}
\item $\longer{\codeT{t_1}{\gamma_1}}{\codeT{t_2}{\gamma_2}}$
  は、「$t_1=t_2$ かつ $\longer{\gamma_1}{\gamma_2}$
\item $\codeT{t}{\gamma}$の形でない$t_1,t_2$に対しては、
  $t_1 = t_2$。
\end{itemize}
(型推論のプロセスの最中では、$t_1, t_2$ はメタ型変瑞。洽筅キ、譴此その場合、どちらの形かは決定できないので、$\longer{t_1}{t_2}$を
上記の意味にしたがって、「ほどく」ことはできない。なので、
$\longer{t_1}{t_2}$という形のまま制約として生成する。)

(亀山メモ:
ただ、もしかすると、
「レベル0の型変数」と「レベル1の型変数」を最初からわけておく方法もある
かもしれない。そうすると、上記はとける?)

(var1)
\[
  \infer[Constr;~\Gamma \models \longer{t}{t'}]
  {\Gamma \vdash x:t;~\sigma}
  {(x:t') \in \Gamma
  }
\]

\oishi{
  $Constr;~\Gamma \models t \greaterthan t'$ がわからない.
  $\greaterthan$ は 型$t$ ではなく,環境識別子 $\gamma$  にのみ定義されているのでは?
  $\to$ おそらく略記していて,$t$ と $t'$ の肩についている環境識別子 $\gamma$ と $\gamma'$の大なりの関係を表している.\\
  (var1改 : $t$ が $\codeT{t}{\gamma}$ みたいな形じゃない場合)
  \[
    \infer[Constr;~ (x:t) \in \Gamma]
    {\Gamma \vdash x:t;~\sigma}
    {}
  \]
  (var1改 : $t$ が $\codeT{t}{\gamma}$ みたいな形の場合)
  \[
    \infer[Constr;~ (x:t) \in \Gamma, \Gamma \models \gamma \ge \gamma']
    {\Gamma \vdash x:\codeT{t}{\gamma};~\sigma}
    {(x:\codeT{t}{\gamma'}) \in \Gamma}
  \]
}

(var2)
\[
  \infer[Constr;~\Gamma \models \longer{\gamma}{\gamma'}]
  {\Gamma \vdash^{\gamma} u:t;~\sigma}
  {(u:t)^{\gamma'} \in \Gamma
  }
\]

(const)
\[
  \infer[Constr;~\Gamma \models \longer{t}{t^c}]
  {\Gamma \vdash^{L} c:t;~\sigma}
  {}
\]

(app)
\[
  \infer[Constr;~\Gamma \models \longer{t}{t_1}]
  {\Gamma \vdash^L e_1 \, e_2 : t;~\sigma}
  {\Gamma \vdash^L e_1 : t_2 \to t_1;~\sigma
    &\Gamma \vdash^L e_2 : t_2;~\sigma
  }
\]

(lambda0)
\[
  \infer[Constr;~t=\funT{t_1}{t_2}{\sigma'}]
  {\Gamma \vdash \lambda x.e : t;~\sigma}
  {\Gamma,~x:t_1 \vdash e : t_2;~\sigma'}
\]

(lambda1)
\[
  \infer[Constr;~t=\funT{t_1}{t_2}{}]
  {\Gamma \vdash^\gamma \lambda u.e : t;~\sigma}
  {\Gamma,~(u:t_1)^\gamma \vdash^\gamma e : t_2;~\sigma'}
\]

(if)
\[
  \infer[Constr;~(none)]
  {\Gamma \vdash^L
    \textbf{if}~e_1 \textbf{then}~e_2 \textbf{else}~e_3 ~:~t; ~ \sigma}
  {\Gamma \vdash^L e_1 : \Bool;~\sigma
    &\Gamma \vdash^L e_2 : t;~\sigma
    &\Gamma \vdash^L e_3 : t;~\sigma
  }
\]

(code-lambda)
\[
  \infer[Constr;~\longer{t}{\codeT{t_1 \to t_2}{\gamma}}]
  {\Gamma \vdash^L \underline{\lambda}x.e ~:~t;~\sigma}
  {\Gamma,\longer{\gamma'}{\gamma},x:\codeT{t_1}{\gamma'}
    \vdash^L e : \codeT{t_2}{\gamma'};~\sigma
  }
\]

\oishi{追加\\
  (code-lambda改)
  \[
    \infer[Constr;~t=\codeT{t_1 \to t_2}{\gamma}]
    {\Gamma \vdash^L \underline{\lambda}x.e ~:~t;~\sigma}
    {\Gamma,\gamma_1 \greaterthan \gamma,x:\codeT{t_1}{\gamma_1}
      \vdash^L e : \codeT{t_2}{\gamma_1};~\sigma
    }
  \]
}

(reset0)
\[
  \infer[Constr;~\longer{t}{\codeT{t'}{\gamma}}]
  {\Gamma \vdash \textbf{reset0}~e ~:~ t; ~\sigma}
  {\Gamma \vdash e:\codeT{t'}{\gamma};~\codeT{t'}{\gamma},\sigma
  }
\]

(shift0)
\[
  \infer[Constr;~\longer{t}{\codeT{t_1}{\gamma_1}},~ t_2 =\codeT{t_0}{\gamma_0}]
  {\Gamma \vdash \shiftz{k}{e} : t~;~ t_2,\sigma}
  {\Gamma,~k:\contT{\codeT{t_1}{\gamma_1}}{\codeT{t_0}{\gamma_0}}{\sigma}
    \vdash e : \codeT{t_0}{\gamma_0} ; \sigma
    & \Gamma \models \longer{\gamma_1}{\gamma_0}
  }
\]


(throw0)
\[
  \infer[Constr;~\longer{t}{\codeT{t_0}{\gamma_2}}]
  {\Gamma,~k:\contT{\codeT{t_1}{\gamma_1}}{\codeT{t_0}{\gamma_0}}{\sigma}
    \vdash \throw{k}{v} : t ; \sigma}
  {\Gamma
    \vdash v : \codeT{t_1}{\gamma_1 \cup \gamma_2} ; \sigma
    & \Gamma \models \gamma_2 \ord \gamma_0
  }
\]

(code)
\[
  \infer[Constr;~\longer{t}{\codeT{t_1}{\gamma}}]
  {\Gamma \vdash \code{e} : t;~\sigma}
  {\Gamma \vdash^\gamma e : t_1;~\sigma}
\]

\subsection{制約解消}

\end{document}

%%% Local Variables:
%%% mode: japanese-latex
%%% TeX-master: t
%%% End:
