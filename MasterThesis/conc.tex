\chapter{まとめと今後の課題}
本研究では,効率的コード生成に有用な技法であるlet挿入を,型安全に実現す
るための言語と型システムについて述べた.
局所的な代入可能変数を持つ体系に対する須藤らの研究\cite{Sudo2014}などに基づき,
多段階のforループを飛び越えたlet挿入を実現するために,shift0/reset0 を持つコード生成体系を設計した.
須藤らの研究で精密化された環境識別子(Environment Classifier)に join ($\cup$) を導入することで,
計算の順序を変更するようなコントロールオペレータ(shift0/reset0)を扱えるようにし,
安全に多段階の let挿入を行えるように型システムを構築した.
このようなlet挿入が束縛子を越えるケースは,ループにおける不変式の括り出
しなどの有用な最適化を含むが,これまでの研究では一般的なlet挿入を安全
に実現した体系の提案はなく,我々の知る限り本研究がはじめてである.

今後の課題として,まずあげられるのは,進行(Progress)の性質および型推論
アルゴリズムの実装の完成である.また,理論的には Kiselyovらのグローバルな参
照を持つ体系との融合が可能になれば,広い範囲のコード生成技法・最適化技法
をカバーできるため極めて有用である.
また,既存のMetaOCaml との比較においては,
2レベルのみのコード生成に限定している点や
run (生成したコードの実行)や cross-stage persistence (現在ス
テージの値をコードに埋め込む機能)などに対応していない点が欠点であり,
これらの拡張が可能であるかどうかの検討は非常に興味深い将来課題である.

%%% Local Variables:
%%% mode: japanese-latex
%%% TeX-master: "master_oishi"
%%% End:
