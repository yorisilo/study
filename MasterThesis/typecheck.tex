\documentclass[a4j]{jsarticle}

\usepackage[dvipdfmx]{graphicx,color}
\usepackage{ascmac}

\usepackage{theorem}
\usepackage{amsmath,amssymb}
\usepackage{ascmac}
\usepackage{mathtools}
\usepackage{proof}
\usepackage{stmaryrd}
\usepackage{listings,jlisting}

\definecolor{DarkGreen}{rgb}{0,0.5,0}
\definecolor{Magenta}{rgb}{1.0, 0.0, 1.0}

\newcommand\too{\leadsto^*}
\newcommand\pink[1]{\textcolor{pink}{#1}}
\newcommand\red[1]{\textcolor{red}{#1}}
\newcommand\green[1]{\textcolor{green}{#1}}
\newcommand\magenta[1]{\textcolor{magenta}{#1}}
\newcommand\blue[1]{\textcolor{blue}{#1}}

\newcommand\fun[2]{\lambda{#1}.{#2}}

\newcommand\Resetz{\textbf{reset0}}
\newcommand\Shiftz{\textbf{shift0}}
\newcommand\Throw{\textbf{throw}}
\newcommand\resetz[1]{\Resetz~{#1}}
\newcommand\shiftz[2]{\Shiftz~{#1}\to{#2}}
\newcommand\throw[2]{\Throw~{#1}~{#2}}

\newcommand\cfun[2]{\underline{\lambda}{#1}.{#2}}
\newcommand\ccfun[2]{\underline{\underline{\lambda}}{#1}.{#2}}

\newcommand\cResetz{\underline{\textbf{reset0}}}
\newcommand\cShiftz{\underline{\textbf{shift0}}}
\newcommand\cThrow{\underline{\textbf{throw}}}
\newcommand\cresetz[1]{\cResetz~{#1}}
\newcommand\cshiftz[2]{\cShiftz~{#1}\to{#2}}
\newcommand\cthrow[2]{\cThrow~{#1}~{#2}}

\newcommand\cPlus{\underline{\textbf{+}}}
\newcommand\Plus{\textbf{+}}

\newcommand\cLet{\underline{\textbf{let}}}
\newcommand\cIn{\underline{\textbf{in}}}
\newcommand\clet[3]{\cLet~{#1}={#2}~\cIn~{#3}}
\newcommand\csp[1]{\texttt{\%}{#1}}
\newcommand\cint{\underline{\textbf{int}}}
\newcommand\code[1]{\texttt{<}{#1}\texttt{>}}
\newcommand\codebegin{\texttt{<}}
\newcommand\codeend{\texttt{>}}

\newcommand\intT{\mbox{\texttt{int}}}
\newcommand\boolT{\mbox{\texttt{bool}}}

\newcommand\codeT[2]{\langle{#1}\rangle^{#2}}
\newcommand\funT[3]{{#1} \stackrel{#3}{\rightarrow} {#2}}
\newcommand\contT[3]{{#1} \stackrel{#3}{\Rightarrow} {#2}}

\newcommand\ord{\ge}

\newcommand\Let{\textbf{let}}
\newcommand\In{\textbf{in}}
\newcommand\letin[3]{\Let~{#1}={#2}~\In~{#3}}

\newcommand\ift[3]{\textbf{if}~{#1}~\textbf{then}~{#2}~\textbf{else}~{#3}}
\newcommand\cif[3]{\underline{\textbf{if}}~\code{{#1}}~\code{{#2}}~\code{{#3}}}
\newcommand\cIf{\underline{\textbf{if}}}

\newcommand\fix{\textbf{fix}}
\newcommand\cfix{\underline{\textbf{fix}}}

\newcommand\lto{\leadsto}
\newcommand\cat{~\underline{@}~}

\newcommand\ksubst[2]{\{{#1}\Leftarrow{#2}\}}

\newcommand\cFor{\underline{\textbf{for}}}
\newcommand\forin[2]{\textbf{for}~{#1}~\textbf{to}~{#2}~\textbf{do}}
\newcommand\cforin[2]{\underline{\textbf{for}}~{#1}~\underline{\textbf{to}}~{#2}~\underline{\textbf{do}}}
\newcommand\cArray[1]{\underline{[{#1}]}}
\newcommand\cArrays[2]{\underline{[{#1}][{#2}]}}
\newcommand\aryset[3]{{#1}[{#2}]\leftarrow {#3}}
\newcommand\caryset[3]{\underline{\textbf{aryset}}~{#1}~{#2}~{#3}}

% コメントマクロ
\newcommand\kam[1]{\red{kam said: {#1}}}
\newcommand\oishi[1]{\blue{oishi said: {#1}}}

\theoremstyle{break}

\newtheorem{theo}{定理}[section]
\newtheorem{defi}{定義}[section]
\newtheorem{lemm}{補題}[section]

\algnewcommand\algorithmicforeach{\textbf{for each}}
\algdef{S}[FOR]{ForEach}[1]{\algorithmicforeach\ #1\ \algorithmicdo}


\title{型検査器について}
\author{大石 純平}
\date{\today}

\begin{document}
\maketitle
\section{型検査器}

\subsection{制約を導く}
入力: $(\Gamma, e, t, \sigma, l, C)$
出力: $C_t \, C_{\sigma}$

入力
\begin{itemize}
\item $\Gamma$ : 型文脈
\item $e$ : 項
\item $t$ : 型
\item $\sigma$ : answer type の列
\item $l$ : レベルを表す(現在レベル 0, or コードレベル 1)
\item $C$ : constraint(制約):
\end{itemize}

環境識別子(Environment Classifier)$\gamma$
\begin{align*}
  \gamma & ::= \gamma_x \mid \gamma \cup \gamma
\end{align*}

型文脈$\Gamma$
\begin{align*}
  \Gamma ::= \emptyset
  \mid \Gamma, (\gamma \ord \gamma)
  \mid \Gamma, (x : t)
  \mid \Gamma, (u : t)^{\gamma}
\end{align*}

型の有限列(\Shiftz の answer typeの列)$\sigma$
\begin{align*}
  \sigma & ::= \epsilon \mid \sigma,t^0 \\
\end{align*}


出力
\begin{itemize}
\item $C_t$ : 型の集合
\item $C_{\sigma}$ : ec や ec の 不等式 の集合

\end{itemize}

\subsection{制約を解く}
型に対しては $C_t$ が分かればすぐわかるが,
ecに対しては すぐわからない.

$C_{\sigma}$は,environment classifier の 不等式,等式 などの集合である.
その制約を解いて,解$\theta$があれば,$\theta$ のもとで,
$\Gamma \vdash^{L} e : t ;~\sigma$ がおk.

\begin{align*}
  C_{x} ::= \gamma_x \mid C_{\gamma} \\
  C_{\gamma} ::= \gamma_x \mid C_{\gamma} \cup C_{\gamma} \mid C_{x} \\
  \text{\oishi{ここはもう少し考える.}}
\end{align*}

$\gamma$ は 以下のようにツリーの位置によって,包含関係は決まる.
\[
  \infer{(.) \cfun ...}
  {\infer{(1) \cfun ... }
    {(1,1) \cfun ... & (1,2) \cfun ...}
    &
    \infer{(2) \cfun ... }
    {(2,1) \cfun ...}
  }
\]
各$(i,j)$に対して $(i,j)\# C_{\gamma}$ であるかどうかを調べれば良い


\begin{align*}
  \Gamma \vDash C_{\gamma} \ord C_{\gamma} \\
  C_{\gamma} \ord C_{\gamma},... \vDash C_{\gamma} \ord C_{\gamma} \\
  \gamma_x \# C_{\gamma}
\end{align*}
$\gamma_x \# C_{\gamma}$ は $\gamma_x$ は $C_{\gamma}$に含まれないという意味

\end{document}
%%% Local Variables:
%%% mode: japanese-latex
%%% TeX-master: t
%%% End:
