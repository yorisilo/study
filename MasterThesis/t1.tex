%
% Type inference algorithm draft (for Oishi System)
%
\documentclass[dvipdfmx]{jsarticle}

\usepackage[dvipdfmx]{graphicx,color}
\addtolength{\topmargin}{-2cm}
\addtolength{\textwidth}{3cm}
\addtolength{\oddsidemargin}{-1.5cm}

\usepackage{theorem}
\usepackage{amsmath,amssymb}
\usepackage{ascmac}
\usepackage{mathtools}
\usepackage{proof}
\usepackage{stmaryrd}
\usepackage{listings,jlisting}
\usepackage{here}

\newtheorem{theorem}{theorem}[section]

\definecolor{DarkGreen}{rgb}{0,0.5,0}
\definecolor{Magenta}{rgb}{1.0, 0.0, 1.0}

\newcommand\cbra{\texttt{<}}
\newcommand\cket{\texttt{>}}

\newcommand\lam{\lambda}

\newcommand\codeTs[2]{\langle{#1}\rangle{\textbf{\textasciicircum}{#2}}}

\newcommand\fordo[2]{\textbf{for}~{#1}~\textbf{to}~{#2}~\textbf{do}}
\newcommand\cfordo[2]{\underline{\textbf{for}}~{#1}~\underline{\textbf{to}}~{#2}~\underline{\textbf{do}}}
\newcommand\cfor[1]{\underline{\textbf{for}}~{#1}}

\newcommand\too{\leadsto^*}
\newcommand\downtoo{\rotatebox{-90}{$\leadsto^*$}}
\newcommand\pink[1]{\textcolor{pink}{#1}}
\newcommand\red[1]{\textcolor{red}{#1}}
\newcommand\green[1]{\textcolor{green}{#1}}
\newcommand\magenta[1]{\textcolor{magenta}{#1}}
\newcommand\blue[1]{\textcolor{blue}{#1}}

\newcommand\fun[2]{\lambda{#1}.{#2}}

\newcommand\Resetz{\textbf{reset0}}
\newcommand\Shiftz{\textbf{shift0}}
\newcommand\Throw{\textbf{throw}}
\newcommand\resetz[1]{\Resetz~{#1}}
\newcommand\shiftz[2]{\Shiftz~{#1}\to{#2}}
\newcommand\throw[2]{\Throw~{#1}~{#2}}

\newcommand\cfun[2]{\underline{\lambda}{#1}.{#2}}
\newcommand\ccfun[2]{\underline{\underline{\lambda}}{#1}.{#2}}

\newcommand\cResetz{\underline{\textbf{reset0}}}
\newcommand\cShiftz{\underline{\textbf{shift0}}}
\newcommand\cThrow{\underline{\textbf{throw}}}
\newcommand\cresetz[1]{\cResetz~{#1}}
\newcommand\cshiftz[2]{\cShiftz~{#1}\to{#2}}
\newcommand\cthrow[2]{\cThrow~{#1}~{#2}}

\newcommand\cPlus{\underline{\textbf{+}}}
\newcommand\Plus{\textbf{+}}

\newcommand\cLet{\underline{\textbf{let}}}
\newcommand\cIn{\underline{\textbf{in}}}
\newcommand\clet[3]{\cLet~{#1}={#2}~\cIn~{#3}}
\newcommand\csp[1]{\texttt{\%}{#1}}
\newcommand\cint{\underline{\textbf{int}}}
\newcommand\code[1]{\texttt{<}{#1}\texttt{>}}
\newcommand\codebegin{\texttt{<}}
\newcommand\codeend{\texttt{>}}

\newcommand\intT{\mbox{\texttt{int}}}
\newcommand\boolT{\mbox{\texttt{bool}}}

\newcommand\codeT[2]{\langle{#1}\rangle^{#2}}
\newcommand\funT[3]{{#1} \stackrel{#3}{\rightarrow} {#2}}
\newcommand\contT[3]{{#1} \stackrel{#3}{\Rightarrow} {#2}}

\newcommand\ord{\ge}

\newcommand\Let{\textbf{let}}
\newcommand\In{\textbf{in}}
\newcommand\letin[3]{\Let~{#1}={#2}~\In~{#3}}

\newcommand\iif{\textbf{if}}
\newcommand\then{\textbf{then}}
\newcommand\eelse{\textbf{else}}
\newcommand\ift[3]{\textbf{if}~{#1}~\textbf{then}~{#2}~\textbf{else}~{#3}}
\newcommand\cif[3]{\underline{\textbf{if}}~\code{{#1}}~\code{{#2}}~\code{{#3}}}
\newcommand\cIf{\underline{\textbf{if}}}

\newcommand\fix{\textbf{fix}}
\newcommand\cfix{\underline{\textbf{fix}}}

\newcommand\lto{\leadsto}
\newcommand\cat{~\underline{@}~}

\newcommand\ksubst[2]{\{{#1}\Leftarrow{#2}\}}

\newcommand\cFor{\underline{\textbf{for}}}
\newcommand\forin[2]{\textbf{for}~{#1}~\textbf{to}~{#2}~\textbf{do}}
\newcommand\cforin[2]{\underline{\textbf{for}}~{#1}~\underline{\textbf{to}}~{#2}~\underline{\textbf{do}}}
\newcommand\cArray[1]{\underline{[{#1}]}}
\newcommand\cArrays[2]{\underline{[{#1}][{#2}]}}
\newcommand\aryset[3]{{#1}[{#2}]\leftarrow {#3}}
\newcommand\caryset[3]{\underline{\textbf{aryset}}~{#1}~{#2}~{#3}}
\newcommand\set{\underline{\textbf{set}}}

% コメントマクロ
\newcommand\kam[1]{\red{kam said: {#1}}}
\newcommand\oishi[1]{\blue{oishi said: {#1}}}

\theoremstyle{break}

\newtheorem{theo}{定理}[section]
\newtheorem{defi}{定義}[section]
\newtheorem{lemm}{補題}[section]

\algnewcommand\algorithmicforeach{\textbf{for each}}
\algdef{S}[FOR]{ForEach}[1]{\algorithmicforeach\ #1\ \algorithmicdo}


\newcommand\longer[2]{{#1} \ord {#2}}
% \newcommand*\defeq{\stackrel{\text{def}}{=}}
\newcommand\uni{\cup} % !!! 現在の順序では「∪」

\overfullrule=0pt

\begin{document}

\begin{center}
  型システム
  2016/12/06
\end{center}

基本型$b$,環境識別子(Environment Classifier)$\gamma$を以下の通り定義する.

\begin{figure}[H]
  \centering
  \begin{align*}
    b & ::= \intT \mid \boolT \\
    \gamma & ::= \gamma_x \mid \gamma \cup \gamma
  \end{align*}
  \caption{基本型,環境識別子の定義}
  \label{fig:bec_def}
\end{figure}

$\gamma$の定義における$\gamma_x$は環境識別子の変数を表す.
すなわち,環境識別子は,変数であるかそれらを$\cup$で結合した形である.
以下では,メタ変数と変数を区別せず$\gamma_x$を$\gamma$と表記する.
ここで環境識別子として$\cup$を導入した理由は後述する.

$L ::= \empty \mid \gamma$ は現在ステージと将来ステージをまとめ
て表す記号である.たとえば,$\Gamma \vdash^L
e:t~;~\sigma$は,$L=\empty$のとき現在ステージの判断で,
$L=\gamma$のとき将来ステージの判断となる.

レベル0の型$t^0$,レベル1の型$t^1$,(レベル0の)型の有限列$\sigma$,
(レベル0の)継続の型$\kappa$を次の通り定義する.

\begin{figure}[H]
  \centering
  \begin{align*}
    t^0 & ::= b \mid \funT{t^0}{t^0}{\sigma} \mid \codeT{t^1}{\gamma} \\
    t^1 & ::= b \mid t^1 \to t^1 \\
    \sigma & ::= \epsilon \mid t^0, \sigma \\
    \kappa^0 & ::= \contT{\codeT{t^1}{\gamma}}{\codeT{t^1}{\gamma}}{\sigma}
  \end{align*}
  \caption{(レベル0の)継続の型の定義}
  \label{fig:k_def}
\end{figure}

レベル0の関数型$\funT{t^0}{t^0}{\sigma}$は,
エフェクトをあらわす列$\sigma$を伴っている.これは,その関数型をもつ項
を引数に適用したときに生じる計算エフェクトであり,具体的には,
\Shiftz の answer type の列である.前述したようにshift0 は多段
階の reset0 にアクセスできるため,$n$個先のreset0 の answer typeまで記
憶するため,このように型の列$\sigma$で表現している.
ただし,本研究の範囲では,answer type modification に対応する必要はな
いので,エフェクトはシンプルに型の列($n$個先の reset0 のanswer type を
$n=1,\cdots,k$に対して並べた列)で表現している.
この型システムの詳細は,Materzokら\cite{Materzok2011}の研究を参照されたい.

本稿の範囲では,コントロールオペレータは現在ステージのみにあらわれ,生
成されるコードの中にはあらわないため,レベル1の関数型は,エフェクトを
表す列を持たない.
また,本項では,shift0/reset0 はコードを操作する目的にのみ使うため,継
続の型は,コードからコードへの関数の形をしている.
ここでは,後の定義を簡略化するため,継続を,通常の関数とは区別しており,
そのため,継続の型も通常の関数の型とは区別して二重の横線で表現している.

型判断は,以下の2つの形である.

\begin{figure}[H]
  \centering
  \begin{align*}
    \Gamma \vdash^{L} e : t ;~\sigma \\
    \Gamma \models \gamma \ord \gamma
  \end{align*}
  \caption{型判断の定義}
  \label{fig:judgement_def}
\end{figure}

ここで,型文脈$\Gamma$は次のように定義される.

\begin{figure}[H]
  \centering
  \begin{align*}
    \Gamma ::= \emptyset
    \mid \Gamma, (\gamma \ord \gamma)
    \mid \Gamma, (x : t)
    \mid \Gamma, (u : t)^{\gamma}
  \end{align*}
  \caption{型文脈の定義}
  \label{fig:type_context_def}
\end{figure}


型判断の導出規則を与える.まず,$\Gamma \models \gamma \ord \gamma$の
形に対する規則である.

\begin{figure}[H]
  \centering
  \[
    \infer
    {\Gamma \models \gamma_1 \ord \gamma_1}
    {}
    \quad
    \infer
    {\Gamma, \gamma_1 \ord \gamma_2 \models \gamma_1 \ord \gamma_2}
    {}
  \]

  \[
    \infer
    {\Gamma \models \gamma_1 \ord \gamma_3}
    {\Gamma \models \gamma_1 \ord \gamma_2 & \Gamma \models \gamma_2 \ord \gamma_3}
  \]

  \[
    \infer
    {\Gamma \models \gamma_1 \cup \gamma_2 \ord \gamma_1}
    {}
    \quad
    \infer
    {\Gamma \models \gamma_1 \cup \gamma_2 \ord \gamma_2}
    {}
  \]

  \[
    \infer
    {\Gamma \models \gamma_3 \ord \gamma_1 \cup \gamma_2}
    {\Gamma \models \gamma_3 \ord \gamma_1
      &\Gamma \models \gamma_3 \ord \gamma_2}
  \]
  \caption{$\Gamma \models \gamma \ord \gamma$の形に対する型導出規則}
  \label{fig:gmg_rule}
\end{figure}



次に,$\Gamma \vdash^{L} e : t ;~\sigma$ の形に対する型導出規則を与える.
まずは,レベル0における単純な規則である.

\begin{figure}[H]
  \centering
  \[
    \infer
    {\Gamma, x : t \vdash x : t ~;~ \sigma}
    {}
    \quad
    \infer
    {\Gamma, (u : t)^\gamma \vdash^\gamma u : t ~;~ \sigma}
    {}
  \]

  \[
    \infer
    {\Gamma \vdash^{L} c : t^c ~;~\sigma}
    {}
  \]

  \[
    \infer
    {\Gamma \vdash^{\gamma} e_1~ e_2 : t_1 ; \sigma}
    {\Gamma \vdash^{\gamma} e_1 : \funT{t_2}{t_1}{\sigma};~ \sigma
      & \Gamma \vdash^{\gamma} e_2 : t_2  ; \sigma
    }
    \quad
    \infer
    {\Gamma \vdash e_1 \, e_2 : t;~\sigma}
    {\Gamma \vdash e_1 : t_2 \to t_1;~\sigma
      &\Gamma \vdash e_2 : t_2;~\sigma
    }
  \]

  \[
    \infer
    {\Gamma \vdash \fun{x}{e} : \funT{t_1}{t_2}{\sigma} ~;~\sigma}
    {\Gamma,~x : t_1 \vdash e : t_2 ~;~ \sigma}
    \quad
    \infer
    {\Gamma \vdash^\gamma \fun{u}{e} : \funT{t_1}{t_2}{} ~;~\sigma'}
    {\Gamma,~(u : t_1)^\gamma \vdash^\gamma e : t_2 ~;~ \sigma}
  \]

  \[
    \infer
    {\Gamma \vdash^{L} \ift{e_1}{e_2}{e_3} : t ~;~ \sigma}
    {\Gamma \vdash^{L} e_1 : \boolT ;~ \sigma
      & \Gamma \vdash^{L} e_2 : t ; \sigma
      & \Gamma \vdash^{L} e_3 : t ; \sigma}
  \]
  \caption{(レベル0,レベル1の)$\Gamma \vdash^{L} e : t ;~\sigma$ の単純な形に対する型導出規則}
  \label{fig:gvs_rule}
\end{figure}


次にコードレベル変数に関するラムダ抽象の規則である.

\begin{figure}[H]
  \centering
  \[
    \infer[(\gamma_1~\text{is eigen var})]
    {\Gamma \vdash \cfun{x}{e} : \codeT{t_1\to t_2}{\gamma} ~;~ \sigma}
    {\Gamma,~\gamma_1 \ord \gamma,~x:\codeT{t_1}{\gamma_1} \vdash e
      : \codeT{t_2}{\gamma_1}; \sigma}
  \]
  \oishi{\\$\ccfun{}{}$ は必要あるかどうか}
  \[
    \infer
    {\Gamma \vdash \ccfun{u^1}{e} : \codeT{t_1 \to t_2}{\gamma} ; \sigma}
    {\Gamma, \gamma_1 \ord \gamma, x : (u : t_1)^{\gamma_1} \vdash e : \codeT{t_2}{\gamma_1} ; \sigma}
  \]
  \caption{コードレベルのラムダ抽象の型導出規則}
  \label{fig:code_abs_type_rule}
\end{figure}


コントロールオペレータに対する型導出規則である.

\begin{figure}[H]
  \centering
  \[
    \infer{\Gamma \vdash \resetz{e} : \codeT{t}{\gamma} ~;~ \sigma}
    {\Gamma \vdash e : \codeT{t}{\gamma} ~;~ \codeT{t}{\gamma}, \sigma}
  \]

  \[
    \infer{\Gamma \vdash \shiftz{k}{e} : \codeT{t_1}{\gamma_1} ~;~ \codeT{t_0}{\gamma_0},\sigma}
    {\Gamma,~k:\contT{\codeT{t_1}{\gamma_1}}{\codeT{t_0}{\gamma_0}}{\sigma}
      \vdash e : \codeT{t_0}{\gamma_0} ; \sigma
      & \Gamma \models \gamma_1 \ord \gamma_0
    }
  \]

  \[
    \infer
    {\Gamma,~ k:\contT{\codeT{t_1}{\gamma_1}}{\codeT{t_0}{\gamma_0}}{\sigma}
      \vdash \throw{k}{v} : \codeT{t_0}{\gamma_2} ; \sigma}
    {\Gamma,~ k:\contT{\codeT{t_1}{\gamma_1}}{\codeT{t_0}{\gamma_0}}{\sigma}
      \vdash v : \codeT{t_1}{\gamma_1 \cup \gamma_2} ; \sigma
      & \Gamma \models \gamma_2 \ord \gamma_0
    }
  \]
  \caption{コントロールオペレータに対する型導出規則}
  \label{fig:controlop_type_rule}
\end{figure}


コード生成に関する補助的な規則として,Subsumptionに相当する規則等がある.

\begin{figure}[H]
  \centering
  \[
    \infer
    {\Gamma \vdash e : \codeT{t}{\gamma_2} ; \sigma}
    {\Gamma \vdash e : \codeT{t}{\gamma_1} ; \sigma
      & \Gamma \models \gamma_2 \ord \gamma_1
    }
  \]

  \[
    \infer
    {\Gamma \vdash^{\gamma_2} e : t ~;~ \sigma}
    {\Gamma \vdash^{\gamma_1} e : t ~;~ \sigma
      & \Gamma \models \gamma_2 \ord \gamma_1
    }
  \]


  \[
    \infer
    {\Gamma \vdash \code{e} : \codeT{t^1}{\gamma} ; \sigma}
    {\Gamma \vdash^{\gamma} e : t^1 ; \sigma}
  \]

  \[
    \infer
    {\Gamma \vdash e : t;~\codeT{t'}{\gamma_2},\sigma}
    {\Gamma \vdash e : t;~\codeT{t'}{\gamma_1},\sigma
      & \Gamma \models \gamma_2 \ord \gamma_1
    }
  \]

  \caption{コード生成に関するSubsumptionの型導出規則}
  \label{fig:code_gen_subs_type_rule}
\end{figure}


\end{document}
%%% Local Variables:
%%% mode: japanese-latex
%%% TeX-master: t
%%% End:
