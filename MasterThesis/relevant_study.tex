\chapter{関連研究}
表現力と安全性を兼ね備えたコード生成の体系としては,
2009年の亀山らの研究\cite{Kameyama2009}が最初である.
彼らは,MetaOCamlにおいてshift/resetとよばれるコントロールオペレータを
使うスタイルでのプログラミングを提案するとともに,
コントロールオペレータの影響が変数スコープを越えることを制限する型シス
テムを構築し,安全性を厳密に保証した.
Westbrookら\cite{Westbrook}は同様の研究を Java のサブセットを対象におこなった.
須藤ら\cite{Sudo2014}は,書換え可能変数を持つコード生成体系に対して,
部分型付けを導入した型システムを提案して,安全性を保証した.
これらの体系は,安全性の保証を最優先した結果,表現力の上での制限が強く
なっている.特に,let挿入とよばれるコード生成技法をシミュレートするた
めには,shift/reset が必要であるが,複数の場所へのlet挿入を許すために
は,複数の種類のshift/resetを組み合わせる必要がある.
この目的のため,階層的shift/resetやマルチプロンプト
shift/resetといった,shift/reset を複雑にしたコントロールオペレータを
考えることができるが,その場合の型システムは非常に複雑になることが予想
され,安全性を保証するための条件も容易には記述できない,等の問題点がある.

本研究では,このような問題点を克服するため,shift/reset の意味論をわず
かに変更した shift0/reset0 というコントロールオペレータに着目する.
このコントロールオペレータは,長い間,研究対象となってこなかったが
2011年以降,Materzok らは,部分型付けに基づく型システムや,
関数的なCPS変換を与えるなど,簡潔で拡張が容易な理論的基盤をもつことを
解明した\cite{Materzok2011,materzok2012}.
特に,shift0/reset0 は shift/reset と同様のコントロールオペレータであ
りながら,階層的shift/reset を表現することができる,という点で,
表現力が高い.本研究では,これらの事実に基づき,これまでのshift/reset
を用いたコード生成体系の知見を,shift0/reset0 を用いたコード生成体系の
構築に活用するものである.


\oishi{let-insertion の参考論文 : Olivier Danvy 最初って言ってるやつ,
  MetaOcaml : Gentle to introduction part1, part2 Taha,
  varmeta Flops : oleg,
  lightweight-module scala : romps stagingのやつ
  を参考文献に加える
}

%%% Local Variables:
%%% mode: japanese-latex
%%% TeX-master: "master_oishi"
%%% End:
