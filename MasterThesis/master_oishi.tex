\documentclass[a4paper,11pt]{jreport}

\setcounter{tocdepth}{3}
\setcounter{page}{-1}

\setlength{\oddsidemargin}{0.1in}
\setlength{\evensidemargin}{0.1in}
\setlength{\topmargin}{0in}
\setlength{\textwidth}{6in}
% \setlength{\textheight}{10.1in}
\setlength{\parskip}{0em}
\setlength{\topsep}{0em}

\renewcommand{\baselinestretch}{1.1}

% \newcommand{\zu}[1]{{\gt \bf 図\ref{#1}}}

%% タイトル生成用パッケージ(重要)
\usepackage{sie-jp-utf8}

\usepackage[dvipdfmx]{graphicx,color}
\usepackage{ascmac}

\usepackage{theorem}
\usepackage{amsmath,amssymb}
\usepackage{ascmac}
\usepackage{mathtools}
\usepackage{proof}
\usepackage{stmaryrd}
\usepackage{listings,jlisting}
\usepackage{here}
\usepackage{verbatim}
\usepackage{framed}
\usepackage{algorithm}
\usepackage{algpseudocode}


\lstset{
  basicstyle=\ttfamily,
  columns=fullflexible,
  keepspaces=true,
}

\newenvironment{vq}
{%begin
  \VerbatimEnvironment \begin{screen} \begin{quote} \begin{Verbatim}
      }
      {%end
      \end{Verbatim} \end{quote} \end{screen}
}
\newtheorem{theorem}{theorem}[section]

\definecolor{DarkGreen}{rgb}{0,0.5,0}
\definecolor{Magenta}{rgb}{1.0, 0.0, 1.0}

\newcommand\too{\leadsto^*}
\newcommand\pink[1]{\textcolor{pink}{#1}}
\newcommand\red[1]{\textcolor{red}{#1}}
\newcommand\green[1]{\textcolor{green}{#1}}
\newcommand\magenta[1]{\textcolor{magenta}{#1}}
\newcommand\blue[1]{\textcolor{blue}{#1}}

\newcommand\fun[2]{\lambda{#1}.{#2}}

\newcommand\Resetz{\textbf{reset0}}
\newcommand\Shiftz{\textbf{shift0}}
\newcommand\Throw{\textbf{throw}}
\newcommand\resetz[1]{\Resetz~{#1}}
\newcommand\shiftz[2]{\Shiftz~{#1}\to{#2}}
\newcommand\throw[2]{\Throw~{#1}~{#2}}

\newcommand\cfun[2]{\underline{\lambda}{#1}.{#2}}
\newcommand\ccfun[2]{\underline{\underline{\lambda}}{#1}.{#2}}

\newcommand\cResetz{\underline{\textbf{reset0}}}
\newcommand\cShiftz{\underline{\textbf{shift0}}}
\newcommand\cThrow{\underline{\textbf{throw}}}
\newcommand\cresetz[1]{\cResetz~{#1}}
\newcommand\cshiftz[2]{\cShiftz~{#1}\to{#2}}
\newcommand\cthrow[2]{\cThrow~{#1}~{#2}}

\newcommand\cPlus{\underline{\textbf{+}}}
\newcommand\Plus{\textbf{+}}

\newcommand\cLet{\underline{\textbf{let}}}
\newcommand\cIn{\underline{\textbf{in}}}
\newcommand\clet[3]{\cLet~{#1}={#2}~\cIn~{#3}}
\newcommand\csp[1]{\texttt{\%}{#1}}
\newcommand\cint{\underline{\textbf{int}}}
\newcommand\code[1]{\texttt{<}{#1}\texttt{>}}
\newcommand\codebegin{\texttt{<}}
\newcommand\codeend{\texttt{>}}

\newcommand\intT{\mbox{\texttt{int}}}
\newcommand\boolT{\mbox{\texttt{bool}}}

\newcommand\codeT[2]{\langle{#1}\rangle^{#2}}
\newcommand\funT[3]{{#1} \stackrel{#3}{\rightarrow} {#2}}
\newcommand\contT[3]{{#1} \stackrel{#3}{\Rightarrow} {#2}}

\newcommand\ord{\ge}

\newcommand\Let{\textbf{let}}
\newcommand\In{\textbf{in}}
\newcommand\letin[3]{\Let~{#1}={#2}~\In~{#3}}

\newcommand\ift[3]{\textbf{if}~{#1}~\textbf{then}~{#2}~\textbf{else}~{#3}}
\newcommand\cif[3]{\underline{\textbf{if}}~\code{{#1}}~\code{{#2}}~\code{{#3}}}
\newcommand\cIf{\underline{\textbf{if}}}

\newcommand\fix{\textbf{fix}}
\newcommand\cfix{\underline{\textbf{fix}}}

\newcommand\lto{\leadsto}
\newcommand\cat{~\underline{@}~}

\newcommand\ksubst[2]{\{{#1}\Leftarrow{#2}\}}

\newcommand\cFor{\underline{\textbf{for}}}
\newcommand\forin[2]{\textbf{for}~{#1}~\textbf{to}~{#2}~\textbf{do}}
\newcommand\cforin[2]{\underline{\textbf{for}}~{#1}~\underline{\textbf{to}}~{#2}~\underline{\textbf{do}}}
\newcommand\cArray[1]{\underline{[{#1}]}}
\newcommand\cArrays[2]{\underline{[{#1}][{#2}]}}
\newcommand\aryset[3]{{#1}[{#2}]\leftarrow {#3}}
\newcommand\caryset[3]{\underline{\textbf{aryset}}~{#1}~{#2}~{#3}}

% コメントマクロ
\newcommand\kam[1]{\red{kam said: {#1}}}
\newcommand\oishi[1]{\blue{oishi said: {#1}}}

\theoremstyle{break}

\newtheorem{theo}{定理}[section]
\newtheorem{defi}{定義}[section]
\newtheorem{lemm}{補題}[section]

\algnewcommand\algorithmicforeach{\textbf{for each}}
\algdef{S}[FOR]{ForEach}[1]{\algorithmicforeach\ #1\ \algorithmicdo}

\renewcommand{\topfraction}{.85}
\renewcommand{\bottomfraction}{.60}
\renewcommand{\textfraction}{.15}
\renewcommand{\floatpagefraction}{.6}

\newcommand\smallerscope[2]{#1 \ord #2}
\newcommand\greaterscope[2]{#2 \ord #1}
\newcommand\longer[2]{{#1} \ord {#2}}
% \newcommand*\defeq{\stackrel{\text{def}}{=}}
\newcommand\Int{\mbox{\texttt{Int}}}
\newcommand\Bool{\mbox{\texttt{Bool}}}

\newcommand\uni{\cup} % !!! 現在の順序では「∪」

%% タイトル
%% 【注意】タイトルの最後に\\ を入れるとエラーになります
\title{安全なコード移動が可能なコード生成言語の\\型システムの設計と実装}

%% 著者
\author{大石 純平}
%% 学位 (2012/11 追加)
\degree{修士(工学)}
%% 指導教員
\advisor{亀山 幸義}

%% 専攻名 と 年月
%% 年月は必要に応じて書き替えてください。
\majorfield{コンピュータサイエンス} \yearandmonth{2017年 3月}

\begin{document}
\maketitle
\thispagestyle{empty}
\newpage

\thispagestyle{empty}
\vspace*{20pt plus 1fil}
\parindent=1zw
\noindent
%%
%% 論文の概要(Abstract)
%%
\begin{center}
  {\bf 概要}
  \vspace{5mm}
\end{center}
コード生成法は,プログラムの実行性能の高さと保守性・再利用性を両立でき
るプログラミング手法として有力なものである.本研究は,コード生成法で必
要とされる「多段階let挿入」等を簡潔に表現できるコントロールオペレータ
である shift0/reset0を持つコード生成言語とその型システムを構築し,
生成されたコードの型安全性を保証するための型システムを構築した.多段階let挿入は,入れ子になった
forループを飛び越えたコード移動を許す仕組みであり,ループ不変式の移動
などのために必要である.コード生成言語の型安全性に関して,破壊的代入
を持つ体系に対するSudoらの研究等があるが,本研究は,彼らの環境識別子
にジョインを追加するという拡張により,shift0/reset0 を持つコード生成言
語に対する型システムが構築できることを示した.

%%%%%
\par
\vspace{0pt plus 1fil}
\newpage

\pagenumbering{roman} % I, II, III, IV
\tableofcontents
\listoffigures
% \listoftables

\pagebreak \setcounter{page}{1}
\pagenumbering{arabic} % 1,2,3

\input{intro}
\section{準備}

\subsection{コード生成の例}
\begin{frame}
  \frametitle{コード生成言語による記述例}

  \begin{align*}
    \visible<1->{\text{コード生成器}} \visible<1->{&\phantom{\too} \text{生成されるコード}} \\
    \visible<1->{(\cint~ 3)} \visible<1->{&\too \code{3}} \\
    \visible<1->{(\cint~ 3)~ \cPlus~ (\cint~ 5)} \visible<1->{&\too \code{3 + 5}} \\
    \visible<1->{\cfun{x}{~(x~ \cPlus~ (\cint~ 3))}} \visible<1->{&\too \code{\fun{x'}{~(x' + 3)}}} \\
    \visible<1->{\cfordo{x = \cdots}{\cdots}~ \cdots}
    \visible<1->{&\too \code{\fordo{x' = \cdots}{\cdots}~ \cdots}}
  \end{align*}

  % \begin{visibleenv}<2>
  %   \begin{exampleblock}{コードコンビネータ}
  %     \begin{itemize}
  %     \item 下線つきの演算子
  %     \item コードを引数にとり,コードを返す
  %     \end{itemize}
  %   \end{exampleblock}
  % \end{visibleenv}

\end{frame}

\subsection{コード生成器と生成されるコード}

\begin{frame}
  \frametitle{let挿入(コード移動)の実現方法}

  \begin{visibleenv}<1->
    \begin{columns}
      \begin{column}{0.5\textwidth}%% [横幅] 0.2\textwidth = ページ幅の 20 %
        コード生成器
        \begin{align*}
          & ~ \cfordo{x = e1}{e2} \\
          & ~~~ \cfordo{y = e3}{e4} \\
          & ~~~~~~\caryset{\code{a}}{(x,y)}~ ~ \\
          & ~~~~~~~~\magenta{\cLet ~u ~= ~\text{cc} ~\cIn}~~ \text{u}
        \end{align*}
      \end{column}
      $\too$
      \begin{column}{0.5\textwidth}%% [横幅] 0.2\textwidth = ページ幅の 20 %
        生成したいコード
        \begin{align*}
          & \cbra \fordo{x' = e1'}{e2'} \\
          & ~~\magenta{\Let ~u' ~= ~\text{cc}' ~\In} \\
          & ~~~~\fordo{y' = e3'}{e4'} \\
          & ~~~~~~\aryset{a}{x',y'}{u'} \cket
        \end{align*}
      \end{column}
    \end{columns}
  \end{visibleenv}

  \begin{visibleenv}<2>
    \begin{exampleblock}{shift0/reset0の導入}
      \red{shift0/reset0} 等を用いることで,(多段階)let挿入等を行う
    \end{exampleblock}
  \end{visibleenv}
\end{frame}

% \subsection{多段階let 挿入}

% \begin{frame}
%   \frametitle{shift0/reset0 によるlet挿入}
%   \noindent
%   \begin{align*}
%     \uncover<3->{\Resetz ~(E[\Shiftz~ k \to e]) ~\leadsto~ e \ksubst{k}{E}}
%   \end{align*}

%   \noindent

%   %   \begin{align*}
%   %     \text{コード生成器:}~~
%   %     & \uncover<4->{\blue{\Resetz}} ~~\cfordo{x = e1}{e2} \\
%   %     & ~~\uncover<2-3>{\blue{\Resetz}} ~~\cfordo{y = e3}{e4} \\
%   %     & ~~~~\uncover<2->{\blue{\Shiftz}~\blue{k}~\to}~
%   %     \magenta{\cLet~u=t~\cIn} \\
%   %     & ~~~~~~\uncover<2->{(\blue{\Throw~k}}~(\caryset{a}{(x,y)}{u})
%   %     \uncover<2->{)} \\
%   %     &   \uncover<3,5->{\blue{k}}
%   %     \only<3>{\Leftarrow ~~\cfordo{y = e3}{e4}~[\ ]}
%   %     \only<5->{\Leftarrow ~~\cfordo{x = e1}{e2} ~~\cfordo{y = e3}{e4} ~~[\ ]} \\
%   %     \text{生成コード:}~~
%   %     & \uncover<3,5->{\cbra~}
%   %     \only<3>{\fordo{x' = e1'}{e2'}} \only<5->{\magenta{\Let ~u' ~= ~t' ~\In}} \\
%   %     & ~~
%   %     \only<3>{\magenta{\Let ~u' ~= ~t' ~\In}} \only<5->{\fordo{x' = e1'}{e2'}} \\
%   %     & ~~~~\uncover<3,5->{\fordo{y' = e3'}{e4'}} \\
%   %     & ~~~~~~\uncover<3,5->{\aryset{a}{x',y'}{u'} ~\cket}
%   %   \end{align*}

%   \begin{align*}
%     \scriptsize{\text{コード生成器:}}~~
%     & \uncover<2-3>{\blue{\Resetz}} \only<1-2>{~\cfordo{x = e1}{e2}} \uncover<3>{\green{~\cfordo{x = e1}{e2}}}\\
%     & ~~~~~~~~~~~ \only<1-2>{\cfordo{y = e3}{e4}} \uncover<3>{\green{\cfordo{y = e3}{e4}}}\\
%     & ~~~~~~~~~~~~ \only<1-2>{\set~a~(x,y)~} \only<3>{\green{\set~a~(x,y)}~} \only<4->{~~~~~~~~~~~~~~~} \only<1>{cc} \uncover<2-3>{\blue{\Shiftz~ k \to}} \uncover<2->{\magenta{\cLet~u = cc~\cIn}~ \blue{\Throw~ k}~ u \\}
%     %     & ~~~~~~~~~~~~~~~~~~~~~~~~~~~~~~~~\uncover<2->{\magenta{\cLet~u = cc~\cIn}~ \blue{\Throw~ k}~ u} \\
%     & \uncover<3->{\blue{k} \Leftarrow ~~\green{\cfordo{x = e1}{e2}} \\
%     & ~~~~~~~~~~~\green{\cfordo{y = e3}{e4}} \\
%     & ~~~~~~~~~~~~~\green{\caryset{a}{(x,y)}{}} [\ ]} \\
%     \uncover<5->{\scriptsize{\text{生成コード:}}~~}
%     & \uncover<5->{\cbra~}
%     \uncover<5->{\magenta{\Let ~u' ~= ~cc' ~\In}} \\
%     & ~~~~\uncover<5->{\fordo{x' = e1'}{e2'}} \\
%     & ~~~~~~\uncover<5->{\fordo{y' = e3'}{e4'}} \\
%     & ~~~~~~~~\uncover<5->{\aryset{a}{x',y'}{u'} ~\cket}
%   \end{align*}

%   %   \begin{align*}
%   %     \scriptsize{\text{コード生成器:}}~~
%   %     & \uncover<2->{\blue{\Resetz}}~ \only<1-2>{\cfordo{x = e1}{e2}} \only<3>{\green{\cfordo{x = e1}{e2}}}\\
%   %     & ~~~~~~~~~~~\only<1-2>{\cfordo{y = e3}{e4}} \only<3>{\green{\cfordo{y = e3}{e4}}}\\
%   %     & ~~~~~~~~~~~~\only<1-2>{\set~a~(x,y)~} \only<3>{\green{\set~a~(x,y)~}} \only<1>{cc} \only<2>{\blue{\Shiftz~ k \to} \magenta{\cLet~u = cc~\cIn}~ \blue{\Throw~ k}~ u} \only<3->{\magenta{\cLet~u = cc~\cIn}~ \blue{\Throw~ k}~ u}\\
%   %   %     & ~~~~~~~~~~~~~~~~~~~~~~~~~~~~~~~~\uncover<2->{\magenta{\cLet~u = cc~\cIn}~ \blue{\Throw~ k}~ u} \\
%   %     & \uncover<3->{\blue{k} \Leftarrow ~~ \green{\cfordo{x = e1}{e2}} \\
%   %     & ~~~~~~~~~~~\green{\cfordo{y = e3}{e4}} \\
%   %     & ~~~~~~~~~~~~~\green{\caryset{a}{(x,y)}{}} [\ ]} \\
%   %     \uncover<4->{\scriptsize{\text{生成コード:}}~~}
%   %     & \uncover<4->{\cbra~}
%   %     \uncover<4->{\magenta{\Let ~u' ~= ~cc' ~\In}} \\
%   %     & ~~~~\uncover<4->{\fordo{x' = e1'}{e2'}} \\
%   %     & ~~~~~~\uncover<4->{\fordo{y' = e3'}{e4'}} \\
%   %     & ~~~~~~~~\uncover<4->{\aryset{a}{x',y'}{u'} ~\cket}
%   %   \end{align*}
% \end{frame}


% \begin{frame}
%   \frametitle{shift0/reset0 による\alert{多段階}let挿入}
%   \noindent
%   \begin{align*}
%     \Resetz ~(E[\Shiftz~ k \to e]) ~\leadsto~ e \ksubst{k}{E}
%   \end{align*}

%   \noindent
%   %   \begin{align*}
%   %     \text{コード生成器:}~~
%   %     & \red{\Resetz} ~~\cfordo{x = e1}{e2} \\
%   %     & ~~\blue{\Resetz} ~~\cfordo{y = e3}{e4} \\
%   %     & ~~~~\blue{\Shiftz}~\blue{k_2}~\to~
%   %     \red{\Shiftz}~\red{k_1}~\to~
%   %     \magenta{\cLet~u=t~\cIn} \\
%   %     & ~~~~~~\red{\Throw~k_1}~
%   %     (\blue{\Throw~k_2}~(\caryset{a}{(x,y)}{u})) \\
%   %     %     & \red{k_1} \Leftarrow ~~\cfordo{x = e1}{e2}~[\ ] \\
%   %     %     & \blue{k_2} \Leftarrow ~~\cfordo{y = e3}{e3}~[\ ] \\
%   %     \text{生成コード:}~~
%   %     & \cbra~\magenta{\Let ~u' ~= ~t' ~\In} \\
%   %     & ~~\fordo{x' = e1'}{e2'} \\
%   %     & ~~~~\fordo{y' = e3'}{e4'} \\
%   %     & ~~~~~~\aryset{a}{x',y'}{u'} ~\cket
%   %   \end{align*}

%   \begin{align*}
%     \scriptsize{\text{コード生成器:}}~~
%     & \red{\Resetz} ~~\cfordo{x = e1}{e2} \\
%     & ~~\blue{\Resetz} ~~\cfordo{y = e3}{e4} \\
%     & ~~~~ \caryset{a}{(x,y)}{\blue{\Shiftz~ k_1 \to}~ \magenta{\cLet~ u = cc1~ \cIn~} \blue{\Throw~ k_1}~ u;} \\
%     & ~~~~ \set~ b~ (x,y)~ \blue{\Shiftz~ k_1 \to~} \red{\Shiftz~ k_2 \to}~ \\
%     & ~~~~~~~~~~~~~~~~~~~~~~~~ \magenta{\cLet~ w = cc2~ \cIn~} \red{\Throw~ k_2} (\blue{\Throw~ k_1}~ w) \\
%     %     & \red{k_1} \Leftarrow ~~\cfordo{x = e1}{e2}~[\ ] \\
%     %     & \blue{k_2} \Leftarrow ~~\cfordo{y = e3}{e3}~[\ ] \\
%     \uncover<2->{\scriptsize{\text{生成コード:}}~~}
%     & \uncover<2->{\cbra~\magenta{\Let ~w' ~= ~cc2' ~\In}} \\
%     & \uncover<2->{~~~~\fordo{x' = e1'}{e2'}} \\
%     & \uncover<2->{~~~~~~\magenta{\Let ~u' ~= ~cc1' ~\In}} \\
%     & \uncover<2->{~~~~~~~~\fordo{y' = e3'}{e4'}} \\
%     & \uncover<2->{~~~~~~~~\aryset{a}{x',y'}{u'}} \\
%     & \uncover<2->{~~~~~~~~\aryset{b}{x',y'}{w'} ~\cket}
%   \end{align*}
% \end{frame}

%%% Local Variables:
%%% mode: latex
%%% TeX-master: "slide_oishi"
%%% End:

\input{idea}
\chapter{対象言語: 構文と意味論}

本研究における対象言語は,ラムダ計算にコード生成機能とコントロールオペ
レータshift0/reset0を追加したものに型システムを導入したものである.

本稿では,最小限の言語のみについて考えるため,コード生成機能の
「ステージ(段階)」は,コード生成段階(レベル0,現在ステージ)と
生成されたコードの実行段階(レベル1,将来ステージ)の2ステージのみを考える.

前述したように,本研究の言語では,
コードコンビネータ(Code Combinator)方式を使い,
コードコンビネータは,
$\cPlus$ や $\cIf$のように下線を引いて表す.

\section{構文の定義}

対象言語の構文を定義する.

変数は,レベル0変数($x$), レベル1変数($u$),
(レベル0の)継続変数($k$)の3種類がある.
レベル0項($e^0$),レベル1項($e^1$)およびレベル0の値($v$)を下の通り定義する.

\begin{figure}[!ht]
  \centering
  \begin{align*}
    c & ::= i \mid b \mid \cint
        \mid \cat \mid + \mid \cPlus \mid \cIf \\
    v & ::= x \mid c \mid \fun{x}{e^0} \mid \code{e^1} \\
    e^0 & ::=  v  \mid e^0~ e^0 \mid \ift{e^0}{e^0}{e^0} \\
      & \mid \cfun{x}{e^0}
        \mid \ccfun{u}{e^0} \\
      & \mid \resetz{e^0}
        \mid \shiftz{k}{e^0}
        \mid \throw{k}{v} \\
    e^1 & ::=  u \mid c \mid \fun{u}{e^1} \mid e^1~ e^1
          \mid \ift{e^1}{e^1}{e^1} \\
  \end{align*}
  \caption{対象言語の構文の定義}
\end{figure}

ここで$i$は整数の定数,$b$は真理値定数である.

定数のうち,下線がついているものはコードコンビネータである.
変数は,ラムダ抽象(下線なし,下線つき,二重下線つき)および shift0 により束縛され,
$\alpha$同値な項は同一視する.
$\letin{x}{e_1}{e_2}$および$\clet{x}{e_1}{e_2}$は,
それぞれ,$(\fun{x}{e_2})e_1$ および $(\cfun{x}{e_2})\cat e_1$の省略形である.
前述の例でのべた$\cFor$は,
コード構築定数とコードレベル適用を用いて導入することとし,
(この導入にあたっての型システムの拡張は容易なので)ここでは省略する.

\section{操作的意味論}

対象言語は,値呼びで left-to-rightの操作的意味論を持つ.
ここでは評価文脈に基づく定義を与える.

評価文脈を以下のように定義する.
\begin{figure}[H]
  \centering
  \begin{align*}
    E & ::= [~] \mid E~ e^0 \mid v~ E \\
      & \mid \ift{E}{e^0}{e^0} \mid \Resetz~ E \mid \ccfun{u}{E}
  \end{align*}
  \caption{評価文脈}
\end{figure}

コード生成言語で特徴的なことは,
コードレベルのラムダ抽象の内部で評価が進行する点である.実際,
上記の定義には,$\ccfun{u}{E}$が含まれている.
たとえば,$\ccfun{u}{\code{u} \cPlus [~]}$ は評価文脈である.

この評価文脈$E$と次に述べる計算規則$r \to l$ により,
評価関係$e \lto e'$ を図\ref{fig:etoe}のように定義する.

\begin{figure}[H]
  \centering
  \[
    \infer{E[r] \lto E[l]}{r \to l}
  \]
  \caption{$e \lto e' $の評価関係}
  \label{fig:etoe}
\end{figure}


計算規則は図\ref{fig:calc_rule}の通り定義する.
\begin{figure}[H]
  \centering
  \begin{align*}
    (\fun{x}{e})~v &\to e\{ x := v \} \\
    \ift{true}{e_1}{e_2} &\to e_1 \\
    \ift{else}{e_1}{e_2} &\to e_2 \\
    \cfun{x}{e} &\to \ccfun{u}{(e\{ x := \code{u} \})} \\
    \ccfun{u}{\code{e}} &\to \code{\fun{u}{e}} \\
    \Resetz~ v &\to v \\
    \Resetz (E[\Shiftz~ k \to e]) &\to e \ksubst{k}{E}
  \end{align*}
  \caption{計算規則}
  \label{fig:calc_rule}
\end{figure}

ただし,4行目の$u$はフレッシュなコードレベル変数とし,
最後の行の$E$は穴の周りに{\Resetz}を含まない評価文脈とする.
また,この行の右辺のトップレベルに{\Resetz}がない点が,
shift/reset の振舞いとの違いである.すなわち,shift0 を1回計算すると,
reset0 が1つはずれるため,shift0 をN個入れ子にすることにより,
N個分外側のreset0 までアクセスすることができ,多段階let挿入を実現でき
るようになる.

上記における継続変数に対する代入$e\ksubst{k}{E}$は図\ref{fig:k_subst}の通り定義する.

\begin{figure}[H]
  \centering
  \begin{align*}
    (\throw{k}{v})\ksubst{k}{E} &\equiv \Resetz (E[v]) \\
    (\throw{k'}{v})\ksubst{k}{E} &\equiv \throw{k'}{(v\ksubst{k}{E})}
    \\
                                & \text{ただし}~k \not= k'
  \end{align*}
  \caption{継続への代入}
  \label{fig:k_subst}
\end{figure}

上記以外の$e$に対する代入の定義は透過的であるとする.
上記の定義の1行目で\Resetz を挿入しているのは{\Shiftz}の意味論に対応し
ており,これを挿入しない場合は別のコントロールオペレータ(Felleisenの
control/promptに類似した control0/prompt0)の振舞いとなる.

コードコンビネータ定数の振舞い(ラムダ計算における$\delta$規則に相当)は
図\ref{fig:comb-rule}のように定義する.

\begin{figure}[H]
  \centering
  \begin{align*}
    \cint~ n &\to \code{n} \\
    \code{e_1}~ \cat~ \code{e_2} &\to \code{e_1~ e_2} \\
    \code{e_1}~ \cPlus~ \code{e_2} &\to \code{e_1 + e_2} \\
    \cif{e_1}{e_2}{e_3} &\to \code{\ift{e_1}{e_2}{e_3}}
  \end{align*}
  \caption{コードコンビネータの規則}
  \label{fig:comb-rule}
\end{figure}

% 計算の例を以下に示す.
% \begin{align*}
    %     e_1 & = \Resetz ~~\cLet~x_1=\csp{3}~\cIn \\
    %               & \phantom{=}~~ \Resetz ~~\cLet~x_2=\csp{5}~\cIn \\
    %               & \phantom{=}~~ \Shiftz~k~\to~\cLet~y=t~\cIn \\
    %               & \phantom{=}~~ \Throw~ k~ (x_1~\cPlus~x_2~\cPlus~y) \\
    %   \end{align*}
    %
    %     \begin{align*}
    %     [ e_1 ] &\lto [ \Resetz (\cLet~x_1=\csp{3}~\cIn \\
    %               &\Resetz~ \cLet~x_2=\csp{5}~\cIn \\
    %               &[ \Shiftz~ k~ \to~ \cLet~ y=t~ \cIn \\
    %               &[ \Throw~ k~(x_1~\cPlus~x_2~\cPlus~y) ] ] ) ] \\
    %               &\lto [ \cLet~ y=t~ \cIn \\
    %               &[ \cfun{x}{\Resetz~ (\cLet~x_1=\csp{3}~ \cIn~ \Resetz~ (\cLet~ x_2=\csp{5}~ \cIn [x]))} (x_1~\cPlus~x_2~\cPlus~y) ]] \\
    %               &\lto [ \cfun{y}{(\cfun{x}{\Resetz~ (\cLet~x_1=\csp{3}~ \cIn~ \Resetz~ (\cLet~ x_2=\csp{5}~ \cIn [x]))} (x_1~\cPlus~x_2~\cPlus~y))}~ \cat~ t ] \\
    %               &\lto [[\cfun{y}{(\cfun{x}{\Resetz~ (\cLet~x_1=\csp{3}~ \cIn~ \Resetz~ (\cLet~ x_2=\csp{5}~ \cIn [x]))} (x_1~\cPlus~x_2~\cPlus~y))}]~ \cat~ t] \\
    %               &\lto [[\ccfun{y_1}{(\cfun{x}{\Resetz~ (\cLet~x_1=\csp{3}~ \cIn~ \Resetz~ (\cLet~ x_2=\csp{5}~ \cIn [x]))} (x_1~\cPlus~x_2~\cPlus~ \code{y_1}))}]~ \cat~ t] \\
    %               &\lto
                      %   \end{align*}

                      %%%   Local Variables:
                      %%%   mode: japanese-latex
                      %%%   TeX-master: "master_oishi"
                      %%%   End:

\chapter{型システム}

本研究での型システムについて述べる.

基本型$b$,環境識別子(Environment Classifier)$\gamma$を以下の通り定義する.

\begin{figure}[H]
  \centering
  \begin{align*}
    b & ::= \intT \mid \boolT \\
    \gamma & ::= \gamma_x \mid \gamma \cup \gamma
  \end{align*}
  \caption{基本型,環境識別子の定義}
  \label{fig:bec_def}
\end{figure}

$\gamma$の定義における$\gamma_x$は環境識別子の変数を表す.
すなわち,環境識別子は,変数であるかそれらを$\cup$で結合した形である.
以下では,メタ変数と変数を区別せず$\gamma_x$を$\gamma$と表記する.
ここで環境識別子として$\cup$を導入した理由は後述する.

$L ::= \empty \mid \gamma$ は現在ステージ(レベル0)と将来ステージ(レベル1)をまとめ
て表す記号である.たとえば,$\Gamma \vdash^L
e:t~;~\sigma$は,$L=\empty$のとき現在ステージ(レベル0)の判断で,
$L=\gamma$のとき将来ステージ(レベル1)の判断となる.

レベル0の型$t^0$,レベル1の型$t^1$,(レベル0の)型の有限列$\sigma$,
(レベル0の)継続の型$\kappa$を次の通り定義する.

\begin{figure}[H]
  \centering
  \begin{align*}
    t^0 & ::= b \mid \funT{t^0}{t^0}{\sigma} \mid \codeT{t^1}{\gamma} \\
    t^1 & ::= b \mid t^1 \to t^1 \\
    \sigma & ::= \epsilon \mid t^0, \sigma \\
    \kappa^0 & ::= \contT{\codeT{t^1}{\gamma}}{\codeT{t^1}{\gamma}}{\sigma}
  \end{align*}
  \caption{(レベル0の)継続の型の定義}
  \label{fig:k_def}
\end{figure}

レベル0の関数型$\funT{t^0}{t^0}{\sigma}$は,
エフェクトをあらわす列$\sigma$を伴っている.これは,その関数型をもつ項
を引数に適用したときに生じる計算エフェクトであり,具体的には,
\Shiftz の answer type の列である.前述したようにshift0 は多段
階の reset0 にアクセスできるため,$n$個先のreset0 の answer typeまで記
憶するため,このように型の列$\sigma$で表現している.
ただし,本研究の範囲では,answer type modification に対応する必要はな
いので,エフェクトはシンプルに型の列($n$個先の reset0 のanswer type を
$n=1,\cdots,k$に対して並べた列)で表現している.
この型システムの詳細は,Materzokら\cite{Materzok2011}の研究を参照されたい.

本稿の範囲では,コントロールオペレータは現在ステージ(レベル0)にのみあらわれ,生
成されるコードの中にはあらわないため,レベル1の関数型は,エフェクトを
表す列を持たない.
また,本項では,shift0/reset0 はコードを操作する目的にのみ使うため,継
続の型は,コードからコードへの関数の形をしている.
ここでは,後の定義を簡略化するため,継続を,通常の関数とは区別しており,
そのため,継続の型も通常の関数の型とは区別して二重の横線で表現している.

型判断は,以下の2つの形である.

\begin{figure}[H]
  \centering
  \begin{align*}
    \Gamma \vdash^{L} e : t ;~\sigma \\
    \Gamma \models \gamma \ord \gamma
  \end{align*}
  \caption{型判断の定義}
  \label{fig:judgement_def}
\end{figure}

ここで,型文脈$\Gamma$は次のように定義される.

\begin{figure}[H]
  \centering
  \begin{align*}
    \Gamma ::= \emptyset
    \mid \Gamma, (\gamma \ord \gamma)
    \mid \Gamma, (x : t)
    \mid \Gamma, (u : t)^{\gamma}
  \end{align*}
  \caption{型文脈の定義}
  \label{fig:type_context_def}
\end{figure}


型判断の導出規則を与える.まず,$\Gamma \models \gamma \ord \gamma$の
形に対する規則である.

\begin{figure}[H]
  \centering
  \[
    \infer
    {\Gamma \models \gamma_1 \ord \gamma_1}
    {}
    \quad
    \infer
    {\Gamma, \gamma_1 \ord \gamma_2 \models \gamma_1 \ord \gamma_2}
    {}
  \]

  \[
    \infer
    {\Gamma \models \gamma_1 \ord \gamma_3}
    {\Gamma \models \gamma_1 \ord \gamma_2 & \Gamma \models \gamma_2 \ord \gamma_3}
  \]

  \[
    \infer
    {\Gamma \models \gamma_1 \cup \gamma_2 \ord \gamma_1}
    {}
    \quad
    \infer
    {\Gamma \models \gamma_1 \cup \gamma_2 \ord \gamma_2}
    {}
  \]

  \[
    \infer
    {\Gamma \models \gamma_3 \ord \gamma_1 \cup \gamma_2}
    {\Gamma \models \gamma_3 \ord \gamma_1
      &\Gamma \models \gamma_3 \ord \gamma_2}
  \]
  \caption{$\Gamma \models \gamma \ord \gamma$の形に対する型導出規則}
  \label{fig:gmg_rule}
\end{figure}



次に,$\Gamma \vdash^{L} e : t ;~\sigma$ の形に対する型導出規則を与える.
まずは,レベル0における単純な規則である.

\begin{figure}[H]
  \centering
  \oishi{下の3つの型付け規則は$\Resetz$に対応する $\Shiftz, \Throw$がない場合でも,型が付いてほしいので,お尻に$\sigma$をつけている.つまり,$\Resetz~(\Resetz~ e)$ みたいなものでも型がつく.
  絶対に$\Resetz, \Shiftz, \Throw$の三位一体を条件として使うのならば,お尻に$\sigma$はいらないはず.($\Resetz$ で付加された answer type は $\Throw$のところで消えるから)}
  \[
    \infer
    {\Gamma, x : t \vdash x : t ~;~ \sigma}
    {}
    \quad
    \infer
    {\Gamma, (u : t)^\gamma \vdash^\gamma u : t ~;~ \sigma}
    {}
  \]

  \[
    \infer
    {\Gamma \vdash^{L} c : t^c ~;~\sigma}
    {}
  \]

  \[
    \infer
    {\Gamma \vdash^{\gamma} e_1~ e_2 : t_1 ; \sigma}
    {\Gamma \vdash^{\gamma} e_1 : \funT{t_2}{t_1}{\sigma};~ \sigma
      & \Gamma \vdash^{\gamma} e_2 : t_2  ; \sigma
    }
    \quad
    \infer
    {\Gamma \vdash e_1 \, e_2 : t;~\sigma}
    {\Gamma \vdash e_1 : t_2 \to t_1;~\sigma
      &\Gamma \vdash e_2 : t_2;~\sigma
    }
  \]

  \[
    \infer
    {\Gamma \vdash \fun{x}{e} : \funT{t_1}{t_2}{\sigma} ~;~\sigma'}
    {\Gamma,~x : t_1 \vdash e : t_2 ~;~ \sigma}
    \quad
    \infer
    {\Gamma \vdash^\gamma \fun{u}{e} : \funT{t_1}{t_2}{} ~;~ }
    {\Gamma,~(u : t_1)^\gamma \vdash^\gamma e : t_2 ~;~  }
  \]

  \[
    \infer
    {\Gamma \vdash^{L} \ift{e_1}{e_2}{e_3} : t ~;~ \sigma}
    {\Gamma \vdash^{L} e_1 : \boolT ;~ \sigma
      & \Gamma \vdash^{L} e_2 : t ; \sigma
      & \Gamma \vdash^{L} e_3 : t ; \sigma}
  \]
  \caption{(レベル0,レベル1の)$\Gamma \vdash^{L} e : t ;~\sigma$ の単純な形に対する型導出規則}
  \label{fig:gvs_rule}
\end{figure}


次にコードレベル変数に関するラムダ抽象の規則である.

\begin{figure}[H]
  \centering
  \[
    \infer[(\gamma_1~\text{is eigen var})]
    {\Gamma \vdash \cfun{x}{e} : \codeT{t_1\to t_2}{\gamma} ~;~ \sigma}
    {\Gamma,~\gamma_1 \ord \gamma,~x:\codeT{t_1}{\gamma_1} \vdash e
      : \codeT{t_2}{\gamma_1}; \sigma}
  \]
  \caption{コードレベルのラムダ抽象の型導出規則}
  \label{fig:code_abs_type_rule}
\end{figure}


コントロールオペレータに対する型導出規則である.

\begin{figure}[H]
  \centering
  \[
    \infer{\Gamma \vdash \resetz{e} : \codeT{t}{\gamma} ~;~ \sigma}
    {\Gamma \vdash e : \codeT{t}{\gamma} ~;~ \codeT{t}{\gamma}, \sigma}
  \]

  % \oishi{$\Throw$ルールの $\sigma$-part がずれる問題は $\Resetz$を $\Throw$の間に入れて対処していたが,$\Shiftz$で,$e$ の$\sigma$-part は先頭要素を減らさず,$\Throw$で減らすようにするのはどうか($\Resetz, \Shiftz, \Throw$は三位一体なので) → だめだった.shift をネストさせると,うまくいかないことがわかった}
  \[
    \infer{\Gamma \vdash \shiftz{k}{e} : \codeT{t_1}{\gamma_1} ~;~ \codeT{t_0}{\gamma_0},\sigma}
    {\Gamma,~k:\contT{\codeT{t_1}{\gamma_1}}{\codeT{t_0}{\gamma_0}}{\sigma}
      \vdash e : \codeT{t_0}{\gamma_0} ; \sigma
      & \Gamma \models \gamma_1 \ord \gamma_0
    }
  \]

  % \oishi{$\Shiftz$ルールでなく,$\Throw$ルールで対応するanswer type($\sigma$の先頭要素$\codeT{t'}{\gamma'}$)を減らす.→ だめだった.shift をネストさせると,うまくいかないことがわかった(多分減らさなくても型は付くと思うけど)}
  \oishi{複数の$\Throw$を使う場合,$\sigma$-partのsub sumptionを使えば良い}
  \[
    \infer
    {\Gamma,~ k:\contT{\codeT{t_1}{\gamma_1}}{\codeT{t_0}{\gamma_0}}{\sigma}
      \vdash \throw{k}{v} : \codeT{t_0}{\gamma_2} ; \sigma}
    {\Gamma,~ k:\contT{\codeT{t_1}{\gamma_1}}{\codeT{t_0}{\gamma_0}}{\sigma}
      \vdash v : \codeT{t_1}{\gamma_1 \cup \gamma_2} ; \sigma
      & \Gamma \models \gamma_2 \ord \gamma_0
    }
  \]
  \caption{コントロールオペレータに対する型導出規則}
  \label{fig:controlop_type_rule}
\end{figure}


コード生成に関する補助的な規則として,Subsumptionに相当する規則等がある.

% \oishi{$\ord$ の左右の $\gamma$ は逆? 一般に項の内部に入るにしたがって,使えるコードレベル変数は増えていくので,逆のような気がする.$ \gamma_1 \ord \gamma_0$ の意味は $\gamma_0$ より $\gamma_1$ のほうが使えるコードレベル変数は多いという意味である}

\begin{figure}[H]
  \centering
  \[
    \infer
    {\Gamma \vdash e : \codeT{t}{\gamma_2} ; \sigma}
    {\Gamma \vdash e : \codeT{t}{\gamma_1} ; \sigma
      & \Gamma \models \gamma_2 \ord \gamma_1
    }
  \]

  \[
    \infer
    {\Gamma \vdash^{\gamma_2} e : t ~;~ \sigma}
    {\Gamma \vdash^{\gamma_1} e : t ~;~ \sigma
      & \Gamma \models \gamma_2 \ord \gamma_1
    }
  \]


  \[
    \infer
    {\Gamma \vdash \code{e} : \codeT{t^1}{\gamma} ; \sigma}
    {\Gamma \vdash^{\gamma} e : t^1 ; \sigma}
  \]

  \[
    \infer
    {\Gamma \vdash e : t;~\codeT{t'}{\gamma_2},\sigma}
    {\Gamma \vdash e : t;~\codeT{t'}{\gamma_1},\sigma
      & \Gamma \models \gamma_2 \ord \gamma_1
    }
  \]

  \caption{コード生成に関するSubsumptionの型導出規則}
  \label{fig:code_gen_subs_type_rule}
\end{figure}

\Shiftz の Answer type の列($\sigma$)に関するSubsumptionに相当する規則がある.

\begin{figure}[H]
  \centering
  \[
    \infer
    {\Gamma \vdash e : t;~\codeT{t'}{\gamma_2},\sigma}
    {\Gamma \vdash e : t;~\codeT{t'}{\gamma_1},\sigma
      & \Gamma \models \gamma_2 \ord \gamma_1
    }
  \]
  \caption{$\sigma$-partに関するSubsumptionの型導出規則}
  \label{fig:sigma_subs_type_rule}
\end{figure}

\section{型付け例}

上記の型システムのもとで,いくつかの項の型付けについて述べる.

\subsection{let挿入の例}
\label{subsec:exam-let}


\begin{align*}
  e_1 & = \Resetz ~~\cLet~x_1= e_1 ~\cIn \\
      & \phantom{=}~~ \Resetz ~~\cLet~x_2= e_2~\cIn \\
      & \phantom{=}~~ \Shiftz~k~\to~\cLet~y=t~\cIn \\
      & \phantom{=}~~ \Throw~ k~ y
\end{align*}

式$e_1$に対して,$t=\cint~ 7$ あるいは $t=x_1$であれば,
$e_1$ は型付け可能である.
一方,$t=x_2$ であれば,$e_1$ は型付けできない.

\newcommand\tzero{\codeT{t}{\gamma_0}}
\newcommand\tone{\codeT{t}{\gamma_1}}
\newcommand\ttwo{\codeT{t}{\gamma_2}}
\newcommand\tthree{\codeT{t}{\gamma_3}}
\newcommand\tonethree{\codeT{t}{\gamma_1\uni\gamma_3}}
\newcommand\tall{\codeT{t}{\gamma_2\uni\gamma_1\uni\gamma_3}}
\newcommand\Gammaone{
  \longer{\gamma_1}{\gamma_0},~x_1:\tone \vdash
  \cLet~x_2=e_2~\cIn~\cdots : \tone;~\tone,\tzero}
\newcommand\Gammatwo{
  \Gamma_1,~k_2:\contT{\ttwo}{\tone}{\tzero},
  ~k_1:\contT{\tone}{\tzero}{\cdot{}}}

\def\proofone{
  \infer
  {\vdash e_1:\tzero;~\cdot{}}
  {
    \infer{\vdash \cLet~x_1=e_1~\cIn~\Resetz~\cLet~x_2=e_2~\cIn~\cdots :
      \tzero;~\tzero
    }
    {\infer{\longer{\gamma_1}{\gamma_0},~x_1:\tone \vdash
        \Resetz~\cLet~x_2=e_2~\cIn~\cdots : \tone;~\tzero
      }
      {\prooftwo}
    }
  }
}
\def\prooftwo{
  \infer{\longer{\gamma_1}{\gamma_0},~x_1:\tone \vdash
    \cLet~x_2=e_2~\cIn~\cdots : \tone;~\tone,\tzero}
  {\infer{\Gamma_1=\longer{\gamma_2}{\gamma_1},~x_2:\ttwo,~\longer{\gamma_1}{\gamma_0},~x_1:\tone \vdash
      \shiftz{k}{\cdots} : \ttwo;~\tone,\tzero
    }
    {\proofthree}
  }
}

\def\proofthree{
  {\infer{\Gamma_2=\Gamma_1,~k:\contT{\ttwo}{\tone}{\tzero}
      \vdash \cLet~y=t~\cIn~\cdots : \tone;~\tzero
    }
    {\prooffour}
  }
}

\def\prooffour{
  \infer{\Gamma_3=\Gamma_2,~\longer{\gamma_3}{\gamma_1},~y:\tthree \vdash \throw{k}{(\Resetz~ y)} :
    \tthree;~\tzero}
  {\infer{\Gamma_3 \vdash y : \tall;~\cdot}{\vdots}
    & \infer{\Gamma_3 \models \longer{\gamma_1\uni\gamma_3}{\gamma_1}}{}
  }
  & \infer[(*)]{\Gamma_2 \vdash t : \tone;~ \tzero}{\vdots}
}

\[
  \proofone
\]

$t$ が $x_1, \cint 7, x_2$ の場合に$\Gamma_2 \vdash t : \tone;~ \tzero$が成り立つかを見ていく.\\
$(*)$ のところに着目すると,
$\Gamma_2 = \longer{\gamma_2}{\gamma_1},~x_2:\ttwo,~\longer{\gamma_1}{\gamma_0},~x_1:\tone,~k:\contT{\ttwo}{\tone}{\tzero}$
より,
\begin{description}
\item[$t=x_1$の時]\mbox{}\\
  $x_1:\tone \vdash x_1:\tone$ が成り立ち,型が付く
\item[$t= \cint 7$の時]\mbox{}\\
  $\cint 7$ は定数であるので,どの Classifier $\gamma_i$ においても型が付く.
\item[$t=x_2$の時]\mbox{}\\
  $\longer{\gamma_2}{\gamma_1},~\longer{\gamma_1}{\gamma_0},~x_2:\ttwo$ より,
  $x_2$ のスコープは$\gamma_2$ であり,
  $\gamma_2$スコープのコード変数は,$\gamma_1$スコープでは一般に使用できないので
  $\longer{\gamma_2}{\gamma_1},~\longer{\gamma_1}{\gamma_0},~x_2:\ttwo \vdash x_2:\tone$ は成り立たず,型がつかない
\end{description}

\subsection{多段階let挿入の例}

\begin{align*}
  e_2 & = \Resetz ~~\cLet~x_1= e_1~\cIn \\
      & \phantom{=}~~ \Resetz ~~\cLet~x_2=e_2~\cIn \\
      & \phantom{=}~~ \Shiftz~k_2~\to~ \Shiftz~k_1~\to~ \cLet~y=t~\cIn \\
      & \phantom{=}~~ \Throw~k_1~ (\Resetz~ (\Throw~k_2~ y))
\end{align*}

式$e_2$に対して,$t=\cint7$であれば$e_1$は型付け可能である.
一方,$t=x_2$ あるいは $t=x_1$であれば,$e_1$は型付けできない.

\def\proofone{
  \infer
  {\vdash e:\tzero;~\cdot{}}
  {
    \infer{\vdash \cLet~x_1=e_1~\cIn~\Resetz~\cLet~x_2=e_2~\cIn~\cdots :
      \tzero;~\tzero
    }
    {\infer{\longer{\gamma_1}{\gamma_0},~x_1:\tone \vdash
        \Resetz~\cLet~x_2=e_2~\cIn~\cdots : \tone;~\tzero
      }
      {\prooftwo}
    }
  }
}
\def\prooftwo{
  \infer{\longer{\gamma_1}{\gamma_0},~x_1:\tone \vdash
    \cLet~x_2=e_2~\cIn~\cdots : \tone;~\tone,\tzero}
  {\infer{\Gamma_1=\longer{\gamma_2}{\gamma_1},~x_2:\ttwo,~\longer{\gamma_1}{\gamma_0},~x_1:\tone \vdash
      \shiftz{k_2}{\shiftz{k_1}{\cdots}} : \ttwo;~\tone,\tzero
    }
    {\proofthree}
  }
}

\def\proofthree{
  \infer{\Gamma_1,~k_2:\contT{\ttwo}{\tone}{\tzero}
    \vdash \shiftz{k_1}{\cdots} : \tone;~ \tzero
  }
  {\infer{\Gamma_2=\Gamma_1,~k_2:\contT{\ttwo}{\tone}{\tzero},
      ~k_1:\contT{\tone}{\tzero}{\cdot{}}
      \vdash \cLet~ y=t~ \cIn~\cdots : \tzero;~\cdot{}
    }
    {\prooffour}
  }
}

\def\prooffour{
  \infer{\Gamma_3=\Gamma_2,~\longer{\gamma_3}{\gamma_0},~y:\tthree \vdash \throw{k_1}{(\Resetz(\throw{k_2}{y}))} : \tthree;~\cdot{}}
  {\prooffive}
  & \infer[(*)]{\Gamma_2 \vdash t : \tzero;~ \cdot}{\vdots}
}

\def\prooffive{
  \infer
  {\Gamma_3 \vdash \Resetz(\throw{k_2}{y}) : \tonethree; ~\cdot{}}
  {\infer[(\#)]
    {\Gamma_3 \vdash \throw{k_2}{y} : \tonethree;~\tonethree}
    {\infer
      {\Gamma_3 \vdash \throw{k_2}{y} : \tonethree;~\tzero}
      {\Gamma_3 \vdash y :
        \codeT{t}{\gamma_2\uni\gamma_1\uni\gamma_3};~\cdot
        & \infer{\Gamma_3 \models
          \longer{\gamma_1\uni\gamma_3}{\gamma_0}}{}
      }
      & \infer{\Gamma_3 \models \longer{\gamma_1\uni\gamma_3}{\gamma_0}}{}
    }
  }
  & \infer{\Gamma_3 \models \longer{\gamma_3}{\gamma_0}}{}
}

\[
  \proofone
\]

この型付けで注意するところは,複数回の$\Throw$ を使うときは その間に $\Resetz$ を入れなければいけないところである.
$\Resetz$ を入れることで,$\sigma$-part のずれを防ぎ,$\Throw$ 規則を適用できる準備ができる.\\
$(\#)$のところに着目すると,$k_2$ の型は $k_2:\contT{\ttwo}{\tone}{\tzero}$ となっているので,$\Throw$規則を適用するには,$\sigma$-part の subsumption 規則を適用して $\tonethree$ から $\tzero$ が導ければ良い.

次に,$t$ が $x_1, x_2, \cint 7$ の場合に$\Gamma_2 \vdash t : \tzero;~ \cdot$が成り立つかを見ていく.\\
$(*)$ のところに着目すると,
$\Gamma_2 = \longer{\gamma_2}{\gamma_1},~x_2:\ttwo,~\longer{\gamma_1}{\gamma_0},~x_1:\tone,k_2:\contT{\ttwo}{\tone}{\tzero},~k_1:\contT{\tone}{\tzero}{\cdot{}}$
より,
\begin{description}
\item[$t=x_1$の時]\mbox{}\\
  $~\longer{\gamma_1}{\gamma_0},~x_1:\tone$ より,
  $x_1$ のスコープは$\gamma_1$ であり,
  $\gamma_1$スコープのコード変数は,$\gamma_0$スコープでは一般に使用できないので
  $~\longer{\gamma_1}{\gamma_0},~x_1:\tone \vdash x_2:\tzero$ は成り立たず,型がつかない
\item[$t=x_2$の時]\mbox{}\\
  $\longer{\gamma_2}{\gamma_1},~\longer{\gamma_1}{\gamma_0},~x_2:\ttwo$ より,
  $x_2$ のスコープは$\gamma_2$ であり,
  $\gamma_2$スコープのコード変数は,$\gamma_0$スコープでは一般に使用できないので
  $\longer{\gamma_2}{\gamma_1},~\longer{\gamma_1}{\gamma_0},~x_2:\ttwo \vdash x_2:\tzero$ は成り立たず,型がつかない
\item[$t= \cint 7$の時]\mbox{}\\
  $\cint 7$ は定数であるので,どの Classifier $\gamma_i$ においても型が付く.
\end{description}

このように,(少なくとも)上記の例については安全な式と危険な式(Scope extrusionが起こる式)を正しく選別できていることがわかった.

\section{型安全性について}

本研究の型システムに対する型保存(Subject Reduction)定理について述べる.
型保存定理は,(証明できれば)
進行(Progress)定理とあわせて型システムの健全性を導く定理である.

\begin{quote}
  (型保存性)
  $\vdash e:t~;~\sigma$ かつ $e \lto e'$ であれば,$\vdash e':t~;~\sigma$
  である.
\end{quote}

この定理は reset0-shift0の計算規則が多相性を持たない場合には容易に証明
できるが,多相性については精密な扱いが必要であり,
現段階では,型保存定理の証明は進行中である.


% \begin{lemm}[不要な仮定の除去]
%   $\Gamma_1,\gamma_2 \ord \gamma_1 \vdash e : t_1 ~;~\sigma$
%   かつ,$\gamma_2$が $\Gamma_1, e, t_1, \sigma$に出現しないなら,
%   $\Gamma_1 \vdash e : t_1 ~;~\sigma$ である.
% \end{lemm}
%
% \begin{lemm}[値に関する性質]
%   $\Gamma_1 \vdash v : t_1 ~;~\sigma$
%   ならば,
%   $\Gamma_1 \vdash v : t_1 ~;~\sigma'$
%   である.
% \end{lemm}
%
% \begin{lemm}[代入]
%   $\Gamma_1, \Gamma_2, x : t_1 \vdash e : t_2 ~;~\sigma$
%   かつ
%   $\Gamma_1 \vdash v : t_1 ~;~\sigma$
%   ならば,
%   $\Gamma_1, \Gamma_2 \vdash e\{x := v\} : t_2~;~\sigma$
% \end{lemm}
%
% これらをもとに型保存定理を証明する.
% 本研究の対象言語は,コントロールオペレータが操作する対象となる式の型を
% コード型に限定するなど,注意深く設計しているので,ほとんどのケースの証
% 明はスムーズであるが,reset0-shift0 に関する計算規則(shift0 が評価文脈
% を捕捉して継続変数$k$に渡す規則)とthrowに関する計算規則では,
% サブタイプ多相性に相当する性質を使っているので,以下の技術的な補題が必
% 要である.
%
% \begin{lemm}[識別子に関する多相性]
%   穴の周りにreset0を含まない評価文脈$E$,変数$x$,
%   そして$\Gamma = (u_1:t_1)^{\gamma_1}, \cdots, (u_n:t_n)^{\gamma_n}$
%   かつ$i=1,\cdots,n$に対して$\Gamma \models \gamma_0 \ord \gamma_i$であるとする.
%   このとき,
%   $\Gamma, x:\codeT{t_0}{\gamma'} \vdash E[x] : \codeT{t_1}{\gamma_0} ~;~\sigma$
%   であれば,フレッシュな$\gamma$に対して,
%   $\Gamma, x:\codeT{t_0}{\gamma'\cup \gamma} \vdash
%   E[x] : \codeT{t_1}{\gamma_0 \cup \gamma} ~;~\sigma$
%   である.
% \end{lemm}
%
% この補題は,評価文脈$E$に対して,穴の型が$\codeT{t_0}{\gamma'}$で
% 評価文脈全体の型が$\codeT{t_1}{\gamma_0}$であれば,
% それぞれの環境識別子に$\gamma_2$を加えて,
% $\codeT{t_0}{\gamma'\cup \gamma}$型から,
% $\codeT{t_1}{\gamma_0\cup \gamma}$型への評価文脈として使ってもよい,
% ということを主張している.ここで $\gamma \ord \gamma_0$ なので,
% $\gamma$と$\gamma_0 \cup \gamma$は$\ord$の意味で等しくなり,
% $\codeT{t_1}{\gamma_0\cup\gamma}$型を持つ項は,
% $\codeT{t_1}{\gamma}$型も持つことがわかる.
% この定理により,shift0が捕捉した継続を(環境識別子について)多相的に使う
% ことが可能となり,reset0-shift0 の計算規則が正当化される.
%
% 上記の補題を証明すれば,型保存定理の証明の残りのケースは比較的容易であ
% る.なお,この補題を使うケースにおいて,定理の言明にあらわれる項$e$が
% 閉じた項であること(環境識別子に関する
%
%
% 進行定理については
% 精密な定式化が必要(reset0がない式でshift0を実行した時など)が必要なので,

% \section{進行}
% \begin{theo}[進行]
%   $\vdash e:t$ が導出可能であれば,$e$ は 値 $v$ である.または,$e \lto e'$ であるような 項 $e'$ が存在する
% \end{theo}
%
% \paragraph{証明}
% $\vdash e:t$ の導出に関する帰納法による.\\
% Const, Abs, Code 規則の場合 $e$ は値である.\\
% Var 規則の場合 $\vdash e:t$ は導出可能でない.\\
% Throw 規則の場合 $\vdash e:t$ は導出可能でない.\\
% Reset0 規則の場合 $e = \Resetz~ e_1$ とする.
% 帰納法の仮定より評価文脈における $\Resetz E$ より簡約が進み,\\
% $e_1$ が値のとき,$e \lto v$ となるような $v$ が存在する.\\
% $e_1$ が値でないとき,

%%% Local Variables:
%%% mode: japanese-latex
%%% TeX-master: "master_oishi"
%%% End:

\chapter{型推論}
\label{chap:type_inference}

この章では本体系の言語によって書かれたプログラムの型を推論するための型推論アルゴリズムを述べる.

本研究の型推論アルゴリズムは
$\Gamma,~ L,~ \sigma,~ e$ が与えられたとき,$\Gamma \vdash^{L} e : t ;~\sigma$ が成立するような $t$ があるかどうか判定し,その型$t$ を返すものである.

型推論アルゴリズムは主に以下の2ステップから構成する
\begin{itemize}
\item 制約生成:与えられた項に対して,型および classifier に関する制約を返す
\item 制約解消:その得られた制約を解消し,その制約を満たす代入$\Theta$ を返す
\end{itemize}

\section{型システム$T_2$の導入}
制約生成のための形ステム$T_2$を導入する.
これは\ref{chap:type_system}章で与えた型システム$T_1$をトップダウンの制約生成に適した形に変形したものである.
% これは,もとの型システム($T_1$とする)を
% 「トップダウンでの制約生成向け型システム($T_2$とする)」に変形することであたえる.

$T_2$の設計指針は以下のとおりである.
\begin{itemize}
\item $T_1$と$T_2$は「型付けできる」という関係として等価である.
\item $T_2$は結論側の式のトップレベルの形だけで,適用可能な型付け規則が一意に定まる.
  % この性質をterm-orientedと呼ぶ
\item $T_2$は,制約生成をする.
  % (結論側の式の要素は変数として、
  % 「それがこういう形でなければいけない」という条件は、制約の形で「生成」する。)
\end{itemize}

$T_2$の設計にあたって解決すべき問題は subsumption規則である.
すなわち,subsumption 規則 は,どのような項に対しても適用ができるので,
上で述べた一意性が成立しない.
そこで,型付け規則と subsumption 規則 を組み合わせた規則とすることで,問題を解決した.
% subsumption 規則 が出現する場所を限定することができる.

以下で,「var1」等といった表記は,
「型システム$T_1$ の var1規則を subsumption 規則と組み合わせた形に改訂し制約を生成する規則」であるということを表す.
また,型付け規則の右側に記述したConstr;... は(型付け規則を下から上にむけて
使うとき),Constr 以下の制約が生成される,という意味である.

また,型$t_1, t_2$に対する$\longer{}{}$の記号は以下の意味である.

\begin{figure}[H]
  \centering
  \begin{itemize}
  \item $\longer{\codeT{t_1}{\gamma_1}}{\codeT{t_2}{\gamma_2}}$
    は,「$t_1=t_2$ かつ $\longer{\gamma_1}{\gamma_2}$
  \item $\codeT{t}{\gamma}$の形でない$t_1,t_2$に対しては,
    $t_1 = t_2$
  \end{itemize}

  \caption{制約生成における $\longer{}{}$ の意味の定義}
  \label{fig:typeinf_longer_def}
\end{figure}

型システム$T_2$は $\longer{t_1}{t_2}$ の形のまま,制約として生成する.

% 以下のfootnote はもう少しちゃんと書くこと
% \footnote{型推論のプロセスの最中では、$t_1, t_2$ はメタ型変数であり,レベル0の型変数,レベル1の型変数どちらの形かは決定できないので、$\longer{t_1}{t_2}$を上記の意味にしたがって、「ほどく」ことはできない.なので,$\longer{t_1}{t_2}$という形のまま制約として生成する。}.


% (亀山メモ:
% ただ、もしかすると、
% 「レベル0の型変数」と「レベル1の型変数」を最初からわけておく方法もある
% かもしれない。そうすると、上記はとける?)


% \oishi{
% 以下の規則において,Constraint の $\Gamma \models \longer{t}{t'}$ の $\Gamma \models$ は規則を適用した瞬間の $\Gamma$ の情報は制約解消において使わないから必要ない?\\
% 一貫性をもたせるのなら, $t = t'$ の形の Constraint も $\Gamma \models t = t'$ とする必要がある?
% }
%   \\

(var0)

\[
  \infer[Constr;~\Gamma \models \longer{t}{t'}]
  {\Gamma \vdash x:t;~\sigma}
  {(x:t') \in \Gamma
  }
\]

(var1)

\[
  \infer[Constr;~\Gamma \models \longer{\gamma}{\gamma'}]
  {\Gamma \vdash^{\gamma} u:t;~\cdot}
  {(u:t)^{\gamma'} \in \Gamma
  }
\]

(const)

\[
  \infer[Constr;~\Gamma \models \longer{t}{t^c}]
  {\Gamma \vdash^{L} c:t;~\sigma}
  {}
\]

(app0)

\[
  \infer[Constr;~\Gamma \models \longer{t}{t_1}]
  {\Gamma \vdash^\gamma e_1 \, e_2 : t;~\cdot}
  {\Gamma \vdash^\gamma e_1 : \funT{t_2}{t_1}{\sigma};~\cdot
    &\Gamma \vdash^\gamma e_2 : t_2;~\cdot
  }
\]

(app1)

\[
  \infer[Constr;~\Gamma \models \longer{t}{t_1}]
  {\Gamma \vdash e_1 \, e_2 : t;~\sigma}
  {\Gamma \vdash e_1 : t_2 \to t_1;~\sigma
    &\Gamma \vdash e_2 : t_2;~\sigma
  }
\]

(lambda0)

\[
  \infer[Constr;~t=\funT{t_1}{t_2}{\sigma'}]
  {\Gamma \vdash \lambda x.e : t;~\sigma}
  {\Gamma,~x:t_1 \vdash e : t_2;~\sigma'}
\]

(lambda1)
% \oishi{
% コードの中で,shift0/reset0 は使わないので, $\sigma$ は $\epsilon$ なはず.
% なので,$\sigma$ は $\epsilon$ としておいてもよい
% }

\[
  \infer[Constr;~t=\funT{t_1}{t_2}{}]
  {\Gamma \vdash^\gamma \lambda u.e : t;~\cdot}
  {\Gamma,~(u:t_1)^\gamma \vdash^\gamma e : t_2;~\cdot}
\]

(if)

\[
  \infer[Constr;~(none)]
  {\Gamma \vdash^L
    \textbf{if}~e_1 \textbf{then}~e_2 \textbf{else}~e_3 ~:~t; ~ \sigma}
  {\Gamma \vdash^L e_1 : \Bool;~\sigma
    &\Gamma \vdash^L e_2 : t;~\sigma
    &\Gamma \vdash^L e_3 : t;~\sigma
  }
\]

(code-lambda)

\[
  \infer[Constr;~\Gamma \models \longer{t}{\codeT{t_1 \to t_2}{\gamma}}]
  {\Gamma \vdash \underline{\lambda}x.e ~:~t;~\sigma}
  {\Gamma,\longer{\gamma'}{\gamma},x:\codeT{t_1}{\gamma'}
    \vdash e : \codeT{t_2}{\gamma'};~\sigma
  }
\]

(code-let)

\[
  \infer[Constr;~\Gamma \models \longer{t}{\codeT{t'}{\gamma_0}}]
  {\Gamma \vdash \cLet~ x = e_0~ \cIn~ e_1 : t ; \sigma}
  { \Gamma \vdash e_0 : \codeT{t'}{\gamma_0} ; \sigma
    &\Gamma, \longer{\gamma_1}{\gamma_0}, x: \codeT{t'}{\gamma_1} \vdash e_1 : \codeT{t'}{\gamma_1} ; \sigma
  }
\]

(reset0)

\[
  \infer[Constr;~\Gamma \models \longer{t}{\codeT{t'}{\gamma}}]
  {\Gamma \vdash \Resetz~e ~:~ t; ~\sigma}
  {\Gamma \vdash e:\codeT{t'}{\gamma};~\codeT{t'}{\gamma},\sigma
  }
\]

(shift0)

% \[
%   \infer[Constr;~\Gamma \models \longer{t}{\codeT{t_1}{\gamma_1}},~ t_2 =\codeT{t_0}{\gamma_0},~ \Gamma \models \longer{\gamma_1}{\gamma_0}]
%   {\Gamma \vdash \shiftz{k}{e} : t~;~ t_2,\sigma}
%   {\Gamma,~k:\contT{\codeT{t_1}{\gamma_1}}{\codeT{t_0}{\gamma_0}}{\sigma}
%     \vdash e : \codeT{t_0}{\gamma_0} ; \sigma
%   }
% \]

\[
  \infer[Constr;~\Gamma \models \longer{t}{\codeT{t_1}{\gamma_1}},~ \Gamma \models \longer{\gamma_1}{\gamma_0},~ \sigma_x = \codeT{t_0}{\gamma_0}, \sigma]
  {\Gamma \vdash \shiftz{k}{e} : t~;~ \sigma_x}
  {\Gamma,~k:\contT{\codeT{t_1}{\gamma_1}}{\codeT{t_0}{\gamma_0}}{\sigma}
    \vdash e : \codeT{t_0}{\gamma_0} ; \sigma
  }
\]

(throw)
% \oishi{
% throw0 規則にのみ $\sigma$ part の subsumption 規則を適用すればおk
% throw0 は今後一考する必要あり
% }
% \\

\[
  \infer[Constr;~ {\scriptstyle \Gamma \models \longer{t}{\codeT{t_0}{\gamma_2}},~ (\Gamma,~ k:~t') \models \gamma_2 \ord \gamma_0,~  t' = \contT{\codeT{t_1}{\gamma_1}}{\codeT{t_0}{\gamma_0}}{\sigma},~ \Gamma \models \gamma_1 \uni \gamma_2 \ord \gamma'}]
  {\Gamma,~ k:t'
    \vdash \throw{k}{v} : t ; \sigma}
  {\Gamma,~ k:t'
    \vdash v : \codeT{t_1}{\gamma'} ; \sigma
  }
\]

(code)

\[
  \infer[Constr;~\longer{t}{\codeT{t_1}{\gamma}}]
  {\Gamma \vdash \code{e} : t;~\sigma}
  {\Gamma \vdash^\gamma e : t_1;~\cdot}
\]


この新しい型システム$T_2$ は$T_1$と同じ型付けをあたえる.
% \oishi{今のところ$T_1$ の$\sigma$-part の subsumption は reset0 に限定していないので,$T_1$ と$T_2$ は同じ型付けになるはず.}

\section{制約生成}

制約生成では,与えられた項$e$に対して,
$T_2$を型付け規則を下から上の向きに適用することで,制約を生成する.
生成する制約は,それぞれの型付け規則の右側に$Constr;~ ...$ と書いてあるものである.
それらの制約を,型付けに従って生成していくことで,制約の集合が生成される
% (もちろん、途中で、つまってしまったら、型推論は失敗する。)

$T_2$の型付け規則を適用する時,型付け規則の下の型判断に存在せず,上の型判断,あるいは制約中にのみ存在する型やclassifierがあるときは,これらを新しい型変数やclassifier変数として生成する.
なお,code-lambda規則での新しいclassifier は,classfier変数ではなく,classifier定数とする\footnote{制約生成のでは,変数と定数に違いはないが,制約を解消するとき,cassifier定数に対する代入はしない.という違いがある}.

\begin{framed}
  制約生成アルゴリズム:
  \begin{itemize}
    \setlength{\itemsep}{-5pt}
  \item $\Gamma$ : 型文脈
  \item $e$ : 項
  \item $t$ : 型
  \item $\sigma$ : answer type の列
  \item $L$ : レベル(現在レベル 0, または コードレベル $\gamma$)
  \item $C$ : 制約
  \end{itemize}

  \begin{description}
  \item[入力] $\Gamma,~ e,~ t,~ \sigma,~ L$
  \item[出力] $C$
  \end{description}

  % \begin{enumerate}
  % \item $T_2$を「下から上」の向きに適用して制約$C$を生成する.
  % \end{enumerate}
  $\Gamma \vdash^L e:t;~ \sigma$ から始めて,$T_2$ のおける型付けを下から上に向かって行う.
  型付けがどこかで失敗するとき,制約生成は失敗する.
  型付けが成功したとき,生成された制約$C_i$ の和集合を$C$としてそれを返す.
\end{framed}

% \begin{algorithm}
%   \caption{Euclid’s algorithm}
%   \label{euclid}
%   \begin{algorithmic}[1] % The number tells where the line numbering should start
%     \Procedure{Euclid}{$a,b$} \Comment{The g.c.d. of a and b}
%     \State $r \gets a \bmod b$
%     \While{$r\not=0$} \Comment{We have the answer if r is 0}
%     \State $a \gets b$
%     \State $b \gets r$
%     \State $r \gets a \bmod b$
%     \EndWhile\label{euclidendwhile}
%     \State \textbf{return} $b$\Comment{The gcd is b}
%     \EndProcedure
%   \end{algorithmic}
% \end{algorithm}

\section{制約の解消}
% 制約生成が成功したとき,以下の性質が成立するはずである.

入力を$\Gamma, L, e, t,\sigma$ として,前章の制約生成アルゴリズムを走
らせ,それが成功して$C$という制約を生成したとき,
\begin{itemize}
\item
  $T_1$ で $\Gamma \vdash^L e: t; ~ \sigma$ が導出可能ならば,
  $C$ を満たす解が存在し,
\item
  $C$ を満たす解が存在すれば、ある代入$\Theta$に対して,
  $\Theta(\Gamma) \vdash^L \Theta(e): \Theta(t) ;~ \Theta(\sigma)$ が $T_1$で導出可能である.
\end{itemize}
という性質が成立する.

つまり,型推論問題を得られた制約を成立するような代入$\Theta$が存在するかどうかという問題に帰着できる.
% (これが成立すれば、もともとの型推論問題を、型制約の解消問題に帰着できたことになる。)

% というわけで、制約の解消をはじめよう。
制約は以下の文法で与えられたものの有限集合である.

\begin{figure}[H]
  \centering
  \begin{align*}
    \Gamma &\models \longer{t^0}{t^0} \\
    \Gamma &\models \longer{c}{c} \\
    \Gamma &\models \longer{\sigma}{\sigma} \\
    t^0 &= t^0 \\
    t^1 &= t^1
  \end{align*}

  ただし、ここで $t_0$, $t_1$, $c$, $\sigma$ は以下の文法で定義される。

  \begin{align*}
    t^0      & ::= \alpha^0 \mid \Int \mid \Bool \mid \funT{t^0}{t^0}{\sigma} \mid \codeT{t^1}{c} \\
    t^1      & ::= \alpha^1 \mid \Int \mid \Bool \mid \funT{t^1}{t^1}{} \\
    c        & ::= \gamma \mid d \mid c \uni c \\
    \sigma   & ::= \sigma_x \mid \epsilon \mid t^0, \sigma
  \end{align*}

  \caption{制約の定義}
  \label{fig:constr_def}
\end{figure}

$t_0$, $t_1$, $c$, $\sigma$ はそれぞれ、
レベル0型,
レベル1型,
レベル0型の列,
classifierをあらわす表現(メタ変数),
answer type の列($\sigma$-part)
である.また,
$\alpha^i$はレベル$i$の型変数,$\gamma$はclassifier変数,$\sigma_x$は$\sigma$-partの変数である.
また,$d$は,固有変数条件をもつclassifier変数のことであり,
型推論のあいだは,これは実質的に定数として扱われる\footnote{つまり,classifier変数$\gamma$に対しては代入するが,$d$に対しては代入しない}.

また,$\Gamma$は,一般の型文脈であるが,不要な情報を落として以下の形にする.
\begin{align*}
  \Delta ::= \emptyset
  \mid \Delta, (\longer{d}{c})
  \mid \Delta, (x : t)
  \mid \Delta, (u : t)^{\Delta}
\end{align*}

% (左辺は、固有変数なので、classifier定数である。右辺は一般のclassifier
% 式がなんでも来る可能性がある。)

制約の解消とは,制約が与えられたとき,その解となる代入$\Theta$を求めることである.
代入$\Theta$は、型変数$\alpha^0,~ \alpha^1$への型の代入と,
classifier変数$\gamma$へのclassifierの代入とから構成される.

% $\sigma$変数 $\sigma_x$ へ answer type の列の代入もあるはず.$\sigma_x$への代入についてよくわかってない

% この代入は「最も一般的」であるべきである。(定義の詳細はいまは省略)
% principal type

\subsection{typeinf1: 制約の解消アルゴリズム(前半)}
$t^0=s^0$ と $t^1=s^1$ の形の制約は,単一化アルゴリズムによって解くことができる.
% 単一化アルゴリズムについての説明が必要っぽい
それを解いた結果,$\alpha^0$,$\alpha^1$, $\gamma$に対する代入が生じる.
もしくは「解なし」という結果となる.

$\longer{\sigma}{\sigma}$ の形の制約も,単一化アルゴリズムによって解くことができる.
それを説いた結果,$\sigma$変数 $\sigma_x$に対する代入や $\longer{t^0}{t^0}$の形の制約が生成される.
もしくは「解なし」という結果となる.

$\Delta \models \longer{t^0}{s^0}$の形の制約は,両方ともが型変数の場合以外は,
簡単に解ける.(その結果として、$t^i=s^i$ の型の制約や,$\Delta \models
\longer{c}{c}$の形の制約を生む可能性があるが,前者は前と同様に解けばよ
く,前者を解いている間にあらたに$\Delta \models \longer{t^0}{s^0}$の形の制約は生じない.)

ここまでの段階で残る制約は,以下のものだけである.

\begin{itemize}
\item $\Delta \models \longer{\alpha^0}{\beta^0}$
\item $\Delta \models \longer{c_1}{c_2}$
% \item $\Delta \models \longer{\sigma_1}{\sigma_2}$
\end{itemize}

ここまでに出てきた代入はすべて,上記の制約に適用済みとする.
つまり,$\alpha:=\Int$という代入がでてきたら,制約中の$\alpha$はすべて$\Int$にしておく.
その結果,「代入における左辺にでてくる型変数やclassifier変数」は,上記の制約には,残っていない.

\begin{oframed}
  制約解消アルゴリズム(前半) unify1
  \begin{itemize}
    \setlength{\itemsep}{-5pt}
  \item $C$ : 制約
  \item $\Theta$ : 代入
  \end{itemize}

  \begin{description}
  \item[入力] $C$, $\Theta$
  \item[出力] $\Theta'$ または,「単一化失敗」
    ここで,$\Theta'$ は$\Theta$に$C$を解いて得られる代入を追加したもの
  \end{description}

  % unify1($C, []$) を呼び出す
  \begin{description}           % \oishi{% $longer{\sigma}{\sigma}$ の形についてはあとで考える}
  \item[1] $C$の中に$t^0=s^0$か$t^1=s^1$か$\Delta \models \longer{t^0}{s^0}$の形の制約がなければ,代入$\Theta$ を返す.
  \item[2] $C$から$t^0=s^0$ か $t^1=s^1$の形の制約を選び,それを$A = B$とする.$C_1 = C - \{A = B\}$ とする.
    \begin{description}
    \item[2-1] $A = B$ のとき,unify1($C_1, \Theta$)を呼び出す.
    \item[2-2] $A \neq B$ で $A$ が型変数のとき,
      \begin{description}
      \item[2-2-1] 型 $B$ に $A$ が現れるなら,「単一化失敗」を返す.
      \item[2-2-2] 型 $B$ に $A$ が現れないなら,$\Theta_1 = [A := B]$とし,unify1($\Theta_1(C_1), \Theta(\Theta_1))$を呼び出す.
      \end{description}
    \item[2-3] $A \neq B$ で $B$ が型変数のとき,$A$ と $B$ を入れ替えて,2-2へ
    \item[2-4] $A \neq B$ で $A = \funT{A_1}{A_2}{}$,$B = \funT{B_1}{B_2}{}$ のとき, $C_2 = C_1 \cup \{A_1 = B_1, A_2 = B_2\}$ とし, unify1($C_2, \Theta$)を呼び出す
    \item[2-5] $A \neq B$ で $A = \funT{A_1}{A_2}{\sigma_1}$,$B = \funT{B_1}{B_2}{\sigma_2}$ のとき, $C_2 = C_1 \cup \{A_1 = B_1, A_2 = B_2, \sigma_1 = \sigma_2\}$ とし, unify1($C_2, \Theta$)を呼び出す
    \item[2-6] $A \neq B$ で $A = \codeT{A_1}{c_1} , B = \codeT{B_1}{c_2}$ のとき,$C_2 = C_1 \cup \{A_1 = B_1, c_1 = c_2\}$ とし, unify1($C_2, \Theta$)を呼び出す
    \item[2-7] $A \neq B$ で ともに$\sigma$変数のときつまり,$A = \sigma_x , B = \sigma_y$ のとき,その制約は残す.
    \item[2-8] $A \neq B$ で $A = \sigma_x , B = \cdot$ のとき,$C_2 = C_1 \cup \{\sigma_x = \cdot\}$ とし, unify1($C_2, \Theta$)を呼び出す
    \item[2-9] $A \neq B$ で $A = \sigma_x , B = t_1, \sigma_2$ のとき,$C_2 = C_1 \cup \{\sigma_x = (t_z,\sigma_2),~ \longer{t_z}{t_1} \}$ とし, unify1($C_2, \Theta$)を呼び出す
    \item[2-10] $A \neq B$ で $A = (t_1, \sigma_1) , B = (t_2, \sigma_2)$ のとき,$C_2 = C_1 \cup \{\longer{t_1}{t_2}, \sigma_1 = \sigma_2 \}$ とし, unify1($C_2, \Theta$)を呼び出す
    \item[2-11] 上記のいずれでもないとき「単一化失敗」を返す.
    \end{description}
  \item[1-3] $C$から$\Delta \models \longer{t^0}{s^0}$ の形の制約を選び,それを$c$とする.$C_1 = C - \{c\}$ とする.
    \begin{description}
    \item[1-3-1] $t^0,s^0$のどちらかが型変数でないとき,詳細は省くが,代入$\theta$ が生成される;
      unify1($C_1, \Theta(\theta)$)を呼び出す
    \item[1-3-2] $t^0,s^0$がともに型変数のときつまり,$c$ が $\Delta \models \longer{\alpha^0}{\beta^0}$のとき,その制約を残す
    \end{description}
  % \item[1-4] $c$ が $\Delta \models \longer{c}{c}$ の形のとき,unify2($\Theta, C$)を呼び出す
  \end{description}
\end{oframed}


% \begin{oframed}
%   制約解消アルゴリズム(前半) typeinf1
%   \begin{itemize}
%     \setlength{\itemsep}{-5pt}
%   \item $C$ : 制約
%   \item $\Theta$ : 代入
%   \end{itemize}

%   \begin{description}
%   \item[入力] $C$ $\Theta$
%   \item[出力] $C$ $\Theta$
%   \end{description}

%   \begin{description}
%   \item[0] typeinf1($C, []$) を呼び出す
%   \item[1] $C$から任意の制約$c$を選び, $C_1 = C - \{c\}$ とする.
%     \begin{description}
%     \item[1-1] $c$ が $t^0=s^0$ の形のとき,
%       $\Theta$ に unify($t^0, s^0$)を加える (unify によって$\alpha^0,~ \alpha^1,~ \gamma$ に対する代入が生じる);
%       typeinf1($C_1, \Theta$) を呼び出す
%     \item[1-2] $c$ が $t^1=s^1$ の形のとき,
%       $\Theta$ に unify($t^1, s^1$)を加える (unify によって$\alpha^0,~ \alpha^1,~ \gamma$ に対する代入が生じる);
%       typeinf1($C_1, \Theta$) を呼び出す
%     \item[1-3] $c$ が $\Delta \models \longer{t^0}{s^0}$ の形のとき,
%       \begin{description}
%       \item[1-3-1] $t^0,s^0$のどちらかが型変数でないとき,詳細は省くが,制約$\theta$ が生成され,それを$\Theta$ に加える;
%         typeinf1($C_1, \Theta$)を呼び出す
%       \item[1-3-2] $t^0,s^0$がともに型変数のときつまり,$c$ が $\Delta \models \longer{\alpha^0}{\beta^0}$のとき,typeinf2($\Theta, C$)を呼び出す
%       \end{description}
%     \item[1-4] $c$ が $\Delta \models \longer{c}{c}$ の形のとき,typeinf2($\Theta, C$)を呼び出す
%     \end{description}
%   \end{description}
% \end{oframed}

% \begin{algorithm}
%   \caption{制約解消アルゴリズム(前半)}

%   \begin{algorithmic}[1] % The number tells where the line numbering should start
%     \Procedure{infertype1}{$C, \Theta = []$}
%     \ForEach{$c \in C$}
%     \State $C_1 \gets C - \{c\}$
%     \If{$c$ が $t^0=s^0$ の形}
%     \State $\Theta$ に unify($t^0, s^0$)を加える \Comment{unify によって$\alpha^0,~ \alpha^1,~ \gamma$ に対する代入が生じる}
%     \State infertype1($C_1, \Theta$)
%     \ElsIf{$c$ が $t^1=s^1$の形}
%     \State $\Theta$ に unify($t^1, s^1$)を加える \Comment{unify によって$\alpha^0,~ \alpha^1,~ \gamma$ に対する代入が生じる}
%     \State infertype1($C_1, \Theta$)
%     \ElsIf{$c$ が $ \Delta \models \longer{t^0}{s^0}$ の形}
%     \If{$t^0$と$s^0$ が型変数でない}
%     \State
%     \Else
%     \State typeinfer2($C, \Theta$) \Comment{$c$が$\longer{\alpha^0}{\beta^0}$の形}
%     \EndIf
%     \Else
%     \State typeinfer2($C, \Theta$) \Comment{$c$が$\longer{c}{c}$の形}
%     \EndIf
%     \EndFor
%     \EndProcedure
%   \end{algorithmic}
% \end{algorithm}

\subsection{typeinf2: 制約の解消アルゴリズム(後半)}
ここまでの段階で残る制約は,上記で述べたように以下のものだけである.
\begin{itemize}
\item $\Delta \models \longer{\alpha^0}{\beta^0}$
\item $\Delta \models \longer{c_1}{c_2}$
% \item $\Delta \models \longer{\sigma_1}{\sigma_2}$
\end{itemize}

まず,$\Delta \models \longer{c}{c}$の形の制約の解消について述べる

この形の制約たちを,
$\Delta_i \models \longer{c_i}{c'_i}$とすると,
それぞれの$\Delta_i$ は両立的であるので,
$\Delta = \Delta_1 \cup \cdots \cup \Delta_n$として,
$\Delta \models \longer{c_i}{c'_i}$を解けばよい.
% それぞれの $\Delta_i$ はcompatible であるはずなので(ここはあとでチェッ
% クが必要)
% $\Delta = \Delta_1 \cup \cdots \cup \Delta_n$ という風に全部を合体させ
% た上で、$\Delta \models \longer{c_i}{c'_i}$を解けばよい。

\begin{description}
\item[(ステップ1: classifier変数の除去)]\mbox{}\\
  制約中のclassifier変数の1つに着目し,それを$\gamma$とする.
  $\longer{c_i}{\gamma}$ の形の制約 ($i=1,2,\cdots,I$)と
  $\longer{\gamma}{c'_j}$ の形の制約 ($j=1,2,\cdots,J$)をすべて消去し,
  以下の制約を,すべての$(i,j)$に対して追加する.

  \[
    \longer{c_i}{c'_j}
  \]

  これにより,classifier変数は1つ減る.% (制約は一般には増えるかもしれない。)
  したがって,ステップ1を繰り返すと,classifier変数はなくなる.

\item[(ステップ2: 右辺の$\uni$の除去)]\mbox{}\\
  $\longer{c_1}{c_2 \uni c_3}$ を
  $\longer{c_1}{c_2}$ と
  $\longer{c_1}{c_3}$ に変換する.

  これにより,不等号の右辺にある$\uni$の個数が1つ減る.
  したがって,ステップ2を繰り返すと,不等号の右辺にある$\uni$はなくなる.

  ステップ2の繰り返しがおわると,制約は,
  $\Delta \models \longer{c}{d}$の形になる.


  % \oishi{ステップ3 は変更する必要あり ここで「$d$は atomic」 という仮定をおく。これについてはあとで吟味する.}\\
  % \oishi{以下は 定理 として証明が必要}
\item[(ステップ3: 左辺の分解)] \mbox{}\\
  $d$は 定数 とすると,
  「$\longer{c_1 \uni c_2}{d}$ ならば
  $\longer{c_1}{d}$または,
  $\longer{c_2}{d}$」ということが言える.

  これを用いて、$\longer{c}{d}$の左辺を分解することができ,

  \[
    \Delta \models \longer{d_1}{d_1'} \vee \cdots \vee \longer{d_n}{d'_n}
  \]

  となる.
  さらに$\Delta$も $\longer{d1}{d2}$の形を「かつ」と「または」で
  つないだ形になる.

  % これは decidable なので、「解があるかどうか」もdecidableである。
\end{description}

$\Delta \models \longer{\alpha^0}{\beta^0}$の形の制約は残す.
% 次に$\Delta \models \longer{\alpha^0}{\beta^0}$の形の制約の解消について述べる.
% この制約は以下の2つの可能性がある.

% \begin{itemize}
% \item $\alpha^0 = \beta^0$
% \item
%   $\alpha^0 = \codeT{t}{\gamma1}$,
%   $\beta^0 = \codeT{t}{\gamma2}$,
%   $\Delta \models \longer{\gamma1}{\gamma2}$
% \end{itemize}

\begin{oframed}
  制約解消アルゴリズム(後半) unify2
  \begin{itemize}
    \setlength{\itemsep}{-5pt}
  \item $C$ : 制約
  \item $\Theta$ : 代入
  \end{itemize}

  \begin{description}
  \item[入力] $C$, $\Theta$ ($\Theta$ には $\Delta \models \longer{\alpha^0}{\beta^0}$, $\Delta \models \longer{c}{c}$ の形の制約のみがある)
  \item[出力] $C$, $\Theta$ ($\Delta \models \longer{\alpha^0}{\beta^0}$の形の制約があればそれも返す)
  \end{description}

  \begin{description}
  \item[1] $C$から任意の制約$c$を選び, $C' = C - \{c\}$ とする.
    \begin{description}
    \item[1-1] $c$ が $\Delta \models \longer{c}{c}$の形のとき,
      \begin{description}
        \item[1-1-1: ステップ1] classifier 変数 $\gamma$ を選び,$\longer{c_i}{\gamma}$ と$\longer{\gamma}{c'_j}$ をすべて消去し,制約に$\longer{c_i}{c'_j}$ を加える.すべての classifier 変数がなくなるまでステップ1を繰り返し,ステップ2へ
        \item[1-1-2: ステップ2] $\longer{c_1}{c_2 \uni c_3}$ を$\longer{c_1}{c_2}$ と $\longer{c_1}{c_3}$ に変換する.右辺に$\uni$ がなくなるまで,ステップ2を繰り返すことによって,制約は, $\longer{c}{d}$となる.ステップ3へ
        \item[1-1-3: ステップ3] $\longer{c_1 \uni c_2}{d}$ を $\longer{c_1}{d} \vee \longer{c_2}{d}$ に変換する.その制約を$\Theta$に加え,unify2($C', \Theta$)を呼び出す
      \end{description}
    % \item[1-2] $c$ が $\Delta \models \longer{\alpha^0}{\beta^0}$の形のとき,
    %   $C'_1$($C'$ に$\alpha^0 = \beta^0$を加えたもの),\\
    %   $C'_2$($C'$ に$\alpha^0 = \codeT{t}{\gamma1},~ \beta^0 = \codeT{t}{\gamma2},~ \Delta \models \longer{\gamma1}{\gamma2}$を加えたもの)とする.\\
    %   unify2($C'_1, \Theta_1$)とunify2($C'_2, \Theta_2$)とを呼び出す.
    \end{description}
    \item[1-2] 解として,$C, \Theta$と $\Delta \models \longer{\alpha^0}{\beta^0}$の形の制約を返す.
  \end{description}
\end{oframed}

% \oishi{ 固有変数d の条件のチェックが必要. Γの中の d たちの順序 と 制約で出てきた順序についてのチェック }

% \subsection{制約の解消アルゴリズム(全体)}

% \begin{framed}
%   制約解消アルゴリズム(全体) typeinf
%   \begin{itemize}
%     \setlength{\itemsep}{-5pt}
%   \item $C$ : 制約
%   \item $\Theta$ : 代入
%   \end{itemize}

%   \begin{description}
%   \item[入力] $\Gamma,~ e,~ t,~ \sigma,~ L$
%   \item[出力] $C$
%   \end{description}

%   \begin{enumerate}
%   \item $T_2$を「下から上」の向きに適用して制約$C$を生成する.
%   \end{enumerate}
% \end{framed}

% 以上によって,制約を解消することで,$\Gamma,~ L,~ \sigma,~ e$ が与えられたとき,$\Gamma \vdash^{L} e : t ;~\sigma$ が成立するような $t$ があるかどうか判定し,その型$t$ のprincipal type が分かる

% \section{型推論アルゴリズム}

%%% Local Variables:
%%% mode: japanese-latex
%%% TeX-master: "master_oishi"
%%% End:

% % \section{実装}

%%% Local Variables:
%%% mode: latex
%%% TeX-master: "slide_oishi"
%%% End:

% \input{proof}
\chapter{関連研究}
表現力と安全性を兼ね備えたコード生成の体系としては,
2009年のKameyamaらの研究\cite{Kameyama2009}が最初である.
彼らは,MetaOCamlにおいてshift/resetとよばれるコントロールオペレータを
使うスタイルでのプログラミングを提案するとともに,
コントロールオペレータの影響が変数スコープを越えることを制限する型シス
テムを構築し,安全性を厳密に保証した.

Westbrookら\cite{Westbrook}は同様の研究を Java のサブセットを対象におこなった.
須藤ら\cite{Sudo2014}は,書換え可能変数を持つコード生成体系に対して,
部分型付けを導入した型システムを提案して,安全性を保証した.
これらの体系は,安全性の保証を最優先した結果,表現力の上での制限が強く
なっている.特に,let挿入とよばれるコード生成技法をシミュレートするた
めには,shift/reset が必要であるが,複数の場所へのlet挿入を許すために
は,複数の種類のshift/resetを組み合わせる必要がある.
この目的のため,階層的shift/resetやマルチプロンプト
shift/resetといった,shift/reset を複雑にしたコントロールオペレータを
考えることができるが,その場合の型システムは非常に複雑になることが予想
され,安全性を保証するための条件も容易には記述できない,等の問題点がある.

本研究では,このような問題点を克服するため,shift/reset の意味論をわず
かに変更した shift0/reset0 というコントロールオペレータに着目する.
このコントロールオペレータは,長い間,研究対象となってこなかったが
2011年以降,Materzok らは,部分型付けに基づく型システムや,
関数的なCPS変換を与えるなど,簡潔で拡張が容易な理論的基盤をもつことを
解明した\cite{Materzok2011,materzok2012}.
特に,shift0/reset0 は shift/reset と同様のコントロールオペレータであ
りながら,階層的shift/reset を表現することができる,という点で,
表現力が高い.本研究では,これらの事実に基づき,これまでのshift/reset
を用いたコード生成体系の知見を,shift0/reset0 を用いたコード生成体系の
構築に活用するものである.


% \oishi{let-insertion の参考論文 : Olivier Danvy 最初って言ってるやつ,
%   MetaOcaml : Gentle to introduction part1, part2 Taha,
%   bermeta Flops : oleg,
%   lightweight-module scala : romps stagingのやつ
%   を参考文献に加える
% }

%%% Local Variables:
%%% mode: japanese-latex
%%% TeX-master: "master_oishi"
%%% End:

\chapter{まとめと今後の課題}
本研究では,効率的コード生成に有用な技法であるlet挿入を,型安全に実現す
るための言語と型システムについて述べた.
局所的な代入可能変数を持つ体系に対する須藤らの研究\cite{Sudo2014}などに基づき,
多段階のforループを飛び越えたlet挿入を実現するために,shift0/reset0 を持つコード生成体系を設計した.
須藤らの研究で精密化された環境識別子(Environment Classifier)に join ($\cup$) を導入することで,
計算の順序を変更するようなコントロールオペレータ(shift0/reset0)を扱えるようにし,
安全に多段階の let挿入を行えるように型システムを構築した.
このようなlet挿入が束縛子を越えるケースは,ループにおける不変式の括り出
しなどの有用な最適化を含むが,これまでの研究では一般的なlet挿入を安全
に実現した体系の提案はなく,我々の知る限り本研究がはじめてである.

今後の課題として,まずあげられるのは,進行(Progress)の性質および型推論
アルゴリズムの実装の完成である.また,理論的には Kiselyovらのグローバルな参
照を持つ体系との融合が可能になれば,広い範囲のコード生成技法・最適化技法
をカバーできるため極めて有用である.
また,既存のMetaOCaml との比較においては,
2レベルのみのコード生成に限定している点や
run (生成したコードの実行)や cross-stage persistence (現在ス
テージの値をコードに埋め込む機能)などに対応していない点が欠点であり,
これらの拡張が可能であるかどうかの検討は非常に興味深い将来課題である.

2017年1月に,プログラミング言語 MetaOCaml の最新版である BER MetaOCaml が バージョンアップされ,コード生成時に自動的に let 挿入を行う genlet という式が導入された.
genlet は,動的に let 挿入を行う点,また,挿入先の選択において変数束縛の安全性を考慮している点,さらに,安全性が保たれる範囲でなるべくトップレベルに近い場所を探す点が特徴であり,人間が手動で let 挿入を行う場所を決めるより,手間が少なく,安全性を考慮しなくてよいなど,優れている点がある.
一方で, if 式や match 式 など条件分岐を行う式の生成では,let 挿入を if 式の外側まで持ち上げるべきか, if 式の内側でとどめておくべきかは,自動的に決まる問題ではなく,プログラマが選択する必要がある.そのような場合は,本研究で述べた通り, shift0/reset0 等のコントロールオペレータを用いて,プログラマが明示的に let挿入の位置を指定することになり,その安全性を保証する本研究が必要となる.
genlet は非常に有用なオペレータであるが,それ自身一種のコントロールオペレータであり,
本研究で述べた手動での let 挿入を行うためのコントロールオペレータと genlet をあわせもつコード生成器の安全性は,新たな将来課題である.

%%% Local Variables:
%%% mode: japanese-latex
%%% TeX-master: "master_oishi"
%%% End:

\chapter*{謝辞}
\addcontentsline{toc}{chapter}{\numberline{}謝辞}
本研究に関して,終始ご指導ご鞭撻を頂きました亀山幸義先生に深く感謝いたします.
また,研究に対しての発表の仕方など有益な助言を下さった海野広志先生に感謝いたします.
最後に,研究に関して様々な議論をして下さった薄井千春君に,
日頃研究を様々な形でサポートして頂きましたプログラム論理研究室の皆様に感謝いたします.
\newpage

\addcontentsline{toc}{chapter}{\numberline{}参考文献}
\renewcommand{\bibname}{参考文献}
\bibliographystyle{junsrt}
\nocite{*}
\bibliography {bibfile}

\appendix
\chapter{シンタックス}
\small
\verbatiminput{../src/metaS0/syntax_tidy.ml}

\chapter{評価器}
\verbatiminput{../src/metaS0/metalamS0_tidy.ml}

\chapter{型推論器}
\verbatiminput{../src/metaS0/typeinf_tidy.ml}

%%% Local Variables:
%%% mode: japanese-latex
%%% TeX-master: "master_oishi"
%%% End:


\end{document}

%%% Local Variables:
%%% mode: japanese-latex
%%% TeX-master: t
%%% End:
