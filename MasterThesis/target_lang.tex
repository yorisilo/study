\chapter{対象言語: 構文と意味論}

本研究における対象言語は,ラムダ計算にコード生成機能とコントロールオペ
レータshift0/reset0を追加したものに型システムを導入したものである.

本稿では,最小限の言語のみについて考えるため,コード生成機能の
「ステージ(段階)」は,コード生成段階(レベル0,現在ステージ)と
生成されたコードの実行段階(レベル1,将来ステージ)の2ステージのみを考える.

前述したように,本研究の言語では,
コードコンビネータ(Code Combinator)方式を使い,
コードコンビネータは,
$\cPlus$ や $\cIf$のように下線を引いて表す.

\section{構文の定義}

対象言語の構文を定義する.

変数は,レベル0変数($x$), レベル1変数($u$),
(レベル0の)継続変数($k$)の3種類がある.
レベル0項($e^0$),レベル1項($e^1$)およびレベル0の値($v$)を下の通り定義する.

\begin{figure}[!ht]
  \centering
  \begin{align*}
    c & ::= i \mid b \mid \cint
        \mid \cat \mid + \mid \cPlus \mid \cIf \\
    v & ::= x \mid c \mid \fun{x}{e^0} \mid \code{e^1} \\
    e^0 & ::=  v  \mid e^0~ e^0 \mid \ift{e^0}{e^0}{e^0} \\
      & \mid \cfun{x}{e^0}
        \mid \ccfun{u}{e^0} \\
      & \mid \resetz{e^0}
        \mid \shiftz{k}{e^0}
        \mid \throw{k}{v} \\
    e^1 & ::=  u \mid c \mid \fun{u}{e^1} \mid e^1~ e^1
          \mid \ift{e^1}{e^1}{e^1} \\
  \end{align*}
  \caption{対象言語の構文の定義}
\end{figure}

ここで$i$は整数の定数,$b$は真理値定数である.

定数のうち,下線がついているものはコードコンビネータである.
変数は,ラムダ抽象(下線なし,下線つき,二重下線つき)および shift0 により束縛され,
$\alpha$同値な項は同一視する.
$\letin{x}{e_1}{e_2}$および$\clet{x}{e_1}{e_2}$は,
それぞれ,$(\fun{x}{e_2})e_1$ および $(\cfun{x}{e_2})\cat e_1$の省略形である.
前述の例でのべた$\cFor$は,
コード構築定数とコードレベル適用を用いて導入することとし,
(この導入にあたっての型システムの拡張は容易なので)ここでは省略する.

\section{操作的意味論}

対象言語は,値呼びで left-to-rightの操作的意味論を持つ.
ここでは評価文脈に基づく定義を与える.

評価文脈を以下のように定義する.
\begin{figure}[H]
  \centering
  \begin{align*}
    E & ::= [~] \mid E~ e^0 \mid v~ E \\
      & \mid \ift{E}{e^0}{e^0} \mid \Resetz~ E \mid \ccfun{u}{E}
  \end{align*}
  \caption{評価文脈}
\end{figure}

コード生成言語で特徴的なことは,
コードレベルのラムダ抽象の内部で評価が進行する点である.実際,
上記の定義には,$\ccfun{u}{E}$が含まれている.
たとえば,$\ccfun{u}{\code{u} \cPlus [~]}$ は評価文脈である.

この評価文脈$E$と次に述べる計算規則$r \to l$ により,
評価関係$e \lto e'$ を図\ref{fig:etoe}のように定義する.

\begin{figure}[H]
  \centering
  \[
    \infer{E[r] \lto E[l]}{r \to l}
  \]
  \caption{$e \lto e' $の評価関係}
  \label{fig:etoe}
\end{figure}


計算規則は図\ref{fig:calc_rule}の通り定義する.
\begin{figure}[H]
  \centering
  \begin{align*}
    (\fun{x}{e})~v &\to e\{ x := v \} \\
    \ift{true}{e_1}{e_2} &\to e_1 \\
    \ift{else}{e_1}{e_2} &\to e_2 \\
    \cfun{x}{e} &\to \ccfun{u}{(e\{ x := \code{u} \})} \\
    \ccfun{u}{\code{e}} &\to \code{\fun{u}{e}} \\
    \Resetz~ v &\to v \\
    \Resetz (E[\Shiftz~ k \to e]) &\to e \ksubst{k}{E}
  \end{align*}
  \caption{計算規則}
  \label{fig:calc_rule}
\end{figure}

ただし,4行目の$u$はフレッシュなコードレベル変数とし,
最後の行の$E$は穴の周りに{\Resetz}を含まない評価文脈とする.
また,この行の右辺のトップレベルに{\Resetz}がない点が,
shift/reset の振舞いとの違いである.すなわち,shift0 を1回計算すると,
reset0 が1つはずれるため,shift0 をN個入れ子にすることにより,
N個分外側のreset0 までアクセスすることができ,多段階let挿入を実現でき
るようになる.

上記における継続変数に対する代入$e\ksubst{k}{E}$は図\ref{fig:k_subst}の通り定義する.

\begin{figure}[H]
  \centering
  \begin{align*}
    (\throw{k}{v})\ksubst{k}{E} &\equiv \Resetz (E[v]) \\
    (\throw{k'}{v})\ksubst{k}{E} &\equiv \throw{k'}{(v\ksubst{k}{E})}
    \\
                                & \text{ただし}~k \not= k'
  \end{align*}
  \caption{継続への代入}
  \label{fig:k_subst}
\end{figure}

上記以外の$e$に対する代入の定義は透過的であるとする.
上記の定義の1行目で\Resetz を挿入しているのは{\Shiftz}の意味論に対応し
ており,これを挿入しない場合は別のコントロールオペレータ(Felleisenの
control/promptに類似した control0/prompt0)の振舞いとなる.

コードコンビネータ定数の振舞い(ラムダ計算における$\delta$規則に相当)は
図\ref{fig:comb-rule}のように定義する.

\begin{figure}[H]
  \centering
  \begin{align*}
    \cint~ n &\to \code{n} \\
    \code{e_1}~ \cat~ \code{e_2} &\to \code{e_1~ e_2} \\
    \code{e_1}~ \cPlus~ \code{e_2} &\to \code{e_1 + e_2} \\
    \cif{e_1}{e_2}{e_3} &\to \code{\ift{e_1}{e_2}{e_3}}
  \end{align*}
  \caption{コードコンビネータの規則}
  \label{fig:comb-rule}
\end{figure}

% 計算の例を以下に示す.
% \begin{align*}
    %     e_1 & = \Resetz ~~\cLet~x_1=\csp{3}~\cIn \\
    %               & \phantom{=}~~ \Resetz ~~\cLet~x_2=\csp{5}~\cIn \\
    %               & \phantom{=}~~ \Shiftz~k~\to~\cLet~y=t~\cIn \\
    %               & \phantom{=}~~ \Throw~ k~ (x_1~\cPlus~x_2~\cPlus~y) \\
    %   \end{align*}
    %
    %     \begin{align*}
    %     [ e_1 ] &\lto [ \Resetz (\cLet~x_1=\csp{3}~\cIn \\
    %               &\Resetz~ \cLet~x_2=\csp{5}~\cIn \\
    %               &[ \Shiftz~ k~ \to~ \cLet~ y=t~ \cIn \\
    %               &[ \Throw~ k~(x_1~\cPlus~x_2~\cPlus~y) ] ] ) ] \\
    %               &\lto [ \cLet~ y=t~ \cIn \\
    %               &[ \cfun{x}{\Resetz~ (\cLet~x_1=\csp{3}~ \cIn~ \Resetz~ (\cLet~ x_2=\csp{5}~ \cIn [x]))} (x_1~\cPlus~x_2~\cPlus~y) ]] \\
    %               &\lto [ \cfun{y}{(\cfun{x}{\Resetz~ (\cLet~x_1=\csp{3}~ \cIn~ \Resetz~ (\cLet~ x_2=\csp{5}~ \cIn [x]))} (x_1~\cPlus~x_2~\cPlus~y))}~ \cat~ t ] \\
    %               &\lto [[\cfun{y}{(\cfun{x}{\Resetz~ (\cLet~x_1=\csp{3}~ \cIn~ \Resetz~ (\cLet~ x_2=\csp{5}~ \cIn [x]))} (x_1~\cPlus~x_2~\cPlus~y))}]~ \cat~ t] \\
    %               &\lto [[\ccfun{y_1}{(\cfun{x}{\Resetz~ (\cLet~x_1=\csp{3}~ \cIn~ \Resetz~ (\cLet~ x_2=\csp{5}~ \cIn [x]))} (x_1~\cPlus~x_2~\cPlus~ \code{y_1}))}]~ \cat~ t] \\
    %               &\lto
                      %   \end{align*}

                      %%%   Local Variables:
                      %%%   mode: japanese-latex
                      %%%   TeX-master: "master_oishi"
                      %%%   End:
