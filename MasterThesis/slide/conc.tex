\section{まとめと今後の課題}

\begin{frame}
  \frametitle{まとめと今後の課題}
  まとめ
  \begin{itemize}
    % \item コードの言語にshift0 reset0 を組み込んだ言語の設計を行った
    % \item コード生成言語の型システムに shift0/reset0 を組み込んだ 型システムの設計を完成させた
  \item コード生成言語にコード移動を許す仕組み(shift0/reset0)を導入し,その安全性を保証するための型システムの設計を行い
    \begin{itemize}
    \item 安全性:Scope extrusion が起きないようにする
    \end{itemize}
  \item 型推論アルゴリズムの開発を行った
  \end{itemize}

  % \vspace{1in}
  \vspace{\baselineskip}

  今後の課題
  \begin{itemize}
    % \item answer type modification に対応した型システムを設計し,(subject reduction 等の)健全性の証明を行う
  \item 設計した型システムの健全性の証明(Subject reduction)
  \item 型推論アルゴリズム(制約解消)の実装
  \item 言語の拡張
    \begin{itemize}
    \item グローバルな参照 (OCamlのref)
    \item 生成したコードの実行 (MetaOCamlの run)
    \end{itemize}
  \end{itemize}
\end{frame}

%%% Local Variables:
%%% mode: latex
%%% TeX-master: "slide_oishi"
%%% End:
