% 先の解決方法で実際になにができるのかの説明

\subsection{型付けの例}

\newcommand\boxterm{\framebox{
    \only<2>{\green{$\cint{3}$}}
    \only<3>{\red{$x~\cPlus~(\cint{3})$}}
    \phantom{A}
  }}

\begin{frame}
  \frametitle{型付けの例(1)}
  \begin{align*}
    e & = \Resetz ~~(\cfordo{x = e1}{e2} \\
      & \phantom{=}~~~ \Shiftz~k~\to~
        {\cLet~u=\green{\boxterm}~\cIn}~\Throw~k~u)
  \end{align*}

  \[
    \infer{\vdash e : \codeTs{t}{\gamma0};~\epsilon}
    {\infer[\blue{(\gamma1^*)}]{\vdash \cfor{x=...} :
        \codeTs{t}{\gamma0};~\codeTs{t}{\gamma0}}
      {\infer{\gamma1 \ord \gamma0,~x:\codeTs{t}{\gamma1}
          \vdash \Shiftz~k~\to~... :
          \codeTs{t}{\gamma1};~\codeTs{t}{\gamma0}}
        {\infer[(\gamma2^*)]{\Gamma a \vdash \cLet~u=...:\codeTs{t}{\gamma0};~\epsilon}
          {\infer{\Gamma b \vdash
              \Throw~k~u:\codeTs{t}{\gamma2};~\epsilon }
            {\infer{\Gamma b \vdash
                u:\codeTs{t}{\gamma1 \cup \gamma2};~ \sigma}
              {}
            }
            &\infer*{\Gamma a \vdash
              \green{\boxterm}:\codeTs{t}{\red{\gamma0}};~\epsilon}
            {}
          }
        }
      }
    }
  \]

  % $[(\gamma1^*)]$ は eigen variable (固有変数)

  {\footnotesize
    \begin{align*}
      \Gamma a &= \gamma1 \ord \gamma0,~x:\codeTs{t}{\red{\gamma1}},
                 ~k: \contT{\codeTs{t}{\gamma1}}{\codeTs{t}{\gamma0}} \\
      \Gamma b &= \Gamma a,~\gamma2 \ord \gamma0,~u:\codeTs{t}{\gamma2}
    \end{align*}
  }

\end{frame}

\begin{frame}
  \frametitle{型付けの例(2)}

  \newcommand\gammaa{\gamma1 \ord \gamma0,~x:\codeTs{t}{\gamma1}}
  \newcommand\gammab{\Gamma a,\gamma2 \ord \gamma1,~y:\codeTs{t}{\gamma2}}
  \newcommand\gammac{\Gamma b,\blue{k_2}:\contT{\codeTs{t}{\gamma2}}{\codeTs{t}{\gamma1}}}
  \newcommand\gammad{\Gamma c,\red{k_1}:\contT{\codeTs{t}{\gamma1}}{\codeTs{t}{\gamma0}}}
  \newcommand\gammae{\Gamma d,\gamma3 \ord \gamma0,\magenta{u}:\codeTs{t}{\gamma3}}

  \footnotesize
  \begin{align*}
    e'= & \red{\Resetz}~(\cfordo{x = e1}{e2}~\blue{\Resetz} ~(\cfordo{y = e3}{e4} \\
        & \blue{\Shiftz}~\blue{k_2}\to~ \red{\Shiftz}~\red{k_1}\to~\magenta{\cLet~u=\green{\framebox{\phantom{x}}}~\cIn}
          ~\red{\Throw~k_1}~(\blue{\Throw~k_2}~e5)))
  \end{align*}

  \[
    \infer{\vdash e'=\red{\Resetz}\cdots : \codeTs{t}{\gamma0};~~~\epsilon}
    {\infer[(\gamma1^*)]
      {\vdash \cfor{x=...} : \codeTs{t}{\gamma0};~~~\codeTs{t}{\gamma0}}
      {\infer{\Gamma a=\gammaa\vdash \blue{\Resetz}\cdots : \codeTs{t}{\gamma1};~~~\codeTs{t}{\gamma0}}
        {\infer[(\gamma2^*)]
          {\Gamma a\vdash \cfor{y=...}: \codeTs{t}{\gamma1};~~~\codeTs{t}{\gamma1},\codeTs{t}{\gamma0}}
          {\infer{\Gamma b=\gammab\vdash \blue{\Shiftz~k_2}... :\codeTs{t}{\gamma2}
              ;~~~\codeTs{t}{\gamma1},\codeTs{t}{\gamma0}}
            {\infer{\Gamma c=\gammac\vdash \red{\Shiftz~k_1}... :\codeTs{t}{\gamma1}
                ;~~~\codeTs{t}{\gamma0}}
              {\infer[(\gamma3^*)]
                {\Gamma d=\gammad\vdash \magenta{\cLet~u=...} : \codeTs{t}{\gamma0};~\epsilon}
                {\infer
                  {\Gamma e=\gammae\vdash \red{\Throw~k_1}~... : \codeTs{t}{\gamma3};~\epsilon}
                  {\infer{\Gamma e\vdash \blue{\Throw~k_2}~e5 :
                      \codeTs{t}{\gamma1\cup\gamma3};~~\epsilon}
                    {\infer*{\Gamma e\vdash e5:
                        \codeTs{t}{\gamma2\cup\gamma1\cup\gamma3};~~~\epsilon}
                      {}
                    }
                  }
                  &\infer*{\Gamma d\vdash \green{\framebox{\phantom{x}}}:\codeTs{t}{\gamma0};\epsilon}{}
                }
              }
            }
          }
        }
      }
    }
  \]

\end{frame}

\begin{frame}
  \frametitle{型推論アルゴリズム}
  \begin{itemize}
  \item 制約生成
  \item 制約解消
  \end{itemize}

\end{frame}

\begin{frame}
  \frametitle{制約生成}
  こういう制約が出てくる
\end{frame}


\begin{frame}
  \frametitle{制約解消}
  先の制約を解消して,型を決定する
\end{frame}
%%% Local Variables:
%%% mode: latex
%%% TeX-master: "slide_oishi"
%%% End:
