\section{問題点}

% コントロールオペレータによるコード移動を安全に行いたい
% コントロールオペレータs0/r0によるコード移動
% 安全に

% \begin{frame}
%   \center
%   \huge{問題点}
% \end{frame}

\begin{frame}
  \frametitle{コード生成前後でコードが移動する}
  \begin{visibleenv}<1->
    \begin{columns}
      \begin{column}{0.5\textwidth}%% [横幅] 0.2\textwidth = ページ幅の 20 %
        コード生成器
        \begin{align*}
          & \cfordo{x = e1}{e2} \\
          & ~~\red{\Resetz~} \cfordo{y = e3}{e4} \\
          & ~~~~\red{\Shiftz~} k \to \magenta{\cLet~ u = y ~ \cIn} \\
          & ~~~~~~\red{\Throw~} k~ \caryset{a}{(x,y)}{u}
        \end{align*}
      \end{column}
      $\too$
      \begin{column}{0.5\textwidth}%% [横幅] 0.2\textwidth = ページ幅の 20 %
        生成されるコード
        \begin{align*}
          & \cbra~ \fordo{x' = e1'}{e2'} \\
          & ~~\magenta{\Let~ u' = y' ~ \In} \\
          & ~~~~\fordo{y' = e3'}{e4'} \\
          & ~~~~~~\aryset{a}{(x',y')}{u'}~ \cket
        \end{align*}
      \end{column}
    \end{columns}
  \end{visibleenv}

  \begin{visibleenv}<2>
    \begin{exampleblock}{Scope Extrusion}
      (コード移動により)意図した束縛から,変数が抜け出てしまうこと
    \end{exampleblock}
  \end{visibleenv}
\end{frame}

% \begin{frame}
%   \frametitle{コード生成前・後でスコープの包含関係が逆転}
%   % sudo らの研究の EC を素朴に適用させると問題があるということ言うスライド
%   % ここに,forループとshift0/reset0 の例を再掲する.
%   % もとのスライドの 24ページの絵をかく.

%   \newcommand\ml{\multicolumn}
%   \begin{columns}
%     \only<1>{コード生成器:}
%     \begin{column}{0.6\textwidth}%% [横幅] 0.2\textwidth = ページ幅の 20 %
%       \center
%       \footnotesize
%       \begin{tabular}{l|l|l|l|l|l|l|l|l|l|l}
%         \cline{2-10}
%         & \ml{9}{|l|}{$\cfordo{x = e1}{e2}$~~~~~~~~~~~~~~~} \\ \cline{3-9}
%         & \footnotesize{\alert{$\mathbf \gamma0$}} & \ml{7}{|l|}{$\Resetz~ \cfordo{y = e3}{e4}$~} & \\ \cline{4-8}
%         & & \footnotesize{\alert{$\mathbf \gamma1$}} & \ml{5}{|l|}{$\Shiftz~ k \to \magenta{\cLet~ u = cc ~ \cIn}$}  & & \\ \cline{5-7}
%         & & & \footnotesize{\alert{$\mathbf \gamma2$}} & \ml{3}{|l|}{$\Throw~ k$}     &   &  &       \\ \cline{6-6}
%         & & & & \footnotesize{\alert{$\mathbf \gamma3$}} & \ml{1}{|l|}{$\caryset{a}{(x,y)}{u}$} & & &  &  \\ \cline{6-6}
%         & & & & \ml{3}{|l|}{\ }  &   &   &           \\ \cline{5-7}
%         & & & \ml{5}{|l|}{\ } &  &               \\ \cline{4-8}
%         & & \ml{7}{|l|}{\ }  & \\ \cline{3-9}
%         & \ml{9}{|l|}{~~~~~~~ } \\ \cline{2-10}
%       \end{tabular}
%     \end{column}

%     \begin{column}{0.5\textwidth}%% [横幅] 0.2\textwidth = ページ幅の 20 %
%       \begin{onlyenv}<2->
%         \flushright
%         \footnotesize
%         \begin{align*}
%           \gamma3 \ord \red{\gamma2} \ord \red{\gamma1} \ord \gamma0 \\
%         \end{align*}
%       \end{onlyenv}

%     \end{column}
%   \end{columns}

%   \bigskip

%   \begin{columns}
%     \only<1>{生成コード:~~}
%     \begin{column}{0.6\textwidth}%% [横幅] 0.2\textwidth = ページ幅の 20 %
%       \footnotesize
%       \begin{tabular}{l|l|l|l|l|l|l|l|l|l|l|l|l|}
%         \cline{2-10}
%         & \ml{9}{|l|}{$\cbra$ $\fordo{x' = e1'}{e2'}$~~~~~~~~~~~~~~~} \\ \cline{3-9}
%         & \footnotesize{\alert{$\mathbf \gamma0$}} & \ml{7}{|l|}{\magenta{$\Let~ u' = cc' ~ \In$}~} & \\ \cline{4-8}
%         & & \footnotesize{\alert{$\mathbf \gamma2$}} & \ml{5}{|l|}{$\fordo{y' = e3'}{e4'}$}  & & \\ \cline{5-7}
%         & & & \footnotesize{\alert{$\mathbf \gamma1$}} & \ml{3}{|l|}{\ }     &   &  &       \\ \cline{6-6}
%         & & & & \footnotesize{\alert{$\mathbf \gamma3$}} & \ml{1}{|l|}{$\aryset{a}{(x',y')}{u'}$ $\cket$} & & &  &  \\ \cline{6-6}
%         & & & & \ml{3}{|l|}{\ }  &   &   &           \\ \cline{5-7}
%         & & & \ml{5}{|l|}{\ } &  &               \\ \cline{4-8}
%         & & \ml{7}{|l|}{\ }  & \\ \cline{3-9}
%         & \ml{9}{|l|}{~~~~~~~ } \\ \cline{2-10}
%       \end{tabular}
%     \end{column}

%     \begin{column}{0.5\textwidth}%% [横幅] 0.2\textwidth = ページ幅の 20 %
%       \begin{onlyenv}<2->
%         \flushright
%         \footnotesize
%         \begin{align*}
%           \gamma3 \ord \red{\gamma1} \ord \red{\gamma2} \ord \gamma0 \\
%         \end{align*}
%       \end{onlyenv}
%     \end{column}
%   \end{columns}
% \end{frame}

%%% Local Variables:
%%% mode: latex
%%% TeX-master: "slide_oishi"
%%% End:
