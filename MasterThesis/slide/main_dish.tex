\begin{frame}
  \center
  \huge{型推論アルゴリズム}
\end{frame}

\begin{frame}
  \frametitle{型推論アルゴリズム}

  $\Gamma,~ L,~ \sigma,~ e$ が与えられたとき,$\Gamma \vdash^{L} e : t ;~\sigma$ が成立するような $t$ があるかどうか判定し,その型$t$ を返す

  \begin{exampleblock}{制約生成}
    与えられた項に対して,型,EC,エフェクトに関する制約を返す
  \end{exampleblock}
  \begin{exampleblock}{制約解消}
    その得られた制約を解消し,その制約を満たす代入$\Theta$ を返す
  \end{exampleblock}
\end{frame}

% \begin{frame}
%   \frametitle{型推論アルゴリズム}

% \end{frame}

\begin{frame}
  \frametitle{制約生成}
  \begin{exampleblock}{制約生成用の型システム$T_2$の導入}
    subsumption 規則をあらゆる規則に付加させて型推論用(制約を生成する)の型システムを作成.
    型付け規則を一意に適用できるようにした
  \end{exampleblock}
  \begin{exampleblock}{型に関する順序 $\longer{t_1}{t_2}$ の導入}
    制約生成時において,コード型か普通の型か判断することができないため
    その2つを同時に表す$\longer{}{}$ を導入した
  \end{exampleblock}
  % こういう制約が出てくる
\end{frame}

\begin{frame}
  \frametitle{制約解消}

  生成された制約 $\Delta \models C$

  仮定 $\Delta$
  \begin{align*}
    &\text{ECに対する順序} & \quad d \ord e \\
  \end{align*}
  制約 $C$
  \begin{align*}
    &\text{型} & \quad t0 = t1 & \quad t0 \ord t1 \\
    &\text{EC} & \quad \gamma0 = \gamma1 & \quad \gamma0 \ord \gamma1 \\
    &\text{エフェクト (型の列)} & \quad \sigma0 = \sigma1
  \end{align*}

  \begin{exampleblock}{制約に対する解の存在判定}
    型に対する単一化等をおこなう \\ % 単一化については質問が出そうなのでappendixを作成する
  \end{exampleblock}
  ここでは,ECの不等式制約の解消について説明をする
  % 先の制約を解消して,型を決定する
\end{frame}

\begin{frame}
  \frametitle{制約解消:ECの不等式制約の解消}
  この時点で残る制約 $\Delta \models C$
  \begin{align*}
    &\text{仮定 $\Delta$} & \quad d \ord e \text{の有限集合 ($d$はEC定数)}\\
    &\text{制約 $C$} & \quad e1 \ord e2 \text{の有限集合}\\
    &\text{$e,e1,e2,...$ ECを表す式} & \quad e ::= d \mid x \mid e \uni e\\
  \end{align*}

  \vspace{-2.5zh} %% odd

  \begin{onlyenv}<1>
    \begin{exampleblock}{制約解消アルゴリズム(の一部)}
      \begin{description}
      \item[$d1 \ord d1$] $\Longrightarrow  \text{$\Delta$を使って判定}$
      \item[$e1 \ord e2 \uni e3$] $\Longrightarrow \text{$e1 \ord e2$ かつ $e1 \ord e3$}$
      \item[$e1 \uni e2 \ord d$] $\Longrightarrow \text{$e1 \ord d$ または $e2 \ord d$}$
      \item[$\text{変数 $x$ の除去}$] $[x := d1 \uni d2 \uni y ]$ \mbox{} \\
        \vspace{-2zh} %% odd
        \begin{align*}
          &e1 \ord x & \quad x \ord d1 & \quad \phantom{\Longrightarrow\,\,\,} e1 \ord d1 & \quad e2 \ord d1 \\
          &e2 \ord x & \quad x \ord d2 & \quad \Longrightarrow e1 \ord d2 & \quad e2 \ord d2 \\
          &          & \quad x \ord y  & \quad \phantom{\Longrightarrow\,\,\,} e1 \ord y  & \quad e2 \ord y
        \end{align*}
      \end{description}
    \end{exampleblock}
  \end{onlyenv}

  % 制約を満たす解が存在するかどうかは判定できる
  % 解は1つとは限らない

\end{frame}

% \begin{frame}
%   \frametitle{制約解消:ECの不等式制約の解消の例}

% \end{frame}

%%% Local Variables:
%%% mode: latex
%%% TeX-master: "slide_oishi"
%%% End:
