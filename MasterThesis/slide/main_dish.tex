\begin{frame}
  \center
  \huge{型推論アルゴリズム}
\end{frame}

\begin{frame}
  \frametitle{型推論アルゴリズム}

  $\Gamma,~ L,~ \sigma,~ t,~ e$ が与えられたとき,$\Theta(\Gamma \vdash^{L} e : t ;~\sigma)$ が成立するような代入$\Theta$があるかどうか判定する

  \begin{exampleblock}{制約生成}
    与えられた項に対して,型,EC,エフェクトに関する制約を返す
  \end{exampleblock}
  \begin{exampleblock}{制約解消}
    その得られた制約を解消し,その制約を満たす代入$\Theta$またはfailを返す
  \end{exampleblock}
\end{frame}

% \begin{frame}
%   \frametitle{型推論アルゴリズム}

% \end{frame}

\begin{frame}
  \frametitle{制約生成}
  \begin{exampleblock}{制約生成用の型システム$T_2$}
    \footnotesize
    $T_1$
    \vspace{-0.5zh}
    \[
      \infer[]
      {\Gamma \vdash u~ \cPlus~ w: \codeTs{\intT}{\gamma}; \sigma}
      {\Gamma \vdash u: \codeTs{\intT}{\gamma}; \sigma & \Gamma \vdash w: \codeTs{\intT}{\gamma}; \sigma}
      +
      \quad
      \infer
      {\Gamma \vdash e : \codeTs{t}{\gamma2} ; \sigma}
      {\Gamma \vdash e : \codeTs{t}{\gamma1} ; \sigma
        & \Gamma \models \gamma2 \ord \gamma1
      }
    \]
    \vspace{-2zh}
    \begin{visibleenv}<2->
      $T_2$\\
      \[
        \infer[Constr;~ \Gamma \models \longer{t}{\codeTs{\intT}{\gamma}}]
        {\Gamma \vdash u~ \cPlus~ w: t; \sigma}
        {\Gamma \vdash u: \codeTs{\intT}{\gamma}; \sigma & \Gamma \vdash w: \codeTs{\intT}{\gamma}; \sigma}
      \]
    \end{visibleenv}
    \vspace{-2zh}
    \begin{itemize}
    \item 下から上に一意的に適用
    \item 規則適用時に制約を生成
    \end{itemize}
    % subsumption 規則をあらゆる規則に付加させて型推論用(制約を生成する)の型システムを作成.
    % 型付け規則を一意に適用できるようにした
  \end{exampleblock}
  \begin{exampleblock}{型に関する順序 $\longer{t_1}{t_2}$}
    % 制約生成時において,コード型か普通の型か判断することができないため
    % その2つを同時に表す$\longer{}{}$ を導入した
    コード型か普通の型か判断できないため,型に関する順序$\ord$の導入を行った
  \end{exampleblock}
  % こういう制約が出てくる
\end{frame}

\begin{frame}
  \frametitle{制約解消}

  % 生成された制約 $\Delta \models C$

  % 仮定 $\Delta$
  % \begin{align*}
  %   &\text{ECに対する順序} & \quad d \ord \gamma \\
  % \end{align*}
  制約 $C$
  \begin{align*}
    &\text{型} & t0 = t1 & \quad t0 \ord t1 \\
    &\text{EC} & \gamma0 = \gamma1 & \quad \only<2->{\red{\gamma0 \ord \gamma1}}\only<1>{\gamma0 \ord \gamma1} \\
    &\text{エフェクト (型の列)} & \sigma0 = \sigma1
  \end{align*}

  \begin{exampleblock}{制約に対する解の存在判定}
    型,EC,エフェクトに対する単一化等をおこなう \\ % 単一化については質問が出そうなのでappendixを作成する
  \end{exampleblock}
  \visible<2->{ここでは,ECの不等式制約の解消について説明をする}
  % 先の制約を解消して,型を決定する
\end{frame}

% \begin{frame}
%   \frametitle{制約解消:ECの不等式制約の解消}
%   この時点で残る制約 $\Delta \models C$
%   \begin{align*}
%     &\text{仮定 $\Delta$} & \quad d \ord e \text{の有限集合 ($d$はEC定数)}\\
%     &\text{制約 $C$} & \quad e1 \ord e2 \text{の有限集合}\\
%     &\text{$e,e1,e2,...$ ECを表す式} & \quad e ::= d \mid x \mid e \uni e\\
%   \end{align*}

%   \vspace{-2.5zh} %% odd

%   \begin{onlyenv}<1>
%     \begin{exampleblock}{制約解消アルゴリズム(の一部)}
%       \begin{description}
%       \item[$d1 \ord d1$] $\Longrightarrow  \text{$\Delta$を使って判定}$
%       \item[$e1 \ord e2 \uni e3$] $\Longrightarrow \text{$e1 \ord e2$ かつ $e1 \ord e3$}$
%       \item[$e1 \uni e2 \ord d$] $\Longrightarrow \text{$e1 \ord d$ または $e2 \ord d$}$
%       \item[$\text{変数 $x$ の除去}$] $[x := d1 \uni d2 \uni y ]$ \mbox{} \\
%         \vspace{-2zh} %% odd
%         \begin{align*}
%           &e1 \ord x & \quad x \ord d1 & \quad \phantom{\Longrightarrow\,\,\,} e1 \ord d1 & \quad e2 \ord d1 \\
%           &e2 \ord x & \quad x \ord d2 & \quad \Longrightarrow e1 \ord d2 & \quad e2 \ord d2 \\
%           &          & \quad x \ord y  & \quad \phantom{\Longrightarrow\,\,\,} e1 \ord y  & \quad e2 \ord y
%         \end{align*}
%       \end{description}
%     \end{exampleblock}
%   \end{onlyenv}

%   % 制約を満たす解が存在するかどうかは判定できる
%   % 解は1つとは限らない

% \end{frame}

\begin{frame}
  \frametitle{制約解消:ECの不等式制約の解消の例}
  \[
    \cfun{x}{(\cint{1}~ \cPlus~ x)} : \visible<-4>{t1} \visible<5->{\codeTs{\funT{\intT}{\intT}{}}{d0}}
  \]

  \begin{align*}
    % & C = \{\invisible<4->{\longer{\gamma2}{\gamma3},~} \longer{d1}{\gamma2},~ \invisible<3->{\longer{\gamma2}{d1},~} \invisible<2->{\longer{\gamma'}{\gamma1}}\} \\
    & C = \{\invisible<4->{\longer{\gamma2}{\gamma3},~} \invisible<3->{\longer{\gamma2}{d1},~} \invisible<2->{\longer{\gamma'}{\gamma1}}\} \\
    & \Theta = \{t1 := \codeTs{\funT{t2}{t3}{}}{\gamma'},~ t2 := \intT,~ t3 := \intT,~ t4 := \intT,~\\
    &\phantom{\Theta =~}\gamma' := d0 \only<2->{,~ \gamma1 := d0} \only<3->{,~ \gamma2 := d1} \only<4->{,~ \gamma3 := d0}\}
  \end{align*}

  \begin{exampleblock}{ECの変数の除去を行う}
    \begin{description}
    \item[$\gamma1$ を選ぶ]<2-> $C$から $\longer{\gamma'}{\gamma1}$ を消去し $\Theta$ に代入$\gamma1 := d0$ を追加
    \item[$\gamma2$ を選ぶ]<3-> $C$から $\longer{\gamma2}{d1}$ を消去し $\Theta$ に代入$\gamma2 := d1$ を追加
    \item[$\gamma3$ を選ぶ]<4-> $C$から $\longer{\gamma2}{\gamma3}$ を消去し $\Theta$ に代入$\gamma3 := d0$ を追加
    \end{description}
  \end{exampleblock}

\end{frame}

\begin{frame}
  \frametitle{研究成果}
  \begin{block}{コード生成 + shift0/reset0 の体系に対する}
    \begin{itemize}
    \item 型システムの設計
    \item 型推論アルゴリズムの開発
    \end{itemize}
  \end{block}
\end{frame}

%%% Local Variables:
%%% mode: latex
%%% TeX-master: "slide_oishi"
%%% End:
