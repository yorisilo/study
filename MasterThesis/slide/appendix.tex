\section{shift/reset と shift0/reset0 の意味論}

\begin{frame}[fragile]
  \frametitle{shift/reset と shift0/reset0 の意味論}

  shift/reset
  \noindent
  \begin{align*}
    \Resett ~(E[\Shiftt~ k \to e]) &~\leadsto~ \Resett~ e \ksubst{k}{E} \\
    \uncover<2->{\Resett ~(E[\Shiftt~ k \to e]) &~\leadsto~ \Resett~ e[k := \fun{x}{\Resett~ E[x]}]}
  \end{align*}

  shift0/reset0
  \noindent
  \begin{align*}
    \Resetz ~(E[\Shiftz~ k \to e]) &~\leadsto~ e \ksubst{k}{E} \\
    \uncover<2->{\Resetz ~(E[\Shiftz~ k \to e]) &~\leadsto~ e[k := \fun{x}{\Resetz~ E[x]}]}
  \end{align*}

  \begin{visibleenv}<3->
    \begin{exampleblock}{s0/r0を使う理由}
      reset0は評価後に reset0 が付かないので,/shift0/reset0 をネストさせたときに,複数のreset0をとびこえることができるので,
      今回の主目的である多段階let挿入が行える
    \end{exampleblock}
  \end{visibleenv}
\end{frame}

\section{Union を追加した理由}

\begin{frame}
  \frametitle{コード生成前・後でスコープの包含関係が逆転}
  % sudo らの研究の EC を素朴に適用させると問題があるということ言うスライド
  % ここに,forループとshift0/reset0 の例を再掲する.
  % もとのスライドの 24ページの絵をかく.

  \newcommand\ml{\multicolumn}
  \begin{columns}
    \only<1>{コード生成器:}
    \begin{column}{0.6\textwidth}%% [横幅] 0.2\textwidth = ページ幅の 20 %
      \center
      \footnotesize
      \begin{tabular}{l|l|l|l|l|l|l|l|l|l|l}
        \cline{2-10}
        & \ml{9}{|l|}{$\cfordo{x = e1}{e2}$~~~~~~~~~~~~~~~} \\ \cline{3-9}
        & \footnotesize{\alert{$\mathbf \gamma0$}} & \ml{7}{|l|}{$\Resetz~ \cfordo{y = e3}{e4}$~} & \\ \cline{4-8}
        & & \footnotesize{\alert{$\mathbf \gamma1$}} & \ml{5}{|l|}{$\Shiftz~ k \to \magenta{\cLet~ u = cc ~ \cIn}$}  & & \\ \cline{5-7}
        & & & \footnotesize{\alert{$\mathbf \gamma2$}} & \ml{3}{|l|}{$\Throw~ k$}     &   &  &       \\ \cline{6-6}
        & & & & \footnotesize{\alert{$\mathbf \gamma3$}} & \ml{1}{|l|}{$\caryset{a}{(x,y)}{u}$} & & &  &  \\ \cline{6-6}
        & & & & \ml{3}{|l|}{\ }  &   &   &           \\ \cline{5-7}
        & & & \ml{5}{|l|}{\ } &  &               \\ \cline{4-8}
        & & \ml{7}{|l|}{\ }  & \\ \cline{3-9}
        & \ml{9}{|l|}{~~~~~~~ } \\ \cline{2-10}
      \end{tabular}
    \end{column}

    \begin{column}{0.5\textwidth}%% [横幅] 0.2\textwidth = ページ幅の 20 %
      \begin{onlyenv}<2>
        \flushright
        \footnotesize
        \begin{align*}
          &\gamma3 \ord \red{\gamma2} \ord \red{\gamma1} \ord \gamma0
        \end{align*}
      \end{onlyenv}
      \begin{onlyenv}<3>
        \flushright
        \footnotesize
        \begin{align*}
          &\text{使用できるコード変数}\\
          &\gamma0 = \{x\},~ \\
          &\gamma1 = \{x,y\},~\\
          &\gamma2 = \{u,x\},~\\
          &\gamma3 = \{u,x,y\}
        \end{align*}
      \end{onlyenv}
    \end{column}
  \end{columns}

  \bigskip

  \begin{columns}
    \only<1>{生成コード:~~}
    \begin{column}{0.6\textwidth}%% [横幅] 0.2\textwidth = ページ幅の 20 %
      \footnotesize
      \begin{tabular}{l|l|l|l|l|l|l|l|l|l|l|l|l|}
        \cline{2-10}
        & \ml{9}{|l|}{$\cbra$ $\fordo{x' = e1'}{e2'}$~~~~~~~~~~~~~~~} \\ \cline{3-9}
        & \footnotesize{\alert{$\mathbf \gamma0$}} & \ml{7}{|l|}{\magenta{$\Let~ u' = cc' ~ \In$}~} & \\ \cline{4-8}
        & & \footnotesize{\alert{$\mathbf \gamma2$}} & \ml{5}{|l|}{$\fordo{y' = e3'}{e4'}$}  & & \\ \cline{5-7}
        & & & \footnotesize{\alert{$\mathbf \gamma1$}} & \ml{3}{|l|}{\ }     &   &  &       \\ \cline{6-6}
        & & & & \footnotesize{\alert{$\mathbf \gamma3$}} & \ml{1}{|l|}{$\aryset{a}{(x',y')}{u'}$ $\cket$} & & &  &  \\ \cline{6-6}
        & & & & \ml{3}{|l|}{\ }  &   &   &           \\ \cline{5-7}
        & & & \ml{5}{|l|}{\ } &  &               \\ \cline{4-8}
        & & \ml{7}{|l|}{\ }  & \\ \cline{3-9}
        & \ml{9}{|l|}{~~~~~~~ } \\ \cline{2-10}
      \end{tabular}
    \end{column}

    \begin{column}{0.5\textwidth}%% [横幅] 0.2\textwidth = ページ幅の 20 %
      \begin{onlyenv}<2>
        \flushright
        \footnotesize
        \begin{align*}
          \gamma3 \ord \red{\gamma1} \ord \red{\gamma2} \ord \gamma0
        \end{align*}
      \end{onlyenv}
    \end{column}
  \end{columns}
\end{frame}

\begin{frame}
  \frametitle{Union}
  \flushleft
  \includegraphics[clip,height=3cm]{./img/ecgraph.png}
  \begin{itemize}
  \item $\gamma1$ のコード変数は $\gamma2$ では使ってはいけない
  \item $\gamma2$ のコード変数は $\gamma1$ では使ってはいけない
  \item [$\Rightarrow$] \red{$\gamma1$ と $\gamma2$ の間に順序を付けない}
  \end{itemize}

  \begin{itemize}
  \item $\gamma1, \gamma2$ のコード変数は $\gamma3$ で使ってよい
  \item [$\Rightarrow$] \red{Sudoらの体系に $\cup$ (ユニオン) を追加} % ジョイン
  \end{itemize}
\end{frame}

\section{単一化とは}

\begin{frame}
  \frametitle{型の単一化の例}
  \begin{align*}
    &y \to (\intT \to w) \to x \\
    &\stackrel?= (x \to z) \to (x \to z)
  \end{align*}

  \begin{exampleblock}{単一化問題}
    両辺が同じ型になるような型変数への代入が存在するかどうかを判定することである.
  \end{exampleblock}

  \vspace{-1zh}
  \begin{visibleenv}<2>
    \begin{align*}
      \Theta &= \{x := \intT \to w,~ y := (\intT \to w) \to (\intT \to w),~\\
             &~~~z := \intT \to w \} \\
      \\
             &  (\intT \to w) \to (\intT \to w) \to (\intT \to w) \to (\intT \to w) \\
             &= (\intT \to w) \to (\intT \to w ) \to (\intT \to w ) \to (\intT \to w )
    \end{align*}
  \end{visibleenv}
\end{frame}

\section{型に関する順序の導入}

\begin{frame}
  \frametitle{型に関する順序}
  \[
    \infer
    {\Gamma \vdash e : \codeTs{t}{\gamma2} ; \sigma}
    {\Gamma \vdash e : \codeTs{t}{\gamma1} ; \sigma
      & \Gamma \models \gamma2 \ord \gamma1
    }
    \quad
    \infer
    {\Gamma \vdash e : t ; \sigma}
    {\Gamma \vdash e : t ; \sigma}
  \]
  元の型システム$T_1$ではこの普通の型,コード型それぞれを一つで表すような表現はなかったため,
  $T_2$ではこのような
  \[
    \infer[Constr;~ \Gamma \models \longer{t1}{t2}]
    {\Gamma \vdash e : t1 ; \sigma}
    {\Gamma \vdash e : t2 ; \sigma}
  \]
  型に関する順序を導入して一つで表せるようにしている
  規則を適用するたびに分岐が増えていくために一つの表現で表している
\end{frame}

\section{固有変数条件について}
\begin{frame}
  \frametitle{型システムでコード変数のスコープを表現}
  % 大なり記号はスコープの大きさを表しているのではなく,使える自由変数の集合の大きさを表している
  \vspace{-3zw}
  \center
  \begin{align*}
    & \Gamma = \gamma2 \ord \gamma1,~
      x : \codeTs{\textbf{int}}{\gamma1},~
      y : \codeTs{\textbf{int}}{\gamma2}
  \end{align*}

  \begin{tabular}{c|c}
    $\gamma1$ & $\gamma2$ \\ \hline \hline
    \uncover<1->{$\Gamma ~\vdash~ x : \codeTs{\textbf{int}}{\gamma1}~~ \alert{\text{OK}}$} & \uncover<1->{$\Gamma ~\vdash~ x : \codeTs{\textbf{int}}{\gamma2}~~ \alert{\text{OK}}$} \\ \hline

    \uncover<1->{$\Gamma ~\vdash~ y : \codeTs{\textbf{int}}{\gamma1}~~ \alert{\text{NG}}$} & \uncover<1->{$\Gamma ~\vdash~ y : \codeTs{\textbf{int}}{\gamma2}~~ \alert{\text{OK}}$} \\ \hline

    \uncover<1->{$\Gamma ~\vdash~ x\cPlus y : \codeTs{\textbf{int}}{\gamma1}~~  \alert{\text{NG}}$} & \uncover<1->{$\Gamma ~\vdash~ x\cPlus y : \codeTs{\textbf{int}}{\gamma2}~~  \alert{\text{OK}}$}
  \end{tabular}

  \bigskip

  \begin{uncoverenv}<2->
    コードレベルのラムダ抽象の型付け規則で固有変数条件を利用:

    \[
      \infer[(\gamma_2~\text{is eigen var})]
      {\Gamma \vdash \cfun{x}{e} : \codeTs{t_1\to t_2}{\gamma_1} }
      {\Gamma,~\gamma_2 \ord \gamma_1,~x:\codeTs{t_1}{\gamma_2} \vdash
        e : \codeTs{t_2}{\gamma_2}}
    \]

    \flushleft
    \footnotesize
    この「$y$ は$\gamma1$スコープのコード変数としては使えない」という条件を,
    Sudoらは,自然演繹の固有変数条件を用いて表現した.
    この型付け規則で,
    $\gamma2$ が固有変数であり,規則の結論に$\gamma2$が出現してはいけないこと,
    いいかえれば,下から上に見たとき,$\gamma2$ がfreshな変数であるという条件となる
  \end{uncoverenv}
\end{frame}

% x+y の形のコード生成器は、gamma1スコープはもたず、gamma2スコープをもつことがわかります。

% この「y はgamma1スコープのコード変数としては使えない」という条件を、
% Sudoらは、自然演繹(えんえき)の固有変数条件を用いて表現しました。
% この型付け規則(といって、p.23一番下の型付け規則を指す)で、
% gamma2 が固有変数であり、規則の結論にgamma2が出現してはいけないこと、
% いいかえれば、下から上に見たとき、gamma2 がfreshな変数であるという条件となります。


\section{健全性の証明}

\begin{frame}[fragile]
  \frametitle{健全性の証明 (Subject Reduction)}

  型安全性(型システムの健全性; Subject Reduction等の性質)を厳密に証明する.

  \begin{block}{Subject Redcution Property}
    $\Gamma \vdash M: \tau$ が導ければ(プログラム$M$が型検査を通れば),
    $M$を計算して得られる任意の$N$に対して,
    $\Gamma \vdash N: \tau$ が導ける($N$も型検査を通り,$M$と同じ型,
    同じ自由変数を持つ)
  \end{block}

  EC がなければ,既存手法で証明ができる.
  今回の場合をカバーするギリギリの型システムを試行錯誤で模索して作成した.
  あまり強力だと型推論が大変
  弱いと書きたいものがかけない
  そこを何度もやりなおしたのでそこの精密な検証ができていない.
  型推論ができるような型システムのギリギリのところを作成した

  answertype のsubsumption 規則をreset0 に付けるか throwに付けるか色々と模索を行って型システムを直前まで型推論ができるようにしていたので,そこの精密な検証ができていない
\end{frame}

% \section{answer type の列 (エフェクト)}

% \begin{frame}
%   \frametitle{answer type の列(エフェクト)}

% \end{frame}


%%% Local Variables:
%%% mode: latex
%%% TeX-master: "slide_oishi"
%%% End:
