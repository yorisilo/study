% 先の解決方法で実際になにができるのかの説明

\subsection{型付けの例}

\newcommand\boxterm{\framebox{
    \only<2>{\green{$\cint{3}$}}
    \only<3>{\red{$x~\cPlus~(\cint{3})$}}
    \phantom{A}
  }}

\begin{frame}
  \frametitle{型付けの例(1)}
  \begin{align*}
    e & = \Resetz ~~(\cfordo{x = e1}{e2} \\
      & \phantom{=}~~~ \Shiftz~k~\to~
        {\cLet~u=\green{\boxterm}~\cIn}~\Throw~k~u)
  \end{align*}

  \[
    \infer{\vdash e : \codeTs{t}{\gamma0};~\epsilon}
    {\infer[\blue{(\gamma1^*)}]{\vdash \cfor{x=...} :
        \codeTs{t}{\gamma0};~\codeTs{t}{\gamma0}}
      {\infer{\gamma1 \ord \gamma0,~x:\codeTs{t}{\gamma1}
          \vdash \Shiftz~k~\to~... :
          \codeTs{t}{\gamma1};~\codeTs{t}{\gamma0}}
        {\infer[(\gamma2^*)]{\Gamma a \vdash \cLet~u=...:\codeTs{t}{\gamma0};~\epsilon}
          {\infer{\Gamma b \vdash
              \Throw~k~u:\codeTs{t}{\gamma2};~\epsilon }
            {\infer{\Gamma b \vdash
                u:\codeTs{t}{\gamma1 \cup \gamma2};~ \sigma}
              {}
            }
            &\infer*{\Gamma a \vdash
              \green{\boxterm}:\codeTs{t}{\red{\gamma0}};~\epsilon}
            {}
          }
        }
      }
    }
  \]

  % $[(\gamma1^*)]$ は eigen variable (固有変数)

  {\footnotesize
    \begin{align*}
      \Gamma a &= \gamma1 \ord \gamma0,~x:\codeTs{t}{\red{\gamma1}},
                 ~k: \contT{\codeTs{t}{\gamma1}}{\codeTs{t}{\gamma0}} \\
      \Gamma b &= \Gamma a,~\gamma2 \ord \gamma0,~u:\codeTs{t}{\gamma2}
    \end{align*}
  }

\end{frame}

\begin{frame}
  \frametitle{型付けの例(2)}

  \newcommand\gammaa{\gamma1 \ord \gamma0,~x:\codeTs{t}{\gamma1}}
  \newcommand\gammab{\Gamma a,\gamma2 \ord \gamma1,~y:\codeTs{t}{\gamma2}}
  \newcommand\gammac{\Gamma b,\blue{k_2}:\contT{\codeTs{t}{\gamma2}}{\codeTs{t}{\gamma1}}}
  \newcommand\gammad{\Gamma c,\red{k_1}:\contT{\codeTs{t}{\gamma1}}{\codeTs{t}{\gamma0}}}
  \newcommand\gammae{\Gamma d,\gamma3 \ord \gamma0,\magenta{u}:\codeTs{t}{\gamma3}}

  \newcommand\boxterms{\framebox{
      \only<2>{\red{$x$}}
      \only<3>{\red{$y$}}
      \phantom{a}
    }}

  \vspace{-1zh} % odd

  \footnotesize
  \begin{align*}
    e'= & \red{\Resetz}~(\cfordo{x = e1}{e2}~\blue{\Resetz} ~(\cfordo{y = e3}{e4} \\
    % & \blue{\Shiftz}~\blue{k_2}\to~ \red{\Shiftz}~\red{k_1}\to~\magenta{\cLet~u=\green{\framebox{\phantom{x}}}~\cIn}
      & \blue{\Shiftz}~\blue{k_2}\to~ \red{\Shiftz}~\red{k_1}\to~\magenta{\cLet~u=\green{\boxterms}~\cIn}
          ~\red{\Throw~k_1}~(\blue{\Throw~k_2}~e5)))
  \end{align*}

  \vspace{-2zh} % odd

  \[
    \infer{\vdash e'=\red{\Resetz}\cdots : \codeTs{t}{\gamma0};~~~\epsilon}
    {\infer[(\gamma1^*)]
      {\vdash \cfor{x=...} : \codeTs{t}{\gamma0};~~~\codeTs{t}{\gamma0}}
      {\infer{\Gamma a=\gammaa\vdash \blue{\Resetz}\cdots : \codeTs{t}{\gamma1};~~~\codeTs{t}{\gamma0}}
        {\infer[(\gamma2^*)]
          {\Gamma a\vdash \cfor{y=...}: \codeTs{t}{\gamma1};~~~\codeTs{t}{\gamma1},\codeTs{t}{\gamma0}}
          {\infer{\Gamma b=\gammab\vdash \blue{\Shiftz~k_2}... :\codeTs{t}{\gamma2}
              ;~~~\codeTs{t}{\gamma1},\codeTs{t}{\gamma0}}
            {\infer{\Gamma c=\gammac\vdash \red{\Shiftz~k_1}... :\codeTs{t}{\gamma1}
                ;~~~\codeTs{t}{\gamma0}}
              {\infer[(\gamma3^*)]
                {\Gamma d=\gammad\vdash \magenta{\cLet~u=...} : \codeTs{t}{\gamma0};~\epsilon}
                {\infer
                  {\Gamma e=\gammae\vdash \red{\Throw~k_1}~... : \codeTs{t}{\gamma3};~\epsilon}
                  {\infer{\Gamma e\vdash \blue{\Throw~k_2}~e5 :
                      \codeTs{t}{\gamma1\cup\gamma3};~~\epsilon}
                    {\infer*{\Gamma e\vdash e5:
                        \codeTs{t}{\gamma2\cup\gamma1\cup\gamma3};~~~\epsilon}
                      {}
                    }
                  }
                  % &\infer*{\Gamma d\vdash \green{\framebox{\phantom{x}}}:\codeTs{t}{\gamma0};\epsilon}{}
                  &\infer*{\Gamma d\vdash \green{\boxterms}:\codeTs{t}{\gamma0};\epsilon}{}
                }
              }
            }
          }
        }
      }
    }
  \]

  {\footnotesize
    \begin{align*}
      \Gamma d &= \cdots,~ x:\codeTs{t}{\red{\gamma1}},~ y:\codeTs{t}{\red{\gamma2}},~ \gamma1 \ord \gamma0,~ \gamma2 \ord \gamma1,~ \cdots
    \end{align*}
  }

\end{frame}

\begin{frame}
  \center
  \huge{型推論アルゴリズム}
\end{frame}

\begin{frame}
  \frametitle{型推論アルゴリズム}

  $\Gamma,~ L,~ \sigma,~ e$ が与えられたとき,$\Gamma \vdash^{L} e : t ;~\sigma$ が成立するような $t$ があるかどうか判定し,その型$t$ を返す

  \begin{exampleblock}{制約生成}
    与えられた項に対して,型,EC,エフェクトに関する制約を返す
  \end{exampleblock}
  \begin{exampleblock}{制約解消}
    その得られた制約を解消し,その制約を満たす代入$\Theta$ を返す
  \end{exampleblock}
\end{frame}

% \begin{frame}
%   \frametitle{型推論アルゴリズム}

% \end{frame}

\begin{frame}
  \frametitle{制約生成}
  \begin{exampleblock}{制約生成用の型システム$T_2$の導入}
    subsumption 規則をあらゆる規則に付加させて型推論用(制約を生成する)の型システムを作成.
    型付け規則を一意に適用できるようにした
  \end{exampleblock}
  \begin{exampleblock}{型に関する順序 $\longer{t_1}{t_2}$ の導入}
    制約生成時において,コード型か普通の型か判断することができないため
    その2つを同時に表す$\longer{}{}$ を導入した
  \end{exampleblock}
  % こういう制約が出てくる
\end{frame}

\begin{frame}
  \frametitle{制約解消}

  生成された制約 $\Delta \models C$

  仮定 $\Delta$
  \begin{align*}
    &\text{ECに対する順序} & \quad d \ord e \\
  \end{align*}
  制約 $C$
  \begin{align*}
    &\text{型} & \quad t0 = t1 & \quad t0 \ord t1 \\
    &\text{EC} & \quad \gamma0 = \gamma1 & \quad \gamma0 \ord \gamma1 \\
    &\text{エフェクト (型の列)} & \quad \sigma0 = \sigma1
  \end{align*}

  \begin{exampleblock}{制約に対する解の存在判定}
    型に対する単一化等をおこなう \\ % 単一化については質問が出そうなのでappendixを作成する
  \end{exampleblock}
  ここでは,ECの不等式制約の解消について説明をする
  % 先の制約を解消して,型を決定する
\end{frame}

\begin{frame}
  \frametitle{制約解消:ECの不等式制約の解消}
  この時点で残る制約 $\Delta \models C$
  \begin{align*}
    &\text{仮定 $\Delta$} & \quad d \ord e \text{の有限集合 ($d$はEC定数)}\\
    &\text{制約 $C$} & \quad e1 \ord e2 \text{の有限集合}\\
    &\text{$e,e1,e2,...$ ECを表す式} & \quad e ::= d \mid x \mid e \uni e\\
  \end{align*}

  \vspace{-2.5zh} %% odd

  \begin{onlyenv}<1>
    \begin{exampleblock}{制約解消アルゴリズム(の一部)}
      \begin{description}
      \item[$d1 \ord d1$] $\Longrightarrow  \text{$\Delta$を使って判定}$
      \item[$e1 \ord e2 \uni e3$] $\Longrightarrow \text{$e1 \ord e2$ かつ $e1 \ord e3$}$
      \item[$e1 \uni e2 \ord d$] $\Longrightarrow \text{$e1 \ord d$ または $e2 \ord d$}$
      \item[$\text{変数 $x$ の除去}$] $[x := d1 \uni d2 \uni y ]$ \mbox{} \\
        \vspace{-2zh} %% odd
        \begin{align*}
          &e1 \ord x & \quad x \ord d1 & \quad \phantom{\Longrightarrow\,\,\,} e1 \ord d1 & \quad e2 \ord d1 \\
          &e2 \ord x & \quad x \ord d2 & \quad \Longrightarrow e1 \ord d2 & \quad e2 \ord d2 \\
          &          & \quad x \ord y  & \quad \phantom{\Longrightarrow\,\,\,} e1 \ord y  & \quad e2 \ord y
        \end{align*}
      \end{description}
    \end{exampleblock}
  \end{onlyenv}

  % 制約を満たす解が存在するかどうかは判定できる
  % 解は1つとは限らない

\end{frame}

% \begin{frame}
%   \frametitle{制約解消:ECの不等式制約の解消の例}

% \end{frame}

%%% Local Variables:
%%% mode: latex
%%% TeX-master: "slide_oishi"
%%% End:
