\chapter{実装}
この章では,コード生成言語にshift0/reset0 を導入した言語の評価器と
その言語で記述した項に対して,型推論を行い,自動的に型を付けるために必要な制約を生成する制約生成器の実装を見ていく.なお,実際のプログラムソースは付録に掲載している.
それらの実装は OCaml を用いて実装を行っている.

\section{評価器}
\begin{itemize}
\item 評価器 \lstinline|eval1|は,項$e$を受け取り,値$v$を返す関数である.
\item プリティプリンタ \lstinline|print_expr'| は,項$e$ を受け取り,項$e$のASTを整形する関数である.
\item プリティプリンタ \lstinline|print_value'| は,値$v$ を受け取り,値$v$のASTを整形する関数である.
\end{itemize}

以下で,いくつか具体例を見ていく

\begin{lstlisting}
 let e1 =
  R0
   (PrimOp2 ("Add_", Code (Int 10),
     R0
      (PrimOp2 ("Add_", Code (Int 20),
        S0 ("k1",
         S0 ("k2",
          App (Var "k1", App (Var "k2", Code (Int 1)))))))));;

# print_expr' e1
R0(<10> +_ R0(<20> +_ (S0 k1 -> (S0 k2 -> (k1 (k2 <1>))))))- : unit = ()

#  print_value' (eval1 e1)
<20 + 10 + 1>- : unit = ()

\end{lstlisting}

\section{制約生成器}
制約生成器は, $T_2$の型付け規則を用い,下から上に型付けを行い制約を生成していく.
それによって,型とclassifierと$\sigma$-partに関する制約が生成され,
対応する型付け規則がなければ,そこで止まる.
\begin{itemize}
\item 制約生成器 \lstinline|pp_cnstrl|は,項$e$を受け取り,制約を整形した結果を返す関数である.
% \item プリティプリンタ \lstinline|print_cnstr'| は,制約を受け取り,そのASTを整形する関数である.
\end{itemize}

\ref{subsec:exam-let} のlet挿入の例に対して,どのような制約が現れるか以下に実行例を記す.

\tiny
\begin{lstlisting}
# pp_cnstrl (R0(Let_("x", Code(Int 1), R0(Let_("y", Code(Int 2), S0("k", Let_("z", Var "x", T0("k", App(Var "k", Var "z")))))))));;
t_161 = Int,
 |= <t_150>^(γ_159) > <t_161>^(γ_173),
 |= <t_149>^(γ_158) > <t_150>^(γ_159),
 |= t_148 > <t_149>^(γ_158),
t_160 = Int,
γ_160>γ_159,x: (<t_150>^(γ_160))^(lv0), |= <t_152>^(γ_162) >
 <t_160>^(γ_172),
γ_160>γ_159,x: (<t_150>^(γ_160))^(lv0), |= <t_151>^(γ_161) >
 <t_152>^(γ_162),
γ_160>γ_159,x: (<t_150>^(γ_160))^(lv0), |= <t_150>^(γ_160) >
 <t_151>^(γ_161),
 |= <t_149>^(γ_158) > <t_150>^(γ_159),
 |= t_148 > <t_149>^(γ_158),
k: (<t_154>^(γ_165) = <t_149>^(γ_158),ε  => <t_153>^(γ_164))^(lv0),
γ_163>γ_162,
y: (<t_152>^(γ_163))^(lv0),γ_160>γ_159,x: (<t_150>^(γ_160))^(lv0), |=
 <t_155>^(γ_166) > <t_150>^(γ_160),
k: (<t_154>^(γ_165) = <t_149>^(γ_158),ε  => <t_153>^(γ_164))^(lv0),
γ_163>γ_162,
y: (<t_152>^(γ_163))^(lv0),γ_160>γ_159,x: (<t_150>^(γ_160))^(lv0), |=
 <t_153>^(γ_164) > <t_155>^(γ_166),
γ_163>γ_162,
y: (<t_152>^(γ_163))^(lv0),γ_160>γ_159,x: (<t_150>^(γ_160))^(lv0), |=
 <t_152>^(γ_163) > <t_154>^(γ_165),
<t_151>^(γ_161) = <t_153>^(γ_164),
γ_163>γ_162,
y: (<t_152>^(γ_163))^(lv0),γ_160>γ_159,x: (<t_150>^(γ_160))^(lv0), |=
 γ_165 > γ_164,
γ_160>γ_159,x: (<t_150>^(γ_160))^(lv0), |= <t_151>^(γ_161) >
 <t_152>^(γ_162),
γ_160>γ_159,x: (<t_150>^(γ_160))^(lv0), |= <t_150>^(γ_160) >
 <t_151>^(γ_161),
 |= <t_149>^(γ_158) > <t_150>^(γ_159),
 |= t_148 > <t_149>^(γ_158),
γ_167>γ_166,
z: (<t_155>^(γ_167))^(lv0),
k: (<t_154>^(γ_165) = <t_149>^(γ_158),ε  => <t_153>^(γ_164))^(lv0),
γ_163>γ_162,
y: (<t_152>^(γ_163))^(lv0),γ_160>γ_159,x: (<t_150>^(γ_160))^(lv0), |=
 t_159 -> t_158 > <t_154>^(γ_165) = <t_149>^(γ_158),ε  => <t_153>^(γ_164),
γ_167>γ_166,
z: (<t_155>^(γ_167))^(lv0),
k: (<t_154>^(γ_165) = <t_149>^(γ_158),ε  => <t_153>^(γ_164))^(lv0),
γ_163>γ_162,
y: (<t_152>^(γ_163))^(lv0),γ_160>γ_159,x: (<t_150>^(γ_160))^(lv0), |=
 <t_157>^(γ_171) > t_158,
γ_167>γ_166,
z: (<t_155>^(γ_167))^(lv0),
k: (<t_154>^(γ_165) = <t_149>^(γ_158),ε  => <t_153>^(γ_164))^(lv0),
γ_163>γ_162,
y: (<t_152>^(γ_163))^(lv0),γ_160>γ_159,x: (<t_150>^(γ_160))^(lv0), |=
 <t_155>^(γ_167) > <t_156>^(γ_170),
γ_167>γ_166,
z: (<t_155>^(γ_167))^(lv0),
k: (<t_154>^(γ_165) = <t_149>^(γ_158),ε  => <t_153>^(γ_164))^(lv0),
γ_163>γ_162,
y: (<t_152>^(γ_163))^(lv0),γ_160>γ_159,x: (<t_150>^(γ_160))^(lv0), |=
 γ_170 > γ_168,
<t_154>^(γ_165) = <t_149>^(γ_158),ε  => <t_153>^(γ_164) =
 <t_157>^(γ_169) = <t_151>^(γ_161),<t_149>^(γ_158),ε   => <t_156>^(γ_168),
γ_167>γ_166,
z: (<t_155>^(γ_167))^(lv0),
k: (<t_154>^(γ_165) = <t_149>^(γ_158),ε  => <t_153>^(γ_164))^(lv0),
γ_163>γ_162,
y: (<t_152>^(γ_163))^(lv0),γ_160>γ_159,x: (<t_150>^(γ_160))^(lv0), |=
 γ_169 U γ_170 > γ_171,
k: (<t_154>^(γ_165) = <t_149>^(γ_158),ε  => <t_153>^(γ_164))^(lv0),
γ_163>γ_162,
y: (<t_152>^(γ_163))^(lv0),γ_160>γ_159,x: (<t_150>^(γ_160))^(lv0), |=
 <t_153>^(γ_164) > <t_155>^(γ_166),
γ_163>γ_162,
y: (<t_152>^(γ_163))^(lv0),γ_160>γ_159,x: (<t_150>^(γ_160))^(lv0), |=
 <t_152>^(γ_163) > <t_154>^(γ_165),
<t_151>^(γ_161) = <t_153>^(γ_164),
γ_163>γ_162,
y: (<t_152>^(γ_163))^(lv0),γ_160>γ_159,x: (<t_150>^(γ_160))^(lv0), |=
 γ_165 > γ_164,
γ_160>γ_159,x: (<t_150>^(γ_160))^(lv0), |= <t_151>^(γ_161) >
 <t_152>^(γ_162),
γ_160>γ_159,x: (<t_150>^(γ_160))^(lv0), |= <t_150>^(γ_160) >
 <t_151>^(γ_161),
 |= <t_149>^(γ_158) > <t_150>^(γ_159),
 |= t_148 > <t_149>^(γ_158),
γ_167>γ_166,
z: (<t_155>^(γ_167))^(lv0),
k: (<t_154>^(γ_165) = <t_149>^(γ_158),ε  => <t_153>^(γ_164))^(lv0),
γ_163>γ_162,
y: (<t_152>^(γ_163))^(lv0),γ_160>γ_159,x: (<t_150>^(γ_160))^(lv0), |=
 t_159 > <t_155>^(γ_167),
γ_167>γ_166,
z: (<t_155>^(γ_167))^(lv0),
k: (<t_154>^(γ_165) = <t_149>^(γ_158),ε  => <t_153>^(γ_164))^(lv0),
γ_163>γ_162,
y: (<t_152>^(γ_163))^(lv0),γ_160>γ_159,x: (<t_150>^(γ_160))^(lv0), |=
 <t_157>^(γ_171) > t_158,
γ_167>γ_166,
z: (<t_155>^(γ_167))^(lv0),
k: (<t_154>^(γ_165) = <t_149>^(γ_158),ε  => <t_153>^(γ_164))^(lv0),
γ_163>γ_162,
y: (<t_152>^(γ_163))^(lv0),γ_160>γ_159,x: (<t_150>^(γ_160))^(lv0), |=
 <t_155>^(γ_167) > <t_156>^(γ_170),
γ_167>γ_166,
z: (<t_155>^(γ_167))^(lv0),
k: (<t_154>^(γ_165) = <t_149>^(γ_158),ε  => <t_153>^(γ_164))^(lv0),
γ_163>γ_162,
y: (<t_152>^(γ_163))^(lv0),γ_160>γ_159,x: (<t_150>^(γ_160))^(lv0), |=
 γ_170 > γ_168,
<t_154>^(γ_165) = <t_149>^(γ_158),ε  => <t_153>^(γ_164) =
 <t_157>^(γ_169) = <t_151>^(γ_161),<t_149>^(γ_158),ε   => <t_156>^(γ_168),
γ_167>γ_166,
z: (<t_155>^(γ_167))^(lv0),
k: (<t_154>^(γ_165) = <t_149>^(γ_158),ε  => <t_153>^(γ_164))^(lv0),
γ_163>γ_162,
y: (<t_152>^(γ_163))^(lv0),γ_160>γ_159,x: (<t_150>^(γ_160))^(lv0), |=
 γ_169 U γ_170 > γ_171,
k: (<t_154>^(γ_165) = <t_149>^(γ_158),ε  => <t_153>^(γ_164))^(lv0),
γ_163>γ_162,
y: (<t_152>^(γ_163))^(lv0),γ_160>γ_159,x: (<t_150>^(γ_160))^(lv0), |=
 <t_153>^(γ_164) > <t_155>^(γ_166),
γ_163>γ_162,
y: (<t_152>^(γ_163))^(lv0),γ_160>γ_159,x: (<t_150>^(γ_160))^(lv0), |=
 <t_152>^(γ_163) > <t_154>^(γ_165),
<t_151>^(γ_161) = <t_153>^(γ_164),
γ_163>γ_162,
y: (<t_152>^(γ_163))^(lv0),γ_160>γ_159,x: (<t_150>^(γ_160))^(lv0), |=
 γ_165 > γ_164,
γ_160>γ_159,x: (<t_150>^(γ_160))^(lv0), |= <t_151>^(γ_161) >
 <t_152>^(γ_162),
γ_160>γ_159,x: (<t_150>^(γ_160))^(lv0), |= <t_150>^(γ_160) >
 <t_151>^(γ_161),
 |= <t_149>^(γ_158) > <t_150>^(γ_159),
 |= t_148 > <t_149>^(γ_158),
- : unit = ()
\end{lstlisting}
\normalsize

制約解消についての実装は進行中である.
% \oishi{制約がすごく多く出てしまうはずで,どのように記せばよいのか...,}
%%% Local Variables:
%%% mode: japanese-latex
%%% TeX-master: "master_oishi"
%%% End:
