\chapter{型推論}

概要: 以下の2ステップから構成
\begin{itemize}
\item 制約生成:与えられた項にたいして、(型およびクラシファイアに関する)制約を返す。
\item 制約を解く。
\end{itemize}

\section{制約生成}

これは、もの型システム($T_1$とする)を
「トップダウンでの制約生成向け型システム($T_2$とする)」に変
形することであたえる。

$T_2$の設計指針:
\begin{itemize}
\item $T_1$と$T_2$は「型付けできる」という関係として等価である。
\item $T_2$は、term-oriented である。
  (結論側の式のトップレベルの形だけで、どの型付けルールを適用可能か、一意的にわかる。)
\item $T_2$は、制約生成をする。
  (結論側の式の要素は変数として、
  「それがこういう形でなければいけない」という条件は、制約の形で「生成」する。)
\end{itemize}

以上をどう満たすか? ポイントは、subsumption rule の適用タイミング(な
るべく subsumption rule を適用するのを避けたい)である。

\section{型システム$T_2$の導入}

subsumption rule が出現する場所を限定することができる.
特に,ルールと,その直後に subsumption がつかわれる場合を考えてみよう.
以下で,「var1」等といった表記は,
「もともとあるver1ルールを subsumption規則と組み合わせた形に改訂したも
の」である.
また,横棒の右に書いてあるConstr;... は(ルールを下から上にむけて
使うとき),Constr 以下の制約が生成される,という意味である.

また、型$t_1, t_2$に対する$\longer{}{}$の記号は以下の意味であるが、
とりあえず、(以下の意味にしたがって分解はせずに)
$\longer{t_1}{t_2}$ の形のまま、制約として生成する。
\begin{itemize}
\item $\longer{\codeT{t_1}{\gamma_1}}{\codeT{t_2}{\gamma_2}}$
  は、「$t_1=t_2$ かつ $\longer{\gamma_1}{\gamma_2}$
\item $\codeT{t}{\gamma}$の形でない$t_1,t_2$に対しては、
  $t_1 = t_2$。
\end{itemize}
(型推論のプロセスの最中では、$t_1, t_2$ はメタ型変瑞。洽筅キ、譴此その場合、どちらの形かは決定できないので、$\longer{t_1}{t_2}$を
上記の意味にしたがって、「ほどく」ことはできない。なので、
$\longer{t_1}{t_2}$という形のまま制約として生成する。)

(亀山メモ:
ただ、もしかすると、
「レベル0の型変数」と「レベル1の型変数」を最初からわけておく方法もある
かもしれない。そうすると、上記はとける?)


\oishi{
  以下のルールにおいて,Constraint の $\Gamma \models \longer{t}{t'}$ の $\Gamma \models$ はルールを適用した瞬間の $\Gamma$ の情報は制約解消において使わないから必要ない?\\
  一貫性をもたせるのなら, $t = t'$ の形の Constraint も $\Gamma \models t = t'$ とする必要がある?
}
\\
(var0)

\[
  \infer[Constr;~\Gamma \models \longer{t}{t'}]
  {\Gamma \vdash x:t;~\sigma}
  {(x:t') \in \Gamma
  }
\]

(var1)

\[
  \infer[Constr;~\Gamma \models \longer{\gamma}{\gamma'}]
  {\Gamma \vdash^{\gamma} u:t;~\sigma}
  {(u:t)^{\gamma'} \in \Gamma
  }
\]

(const)

\[
  \infer[Constr;~\Gamma \models \longer{t}{t^c}]
  {\Gamma \vdash^{L} c:t;~\sigma}
  {}
\]

(app0)

\[
  \infer[Constr;~\Gamma \models \longer{t}{t_1}]
  {\Gamma \vdash^\gamma e_1 \, e_2 : t;~\sigma}
  {\Gamma \vdash^\gamma e_1 : \funT{t_2}{t_1}{\sigma};~\sigma
    &\Gamma \vdash^\gamma e_2 : t_2;~\sigma
  }
\]

(app1)

\[
  \infer[Constr;~\Gamma \models \longer{t}{t_1}]
  {\Gamma \vdash e_1 \, e_2 : t;~\sigma}
  {\Gamma \vdash e_1 : t_2 \to t_1;~\sigma
    &\Gamma \vdash e_2 : t_2;~\sigma
  }
\]

(lambda0)

\[
  \infer[Constr;~t=\funT{t_1}{t_2}{\sigma'},~ \Gamma \models \sigma \ord \sigma' ]
  {\Gamma \vdash \lambda x.e : t;~\sigma}
  {\Gamma,~x:t_1 \vdash e : t_2;~\sigma'}
\]

(lambda1) \\
\oishi{
  コードの中で,shift0/reset0 は使わないので, $\sigma$ は $\epsilon$ なはず.
  なので,$\sigma$ は $\epsilon$ としておいてもよい
}

\[
  \infer[Constr;~t=\funT{t_1}{t_2}{}]
  {\Gamma \vdash^\gamma \lambda u.e : t;~\sigma}
  {\Gamma,~(u:t_1)^\gamma \vdash^\gamma e : t_2;~\sigma}
\]

(if)

\[
  \infer[Constr;~(none)]
  {\Gamma \vdash^L
    \textbf{if}~e_1 \textbf{then}~e_2 \textbf{else}~e_3 ~:~t; ~ \sigma}
  {\Gamma \vdash^L e_1 : \Bool;~\sigma
    &\Gamma \vdash^L e_2 : t;~\sigma
    &\Gamma \vdash^L e_3 : t;~\sigma
  }
\]

(code-lambda)

\[
  \infer[Constr;~\Gamma \models \longer{t}{\codeT{t_1 \to t_2}{\gamma}}]
  {\Gamma \vdash \underline{\lambda}x.e ~:~t;~\sigma}
  {\Gamma,\longer{\gamma'}{\gamma},x:\codeT{t_1}{\gamma'}
    \vdash e : \codeT{t_2}{\gamma'};~\sigma
  }
\]

(reset0)

\[
  \infer[Constr;~\Gamma \models \longer{t}{\codeT{t'}{\gamma}}]
  {\Gamma \vdash \Resetz~e ~:~ t; ~\sigma}
  {\Gamma \vdash e:\codeT{t'}{\gamma};~\codeT{t'}{\gamma},\sigma
  }
\]

(shift0)

\[
  \infer[Constr;~\Gamma \models \longer{t}{\codeT{t_1}{\gamma_1}},~ t_2 =\codeT{t_0}{\gamma_0},~ \Gamma \models \longer{\gamma_1}{\gamma_0}]
  {\Gamma \vdash \shiftz{k}{e} : t~;~ t_2,\sigma}
  {\Gamma,~k:\contT{\codeT{t_1}{\gamma_1}}{\codeT{t_0}{\gamma_0}}{\sigma}
    \vdash e : \codeT{t_0}{\gamma_0} ; \sigma
  }
\]


(throw0)\\
\oishi{
  throw0 ルールにのみ $\sigma$ part の subsumption ルールを適用すればおk?
}
\\
変更前
\[
  \infer[Constr;~ \Gamma \models \longer{t}{\codeT{t_0}{\gamma_2}},~ (\Gamma,~ k:t') \models \gamma_2 \ord \gamma_0,~  t' = \contT{\codeT{t_1}{\gamma_1}}{\codeT{t_0}{\gamma_0}}{\sigma}]
  {\Gamma,~ k:t'
    \vdash \throw{k}{v} : t ; \sigma}
  {\Gamma,~ k:t'
    \vdash v : \codeT{t_1}{\gamma_1 \uni \gamma_2} ; \sigma
  }
\]

変更後
\[
  \infer[Constr;~ \Gamma \models \longer{t}{\codeT{t_0}{\gamma_2}},~ (\Gamma,~ k:t') \models \gamma_2 \ord \gamma_0,~  t' = \contT{\codeT{t_1}{\gamma_1}}{\codeT{t_0}{\gamma_0}}{\sigma},~ \Gamma \models \gamma_1 \uni \gamma_2 \ord \gamma']
  {\Gamma,~ k:t'
    \vdash \throw{k}{v} : t ; \sigma}
  {\Gamma,~ k:t'
    \vdash v : \codeT{t_1}{\gamma'} ; \sigma
  }
\]

(code)

\[
  \infer[Constr;~\longer{t}{\codeT{t_1}{\gamma}}]
  {\Gamma \vdash \code{e} : t;~\sigma}
  {\Gamma \vdash^\gamma e : t_1;~\sigma}
\]


この新しい型システム$T_2$ は(hopefully) $T_1$と同じ型付けをあたえる。

型推論では、与えられた項$e$に対して、
$T_2$を「下から上」の向きに適用して、Constraintを生成する。
(もちろん、途中で、つまってしまったら、型推論は失敗する。

下から上に行くときに、「横棒の下には存在しないで、横棒の上、あるいは、
Constraintの中にのみ存在する」型やclassifier があるときは、
これらを型変数やclassifier変数として生成する。

なお、code-lambda規則での新しいclassifier は、classfier変数ではなく、
classifier定数とする。(この段階では、変数も定数も差はないが、
制約を解消するとき、cassifier定数に対する代入はしない、という違いがある。)

\section{制約の解消}
\oishi{この節をどうするか}

制約生成が成功したとき(つまり、途中で「つっかかったり」しないとき)、以
下の性質が成立するはずである。

入力を$\Gamma, L, e, t,\sigma$ として、前章の制約生成アルゴリズムを走
らせ、それが成功して$C$という制約を生成したとき、
\begin{itemize}
\item
  $\Gamma \vdash^L e: t; ~ \sigma$ が導出可能ならば、
  $C$ を満たす解が存在し、
\item
  $C$ を満たす解が存在すれば、ある代入$\theta$に対して、
  $\Gamma\theta \vdash^L e\theta: t\theta; ~ \sigma\theta$ が導出可能である。
\end{itemize}
という性質が成立することができる。
(これが成立すれば、もともとの型推論問題を、型制約の解消問題に帰着できたことになる。)

というわけで、制約の解消をはじめよう。制約は以下の文法で与えられたものの有限集合である。

\begin{itemize}
\item $\Gamma \models \longer{t^0}{t^0}$
\item $\Gamma \models \longer{c}{c}$
\item $\Gamma \models \longer{\sigma}{\sigma}$
\item $t^0=t^0$
\item $t^1=t^1$
\end{itemize}
ただし、ここで $t_0$, $t_1$, $c$, $\sigma$ は以下の文法で定義される。

\begin{align*}
  t^0      & ::= t^0_x \mid \Int \mid \Bool \mid \funT{t^0}{t^0}{c} \mid \codeT{t^1}{c} \\
  t^1      & ::= t^1_x \mid \Int \mid \Bool \mid \funT{t^1}{t^1}{} \\
  c        & ::= \gamma \mid d \mid c \uni c \\
  \sigma   & ::= \epsilon \mid t^0, \sigma
\end{align*}


$t_0$, $t_1$, $c$, $\sigma$ はそれぞれ、
レベル0型、
レベル1型、
レベル0型の列,
classifierをあらわす表現(メタ変数),
である。また、
$t^i_x$はそれぞれのレベルの型変数、$\gamma$はclassifier変数である。
また、$d$は、固有変数条件をもつclassifier変数のことであり、
型推論のあいだは、これは(実質的に)定数として扱われる(つまり、
$d$に対しては代入しない。classifier変数$\gamma$に対しては代入する。)

また、$\Gamma$は、一般の型文脈であるが、不要な情報を落として以下の形にしてよい。
\begin{align*}
  \Delta & ::= \cdots \mid \Delta,\longer{d}{c}
\end{align*}
(左辺は、固有変数なので、classifier定数である。右辺は一般のclassifier
式がなんでも来る可能性がある。)

制約の解消とは、制約(上記の形の有限集合)が与えられたとき、
それの「解」となる代入をともとめることである。代入$S$は、
型変数$\alpha^0, \alpha^1$への型の代入と、
classifier変数$\gamma$へのclassifierの代入とから構成される。
この代入は「最も一般的」であるべきである。(定義の詳細はいまは省略)

\subsection{制約の解消アルゴリズム(前半)}

$t^0=s^0$ と $t^1=s^1$ の形の制約は、普通の型推論で解ける。
それを解いた結果、$\alpha^0$、$\alpha^1$, $\gamma$に対する代入が生じ
る(か、あるいは、「解なし」という結果になる)。

$\Delta \models \longer{t^0}{s^0}$の形の制約は、両方ともが型変数の場合以外は、
簡単に解ける。(その結果として、$t^i=s^i$ の型の制約や、$\Delta \models
\longer{c}{c}$の形の制約を生む可能性があるが、前者は前と同様に解けばよ
く、前者を解いている間にあらたに$\Delta \models \longer{t^0}{s^0}$の形の制約は生じない。)

ここまでの段階で(代入がいくつか出たほか)、残る制約は、以下のものだけである。

\begin{itemize}
\item $\Delta \models \longer{\alpha^0}{\beta^0}$
\item $\Delta \models \longer{c}{c}$
\end{itemize}

ここまでにでてきた代入はすべて、上記の制約たちに適用済みとする。」
(つまり、$\alpha:=\Int$という代入がでてきたら、上の式にある$\alpha$は
すべて$\Int$にしておく。その結果、「代入における左辺にでてくる型変数やclassifier変数」
は、上記の制約たちには、のこっていない。)


\subsection{制約の解消アルゴリズム(後半)}

$\Delta \models \longer{c}{c}$の形の制約を解こう。
($\Delta \models \longer{\alpha^0}{\beta^0}$については次のsectionで
考えるので、ここでは無私する。)

この形の制約たちを、
$\Delta_i \models \longer{c_i}{c'_i}$とすると、
それぞれの $\Delta_i$ はcompatible であるはずなので(ここはあとでチェッ
クが必要)
$\Delta = \Delta_1 \cup \cdots \cup \Delta_n$ という風に全部を合体させ
た上で、$\Delta \models \longer{c_i}{c'_i}$を解けばよい。

(ステップ1: classifier変数の除去)
使われているclassifier変数の1つに着目する。(どれでもよい。)それを
$\gamma$とする。
$\longer{c_i}{\gamma}$ の形の制約 ($i=1,2,\cdots,I$)とを全部あつめる。
$\longer{\gamma}{c'_j}$ の形の制約 ($j=1,2,\cdots,J$)と、
これらを消去して、かわりに、以下の制約を、すべての$(i,j)$に対して追加する。

\[
  \longer{c_i}{c'_j}
\]

これにより、classifier変数は1つ減る。(制約は一般には増えるかもしれない。)
ステップ1を繰返すと、classifier変数はなくなる。

(ステップ2) $\longer{c_1 \uni c_2}{c_3}$ を
$\longer{c_1}{c_3}$ と
$\longer{c_2}{c_3}$ に変換する。

これにより、不等号の左辺にある$\uni$の個数が1つ減る。
ステップ2を繰返すと、不等号の左辺にある$\uni$はなくなる。

ステップ2の繰返しがおわると、制約は、
$\Delta \models \longer{d}{c}$の形になる。

\oishi{ステップ3 は変更する必要あり }\\
(ステップ3) ここで「$d$は atomic」 という仮定をおく。これについては
あとで吟味する.
これは、
「$\longer{d}{c_1 \uni c_2}$ ならば
$\longer{d}{c_1}$または、
$\longer{d}{c_2}$」という内容である。
これらの逆向きは、いつでも成立するので、結局、ここの「ならば」は「同値」といってもよい。

これをもちいて、$\longer{d}{c}$の右辺も分解できて、

\[
  \Delta \models \longer{d_1}{d_1'} \vee \cdots \vee \longer{d_n}{d'_n}
\]

となり、さらに$\Delta$も $\longer{d1}{d2}$の形を「かつ」と「または」で
つないだ形になるはずである。

これは decidable なので、「解があるかどうか」もdecidableである。

\subsection{制約の解消アルゴリズム(後半のおまけ)}

実は、まだ、
$\Delta \models \longer{\alpha^0}{\beta^0}$という制約がのこっていた。

これを最後まで残した(解かなかった)のは、以下の2つの可能性があるからで
ある。これはどうするか?

\begin{itemize}
\item $\alpha^0 = \beta^0$
\item
  $\alpha^0 = \codeT{t}{\gamma1}$,
  $\beta^0 = \codeT{t}{\gamma2}$,
  $\Delta \models \longer{\gamma1}{\gamma2}$
\end{itemize}


%%% Local Variables:
%%% mode: japanese-latex
%%% TeX-master: "master_oishi"
%%% End:
