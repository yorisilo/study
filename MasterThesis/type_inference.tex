\chapter{型推論}
\label{chap:type_inference}

この章では本体系の言語によって書かれたプログラムの型を推論するための型推論アルゴリズムを述べる.

本研究の型推論アルゴリズムは
$\Gamma,~ L,~ \sigma,~ e$ が与えられたとき,$\Gamma \vdash^{L} e : t ;~\sigma$ が成立するような $t$ があるかどうか判定し,その型$t$ を返すものである.

型推論アルゴリズムは主に以下の2ステップから構成する
\begin{itemize}
\item 制約生成:与えられた項に対して,型および classifier に関する制約を返す
\item 制約解消:その得られた制約を解消し,その制約を満たす代入$\Theta$ を返す
\end{itemize}

\section{型システム$T_2$の導入}
制約生成のための形ステム$T_2$を導入する.
これは\ref{chap:type_system}章で与えた型システム$T_1$をトップダウンの制約生成に適した形に変形したものである.
% これは,もとの型システム($T_1$とする)を
% 「トップダウンでの制約生成向け型システム($T_2$とする)」に変形することであたえる.

$T_2$の設計指針は以下のとおりである.
\begin{itemize}
\item $T_1$と$T_2$は「型付けできる」という関係として等価である.
\item $T_2$は結論側の式のトップレベルの形だけで,適用可能な型付け規則が一意に定まる.
  % この性質をterm-orientedと呼ぶ
\item $T_2$は,制約生成をする.
  % (結論側の式の要素は変数として、
  % 「それがこういう形でなければいけない」という条件は、制約の形で「生成」する。)
\end{itemize}

$T_2$の設計にあたって解決すべき問題は subsumption規則である.
すなわち,subsumption 規則 は,どのような項に対しても適用ができるので,
上で述べた一意性が成立しない.
そこで,型付け規則と subsumption 規則 を組み合わせた規則とすることで,問題を解決した.
% subsumption 規則 が出現する場所を限定することができる.

以下で,「var1」等といった表記は,
「型システム$T_1$ の var1規則を subsumption 規則と組み合わせた形に改訂し制約を生成する規則」であるということを表す.
また,型付け規則の右側に記述したConstr;... は(型付け規則を下から上にむけて
使うとき),Constr 以下の制約が生成される,という意味である.

また,型$t_1, t_2$に対する$\longer{}{}$の記号は以下の意味である.

\begin{figure}[H]
  \centering
  \begin{itemize}
  \item $\longer{\codeT{t_1}{\gamma_1}}{\codeT{t_2}{\gamma_2}}$
    は,「$t_1=t_2$ かつ $\longer{\gamma_1}{\gamma_2}$
  \item $\codeT{t}{\gamma}$の形でない$t_1,t_2$に対しては,
    $t_1 = t_2$
  \end{itemize}

  \caption{制約生成における $\longer{}{}$ の意味の定義}
  \label{fig:typeinf_longer_def}
\end{figure}

型システム$T_2$は $\longer{t_1}{t_2}$ の形のまま,制約として生成する.

% 以下のfootnote はもう少しちゃんと書くこと
% \footnote{型推論のプロセスの最中では、$t_1, t_2$ はメタ型変数であり,レベル0の型変数,レベル1の型変数どちらの形かは決定できないので、$\longer{t_1}{t_2}$を上記の意味にしたがって、「ほどく」ことはできない.なので,$\longer{t_1}{t_2}$という形のまま制約として生成する。}.


% (亀山メモ:
% ただ、もしかすると、
% 「レベル0の型変数」と「レベル1の型変数」を最初からわけておく方法もある
% かもしれない。そうすると、上記はとける?)


% \oishi{
% 以下の規則において,Constraint の $\Gamma \models \longer{t}{t'}$ の $\Gamma \models$ は規則を適用した瞬間の $\Gamma$ の情報は制約解消において使わないから必要ない?\\
% 一貫性をもたせるのなら, $t = t'$ の形の Constraint も $\Gamma \models t = t'$ とする必要がある?
% }
%   \\

(var0)

\[
  \infer[Constr;~\Gamma \models \longer{t}{t'}]
  {\Gamma \vdash x:t;~\sigma}
  {(x:t') \in \Gamma
  }
\]

(var1)

\[
  \infer[Constr;~\Gamma \models \longer{\gamma}{\gamma'}]
  {\Gamma \vdash^{\gamma} u:t;~\cdot}
  {(u:t)^{\gamma'} \in \Gamma
  }
\]

(const)

\[
  \infer[Constr;~\Gamma \models \longer{t}{t^c}]
  {\Gamma \vdash^{L} c:t;~\sigma}
  {}
\]

(app0)

\[
  \infer[Constr;~\Gamma \models \longer{t}{t_1}]
  {\Gamma \vdash^\gamma e_1 \, e_2 : t;~\cdot}
  {\Gamma \vdash^\gamma e_1 : \funT{t_2}{t_1}{\sigma};~\cdot
    &\Gamma \vdash^\gamma e_2 : t_2;~\cdot
  }
\]

(app1)

\[
  \infer[Constr;~\Gamma \models \longer{t}{t_1}]
  {\Gamma \vdash e_1 \, e_2 : t;~\sigma}
  {\Gamma \vdash e_1 : t_2 \to t_1;~\sigma
    &\Gamma \vdash e_2 : t_2;~\sigma
  }
\]

(lambda0)

\[
  \infer[Constr;~t=\funT{t_1}{t_2}{\sigma'}]
  {\Gamma \vdash \lambda x.e : t;~\sigma}
  {\Gamma,~x:t_1 \vdash e : t_2;~\sigma'}
\]

(lambda1)
% \oishi{
% コードの中で,shift0/reset0 は使わないので, $\sigma$ は $\epsilon$ なはず.
% なので,$\sigma$ は $\epsilon$ としておいてもよい
% }

\[
  \infer[Constr;~t=\funT{t_1}{t_2}{}]
  {\Gamma \vdash^\gamma \lambda u.e : t;~\cdot}
  {\Gamma,~(u:t_1)^\gamma \vdash^\gamma e : t_2;~\cdot}
\]

(if)

\[
  \infer[Constr;~(none)]
  {\Gamma \vdash^L
    \textbf{if}~e_1 \textbf{then}~e_2 \textbf{else}~e_3 ~:~t; ~ \sigma}
  {\Gamma \vdash^L e_1 : \Bool;~\sigma
    &\Gamma \vdash^L e_2 : t;~\sigma
    &\Gamma \vdash^L e_3 : t;~\sigma
  }
\]

(code-lambda)

\[
  \infer[Constr;~\Gamma \models \longer{t}{\codeT{t_1 \to t_2}{\gamma}}]
  {\Gamma \vdash \underline{\lambda}x.e ~:~t;~\sigma}
  {\Gamma,\longer{\gamma'}{\gamma},x:\codeT{t_1}{\gamma'}
    \vdash e : \codeT{t_2}{\gamma'};~\sigma
  }
\]

(code-let)

\[
  \infer[Constr;~\Gamma \models \longer{t}{\codeT{t'}{\gamma_0}}]
  {\Gamma \vdash \cLet~ x = e_0~ \cIn~ e_1 : t ; \sigma}
  { \Gamma \vdash e_0 : \codeT{t'}{\gamma_0} ; \sigma
    &\Gamma, \longer{\gamma_1}{\gamma_0}, x: \codeT{t'}{\gamma_1} \vdash e_1 : \codeT{t'}{\gamma_1} ; \sigma
  }
\]

(reset0)

\[
  \infer[Constr;~\Gamma \models \longer{t}{\codeT{t'}{\gamma}}]
  {\Gamma \vdash \Resetz~e ~:~ t; ~\sigma}
  {\Gamma \vdash e:\codeT{t'}{\gamma};~\codeT{t'}{\gamma},\sigma
  }
\]

(shift0)

% \[
%   \infer[Constr;~\Gamma \models \longer{t}{\codeT{t_1}{\gamma_1}},~ t_2 =\codeT{t_0}{\gamma_0},~ \Gamma \models \longer{\gamma_1}{\gamma_0}]
%   {\Gamma \vdash \shiftz{k}{e} : t~;~ t_2,\sigma}
%   {\Gamma,~k:\contT{\codeT{t_1}{\gamma_1}}{\codeT{t_0}{\gamma_0}}{\sigma}
%     \vdash e : \codeT{t_0}{\gamma_0} ; \sigma
%   }
% \]

\[
  \infer[Constr;~\Gamma \models \longer{t}{\codeT{t_1}{\gamma_1}},~ \Gamma \models \longer{\gamma_1}{\gamma_0},~ \sigma_x = \codeT{t_0}{\gamma_0}, \sigma]
  {\Gamma \vdash \shiftz{k}{e} : t~;~ \sigma_x}
  {\Gamma,~k:\contT{\codeT{t_1}{\gamma_1}}{\codeT{t_0}{\gamma_0}}{\sigma}
    \vdash e : \codeT{t_0}{\gamma_0} ; \sigma
  }
\]

(throw)
% \oishi{
% throw0 規則にのみ $\sigma$ part の subsumption 規則を適用すればおk
% throw0 は今後一考する必要あり
% }
% \\

\[
  \infer[Constr;~ {\scriptstyle \Gamma \models \longer{t}{\codeT{t_0}{\gamma_2}},~ (\Gamma,~ k:~t') \models \gamma_2 \ord \gamma_0,~  t' = \contT{\codeT{t_1}{\gamma_1}}{\codeT{t_0}{\gamma_0}}{\sigma},~ \Gamma \models \gamma_1 \uni \gamma_2 \ord \gamma'}]
  {\Gamma,~ k:t'
    \vdash \throw{k}{v} : t ; \sigma}
  {\Gamma,~ k:t'
    \vdash v : \codeT{t_1}{\gamma'} ; \sigma
  }
\]

(code)

\[
  \infer[Constr;~\longer{t}{\codeT{t_1}{\gamma}}]
  {\Gamma \vdash \code{e} : t;~\sigma}
  {\Gamma \vdash^\gamma e : t_1;~\cdot}
\]


この新しい型システム$T_2$ は$T_1$と同じ型付けをあたえる.
% \oishi{今のところ$T_1$ の$\sigma$-part の subsumption は reset0 に限定していないので,$T_1$ と$T_2$ は同じ型付けになるはず.}

\section{制約生成}

制約生成では,与えられた項$e$に対して,
$T_2$を型付け規則を下から上の向きに適用することで,制約を生成する.
生成する制約は,それぞれの型付け規則の右側に$Constr;~ ...$ と書いてあるものである.
それらの制約を,型付けに従って生成していくことで,制約の集合が生成される
% (もちろん、途中で、つまってしまったら、型推論は失敗する。)

$T_2$の型付け規則を適用する時,型付け規則の下の型判断に存在せず,上の型判断,あるいは制約中にのみ存在する型やclassifierがあるときは,これらを新しい型変数やclassifier変数として生成する.
なお,code-lambda規則での新しいclassifier は,classfier変数ではなく,classifier定数とする\footnote{制約生成のでは,変数と定数に違いはないが,制約を解消するとき,cassifier定数に対する代入はしない.という違いがある}.

\begin{framed}
  制約生成アルゴリズム:
  \begin{itemize}
    \setlength{\itemsep}{-5pt}
  \item $\Gamma$ : 型文脈
  \item $e$ : 項
  \item $t$ : 型
  \item $\sigma$ : answer type の列
  \item $L$ : レベル(現在レベル 0, または コードレベル $\gamma$)
  \item $C$ : 制約
  \end{itemize}

  \begin{description}
  \item[入力] $\Gamma,~ e,~ t,~ \sigma,~ L$
  \item[出力] $C$
  \end{description}

  % \begin{enumerate}
  % \item $T_2$を「下から上」の向きに適用して制約$C$を生成する.
  % \end{enumerate}
  $\Gamma \vdash^L e:t;~ \sigma$ から始めて,$T_2$ のおける型付けを下から上に向かって行う.
  型付けがどこかで失敗するとき,制約生成は失敗する.
  型付けが成功したとき,生成された制約$C_i$ の和集合を$C$としてそれを返す.
\end{framed}

% \begin{algorithm}
%   \caption{Euclid’s algorithm}
%   \label{euclid}
%   \begin{algorithmic}[1] % The number tells where the line numbering should start
%     \Procedure{Euclid}{$a,b$} \Comment{The g.c.d. of a and b}
%     \State $r \gets a \bmod b$
%     \While{$r\not=0$} \Comment{We have the answer if r is 0}
%     \State $a \gets b$
%     \State $b \gets r$
%     \State $r \gets a \bmod b$
%     \EndWhile\label{euclidendwhile}
%     \State \textbf{return} $b$\Comment{The gcd is b}
%     \EndProcedure
%   \end{algorithmic}
% \end{algorithm}

\section{制約の解消}
% 制約生成が成功したとき,以下の性質が成立するはずである.

入力を$\Gamma, L, e, t,\sigma$ として,前章の制約生成アルゴリズムを走
らせ,それが成功して$C$という制約を生成したとき,
\begin{itemize}
\item
  $T_1$ で $\Gamma \vdash^L e: t; ~ \sigma$ が導出可能ならば,
  $C$ を満たす解が存在し,
\item
  $C$ を満たす解が存在すれば、ある代入$\Theta$に対して,
  $\Theta(\Gamma) \vdash^L \Theta(e): \Theta(t) ;~ \Theta(\sigma)$ が $T_1$で導出可能である.
\end{itemize}
という性質が成立する.

つまり,型推論問題を得られた制約を成立するような代入$\Theta$が存在するかどうかという問題に帰着できる.
% (これが成立すれば、もともとの型推論問題を、型制約の解消問題に帰着できたことになる。)

% というわけで、制約の解消をはじめよう。
制約は以下の文法で与えられたものの有限集合である.

\begin{figure}[H]
  \centering
  \begin{align*}
    \Gamma &\models \longer{t^0}{t^0} \\
    \Gamma &\models \longer{c}{c} \\
    \sigma &= \sigma \\
    t^0 &= t^0 \\
    t^1 &= t^1
  \end{align*}

  ただし、ここで $t_0$, $t_1$, $c$, $\sigma$ は以下の文法で定義される。

  \begin{align*}
    t^0      & ::= \alpha^0 \mid \Int \mid \Bool \mid \funT{t^0}{t^0}{\sigma} \mid \codeT{t^1}{c} \\
    t^1      & ::= \alpha^1 \mid \Int \mid \Bool \mid \funT{t^1}{t^1}{} \\
    c        & ::= \gamma_x \mid d \mid c \uni c \\
    \sigma   & ::= \sigma_x \mid \epsilon \mid t^0, \sigma
  \end{align*}

  \caption{制約の定義}
  \label{fig:constr_def}
\end{figure}

$t_0$, $t_1$, $c$, $\sigma$ はそれぞれ、
レベル0型,
レベル1型,
レベル0型の列,
classifierをあらわす表現(メタ変数),
answer type の列($\sigma$-part)
である.また,
$\alpha^i$はレベル$i$の型変数,$\gamma_x$はclassifier変数,$\sigma_x$は$\sigma$-partの変数($\sigma$変数)である.
また,$d$は,固有変数条件をもつclassifier変数のことであり,
型推論のあいだは,これは実質的に定数として扱われる\footnote{つまり,classifier変数$\gamma$に対しては代入するが,$d$に対しては代入しない}.

また,$\Gamma$は,一般の型文脈であるが,不要な情報を落として以下の形にする.
\begin{align*}
  \Delta ::= \emptyset
  \mid \Delta, (\longer{d}{c})
  \mid \Delta, (x : t)
  \mid \Delta, (u : t)^{\Delta}
\end{align*}

% (左辺は、固有変数なので、classifier定数である。右辺は一般のclassifier
% 式がなんでも来る可能性がある。)

制約の解消とは,制約が与えられたとき,その解となる代入$\Theta$を求めることである.
代入$\Theta$は、型変数$\alpha^0,~ \alpha^1$への型の代入と,
classifier変数$\gamma_x$へのclassifierの代入とから構成される.

% $\sigma$変数 $\sigma_x$ へ answer type の列の代入もあるはず.$\sigma_x$への代入についてよくわかってない

% この代入は「最も一般的」であるべきである。(定義の詳細はいまは省略)
% principal type

\subsection{typeinf1: 制約の解消アルゴリズム(前半)}
$t^0=s^0$ と $t^1=s^1$ の形の制約は,単一化アルゴリズムによって解くことができる.
% 単一化アルゴリズムについての説明が必要っぽい
それを解いた結果,$\alpha^0$,$\alpha^1$, $\gamma_x$に対する代入が生じる.
もしくは「解なし」という結果となる.
また,$\sigma_1 = \sigma_2$ の形の制約も,単一化アルゴリズムで解くことができ,
それを解いた結果,$\sigma_x$ に対する代入が生じる.もしくは「解なし」という結果となる.

$\longer{\sigma}{\sigma}$ の形の制約も,単一化アルゴリズムによって解くことができる.
それを説いた結果,$\sigma$変数 $\sigma_x$に対する代入や $\longer{t^0}{t^0}$の形の制約が生成される.
もしくは「解なし」という結果となる.

$\Delta \models \longer{t^0}{s^0}$の形の制約は,両方ともが型変数の場合以外は,
簡単に解ける.
(その結果として、$t^i=s^i$ の型の制約や,
$\Delta \models \longer{c}{c}$の形の制約を生む可能性があるが,
前者は前と同様に解けばよく,
前者を解いている間にあらたに$\Delta \models \longer{t^0}{s^0}$の形の制約は生じない.)

ここまでの段階で残る制約は,以下のものだけである.

\begin{itemize}
\item $\Delta \models \longer{\alpha^0}{\beta^0}$
\item $\Delta \models \longer{c_1}{c_2}$
% \item $\Delta \models \longer{\sigma_1}{\sigma_2}$
\end{itemize}

ここまでに出てきた代入はすべて,上記の制約に適用済みとする.
つまり,$\alpha:=\Int$という代入がでてきたら,制約中の$\alpha$はすべて$\Int$にしておく.
その結果,「代入における左辺にでてくる型変数やclassifier変数」は,上記の制約には,残っていない.

\begin{oframed}
  制約解消アルゴリズム(前半) unify1
  \begin{itemize}
    \setlength{\itemsep}{-5pt}
  \item $C$ : 制約
  \item $\Theta$ : 代入
  \end{itemize}

  \begin{description}
  \item[入力] $C$, $\Theta$
  \item[出力] $\Theta'$ または,「単一化失敗」
    ここで,$\Theta'$ は$\Theta$に$C$を解いて得られる代入を追加したもの
  \end{description}

  % unify1($C, []$) を呼び出す
  \begin{description}           % \oishi{% $longer{\sigma}{\sigma}$ の形についてはあとで考える}
  \item[1] $C$の中に$t^0=s^0$か$t^1=s^1$か$\sigma_0 = \sigma_1$か$\Delta \models \longer{t^0}{s^0}$の形の制約がなければ,代入$\Theta$ を返す.
  \item[2] $C$から$t^0=s^0$ か $t^1=s^1$ か $\sigma_0 = \sigma_1$ の形の制約を選び,それを$A = B$とする.$C_1 = C - \{A = B\}$ とする.
    \begin{description}
    \item[2-1] $A = B$ のとき,unify1($C_1, \Theta$)を呼び出す.
    \item[2-2] $A \neq B$ で $A$ が型変数($\sigma$変数)のとき,
      \begin{description}
      \item[2-2-1] 型 $B$ に $A$ が現れるなら,「単一化失敗」を返す.
      \item[2-2-2] 型 $B$ に $A$ が現れないなら,$\Theta_1 = [A := B]$とし,unify1($\Theta_1(C_1), \Theta(\Theta_1))$を呼び出す.
      \end{description}
    \item[2-3] $A \neq B$ で $B$ が型変数($\sigma$変数)のとき,$A$ と $B$ を入れ替えて,2-2へ
    \item[2-4] $A \neq B$ で $A = \funT{A_1}{A_2}{}$,$B = \funT{B_1}{B_2}{}$ のとき, $C_2 = C_1 + \{A_1 = B_1, A_2 = B_2\}$ とし, unify1($C_2, \Theta$)を呼び出す
    \item[2-5] $A \neq B$ で $A = \funT{A_1}{A_2}{\sigma_1}$,$B = \funT{B_1}{B_2}{\sigma_2}$ のとき, $C_2 = C_1 + \{A_1 = B_1, A_2 = B_2, \sigma_1 = \sigma_2\}$ とし, unify1($C_2, \Theta$)を呼び出す
    \item[2-6] $A \neq B$ で $A = \codeT{A_1}{c_1} , B = \codeT{B_1}{c_2}$ のとき,$C_2 = C_1 + \{A_1 = B_1, c_1 = c_2\}$ とし, unify1($C_2, \Theta$)を呼び出す
    \item[2-7] $A \neq B$ で ともに$\sigma$変数のときつまり,$A = \sigma_x , B = \sigma_y$ のとき,その制約は残す.
    \item[2-8] $A \neq B$ で $A = \sigma_x , B = \cdot$ のとき,$C_2 = C_1 + \{\sigma_x = \cdot\}$ とし, unify1($C_2, \Theta$)を呼び出す
    \item[2-9] $A \neq B$ で $A = \sigma_x , B = t_1, \sigma_2$ のとき,$C_2 = C_1 + \{\sigma_x = (t_z,\sigma_2),~ \longer{t_z}{t_1} \}$ とし, unify1($C_2, \Theta$)を呼び出す
    \item[2-10] $A \neq B$ で $A = (t_1, \sigma_1) , B = (t_2, \sigma_2)$ のとき,$C_2 = C_1 + \{\longer{t_1}{t_2}, \sigma_1 = \sigma_2 \}$ とし, unify1($C_2, \Theta$)を呼び出す
    \item[2-11] 上記のいずれでもないとき「単一化失敗」を返す.
    \end{description}
  \item[1-3] $C$から$\Delta \models \longer{t^0}{s^0}$ の形の制約を選び,それを$c$とする.$C_1 = C - \{c\}$ とする.
    \begin{description}
    \item[1-3-1] $t^0,s^0$のどちらかが型変数でないとき,詳細は省くが,代入$\theta$ が生成される;
      unify1($C_1, \Theta(\theta)$)を呼び出す
    \item[1-3-2] $t^0,s^0$がともに型変数のときつまり,$c$ が $\Delta \models \longer{\alpha^0}{\beta^0}$のとき,その制約を残す
    \end{description}
  % \item[1-4] $c$ が $\Delta \models \longer{c}{c}$ の形のとき,unify2($\Theta, C$)を呼び出す
  \end{description}
\end{oframed}


% \begin{oframed}
%   制約解消アルゴリズム(前半) typeinf1
%   \begin{itemize}
%     \setlength{\itemsep}{-5pt}
%   \item $C$ : 制約
%   \item $\Theta$ : 代入
%   \end{itemize}

%   \begin{description}
%   \item[入力] $C$ $\Theta$
%   \item[出力] $C$ $\Theta$
%   \end{description}

%   \begin{description}
%   \item[0] typeinf1($C, []$) を呼び出す
%   \item[1] $C$から任意の制約$c$を選び, $C_1 = C - \{c\}$ とする.
%     \begin{description}
%     \item[1-1] $c$ が $t^0=s^0$ の形のとき,
%       $\Theta$ に unify($t^0, s^0$)を加える (unify によって$\alpha^0,~ \alpha^1,~ \gamma$ に対する代入が生じる);
%       typeinf1($C_1, \Theta$) を呼び出す
%     \item[1-2] $c$ が $t^1=s^1$ の形のとき,
%       $\Theta$ に unify($t^1, s^1$)を加える (unify によって$\alpha^0,~ \alpha^1,~ \gamma$ に対する代入が生じる);
%       typeinf1($C_1, \Theta$) を呼び出す
%     \item[1-3] $c$ が $\Delta \models \longer{t^0}{s^0}$ の形のとき,
%       \begin{description}
%       \item[1-3-1] $t^0,s^0$のどちらかが型変数でないとき,詳細は省くが,制約$\theta$ が生成され,それを$\Theta$ に加える;
%         typeinf1($C_1, \Theta$)を呼び出す
%       \item[1-3-2] $t^0,s^0$がともに型変数のときつまり,$c$ が $\Delta \models \longer{\alpha^0}{\beta^0}$のとき,typeinf2($\Theta, C$)を呼び出す
%       \end{description}
%     \item[1-4] $c$ が $\Delta \models \longer{c}{c}$ の形のとき,typeinf2($\Theta, C$)を呼び出す
%     \end{description}
%   \end{description}
% \end{oframed}

% \begin{algorithm}
%   \caption{制約解消アルゴリズム(前半)}

%   \begin{algorithmic}[1] % The number tells where the line numbering should start
%     \Procedure{infertype1}{$C, \Theta = []$}
%     \ForEach{$c \in C$}
%     \State $C_1 \gets C - \{c\}$
%     \If{$c$ が $t^0=s^0$ の形}
%     \State $\Theta$ に unify($t^0, s^0$)を加える \Comment{unify によって$\alpha^0,~ \alpha^1,~ \gamma$ に対する代入が生じる}
%     \State infertype1($C_1, \Theta$)
%     \ElsIf{$c$ が $t^1=s^1$の形}
%     \State $\Theta$ に unify($t^1, s^1$)を加える \Comment{unify によって$\alpha^0,~ \alpha^1,~ \gamma$ に対する代入が生じる}
%     \State infertype1($C_1, \Theta$)
%     \ElsIf{$c$ が $ \Delta \models \longer{t^0}{s^0}$ の形}
%     \If{$t^0$と$s^0$ が型変数でない}
%     \State
%     \Else
%     \State typeinfer2($C, \Theta$) \Comment{$c$が$\longer{\alpha^0}{\beta^0}$の形}
%     \EndIf
%     \Else
%     \State typeinfer2($C, \Theta$) \Comment{$c$が$\longer{c}{c}$の形}
%     \EndIf
%     \EndFor
%     \EndProcedure
%   \end{algorithmic}
% \end{algorithm}

\subsection{typeinf2: 制約の解消アルゴリズム(後半)}
ここまでの段階で残る制約は,上記で述べたように以下のものだけである.
\begin{itemize}
\item $\Delta \models \longer{\alpha^0}{\beta^0}$
\item $\Delta \models \longer{c_1}{c_2}$
% \item $\Delta \models \longer{\sigma_1}{\sigma_2}$
\end{itemize}

まず,$\Delta \models \longer{c}{c}$の形の制約の解消について述べる

$c$とは以下のような式である.
\begin{align*}
  c & ::= d \mid \gamma_x \mid c \cup c
\end{align*}

ここで,$d$ は定数 (固有変数条件をみたすclassifier変数), $\gamma_x$は
classifierをあらわす変数、$c$は classifier式。
また,型推論の対象となった式の一番外側でのclassifierも定数と考えられ
るので,それを$d_0$とする.

% この形の制約たちを,
% $\Delta_i \models \longer{c_i}{c'_i}$とすると,
% それぞれの$\Delta_i$ は両立的であるので,
% $\Delta = \Delta_1 \cup \cdots \cup \Delta_n$として,
% $\Delta \models \longer{c_i}{c'_i}$を解けばよい.

解きたい制約は,以下の形である.
\[
  \Delta \models C
\]

ただし, $\Delta$ は,$d \ge c$ の形の不等式を有限個「かつ」でつなげた
ものである.
また,すべての式$d_i$ ($i>0$)に対して$d_i \ge d_o$が $\Delta$に含まれ
ているとしてよい.
また,$C$は,$c \ge c$ の形の不等式を「かつ」でつなげたものである.

なお,仮定$\Delta$ は,(code-lambda規則等で) 固有変数を導入する際の
$d \ge c$という形の不等式のみがはいっているので,左辺は必ず $d$である.

制約解消問題 $ \Delta \models C $ に対して,
$[\gamma_x := c,\cdots,\gamma_x:=c]$ という形の代入$\Theta$で,
$ \Delta\Theta \models C\Theta $ が(順序に関する規則のみで)導けるとき,
この$\theta$を解と呼ぶ.そのような$\Theta$が存在しないとき,解がないと
いう.

% それぞれの $\Delta_i$ はcompatible であるはずなので(ここはあとでチェッ
% クが必要)
% $\Delta = \Delta_1 \cup \cdots \cup \Delta_n$ という風に全部を合体させ
% た上で、$\Delta \models \longer{c_i}{c'_i}$を解けばよい。

以下で,classifier 式 $c$ が join-irreducible であることを述べる.
\begin{theo}[join-irreducible]
  \label{theo:join-ir}
  $c$はほかのどのような要素 $c_1$, $c_2$に対しても,$c_1 \cup c_2 \ge c$ ならば,
  $c_1 \ge c$ または $c_2 \ge c$ である
\end{theo}

\begin{proof}[\ref{theo:join-ir}の証明]
  $\Delta$ は $d\ge c$の形だけからなる.
  $\Delta$から$c_1 \cup c_2 \ge d$を導く証明のサイズに関する帰納法で,
  $\Delta$から$c_1 \ge d$または$c_2 \ge d$が導けることが言える.
\end{proof}

主に以下の右辺の$\uni$の除去,左辺の分解,classifier変数の除去というアイデアをもとにclassifier不等式の制約解消アルゴリズムを構成する.

\begin{description}
\item[(右辺の$\uni$の除去)]\mbox{}\\
  $\longer{c_1}{c_2 \uni c_3}$ を
  $\longer{c_1}{c_2}$ と
  $\longer{c_1}{c_3}$ に変換する.

  これにより,不等号の右辺にある$\uni$の個数が1つ減る.
  したがって,これを繰り返すと不等号の右辺にある$\uni$はなくなる.

  % ステップ2の繰り返しがおわると,制約は,
  % $\Delta \models \longer{c}{d}$の形になる.


  % \oishi{ステップ3 は変更する必要あり ここで「$d$は atomic」 という仮定をおく。これについてはあとで吟味する.}\\
  % \oishi{以下は 定理 として証明が必要}
\item[(左辺の分解)] \mbox{}\\
  定理 \ref{theo:join-ir} join-irreducible の仮定より
  「$\longer{c_1 \uni c_2}{d}$ ならば
  $\longer{c_1}{d}$または,
  $\longer{c_2}{d}$」ということが言える.

  これを用いて、$\longer{c}{d}$の左辺を分解することができ,

  $C - \{(c_1 \cup c_2 \ge d)\} \cup \{(c_1 \ge d)\}$に対して再帰し,
  さらに,
  $C - \{(c_1 \cup c_2 \ge d)\} \cup \{(c_2 \ge d)\}$に対して再帰し,
  それらの解の両方を解として返す.(一意的には解けない.解は一般的には複数ある.)

  % \[
  %   \Delta \models \longer{d_1}{d_1'} \vee \cdots \vee \longer{d_n}{d'_n}
  % \]

  % となる.
  % さらに$\Delta$も $\longer{d1}{d2}$の形を「かつ」と「または」で
  % つないだ形になる.

  % これは decidable なので、「解があるかどうか」もdecidableである。

\item[(classifier変数の除去)]\mbox{}\\
  制約中のclassifier変数の1つに着目し,それを$\gamma_x$とし,
  $C$の中から以下の形の不等式
  をすべて列挙する。

  \begin{align*}
    c_i & \ge \gamma_x  ~~ & \text{for}~ i=1,2,,...m\\
    \gamma_x & \ge d_j  ~~ & \text{for}~ j=1,2,,...n
  \end{align*}

  ただし $m$や$n$は$0$の可能性があり,また,すべての$c_i$ は $\gamma_x$は含まない.

  \begin{itemize}
  \item $m=0$かつ$n>0$のとき,$\gamma_x:=d_1 \cup \cdots \cup d_n$という
    代入を$\theta$に追加し,$C$からは$\gamma_x$に関する不等式をすべて取
    り除いて再帰する.
  \item $m>0$かつ$n=0$のとき,$\gamma_x:=d_0$という
    代入を$\theta$に追加して、$C$からは$\gamma_x$に関する不等式をすべて取
    り除いて再帰する.
  \item $m>0$かつ$n>0$のとき,
    $C$から$\gamma_x$に関する不等式をすべて取り除いた上で,以下の不等式を
    追加する.
    \[
      c_i \ge d_j   ~~\textrm{(すべての}~i,j~\textrm{に対して)}
    \]
    さらに,$\gamma_x:=d_1 \cup \cdots \cup d_n$という代入を$\Theta$に追加
    した上で再帰する.
  \end{itemize}
  % $\longer{c_i}{\gamma}$ の形の制約 ($i=1,2,\cdots,I$)と
  % $\longer{\gamma}{c'_j}$ の形の制約 ($j=1,2,\cdots,J$)をすべて消去し,
  % 以下の制約を,すべての$(i,j)$に対して追加する.

  % \[
  %   \longer{c_i}{c'_j}
  % \]

  これにより,classifier変数は1つ減る.% (制約は一般には増えるかもしれない。)
  したがって,これを繰り返すとclassifier変数はなくなる.

\end{description}

$\Delta \models \longer{\alpha^0}{\beta^0}$の形の制約は残す.
% 次に$\Delta \models \longer{\alpha^0}{\beta^0}$の形の制約の解消について述べる.
% この制約は以下の2つの可能性がある.

% \begin{itemize}
% \item $\alpha^0 = \beta^0$
% \item
%   $\alpha^0 = \codeT{t}{\gamma1}$,
%   $\beta^0 = \codeT{t}{\gamma2}$,
%   $\Delta \models \longer{\gamma1}{\gamma2}$
% \end{itemize}


以下の制約解消アルゴリズムは$\Delta \models \longer{c}{c}$の形の制約についての制約解消について記述する.
% $\Delta \models \longer{\alpha^0}{\beta^0}$の形の制約は残すだけであるので,特に記述しない.

\begin{oframed}
  制約解消アルゴリズム(後半) unify2
  \begin{itemize}
    \setlength{\itemsep}{-5pt}
  \item $C$ : 制約
  \item $\Theta$ : 代入
  \end{itemize}

  \begin{description}
  \item[入力] $C$, $\Theta$ ($\Theta$ には $\Delta \models \longer{c}{c}$ の形の制約のみがある)
  \item[出力] $C$, $\Theta$
  \end{description}

  \begin{description}
  \item[0] $C$ が空であるならば,$\Theta$ を返して終了する.
  \item[1] $C$から任意の制約$c$を選び, $C' = C - \{c\}$ とする.
    \begin{description}
    \item[1-1] その制約が$c \ge c$ (両辺が同じ不等式)ならば,unify2($C',~ \Theta$)を呼び出す
    \item[1-2] その制約が $d_1 \ge d_2$ (両辺が異なるclassifier定数)のとき,
      さらに,$\Delta$のもとで、$d_1 \ge d_2$ が導けるのであれば,
      unify2($C',~ \Theta$)を呼び出す.
      導けないのであれば,「制約解消失敗」を返して終了する.
    \item[1-3] その制約が $c_1 \ge c_2 \cup c_3$ ならば,
      $C'' = C' + \{c_1 \ge c_2,~ c_1 \ge c_3\}$とし,
      unify2($C'',~ \Theta$)を呼び出す.
    \item[1-4] その制約が $(\cdots \cup \gamma_x \cup \cdots) \ge \gamma_x$ ならば,
      unify2($C',~ \Theta$)を呼び出す.
    \item[1-5] その制約が $c_1 \cup c_2 \ge d$ ならば,
      定理 \ref{theo:join-ir} join-irreducible の仮定より
      $(c_1 \cup c_2 \ge d)$ は,$e_1 \ge d$ または $e_2 \ge d$であるので,
      $C_1 = C' + \{ e_1 \ge d \}$ とし,unify2($C_1,~ \Theta$)を呼び出し,
      $C_2 = C' + \{ e_2 \ge d \}$ とし,unify2($C_2,~ \Theta$)を呼び出し,
      それらの解の両方を解として返す.
    \end{description}
  \item[2] 1-1 から 1-5 のいずれのケースにも当てはまらない場合かつ,$C$ が空でないとき,残っている classifier 変数 $\gamma_x$ を一つ取り出し,$C$の中から,以下の形の不等式すべてを列挙する.
    \begin{align*}
      c_i & \ge \gamma_x  ~~ & \text{for}~ i=1,2,,...m\\
      \gamma_x & \ge d_j  ~~ & \text{for}~ j=1,2,,...n
    \end{align*}
    \begin{description}
    \item[2-1] $m=0$かつ$n>0$の場合,
      $\gamma_x:=d_1 \cup \cdots \cup d_n$という代入を$\Theta$に追加したものを$\Theta'$とし,
      $C$から$\gamma_x$に関する不等式をすべて取り除いたものを $C'$ とし,unify2($C',~\Theta',$)を呼び出す.
    \item[2-2] $m>0$かつ$n=0$の場合,
      $\gamma_x:=d_0$という代入を$\Theta$に追加したものを$\Theta'$とし,
      $C$から$\gamma_x$に関する不等式をすべて取り除いたものを $C'$ とし,unify2($C',~\Theta',$)を呼び出す.
    \item[2-3] $m>0$かつ$n>0$の場合,
      $\gamma_x:=d_1 \cup \cdots \cup d_n$という代入を$\Theta$に追加したものを$\Theta'$とし,
      $C$から$\gamma_x$に関する不等式をすべて取り除いた上で,以下の不等式を
      追加したものを$C'$とし,unify2($C',~\Theta',$)を呼び出す.
      \[
        c_i \ge d_j   ~~\textrm{(すべての}~i,j~\textrm{に対して)}
      \]
    \end{description}
  \end{description}
  % \begin{description}
  % \item[1] $C$から任意の制約$c$を選び, $C' = C - \{c\}$ とする.
  %   \begin{description}
  %   \item[1-1] $c$ が $\Delta \models \longer{c}{c}$の形のとき,
  %     \begin{description}
  %       \item[1-1-1: ステップ1] classifier 変数 $\gamma$ を選び,$\longer{c_i}{\gamma}$ と$\longer{\gamma}{c'_j}$ をすべて消去し,制約に$\longer{c_i}{c'_j}$ を加える.すべての classifier 変数がなくなるまでステップ1を繰り返し,ステップ2へ
  %       \item[1-1-2: ステップ2] $\longer{c_1}{c_2 \uni c_3}$ を$\longer{c_1}{c_2}$ と $\longer{c_1}{c_3}$ に変換する.右辺に$\uni$ がなくなるまで,ステップ2を繰り返すことによって,制約は, $\longer{c}{d}$となる.ステップ3へ
  %       \item[1-1-3: ステップ3] $\longer{c_1 \uni c_2}{d}$ を $\longer{c_1}{d} \vee \longer{c_2}{d}$ に変換する.その制約を$\Theta$に加え,unify2($C', \Theta$)を呼び出す
  %     \end{description}
  %   % \item[1-2] $c$ が $\Delta \models \longer{\alpha^0}{\beta^0}$の形のとき,
  %   %   $C'_1$($C'$ に$\alpha^0 = \beta^0$を加えたもの),\\
  %   %   $C'_2$($C'$ に$\alpha^0 = \codeT{t}{\gamma1},~ \beta^0 = \codeT{t}{\gamma2},~ \Delta \models \longer{\gamma1}{\gamma2}$を加えたもの)とする.\\
  %   %   unify2($C'_1, \Theta_1$)とunify2($C'_2, \Theta_2$)とを呼び出す.
  %   \end{description}
  %   \item[1-2] 解として,$C, \Theta$と $\Delta \models \longer{\alpha^0}{\beta^0}$の形の制約を返す.
  % \end{description}
\end{oframed}

上記のアルゴリズムを繰り返すと,制約解消が途中で失敗しない限り,$\gamma_x$ 変数がすべて消去され,制約$C$が空になり,制約を満たす解の有無が判定可能となる.

% \oishi{ 固有変数d の条件のチェックが必要. Γの中の d たちの順序 と 制約で出てきた順序についてのチェック }

% \subsection{制約の解消アルゴリズム(全体)}

% \begin{framed}
%   制約解消アルゴリズム(全体) typeinf
%   \begin{itemize}
%     \setlength{\itemsep}{-5pt}
%   \item $C$ : 制約
%   \item $\Theta$ : 代入
%   \end{itemize}

%   \begin{description}
%   \item[入力] $\Gamma,~ e,~ t,~ \sigma,~ L$
%   \item[出力] $C$
%   \end{description}

%   \begin{enumerate}
%   \item $T_2$を「下から上」の向きに適用して制約$C$を生成する.
%   \end{enumerate}
% \end{framed}

% 以上によって,制約を解消することで,$\Gamma,~ L,~ \sigma,~ e$ が与えられたとき,$\Gamma \vdash^{L} e : t ;~\sigma$ が成立するような $t$ があるかどうか判定し,その型$t$ のprincipal type が分かる

% \section{型推論アルゴリズム}

%%% Local Variables:
%%% mode: japanese-latex
%%% TeX-master: "master_oishi"
%%% End:
