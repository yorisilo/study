%
% Type inference algorithm draft (for Oishi System)
%
\documentclass[dvipdfmx]{jsarticle}

\usepackage[dvipdfmx]{graphicx,color}
\addtolength{\topmargin}{-2cm}
\addtolength{\textwidth}{3cm}
\addtolength{\oddsidemargin}{-1.5cm}


\usepackage{theorem}
\usepackage{amsmath,amssymb}
\usepackage{ascmac}
%\usepackage{mathtools}
\usepackage{proof}
%\usepackage{stmaryrd}
\usepackage{listings,jlisting}
\usepackage{here}
%\usepackage{verbatim}
%\usepackage{framed}
%\usepackage{algorithm}
%\usepackage{algpseudocode}
%
%\lstset{
%  basicstyle=\ttfamily,
%  columns=fullflexible,
%  keepspaces=true,
%}

%\newenvironment{vq}
%{%begin
%  \VerbatimEnvironment \begin{screen} \begin{quote} \begin{Verbatim}
%      }
%      {%end
%      \end{Verbatim} \end{quote} \end{screen}
%}
%\newtheorem{theorem}{theorem}[section]

%\definecolor{DarkGreen}{rgb}{0,0.5,0}
\definecolor{Magenta}{rgb}{1.0, 0.0, 1.0}

\newcommand\cbra{\texttt{<}}
\newcommand\cket{\texttt{>}}

\newcommand\lam{\lambda}

\newcommand\codeTs[2]{\langle{#1}\rangle{\textbf{\textasciicircum}{#2}}}

\newcommand\fordo[2]{\textbf{for}~{#1}~\textbf{to}~{#2}~\textbf{do}}
\newcommand\cfordo[2]{\underline{\textbf{for}}~{#1}~\underline{\textbf{to}}~{#2}~\underline{\textbf{do}}}
\newcommand\cfor[1]{\underline{\textbf{for}}~{#1}}

\newcommand\too{\leadsto^*}
\newcommand\downtoo{\rotatebox{-90}{$\leadsto^*$}}
\newcommand\pink[1]{\textcolor{pink}{#1}}
\newcommand\red[1]{\textcolor{red}{#1}}
\newcommand\green[1]{\textcolor{green}{#1}}
\newcommand\magenta[1]{\textcolor{magenta}{#1}}
\newcommand\blue[1]{\textcolor{blue}{#1}}

\newcommand\fun[2]{\lambda{#1}.{#2}}

\newcommand\Resetz{\textbf{reset0}}
\newcommand\Shiftz{\textbf{shift0}}
\newcommand\Throw{\textbf{throw}}
\newcommand\resetz[1]{\Resetz~{#1}}
\newcommand\shiftz[2]{\Shiftz~{#1}\to{#2}}
\newcommand\throw[2]{\Throw~{#1}~{#2}}

\newcommand\cfun[2]{\underline{\lambda}{#1}.{#2}}
\newcommand\ccfun[2]{\underline{\underline{\lambda}}{#1}.{#2}}

\newcommand\cResetz{\underline{\textbf{reset0}}}
\newcommand\cShiftz{\underline{\textbf{shift0}}}
\newcommand\cThrow{\underline{\textbf{throw}}}
\newcommand\cresetz[1]{\cResetz~{#1}}
\newcommand\cshiftz[2]{\cShiftz~{#1}\to{#2}}
\newcommand\cthrow[2]{\cThrow~{#1}~{#2}}

\newcommand\cPlus{\underline{\textbf{+}}}
\newcommand\Plus{\textbf{+}}

\newcommand\cLet{\underline{\textbf{let}}}
\newcommand\cIn{\underline{\textbf{in}}}
\newcommand\clet[3]{\cLet~{#1}={#2}~\cIn~{#3}}
\newcommand\csp[1]{\texttt{\%}{#1}}
\newcommand\cint{\underline{\textbf{int}}}
\newcommand\code[1]{\texttt{<}{#1}\texttt{>}}
\newcommand\codebegin{\texttt{<}}
\newcommand\codeend{\texttt{>}}

\newcommand\intT{\mbox{\texttt{int}}}
\newcommand\boolT{\mbox{\texttt{bool}}}

\newcommand\codeT[2]{\langle{#1}\rangle^{#2}}
\newcommand\funT[3]{{#1} \stackrel{#3}{\rightarrow} {#2}}
\newcommand\contT[3]{{#1} \stackrel{#3}{\Rightarrow} {#2}}

\newcommand\ord{\ge}

\newcommand\Let{\textbf{let}}
\newcommand\In{\textbf{in}}
\newcommand\letin[3]{\Let~{#1}={#2}~\In~{#3}}

\newcommand\iif{\textbf{if}}
\newcommand\then{\textbf{then}}
\newcommand\eelse{\textbf{else}}
\newcommand\ift[3]{\textbf{if}~{#1}~\textbf{then}~{#2}~\textbf{else}~{#3}}
\newcommand\cif[3]{\underline{\textbf{if}}~\code{{#1}}~\code{{#2}}~\code{{#3}}}
\newcommand\cIf{\underline{\textbf{if}}}

\newcommand\fix{\textbf{fix}}
\newcommand\cfix{\underline{\textbf{fix}}}

\newcommand\lto{\leadsto}
\newcommand\cat{~\underline{@}~}

\newcommand\ksubst[2]{\{{#1}\Leftarrow{#2}\}}

\newcommand\cFor{\underline{\textbf{for}}}
\newcommand\forin[2]{\textbf{for}~{#1}~\textbf{to}~{#2}~\textbf{do}}
\newcommand\cforin[2]{\underline{\textbf{for}}~{#1}~\underline{\textbf{to}}~{#2}~\underline{\textbf{do}}}
\newcommand\cArray[1]{\underline{[{#1}]}}
\newcommand\cArrays[2]{\underline{[{#1}][{#2}]}}
\newcommand\aryset[3]{{#1}[{#2}]\leftarrow {#3}}
\newcommand\caryset[3]{\underline{\textbf{aryset}}~{#1}~{#2}~{#3}}
\newcommand\set{\underline{\textbf{set}}}

% コメントマクロ
\newcommand\kam[1]{\red{kam said: {#1}}}
\newcommand\oishi[1]{\blue{oishi said: {#1}}}

\theoremstyle{break}

\newtheorem{theo}{定理}[section]
\newtheorem{defi}{定義}[section]
\newtheorem{lemm}{補題}[section]

\algnewcommand\algorithmicforeach{\textbf{for each}}
\algdef{S}[FOR]{ForEach}[1]{\algorithmicforeach\ #1\ \algorithmicdo}


\newcommand\smallerscope[2]{#1 \ord #2}
\newcommand\greaterscope[2]{#2 \ord #1}
\newcommand\longer[2]{{#1} \ord {#2}}
% \newcommand*\defeq{\stackrel{\text{def}}{=}}
\newcommand\Int{\mbox{\texttt{Int}}}
\newcommand\Bool{\mbox{\texttt{Bool}}}

\newcommand\uni{\cup} % !!! 現在の順序では「∪」

\overfullrule=0pt

\begin{document}

\begin{center}
  Oishi Type System に対する型制約「解消」アルゴリズム  \\
  2017/1/21
\end{center}

\section{解きたい型制約}

以下の構文を持つclassifier式を考える。

\begin{align*}
e & ::= d \mid \gamma_x \mid e \cup e 
\end{align*}

ここで、$d$ は定数 (固有変数条件をみたすclassifier変数), $\gamma_x$は
classifierをあらわす変数、$e$は classifier式。
(Oishi論文では、classifier式は$\gamma$と書いていたが、$\gamma_x$と
紛らわしいので、ここでは$e$と書くことにする。)
また、型推論の対象となった式の一番外側でのclassifier も定数と考えられ
るので、それを$d_0$とする。

解きたい制約は、以下の形である。
\[
\Delta \models C
\]
ただし、 $\Delta$ は、$d \ge e$ の形の不等式を有限個「かつ」でつなげた
ものである。
また、すべての式$d_i$ ($i>0$)に対して$d_i \ge d_o$が $\Delta$に含まれ
ているとしてよい。
また、$C$は、$e \ge e$ の形の不等式を「かつ」でつなげたものである。

なお、仮定$\Delta$ は、(code-lambdaルール等で) 固有変数を導入する際の
$d \ge e$という形の不等式のみがはいっているので、左辺が必ず $d$である
ことに注意せよ。


制約解消問題 $ \Delta \models C $ に対して、
$[\gamma_x := e,\cdots,\gamma_x:=e]$ という形の代入$\theta$で、
$ \Delta\theta \models C\theta $ が(順序に関する規則のみで)導けるとき、
この$\theta$を解と呼ぶ。そのような$\theta$が存在しないとき、解がないと
いう。

\section{代数的な準備}

Upper Semilattice (上むき半束)と Join-Irreducible いう言葉を導入する。

まず、lattice(束)とは、順序集合で、$\cap$と$\cup$があるものである。
今回の型推論では、$\cup$はあるが$\cap$はないので、latice ではない。
「上の方向だけ(半分)、lattice の構造がある」という意味で、
semilattice (半束)あるいは、もっと正確には upper semilattice(上向き半
束)という言葉を使う。

要素$e$が、join-irreducible (joinについて還元可能でない)であるとは、

ほかのどんな要素$e_1,e_2$に対しても、$e_1 \cup e_2 \ge e$ ならば、
$e_1 \ge e$ または
$e_2 \ge e$ であることを言う。\footnote{ただし、こまかいことだが、
最小元である$d_0$ は、通常は、join-irreducibleとは言わない。}

今回型推論をおこなう対象の構造は、Upper Semilattice であって、
かつ、すべてのclasifier定数$d$(もともとは固有変数だったもの)が
join-irreducible としてよい。
(理由: $\Delta$ は $d\ge e$の形だけからなる。
$\Delta$から$e_1 \cup e_2 \ge d$を導く証明のサイズに関する帰納法で、
$\Delta$から$e_1 \ge d$または$e_2 \ge d$が導けることが言える。)

\section{制約解消アルゴリズム}

$\Delta$と$C$と代入$\theta$に対して、以下の手順を適用する。
代入$\theta$の初期値は、空代入とする。

\begin{itemize}
\item $C$ が空であるならば、$\theta$を返して終了する。
\item $(e \ge e) \in C$ (両辺が同じ不等式)ならば、
$C$から、この不等式を除去して再帰する。
\item $(d_1 \ge d_2) \in C$ (両辺が異なるclassifier定数)のとき、
さらに、$\Delta$のもとで、$d_1 \ge d_2$ が導けるのであれば、
この不等式を除去して再帰する。
導けないのであれば、「制約解消失敗」を返して終了する。
\item $e_1 \ge e_2 \cup e_3) \in C$ ならば、
$C$から$e_1 \ge e_2 \cup e_3$を除去し、
$e_1 \ge e_2$ と $e_1 \ge e_3$を追加して再帰する。
\item $(\cdots \cup \gamma_x \cup \cdots) \in \gamma_x$ 
ならば、この不等式を除去して再帰する。
\item $(e_1 \cup e_2 \ge d) \in C$ ならば、
join-irreducible の仮定(*)を用いて、
$e_1 \ge d$ または $e_2 \ge d$である。
よって、
$C - \{(e_1 \cup e_2 \ge d)\} \cup \{(e_1 \ge d)\}$に対して再帰し、
さらに、
$C - \{(e_1 \cup e_2 \ge d)\} \cup \{(e_2 \ge d)\}$に対して再帰し、
それらの解の両方を解として返す。(一意的には解けない。解は一般的には複数ある。)
\item 上記のいずれのケースもあてはまらないが、$C$が空でないときは、
残ったclassifier変数$\gamma_x$ をひとつ取り出し、$C$の中から以下の形の不等式
をすべて列挙する。

\begin{align*}
e_i & \ge \gamma_x  ~~ & \text{for}~ i=1,2,,...m\\
\gamma_x & \ge d_j  ~~ & \text{for}~ j=1,2,,...n
\end{align*}

ただし $m$や$n$は0かもしれない。また、上記の変形が終了しているので、
すべての$e_i$ は $\gamma_x$は含まないことに注意。

\begin{itemize}
\item $m=0$かつ$n>0$のとき、$\gamma_x:=d_1 \cup \cdots \cup d_n$という
代入を$\theta$に追加して、$C$からは$\gamma_x$に関する不等式をすべて取
り除いて再帰する。
\item $m>0$かつ$n=0$のとき、$\gamma_x:=d_0$という
代入を$\theta$に追加して、$C$からは$\gamma_x$に関する不等式をすべて取
り除いて再帰する。
\item $m>0$かつ$n>0$のとき、
$C$から$\gamma_x$に関する不等式をすべて取り除いた上で、以下の不等式を
追加する。
\[
e_i \ge d_j   ~~\textrm{(すべての}~i,j~\textrm{に対して)}
\]
さらに、$\gamma_x:=d_1 \cup \cdots \cup d_n$という代入を$\theta$に追加
した上で再帰する。
\end{itemize}
\end{itemize}

上記の手順を繰返すと(この繰返しは、かならず停止することがいえる)、
最終的に、(途中でfailしないかぎり)$\gamma_x$変数がすべて消去され、制約$C$が空になり、
制約を満たす解の有無が判定できる。


\end{document}

