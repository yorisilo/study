\documentclass[9pt,a4paper,onecolumn]{jarticle}
%
\usepackage[dvipdfmx]{graphicx,color,hyperref}
\usepackage{amsmath,amssymb,mathrsfs,amsthm}
\usepackage{ascmac}

\usepackage{centernot}
\usepackage{fancybox}
\usepackage{verbatim}
\usepackage{jtygm}
\usepackage{listings,jlisting}
\usepackage{here,txfonts}
\usepackage{url}
\usepackage{bussproofs}
\usepackage{latexsym}
\usepackage{bm}

\usepackage[margin=3cm]{geometry}

\usepackage{setspace}
% \setstretch{1.08} % ページ全体の行間を設定
% \setlength{\textheight}{38\baselineskip}
\setlength{\columnsep}{12mm}

\theoremstyle{definition}
\newtheorem{theo}{定理}[section]
\newtheorem{defi}{定義}[subsection]
\newtheorem{lemm}{補題}[section]
\renewcommand\proofname{\bf 証明}

\newcommand{\cn}{\centernot}
\newcommand{\la}{\lambda}
\newcommand{\ri}{\longrightarrow}
\newcommand{\map}{\mapsto}
\newcommand{\id}{\text{id }}

\title {\vspace{-3cm}全体ゼミ}
\date{20151208}
\author{筑波大学 プログラム論理研究室 \\ 大石 純平}
% \pagestyle{empty}

\begin {document}
\maketitle
% \thispagestyle{empty}

%\input {readme}

\section{論文紹介します}

\subsection*{読んだ論文}
\begin{itemize}
\item Subtyping Delimited Continuations 2011
\item Marek Materzok, Dariusz Biernacki
\end{itemize}

\subsection*{概要}

Danvy と Filinski のshit/resetを一般化した shift0/reset0 という first-class な限定継続の部分型づけをもつ型システムを提案した.

\subsection*{論文紹介の目的}
\begin{itemize}
\item shift0/reset0 を含んだラムダ計算の体系の紹介
\item answer type や subtyping をもつ型システムの紹介
\end{itemize}

\section{shift/resetの型システム}
Danvy and  Filinski の 型判断は以下のようになっている.

\begin{align}
  \Gamma; \gamma \vdash e : \alpha; \delta
\end{align}
これは,型環境 $\Gamma$ において,$e$ は,$\alpha$ 型を持ち,
context は $\alpha \to \gamma$ 型であり,
answer type を $\gamma$ から $\delta$ へ変える ということを意味する.

また,以下の関数の型について説明する.

\begin{align}
  \beta; \gamma \to \alpha; \delta
\end{align}
これは,$\beta$型の値が与えられて,$\alpha \to \gamma$ というcontext で,評価されると,$\delta$ 型の answer type を得るような関数である.

次に,reset, shift の型付け規則をみてみる.
ホワイトボードで説明する.

\begin{align}
  \frac{\Gamma; \alpha \vdash e : \alpha; \gamma}{\Gamma; \delta \vdash <e> : \gamma; \delta} \text{RESET}
\end{align}

\begin{align}
  \frac{\Gamma; f:(\alpha; \rho \to \gamma; \rho); \beta \vdash e: \beta; \delta}{\Gamma; \gamma \vdash S f.e : \alpha; \delta} \text{SHIFT}
\end{align}

\section{shift0/reset0の型システム}
shift0/reset0 の場合は,アクセスできるのは top context だけでないので,より多くのcontextの情報が必要となる.なので,以下の様なanotationを導入する.

\begin{align}
   anotation \,\, \sigma &::= \epsilon \, |  \, [\tau \sigma] \tau \sigma \\
   type \,\, \tau &::= \alpha \, | \, \tau \overset{\delta} \to \tau
\end{align}

型判断
\begin{align}
  \Gamma \vdash e : \tau'[\tau_1 \sigma_1] ... \tau_n'[\tau_n \sigma_n] \tau \epsilon
\end{align}

%\section{まとめ}


\end{document}

% 参考 http://qiita.com/mountcedar/items/e7603c2eb65661369c3b

%%% Local Variables:
%%% mode: japanese-latex
%%% TeX-master: t
%%% End:
